\documentclass[a4paper,15pt]{article}
\usepackage{amssymb}
\usepackage{amsmath}
\usepackage[english, polish]{babel}
\usepackage[utf8]{inputenc}   % lub utf8
\usepackage[T1]{fontenc}
\usepackage{graphicx}
\usepackage{anysize}
\usepackage{enumerate}
\usepackage{times}
\usepackage{caption}
\usepackage{titlesec}
\usepackage{float}
\usepackage{titleps,kantlipsum}
\usepackage{listings}
\usepackage{xcolor}
\usepackage{hyperref}
\usepackage{framed}
\usepackage{tcolorbox}
\lstloadlanguages{Matlab}
 
\usepackage[justification=centering]{caption}
\titlelabel{\thetitle.\quad}

\pagenumbering{arabic}

\DeclareCaptionFont{white}{\color{white}}
\DeclareCaptionFormat{listing}{%
  \parbox{\textwidth}{\colorbox{darkgreen}{\parbox{\textwidth}{#1#2#3}}\vskip-4pt}}
\captionsetup[lstlisting]{format=listing,labelfont=white,textfont=white}
\lstset{frame=lrb,xleftmargin=\fboxsep,xrightmargin=-\fboxsep}

% Definicja nowego stylu strony
\newpagestyle{mypage}
{
  \headrule
  
  \sethead
  { \MakeUppercase{\thesection\quad \sectiontitle} } 
  {}
  {\thesubsection\quad \subsectiontitle}
  
  \setfoot
  {}
  {}
  {\thepage}
}

\newpagestyle{mypage_1}
{
	\headrule
	
	\sethead
	{  }
	{\MakeUppercase{Systemy Operacyjne - Laboratorium}}
	{}
	
	\setfoot
	{}
	{\thepage}
	{}
}

\settitlemarks{section,subsection,subsubsection}

\pagestyle{mypage_1}

\newcommand{\ask}[2]{
    \begin{tcolorbox}[colback=black!5!white,colframe=gray,title={Pytanie #1}]
        #2
    \end{tcolorbox}
}

\newcommand{\assignment}[2]{
    \begin{tcolorbox}[colback=black!5!white,colframe=black,title={Zadanie #1}]
        #2
    \end{tcolorbox}
}

%\marginsize{left}{right}{top}{bottom}
\marginsize{3cm}{3cm}{3cm}{3cm}
\sloppy
\titleformat{\section}
  {\normalfont\Large\bfseries}{\thesection}{1em}{}[{\titlerule[0.8pt]}]
 
 \definecolor{darkred}{rgb}{0.9,0,0}
\definecolor{grey}{rgb}{0.4,0.4,0.4}
\definecolor{orange}{rgb}{1,0.6,0.05}
\definecolor{darkgreen}{rgb}{0.2,0.5,0.05}
 
\definecolor{mGreen}{rgb}{0,0.6,0}
\definecolor{mGray}{rgb}{0.5,0.5,0.5}
\definecolor{mPurple}{rgb}{0.58,0,0.82}
\definecolor{mKeyword}{RGB}{0,0,242}
\definecolor{backgroundColour}{RGB}{242,242,242}

\lstdefinestyle{CStyle}{
    backgroundcolor=\color{backgroundColour},   
    commentstyle=\color{mGreen},
    keywordstyle=\color{mKeyword},
    numberstyle=\tiny\color{mGray},
    stringstyle=\color{mPurple},
    basicstyle=\footnotesize,
    breakatwhitespace=false,         
    breaklines=true,                 
    %captionpos=b,                    
    keepspaces=true,                 
    numbers=left,                    
    numbersep=5pt,                  
    showspaces=false,                
    showstringspaces=false,
    showtabs=false,                  
    tabsize=2,
    language=C
}


\newcommand{\Hilight}{\makebox[0pt][l]{\color{cyan}\rule[-4pt]{0.65\linewidth}{14pt}}}


\begin{document}

\begin{table}
\begin{center}
\begin{tabular}{|c|c|c|}
\hline
\multicolumn{3}{|c|}{\textbf{Procesy 2 - Procesy i wątki}} \\ \hline Dominik Wróbel & 04 IV 2019 & Czw. 17:00 \\ \hline

\end{tabular}
\end{center}
\end{table}

\tableofcontents

\newpage
\section{Tworzenie wątku}


\subsection{Tworzenia wątku}
\assignment{1}{Zmodyfikuj program tak by tworzył 10 wątków, z których każdy wypisze swój numer przesłany jako argument wywołania funkcji rozpoczęcia.}


\begin{lstlisting}[style=CStyle, label=some-code, caption=Zadania 1 - hello.c]
    #include <pthread.h>
    #include <stdio.h>
    #include <stdlib.h>
    // ----------------------------------------------------------
     
    void *PrintHello(void *arg);
    // ----------------------------------------------------------
     
    int main(int argc, char *argv[]){

         const int THREADS_NUM = 10;
         pthread_t threads[THREADS_NUM];
         int i;

         for(i = 0; i < THREADS_NUM ; i++){ // tworzenie 10 watkow
            
            int rc = pthread_create(&threads[i], NULL, PrintHello, (void*)i);
            if (rc){
                printf("Return code: %d\n", rc);
                exit(-1);
            }
            sleep(1);

         }

        
         return 0;
    }
    // ----------------------------------------------------------
     
    void *PrintHello(void *arg){
         printf("Next boring 'Hello World!' version! Thread number %d\n", ((int*) arg));
         return NULL;
}
\end{lstlisting}

\subsection{Czekanie na zakończenie wątku}
\assignment{2}{Zmodyfikuj program z pierwszego zadania w taki sposób, aby wątek główny (main) czekał na zakończenie pracy przez pozostałe wątki. Aby przetestować poprawność rozwiązania zmień kod poprzedniego programu w następujący sposób:
\begin{itemize}
\item Usuń wywołanie sleep(1); z funkcji main.
\item Dodaj identyczne wywołanie sleep(1); na samym początku funkcji rozpoczęcia.
\item Bezpośrednio przed instrukcją return w funkcji main wypisz informację o zakończeniu wątku głównego.
\item Zmodyfikuj program w celu oczekiwania na zakończenie wątku. Poprawnie zmodyfikowany program powinien wyświetlać taki oto komunikat: \\ 
Next boring 'Hello World!' version! \\
End of the main thread!
\end{itemize}
}

\begin{lstlisting}[style=CStyle, label=some-code, caption=Zadanie 2 - hello.c]
 #include <pthread.h>
    #include <stdio.h>
    #include <stdlib.h>
    // ----------------------------------------------------------
     
    void *PrintHello(void *arg);
    // ----------------------------------------------------------
     
    int main(int argc, char *argv[]){

         const int THREADS_NUM = 10;
         pthread_t threads[THREADS_NUM];
         int i;

         for(i = 0; i < THREADS_NUM ; i++){
            
            int rc = pthread_create(&threads[i], NULL, PrintHello, (void*)i);
            if (rc){
                printf("Return code: %d\n", rc);
                exit(-1);
            }
            
         // sleep(1); I - usun wywolanie sleep(1) z main
         }

        for(i = 0 ; i < THREADS_NUM; i++){ // IV - zmodyfikuj program aby czekal na koniec watkow
            pthread_join(threads[i], NULL); // wait for a thread to finish
        }

         printf("End of the main thread!\n"); // III - przed return wypisz info. o zakoncz. watku glow.
        
         return 0;
    }
    // ----------------------------------------------------------
     
    void *PrintHello(void *arg){
        sleep(1); // II -dodaj identyczne wywolanie sleep(1) na poczatku f. rozpoczecia
         printf("Next boring 'Hello World!' version! Thread number %d\n", ((int*) arg));
         return NULL;
}
\end{lstlisting}

W wyniku działania programu uzyskano następujący wynik:
\begin{lstlisting}[style=CStyle, label=some-code, caption=Zadanie 2 - hello.c - wynik dzialania]
Next boring 'Hello World!' version! Thread number 6
Next boring 'Hello World!' version! Thread number 7
Next boring 'Hello World!' version! Thread number 5
Next boring 'Hello World!' version! Thread number 8
Next boring 'Hello World!' version! Thread number 9
Next boring 'Hello World!' version! Thread number 4
Next boring 'Hello World!' version! Thread number 3
Next boring 'Hello World!' version! Thread number 2
Next boring 'Hello World!' version! Thread number 1
Next boring 'Hello World!' version! Thread number 0
End of the main thread!
\end{lstlisting}

\newpage
\subsection{Synchronizacja wątków}
\assignment{3}{
... \\ 
Uzupełnij poniższy szkielet programu, tak aby realizował powyższą funkcjonalność:
}

\begin{lstlisting}[style=CStyle, label=some-code, caption=Zadanie 3 - func.c]
 #include <pthread.h>
    #include <stdio.h>
    #include <stdlib.h>
     
    #define NUM 4
    #define LENGTH 100
    // ----------------------------------------------------------
     
    typedef struct {
         long* a;
         long sum; 
         int veclen; 
    } CommonData;
    // ----------------------------------------------------------
     
    CommonData data; 
    pthread_t threads[NUM];
    pthread_mutex_t mutex;
     
    void* calc(void* arg); // Funkcja rozpoczecia
    // ----------------------------------------------------------
     
    int main (int argc, char *argv[]){
     
         int threadCreated;
         long i, sum = 0;
         void* status;
         long* a = (long*) malloc (NUM*LENGTH*sizeof(long));     
         pthread_attr_t attr;
     
         //Prepare data structure
         for (i=0; i<LENGTH*NUM; i++) {
              a[i] = i;
              sum += i;
         }
     
         data.veclen = LENGTH; 
         data.a = a; 
         data.sum = 0;
     
         //mutex initialization
         pthread_mutex_init(&mutex, NULL);
     
         //[1] setting thread attribute
         pthread_attr_init(&attr);
         pthread_attr_setdetachstate(&attr, PTHREAD_CREATE_JOINABLE);
     
         for(i=0;i<NUM;i++){ // stworzeniu NUM liczby watkow
            threadCreated = pthread_create(&threads[i], &attr, calc, (void*) i); // tworzenie watkow
         }
     
         //join
         for(i=0;i<NUM;i++) {
              pthread_join(threads[i], &status);
         }

        //[1] destroy - not needed anymore
         pthread_attr_destroy(&attr);
     
         //Print
         printf ("Correct result is: %ld \n", sum);
         printf ("Function result is: %ld \n", data.sum);
     
         //Clean
         free (a);
         pthread_mutex_destroy(&mutex);
         return 0;
    }
    // ----------------------------------------------------------
     
    void* calc(void* arg)
    {
         long* x = data.a;
         long mysum = 0;
         int i;

         int partitionNumber = (int) arg;

         printf("%d  ", partitionNumber);

         int startIndex = partitionNumber * LENGTH;
         int endIndex = startIndex + LENGTH;
 
        printf("s: %d, e: %d", startIndex, endIndex);

         for (i=startIndex; i<endIndex; i++){
              mysum += x[i];
         }
     
        printf("%ld\n", mysum);
        // sekcja krytyczna
         pthread_mutex_lock(&mutex);
         data.sum += mysum;
         pthread_mutex_unlock(&mutex);
        // koniec sekcji krytycznej
     
         pthread_exit((void*) 0);
    }
// ----------------------------------------------------------
\end{lstlisting}

W wyniku działania programu uzyskano: 
\begin{lstlisting}[style=CStyle, label=some-code, caption=Zadanie 3 - func.c - wynik dzialania]
3  s: 300, e: 40034950
2  s: 200, e: 30024950
1  s: 100, e: 20014950
0  s: 0, e: 1004950
Correct result is: 79800 
Function result is: 79800 
\end{lstlisting}



\newpage
\subsection{Zmienne warunkowe}
\assignment{4}{
Rozbuduj poniższy program, który ma realizować prostą funkcjonalność:
\begin{itemize}
\item 2 wątki inkrementują (funkcja increment) wartość zmiennej globalnej globalvariable.
\item Trzeci wątek (funkcja printinfo) oczekuje na sygnał, aby oznajmić, że osiągnięto żądaną wartość MAXVAL.
\item Po wypisaniu informacji wszystkie wątki i cały program kończą działanie.
\end{itemize}
}
\begin{lstlisting}[style=CStyle, label=some-code, caption=Zadanie 4 - cond.c]
#include <pthread.h>
    #include <stdio.h>
    #include <stdlib.h>
    // ----------------------------------------------------------
     
    #define MAXVAL 100
     
    int globalvariable = 0;
    pthread_mutex_t mutex;
    pthread_cond_t cond;
     
    void* increment(void*);
    void* printinfo(void*);
    // ----------------------------------------------------------
     
    int main(){
     
         pthread_t t1, t2, t3;
         pthread_attr_t attr;
     
         // mutex initialization
         pthread_mutex_init(&mutex, NULL);
     
         // conditional initialization
         pthread_cond_init(&cond, NULL);
     
         pthread_attr_init(&attr);
         pthread_attr_setdetachstate(&attr, PTHREAD_CREATE_JOINABLE);
     
         pthread_create(&t1, &attr, increment, NULL); 
         pthread_create(&t2, &attr, increment, NULL);
         pthread_create(&t3, &attr, printinfo, NULL);  
     
         pthread_join(t1, NULL);
         printf("t1 finished!\n");
         pthread_join(t2, NULL);
         printf("t2 finished!\n");
         pthread_join(t3, NULL);
         printf("t3 finished!\n");
     
         printf("Finishing...\n");
         return 0;
    }
    // ----------------------------------------------------------
     
    void* increment(void* arg) {

        while(1){
            pthread_mutex_lock(&mutex); // zamknij Mutex

            if(globalvariable == MAXVAL){ // jesli osiagnieta wartosc max
                pthread_cond_signal(&cond); // poinformuj watek wypisujacy
                pthread_mutex_unlock(&mutex); // otworz mutex
                break;
            }

            globalvariable++; // jesli nie osiagnieto max to inkrementuj
            pthread_mutex_unlock(&mutex); // otworz mutex po inkrementacji
        }

        

         pthread_exit((void*) 0);
    }
    // ----------------------------------------------------------
     
    void* printinfo(void* arg) {
        int val;
        int i;

        pthread_mutex_lock(&mutex); // zamknij mutex

        while(globalvariable < MAXVAL)
            pthread_cond_wait(&cond, &mutex); // czekaj az dwa watki skoncza

        pthread_mutex_unlock(&mutex); // otworz mutex

        printf("Adding values completed, globalvariable is %d \n", globalvariable);

        pthread_exit((void*) 0);
    }
// ----------------------------------------------------------
\end{lstlisting}

W wyniku działania programu uzyskano wynik:

\begin{lstlisting}[style=CStyle, label=some-code, caption=Zadanie 4 - cond.c - wynik dzialania]
Adding values completed, globalvariable is 100 
t1 finished!
t2 finished!
t3 finished!
Finishing...
\end{lstlisting}

\newpage
\subsection{Kasowanie wątku}
\assignment{5}{
Uzupełnij poniższy program o:
\begin{itemize}
\item Przesłanie do tworzonych wątków argumentów (wykorzystaj strukturę thread\_params) zawierających informację o: Indeksie wątku, który jest równy wartości ti w pętli tworzącej wątki. Wartości szukanej.
\item Informację o liczbie iteracji jakie wykonał każdy kończący się wątek w celu odnalezienia wartości (wykorzystaj mechanizmy czyszczące pthread\_ cleanup\_ push, pthread\_ cleanup\_ pop ).
\end{itemize}
}

\begin{lstlisting}[style=CStyle, label=some-code, caption=Zadanie 5 - randomsearch.c]
#include <stdio.h>
    #include <unistd.h>
    #include <stdlib.h>
    #include <sys/types.h>
    #include <pthread.h>
    #include <errno.h>
    // ----------------------------------------------------------
     
    #define NUM_THREADS 5
    #define TARGET 100
    // ----------------------------------------------------------
     

  


    struct thread_params {
         int target; // I. wartosc szukana
         int thread_idx; // I. indeks watku
         int threadNumber;
         int *tries; // liczba iteracji
    };
    // ----------------------------------------------------------


   void thread_cancel_handler(void * arg){ // funkcja wolana gdy watek ginie
        struct thread_params * tp = (struct thread_params*) arg;
        printf("Thread %d, iteration %d\n" , tp->threadNumber, *(tp->tries));
    }

     
    struct thread_params tp[NUM_THREADS];
    pthread_t threads[NUM_THREADS];
    pthread_mutex_t mutex;
    int tries;
     
    void *search(void *arg);
    void cleanup(void *args);
    // ----------------------------------------------------------
     
    int main (int argc, char *argv[]){
         int ti;
         int target=TARGET;
     
         tries = 0;
         pthread_mutex_init(&mutex, NULL); 
     
         printf("Searching for: %d\n", target);
     
         for (ti=0;ti<NUM_THREADS;ti++){
              tp[ti].target = target;
              tp[ti].thread_idx = ti;
              tp[ti].tries = &tries;
              pthread_create(&threads[ti], NULL, search, &(tp[ti]));
         }
     
         for (ti=0;ti<NUM_THREADS;ti++){ 
              pthread_join(threads[ti], NULL);
         }
     
         printf("Number of all iterations: %d.\n", tries);
         pthread_mutex_destroy(&mutex);
         return 0;
    }
    // ----------------------------------------------------------
     
    

    void *search(void *arg){

    struct thread_params * tp = (struct thread_params*) arg;

    pthread_cleanup_push(thread_cancel_handler, &tp); // wykorzystanie mechanizmow czyszczacych
    pthread_cleanup_pop(1);

     int threadIdx = tp->thread_idx; // indeks watku
     int toFind = tp->target;   // wartosc do odszukania
     int ti = 0;
     int rnd;
 
     pthread_t tid = pthread_self();
     srand(tid);
 
     pthread_setcancelstate(PTHREAD_CANCEL_ENABLE, NULL);
     pthread_setcanceltype(PTHREAD_CANCEL_DEFERRED, NULL);
    
 
     while(1){
          while(pthread_mutex_trylock(&mutex) == EBUSY){
               pthread_testcancel();
          }
       
         *(tp->tries)++;

          pthread_mutex_unlock(&mutex);
 
          rnd = (int)(rand()%1000);
          
          if (toFind == rnd) {
               while(pthread_mutex_trylock(&mutex) == EBUSY){
                    pthread_testcancel();
               }
               printf("Number found by %d!\n", threadIdx);
               for(ti=0;ti<NUM_THREADS;ti++){
                    if(ti == threadIdx)      // kasowanie innych watkow - ten watek zakoczy sie normalnie
                         continue;
                    pthread_cancel(threads[ti]);
               }
               sleep(1);
               pthread_mutex_unlock(&mutex);
               break;
          }
     }
 
     return((void *)0);
}
\end{lstlisting}




\end{document}
