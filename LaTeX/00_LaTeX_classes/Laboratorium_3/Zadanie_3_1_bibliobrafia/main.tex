\documentclass[a4paper,12pt]{article}
\usepackage[polish]{babel}
\usepackage{graphicx}
\usepackage[utf8]{inputenc}
\usepackage[T1]{fontenc}
\usepackage{times}
\usepackage{enumerate}

\begin{document}

Z każdym działającym systemem komputerowym powiązane jest oczekiwanie 
{\em poprawności} jego działania (\cite{Fencott:CCS:95}). Istnieje szeroka 
klasa systemów, dla których poprawność powiązana jest nie tylko z 
wynikami ich pracy, ale również z~czasem, w~jakim wyniki te są 
otrzymywane. Systemy takie nazywane są {\em systemami czasu 
rzeczywistego}, a~ponieważ są one rozpatrywane  w~kontekście swojego 
otoczenia, często określane są terminem {\em systemy wbudowane} 
(\cite{Sommerville10}, \cite{formalne}). 

Ze względu na specyficzne cechy takich systemów, weryfikacja jakości 
tworzonego oprogramowania oparta wyłącznie na jego testach jest 
niewystarczająca. Coraz częściej w~takich sytuacjach, weryfikacja 
poprawności tworzonego systemu lub najbardziej istotnych jego 
modułów prowadzona jest z~zastosowaniem metod formalnych 
(\cite{Alur:1990:AMR:90397.90438}, \cite{formalne}). 

\bibliographystyle{plain} %alpha
\bibliography{bibliografia}


\end{document}