\documentclass[a4paper,11pt]{article}
\usepackage[polish]{babel}
\usepackage[utf8]{inputenc}
\usepackage[T1]{fontenc}
\usepackage{times}
\usepackage{graphicx}
\usepackage{anysize}
\usepackage{amsmath}
\usepackage{color}
\usepackage{listings}
\lstloadlanguages{Ada,C++}

%\marginsize{left}{right}{top}{bottom}
\marginsize{2.5cm}{2.5cm}{2.5cm}{2.5cm}

\definecolor{darkred}{rgb}{0.9,0,0}
\definecolor{grey}{rgb}{0.4,0.4,0.4}
\definecolor{orange}{rgb}{1,0.6,0.05}
\definecolor{darkgreen}{rgb}{0.2,0.5,0.05}


\begin{document}

\lstset{
language=C++,
basicstyle=\ttfamily\small,
keywordstyle=\color{darkgreen}\ttfamily\bfseries\small,
stringstyle=\color{red}\ttfamily\small,
commentstyle=\color{grey}\ttfamily\small,
numbers=left,
numberstyle=\color{darkred}\ttfamily\scriptsize,
identifierstyle=\color{blue}\ttfamily\small,
showstringspaces=false,
morekeywords={}
}

\section{Silnia}
Program przedstawiony na listingu \ref{lis1} jest przykładem obliczania silni przy pomocy pętli while.
\begin{lstlisting}[caption=Przykład obliczania silni przy pomocy pętli while, captionpos=b,
label=lis1, firstnumber=12,frame=single]
#include <iostream>
using namespace std;

int main()
{
  int s, n, i;
  cout << "Prosze podac liczbe naturalna n : ";
  cin >> n;
  i = 1;
  s = 1;
  
  while( i < n ) 
  {
  	i++;
  	s *= i;
  }
  
  cout << "Silnia n: " << s << endl;  	
}
\end{lstlisting}
Program przedstawiony na listingu \ref{lis2} jest przykładem obliczania silni przy pomocy pętli for. 
\begin{lstlisting}[caption=Przykład obliczania silni przy pomocy pętli for, captionpos=b,
label=lis2, firstnumber=12,frame=single]
#include <iostream>
using namespace std;

int main()
{
  int s, n, i;
  cout << "Prosze podac liczbe naturalna n : ";
  cin >> n;
  s = 1;
  
  for(i = 2; i <= n ; i++ ) s *= i;
  
  cout << "Silnia n: " << s << endl;  	
}
\end{lstlisting}
\end{document}

