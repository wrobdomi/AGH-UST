\documentclass[a4paper,11pt]{article}
\usepackage{amssymb}
\usepackage{amsmath}
\usepackage[english, polish]{babel}
\usepackage[utf8]{inputenc}   % lub utf8
\usepackage[T1]{fontenc}
\usepackage{graphicx}
\usepackage{anysize}
\usepackage{enumerate}
\usepackage{times}
\usepackage{titlesec}
\usepackage{float}
\usepackage[justification=centering]{caption}
\titlelabel{\thetitle.\quad}
\usepackage{titlesec}
\usepackage{titleps,kantlipsum}
\usepackage{tikz}

\usetikzlibrary{calc,through,backgrounds,positioning,fit}
\usetikzlibrary{shapes,arrows,shadows}

\tikzstyle{place}=[shape=circle, draw, minimum height=10mm]
\tikzstyle{place_1}=[shape=circle, draw, minimum height=5mm]
\tikzstyle{trig}=[shape=circle, draw, dashed, minimum height=10mm]
\tikzstyle{trans}=[shape=rectangle, draw, minimum height=15mm, minimum width=16mm]


\newpagestyle{mypage}
{
  \headrule
  
  \sethead
  { \MakeUppercase{\thesection\quad \sectiontitle} } 
  {}
  {\thesubsection\quad \subsectiontitle}
  
  \setfoot
  {}
  {\thepage}
  {}
}

\newpagestyle{mypage_1}
{
	\headrule
	
	\sethead
	{  }
	{\MakeUppercase{Teoria Sterowania - Laboratorium}}
	{}
	
	\setfoot
	{}
	{}
	{}
}

\pagestyle{mypage_1}
%\marginsize{left}{right}{top}{bottom}
\marginsize{3cm}{3cm}{3cm}{3cm}
\sloppy
 
\begin{document}
\begin{table}
\begin{center}
\begin{tabular}{|l|l|l|}
\hline
\multicolumn{3}{|c|}{\textbf{Optymalizacja parametryczna}} \\ \hline Dominik Wróbel & \textbf{ 22 V 2018} & \textbf{Wt 09:30} \\ \hline

\end{tabular}
\end{center}
\end{table}
\tableofcontents
\section{Cel ćwiczenia}
Celem ćwiczenia jest zapoznanie się z różnymi sposobami doboru optymalnych nastaw regulatorów. W pierwszej części rozważane będą różne całkowe wskaźniki jakości i zbadane zostaną różnice pomiędzy nimi, podjęta zostanie próba wyznaczenia najlepiej działającego wskaźnika dla tego rozważanego problemu. W dalszej części zadania zostaną wyznaczone nastawy dla których obiekt osiąga zadaną wartość zapasu modułu. 
\section{Przebieg ćwiczenia}
\subsection{Zadanie 1}
\subsubsection{Zadanie 1 - Opis zagadnienia}
W zadaniu rozważany jest obiekt opisany transmitancją:
\begin{equation} \label{eq:transfer}
G(s)=\frac{2s+3}{s^{3}+2s^{2}+s}
\end{equation}
Obiekt znajduje się w systemie z ujemnym sprzężeniem zwrotnym i regulatorem PID. Schemat systemu przedstawia Rysunek \ref{fig:opt_par_1}.

\begin{figure}[H]
\centering
\begin{tikzpicture}[scale=1,inner sep=0.4mm]

\node (sum)[place_1, label=below right:{$-$}] at (-1,0) {};
\node (pid) [trans,label=below:{$PID$}] at (2,0) {$K_{p}+K_{i}\frac{1}{s}+K_{d}s$};
\node (sum1)[place_1, label=below right:{}] at (5,0) {};
\node (objekt) [trans,label=below:{$Obiekt$}] at (6.7,0) {$\frac{2s+3}{s^{3}+2s^{2}+s}$};
% \node (env) [trans] at (3.75,1.5) {$env_{1}$};

% \draw [->] (sum) to (pid);
\path[->] (sum) edge node [above ] {$e(t)$} (pid);
\path[->] (pid) edge node [above ] {$u(t)$} (sum1);
\path[->] (-3,0) edge node [above ] {$r(t)$} (sum);
\path[->] (sum1) edge node [above ] {} (objekt);
\path[->] (objekt) edge node [above ] {$y(t)$} (10,0);
\path[->] (5,2) edge node [left ] {$z(t)$} (sum1);
%\path[->]  (10,0) - (10,-3) - (-1,-3) - (sum);
\draw [->] (8,0) -- (8,-2) -- (-1,-2) -- (sum);
\end{tikzpicture}
\caption{Schemat rozważanego układu regulacji}
\label{fig:opt_par_1}
\end{figure}
\pagestyle{mypage}
Dla przedstawionego układu regulacji postawiony jest problem znalezienia optymalnych nastaw regulatora PID minimalizujących przyjęty wskaźnik jakości. 

Rozważane są wskaźniki: \\
\begin{enumerate}
\item \( J_{1} = \int_{0}^{T}e(t)^{2}dt \)
\item \( J_{2} = \int_{0}^{T}e(t)^{2}tdt \)
\item \( J_{3} = \int_{0}^{T}|e(t)|dt \)
\item \( J_{4} = \int_{0}^{T}|e(t)|tdt \)
\item \( J_{5} = \int_{0}^{T}e(t)dt \)
\end{enumerate}

Dla każdego z powyższych wskaźników zostaną wyznaczone programowo optymalne nastawy regulator PID, a następnie zostaną porównane na przykładzie odpowiedzi skokowej układu na sygnał testowy - skok jednostkowy.

W zadaniu przyjęto następujące założenia:
\begin{itemize}
\item Wskaźnik jakości to jeden ze wskaźników całkowych 1-5 wypisanych powyżej,
\item Wartość zadana to skok jednostkowy,
\item Wartości początkowe dla regulatora i obiektu są zerowe,
\item Brak zakłóceń w układzie
\end{itemize}

\subsubsection{Realizacja programowa optymalizacji parametrycznej}
W celu implementacji optymalizacji parametrycznej posłużono się funkcją \textit{fminsearch} w programie Matlab. Jako jej argument użyto funkcji w której zawarto symulowanie modelu układu regulacji z Simulinka. Model układu przedstawia Rysunek \ref{fig:opt_par_2}. W modelu zawarto obliczenia wszystkich pięciu wskaźników przedstawionych w poprzednim punkcie. 
\begin{figure}[H]
\centerline{\includegraphics[scale=1]{opt_par_2.jpg}}
\caption{Model rozważanego układu regulacji w programie Matlab/Simulink}
\label{fig:opt_par_2}
\end{figure}

Przyjęto czas symulacji \(T=25\) jako czas po którym odchyłki od wartości \(y_{\infty}\) są już relatywnie małe i nie mają większego wpływu na wartość całkowania. Warunki początkowe dla nastaw regulatora to \( K_{p} = 1 \), \(K_{i}=0.1\) i \(K_{d}=0.1\).
\subsubsection{Porównanie znalezionych optymalnych nastaw}
Znalezione nastawy regulatora dla wszystkich wskaźników przedstawia poniższa Tabela.
\begin{table}[H]
\begin{center}
\begin{tabular}{|l|p{2cm}|p{2cm}|p{2cm}|}
\hline
\multicolumn{4}{|c|}{\textbf{Znalezione optymalne nastawy regulatora PID}} 
\\ \hline 
Wskaźnik jakości & \( K_{p} \) & \( K_{i} \) & \( K_{d} \) 
\\ \hline
\( J_{1} = \int_{0}^{T}e(t)^{2}dt \) & \(1,0627\) & \( 0,0859 \) & \( 0,1083\) 
\\ \hline
\( J_{2} = \int_{0}^{T}e(t)^{2}tdt \) & \( 1,0345 \) & \( 0,0653 \) & \( 0,1379 \) 
\\ \hline
\( J_{3} = \int_{0}^{T}|e(t)|dt \) & \( 1,0050 \) & \( 0,0395 \) & \( 0,1172 \) 
\\ \hline
\( J_{4} = \int_{0}^{T}|e(t)|tdt \) & \( 1,0020 \) & \( 0,0282 \) & \( 0,1871 \) 
\\ \hline
\( J_{5} = \int_{0}^{T}e(t)dt \) & \( 4.5897 \) & \( 0,4744 \) & \( 0,1993 \) 
\\ \hline

\end{tabular}
\end{center}
\end{table}

Porównanie odpowiedzi skokowych przy znalezionych optymalnych nastawach regulatora PID przedstawia Rysunek \ref{fig:opt_par_3}.

\begin{figure}[H]
\centerline{\includegraphics[scale=0.37]{opt_par_3.jpg}}
\caption{Porównanie odpowiedzi skokowych dla różnych nastaw regulatora PID}
\label{fig:opt_par_3}
\end{figure}

\newpage
\subsubsection{Wnioski do zadania 1}
Po zbadaniu kilku wariantów wskaźników od razu można zauważyć, że przyjęty wskaźnik wpływa na wartość nastaw, co można od razu stwierdzić, po analizie wyników nastaw z tabeli. \\

Z wykresów na Rysunku \ref{fig:opt_par_3} od razu widać, że największe oscylacje otrzymano przy zastosowaniu wskaźnika, który brał pod uwagę tylko uchyb. Wynika to z faktu, że wskaźnik ten w ogólnym  przypadku rozróżnia znak, a więc całka dla ujemnych i dodatnich wartości 'znosi się' i wartość wskaźnika jest zafałszowana.  \\

Znacznie lepiej wyglądają przebiegi uzyskane dla wskaźnika całki z kwadratu uchybu zarówno z czasem jak i bez. Dzieję się ten tak ponieważ wskaźnik ten naprawia wadę wskaźnika, który bierze pod uwagę tylko sam uchyb poprzez podniesienie wartości do kwadratu, a tym samym nie bierze pod uwagę znaku, a jedynie wartość odchyłki. Co więcej widać, że gdy wskaźnik bierze pod uwagę także czas, to znalezione nastawy spowodowały, że dla późniejszych chwil czasu amplituda w stosunku do przebiegu dla parametrów wskaźnika bez czasu maleje tym bardziej względem amplitudy wskaźnika bez czasu im późniejsza chwila.   \\

Najlepiej ze wszystkich przebiegów wygląda przebieg dla wskaźnika biorącego pod uwagę moduł z uchybu. Wskaźnik ten wydaje się być najlepszym w praktyce, ponieważ neutralizuje wadę wskaźnika biorącego pod uwagę tylko uchyb, a jednocześnie nie fałszuje wartości uchybu tak jak robi to wskaźnik kwadratowy poprzez podniesienie do kwadratu. Dodając do wskaźnika czas otrzymujemy podobne zjawisko jak przy analizie wskaźnika kwadratowego. 

\subsection{Zadanie 2}
\subsubsection{Zadanie 2 - Opis zagadnienia}
Problemem w tym zadaniu polega na doborze odpowiedniego wskaźnika jakości, a następnie obliczeniu odpowiednich nastaw dla układu regulacji dla którego chcemy osiągnąć zadany zapas modułu. Zapas ten ma być osiągany dla wyznaczonej częstotliwości. 

W zadaniu przyjęto dane:
\begin{itemize}
\item Zapas modułu \( \Delta L = 20 dB \),
\item Zapas ten osiągany jest dla częstotliwości \( \omega = 100 \) 
\end{itemize}

Analizowanym układem jest ten sam, który był analizowany a Zadaniu 1, jest on przedstawiony na Rysunku \ref{fig:opt_par_1}.

\subsubsection{Przyjęty wskaźnik jakości}
Wskaźnikiem jakości w tym problemie nie jest żaden z typowych wskaźników jakości, ponieważ celem zadania jest osiągnięcie odpowiedniego zapasu modułu układu zamkniętego dla zadanej pulsacji. Chcemy więc aby odległość poziomu modułu od wartości \( -20 dB \) była jak najmniejsza dla pulsacji \( \omega = 100 \), co da zapas modułu równy \( 20 dB \). \\
Aby to osiągnąć przyjęto wskaźnik jakości: 
\begin{equation*}
Q(K_{p},K_{i},K_{d})=|A(\omega) - ( - \Delta L ) | \rightarrow min
\end{equation*}
Wyrażenia \( A(\omega) \) określa poziom modułu transmitancji widmowej  \( G_{F}(s) \) ( zostanie ona wyznaczona w następnym punkcie ) wyrażony w decybelach.

\subsubsection{Wyznaczenie \( A(\omega) \)}
Aby wyznaczyć poziom modułu pozwalający na określenie zapasu stabilności obliczono transmitancję układu otwartego składającego się z regulatora i obiektu:
\begin{equation}\label{eq:2}
G_{F}(s) = G_{o}(s)G_{r}(s) = \frac{(2s+3)}{s^{3}+2s^{2}+s}(K_{p}+K_{i}\frac{1}{s}+K_{d}s)=\frac{(K_{d}s^{2}+K_{p}s+K_{i})(2s+3)}{s(s^{3}+2s^{2}+s)}
\end{equation}
Następnie podstawiając \( s = j \omega = j100\) uzyskano transmitancję widmową. Na podstawie tej transmitancji została wyznaczona część rzeczywista i urojona w zależności od parametrów \( K_{p} \), \(K_{i} \) i \( K_{d} \). Na tej podstawie wyznaczono wielkość \( A( \omega) = 20log_{10} |G_{F}(jw)| \)  Obliczenia te zostały przeprowadzone programowo.

Kolejnym etapem przeprowadzenia obliczeń było napisanie funkcji, która zwraca wartość wskaźnika jakości i zastosowanie optymalizacji przy uzyciu funkcji \textit{fmincon}, na nastawy regulatora narzucono ograniczenie aby były one wartościami dodatnimi.

\subsubsection{Analiza wyniku działania optymalizacji parametrycznej}

W wyniku przeprowadzenia minimalizacji funkcji obliczającej wskaźnik jakości otrzymano następujące nastawy regulatora:
\begin{equation*}
K_{p} = 386,7212 \quad K_{i} = 600,2870, \quad K_{d} = 3,2293
\end{equation*}
Jako, że same wartości tych nastaw nie niosą same w sobie informacji o poprawności działania metody, dla systemu o transmitancji (\ref{eq:2}) z regulatorem PID dla znalezionych optymalnych z punkt widzenia wskaźnika jakości nastaw została narysowana charakterystyka Bodego, którą przedstawia Rysuenk \ref{fig:opt_par_4}.

Na rysunku tym widać, że zamierzony zapas modułu został osiągnięty dla przyjętej pulsacji. Dla porównania ze wskaźnikami całkowymi z Zadania 1 zamieszczono również Rysunek \ref{fig:opt_par_5}, który przedstawia odpowiedź skokową układu zamkniętego.

\begin{figure}[H]
\centerline{\includegraphics[scale=0.5]{opt_par_4.jpg}}
\caption{Charakterystyki Bodego układu o transmitancji \(G_{F}(j \omega )\)  }
\label{fig:opt_par_4}
\end{figure}

\begin{figure}[H]
\centerline{\includegraphics[scale=0.5]{opt_par_5.jpg}}
\caption{Odpowiedź na skok jednostkowy układu zamkniętego}
\label{fig:opt_par_5}
\end{figure}

\subsubsection{Wnioski do zadania 2}
W zadaniu 2 z Rysunku \ref{fig:opt_par_4} widać, że przyjęty wskaźnik jakości okazał się dobry dla zapasu modułu, ale zapas fazy jest o wiele mniejszy, co dowodzi, że aby osiągnąć zarówno zapas modułu jak i fazy należy wziąć pod uwagę we wskaźniku obie te wielkości. Wówczas prawdopodobne jest, że nie byłoby możliwe wyznaczenie nastaw, które w tak dokładny sposób jak w tym przypadku pozwoliłby na osiągnięcie zapasu \(20dB\), ponieważ pod uwagę byłyby brane dwie wielkości i musiałby zostać osiągnięty kompromis pomiędzy nimi.  \\

Wskaźnik jakości przyjmowany dla konkretnej częstotliwości jest dobry dla niewielkiego otoczenia tej częstotliwości - zapas modułu jest równy około 20dB dla niewielkiego otoczenia pulsacji \( \omega = 100 \), można zatem wysunąć wniosek, że przy stosowaniu tego typu podejścia w praktyce konieczne jest także określenie zakresu częstotliwości sygnałów występujących w układzie. \\

W zależności od przyjętego wskaźnika jakości zostaje optymalizowana pewna cecha systemu, co można zaobserwować porównując Rysunki \ref{fig:opt_par_5} oraz \ref{fig:opt_par_3}. W zadaniu 1 gdzie brany był pod uwagę uchyb, nastawy zostały dobrane tak, że odpowiedź skokowa układu szybko zanika do wartości ustalonej wykonując przy tym relatywnie małe oscylacje w porównaniu do oscylacji występujących w układzie z Rysunku \ref{fig:opt_par_5}. 

\end{document}