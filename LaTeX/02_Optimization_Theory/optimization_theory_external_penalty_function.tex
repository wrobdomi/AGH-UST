\documentclass[a4paper,15pt]{article}
\usepackage{amssymb}
\usepackage{amsmath}
\usepackage[english, polish]{babel}
\usepackage[utf8]{inputenc}   % lub utf8
\usepackage[T1]{fontenc}
\usepackage{graphicx}
\usepackage{anysize}
\usepackage{enumerate}
\usepackage{times}
\usepackage{titlesec}
\usepackage{float}
\usepackage{titlesec}
\usepackage{titleps,kantlipsum}
\usepackage{listings}
\usepackage{color}
\lstloadlanguages{Matlab}
 
\usepackage[justification=centering]{caption}
\titlelabel{\thetitle.\quad}



\newpagestyle{mypage}{%
  \headrule
  \sethead{\MakeUppercase{\thesection\quad \sectiontitle}}{}{\thesubsection\quad \subsectiontitle}
  \setfoot{}{}{\thesubsubsection\quad \subsubsectiontitle}
}
\settitlemarks{section,subsection,subsubsection}
\pagestyle{mypage}

%\marginsize{left}{right}{top}{bottom}
\marginsize{3cm}{3cm}{3cm}{3cm}
\sloppy
\titleformat{\section}
  {\normalfont\Large\bfseries}{\thesection}{1em}{}[{\titlerule[0.8pt]}]
 
 \definecolor{darkred}{rgb}{0.9,0,0}
\definecolor{grey}{rgb}{0.4,0.4,0.4}
\definecolor{orange}{rgb}{1,0.6,0.05}
\definecolor{darkgreen}{rgb}{0.2,0.5,0.05}
 
\lstset{
language=Matlab,
basicstyle=\ttfamily\small,
keywordstyle=\color{darkgreen}\ttfamily\bfseries\small,
stringstyle=\color{red}\ttfamily\small,
commentstyle=\color{grey}\ttfamily\small,
numbers=left,
numberstyle=\color{darkred}\ttfamily\scriptsize,
identifierstyle=\color{blue}\ttfamily\small,
showstringspaces=false,
morekeywords={}
}


\begin{document}


\tableofcontents

\begin{table}
\begin{center}
\begin{tabular}{|l|l|l|}
\hline
\multicolumn{3}{|c|}{\textbf{Zewnętrzna funkcja kary}} \\ \hline Dominik Wróbel & \textbf{07 V 2018} & \textbf{Pon 08:00, s. 111} \\ \hline
\multicolumn{2}{|l|}{Numery zadań} & 1, 2 \\ \hline 

\end{tabular}
\end{center}
\end{table}

\section{Cel ćwiczenia}
Celem ćwiczenia jest rozwiązanie zadań optymalizacyjnych z wykorzystaniem podejścia opartego o zewnętrzną funkcje kary. Metoda ta pozwala na rozwiązywanie zadań programowania nieliniowego, polega na generowaniu rozwiązań zadania bez ograniczeń i wprowadzaniu kary za przekroczenie tych graniczeń przy pomocy tzw. zewnętrznej funkcji kary.  \\ Działanie funkcji kary można w uproszczeniu opisać jako funkcję, która ma wartość 0 jeśli dana zmienna \( x \in D \), daje natomiast wartość większą od zera gdy znalezione optimum nie zawiera się w rozwiązaniach dopuszczalnych.
\section{Przebieg ćwiczenia}
\subsection{Zadanie 1 - Minimalizacja funkcjonału z ograniczeniem równościowym }
W zadaniu rozważany jest funkcjonał postaci 
\begin{equation*}
f(x_{1},x_{2})=x_{1}^{2}+x_{2}^{2} \rightarrow min
\end{equation*}  
przy danym ograniczeniu równościowym ( numer na liście n = 16 )
\begin{equation*}
x_{2} = n = 16 
\end{equation*}
Nie trudno zauważyć, że minimum tego funkcjonału przy przyjętym ograniczeniu jest osiągane dla 
\( x = \begin{bmatrix}
& 0 & 16 &
\end{bmatrix} \). 
Zadanie najpierw rozwiązano przy użyciu funkcji matlaba \textit{fmincon}. Funkcja ta znajduje rozwiązanie optymalne przy określonych ograniczeniach nierównościowych. Rozwiązanie zadania przy pomocy tej funkcji odpowiada rozwiązaniu Zadania Programowania Nieliniowego ( ZPN ), które opisuje model matematyczny :
\begin{equation*}
f(x) \rightarrow min
\end{equation*}
\begin{equation*}
x \in X \subset \mathbb{R}^{n}
\end{equation*} 
\begin{equation*}
X = \{ x: g_{i}(x) \leq 0 , \quad i = 1, \dotsc ,m \}
\end{equation*} 
\begin{equation*}
f(x): \mathbb{R}^{n} \rightarrow \mathbb{R}^{1}
\end{equation*}
\begin{equation*}
g_{i}(x): \mathbb{R}^{n} \rightarrow \mathbb{R}^{1} \quad \forall i = 1, \dotsc ,m
\end{equation*}
Następnie zadanie rozwiązano przy użyciu kilku iteracji funkcją \textit{fminunc}. Podejście to odpowiada metodzie zewnętrznej funkcji kary, polega na generowaniu ciągu punktów będących rozwiązaniami zadania minimalizacji bez ograniczeń (ZMB). W zadaniu rozważana jest funkcja 
\begin{equation*}
P(x,k)=f(x)+ k^{j} \cdot \sum_{i=1}^{m}max[0, g_{i}(x)]^{2}
\end{equation*}
z kwadratową funkcją kary:
\begin{equation*}
\phi(x,k) = k^{j} \cdot \sum_{i=1}^{m}max[0, g_{i}(x)]^{2}
\end{equation*}
W ZPN występują ograniczenia nierównościowe z nierównością słabą, na takich ograniczeniach działa też funkcja matlaba \textit{fmincon}, dlatego konieczna jest zamiana ograniczenia równościowego na dwa ograniczenia nierównościowe w następując sposób: 
\begin{equation*}
x_{2} = 1 \iff 
\begin{cases}
x_{2} \leq 1 \\
x_{2} \geq 1
\end{cases} 
\end{equation*}
W implementacji umieszczono następujące wzory :
Funkcja celu: 
\begin{equation*}
f(x) = x_{1}^{2}+x_{2}^{2} \rightarrow min
\end{equation*}
Ograniczenia :
\begin{equation*}
g_{1}(x) = 1 - x_{2} \leq 0
\end{equation*}
\begin{equation*}
g_{2}(x) = x_{2} - 1 \leq 0
\end{equation*}
Dla powyższych ograniczeń funkcja kary z zadania spełnia założenia konieczne do spełnienia przez funkcję kary : 
\begin{itemize}
\item \( \phi(x,k^{j}) = 0 \quad \forall x \in X \)
\item \( \phi(x,k^{j}) > 0 \quad \forall x \not\in X \)
\item \( \phi(x,k^{j}) \rightarrow \infty \quad dla \quad k \rightarrow \infty \quad \forall x \in X \)
\item \( \phi(x,k^{j}) > \phi(x,k^{j+1}) \quad \forall x \not\in X \) 
\end{itemize}
gdzie
\begin{equation*}
\phi(x,k) = k^{j} \cdot max[0, (1-x_{2})]^{2} + k^{j} \cdot max[0, (x_{2} - 1)]^{2}
\end{equation*}
\begin{equation*}
X = \{ (x_{1},x_{2}) : x_{2} = 16 \}
\end{equation*}
Implementacje powyższych obliczeń przedstawiają Listing \ref{lis1} i Listing \ref{lis2}.
\begin{lstlisting}[caption=Znalezienie minimum z ograniczeniami ( fmincon ), captionpos=b,label=lis1, firstnumber=12,frame=single]
% znalezienie minimum z ograniczeniami

x0 = [ 2, 1]; 

fun = @(x)x(1)^2 + x(2)^2;
A = [0,-1;
     0  1];
 
b = [ -1;
       1];

[x_opt, f_opt] = fmincon(fun,x0,A,b);
\end{lstlisting}

\begin{lstlisting}[caption=Znalezienie minimum przy użyciu zewnętrznej funkcji kary ( fminunc ), captionpos=b,label=lis2, firstnumber=12,frame=single]

% znalezienie minimum przy uzyciu zewnetrznej f. kary

x0 = [ 2, 1]; 
max_iter = 10;
points = [ 0 0 ];
f_values = [ 0 ];
diff_table = [ 0 ];

k=1;
k_table = [ k ];
fun_1 = @(x)x(1)^2 + x(2)^2 + k*( ( max( 0, 1-x(2) ) ) ) +  k*( ( max( 0, x(2)-1 ) ) );

for n = 1:max_iter
    [x_opt_k, fval] = fminunc(fun_1,x0);
    
    diff = abs( f_opt - fval );
    diff_table = [ diff_table; diff ];
    f_values = [ f_values ; fval ];  
    points = [ points ; x_opt_k ];
    
    x0 = x_opt_k;
    k= k*1.15;
    k_table = [ k_table ; k ];
    fun_1 = @(x)x(1)^2 + x(2)^2 + k*( ( max( 0, 1-x(2) ) ) ) +  k*( ( max( 0, x(2)-1 ) ) );

end

\end{lstlisting}

Po znalezieniu wartości wyniki zostały przedstawione na wykresie, realizujący to kod przedstawia Listing \ref{lis3}
\begin{lstlisting}[caption=Stworzenie wykresu z wynikami, captionpos=b,label=lis3, firstnumber=12,frame=single]

% Rysowanie punktow 

[x1, x2] = meshgrid( -3:0.1:13, -1:0.1:17 ); 
rys_war = x1 .^ 2 + x2 .^ 2; 
contour(x1, x2, rys_war, ...
[0.1, 1, 2, 10, 20, 30, 50, 100, 200, 300 ], 'ShowText', 'on');
hold on; 
plot(points(:,1),points(:,2),'-o');
xlabel('x1') % x-axis label
ylabel('x2') % y-axis label

\end{lstlisting}

Wyniki wraz z poziomicami przedstawia Rysunek \ref{fig:zkara_1}
\begin{figure}[H]
\centerline{\includegraphics[scale=0.7]{zkara_1.jpg}}
\centering
\caption{Znalezione przez metodę zewnętrznej funkcji kary punkty w kolejnych iteracjach.}
\label{fig:zkara_1}
\end{figure}

Znalezione przez metodą rozwiązania wraz z numerem iteracji przedstawiono w poniższej tabeli.  
\begin{figure}[H]
\centerline{\includegraphics[scale=0.9]{zkara_2.jpg}}
\centering
\caption{Tabela z wynikami działania metody zewnętrznej funkcji kary.}
\label{fig:zkara_2}
\end{figure}

Porównując otrzymany w ostatniej iteracji wynik z analitycznie wyznaczonym minimum stwierdzono, że błąd jest niewielki, rzędu 6 miejsca po przecinku dla \( x_{1} \), narzucone ograniczenie \(x_{2} = 16 \) została natomiast osiągnięte w procesie szukania minimum.  

Z wyników można wyciągnąć wniosek, że im 'dalej' od obszaru dopuszczalnego znajduje się rozwiązanie tym większa wartość funkcji kary, a tym samym większa zmiana rozwiązania w kolejnej iteracji, im bliżej wartości optymalnej tym zmiany i wartość funkcji kary są mniejsze. 
 
\subsection{Zadanie 2 - Minimalizacja odległości od punktu}
W zadaniu należy zminimalizować odległość od punktu \( (0,0) \) przy danych ograniczeniach. W pierwszej kolejności napisano funkcję celu odpowiadającą zagadnieniu. W tym celu wykorzystano metrykę euklidesową: 
\begin{equation*}
f_{1}(x_{1},x_{2}) = \sqrt{(x_{1}-0)^{2}+(x_{2}-0)^{2}}
\end{equation*}
Wyrażenie z pierwiastkiem będzie mieć wartość minimalną gdy wyrażenie pod pierwiastkiem będzie mieć wartość minimalną dlatego w zadaniu rozważana jest funkcja celu :
\begin{equation*}
f(x_{1},x_{2}) = (x_{1}-0)^{2}+(x_{2}-0)^{2}
\end{equation*}
Następnie sprawdzono poprawność postawionego zadania. Zadanie ma dwa ograniczenia równościowe :
\begin{equation*}
g_{1}(x) = x_{1}+x_{2}-1 = 0
\end{equation*}
\begin{equation*}
g_{2}(x) = x_{1}+x_{2}-2 = 0
\end{equation*}
Wyznaczając z pierwszego z tych ograniczeń wartość \( x_{1} \):
\( x_{1}=1-x_{2} \) i wstawiając do drugiego równania otrzymamy sprzeczność, dlatego \textbf{zadanie jest źle postawione}. 

Od razu można więc stwierdzić, że zbiór rozwiązań dopuszczalnych jest zbiorem pustym, co będzie pociągać za sobą błąd w działaniu algorytmu, ponieważ funkcja kary zawsze będzie dawać wartość niezerową. Aby to udowodnić przetestowano działanie algorytmu dla zadania źle postawionego. 

Ograniczenia równościowe zamieniono na ograniczenia nierównościowe analogicznie jak w zadaniu 1. 
Funkcja celu : 
\begin{equation*}
f(x) = (x_{1}-0)^{2}+(x_{2}-0)^{2} \rightarrow min
\end{equation*}
Ograniczenia :
\begin{equation*}
g_{1}(x) = x_{1} + x_{2} -1 \leq 0
\end{equation*}
\begin{equation*}
g_{2}(x) = -x_{1} - x_{2} + 1 \leq 0
\end{equation*}
\begin{equation*}
g_{3}(x) = x_{1} + x_{2} - 2 \leq 0
\end{equation*}
\begin{equation*}
g_{4}(x) = -x_{1} - x_{2} + 2 \leq 0
\end{equation*}
Implementację metod przeprowadzono tak jak w Zadaniu 1, zmieniona została jedynie funkcja kary i celu, co przedstawia Listing \ref{lis4}.
\begin{lstlisting}[caption=Funkcja celu dla zadania 2, captionpos=b,
label=lis4, firstnumber=12,frame=single]

fun_1 = @(x)x(1)^2 + x(2)^2 + k*( ( max( 0, x(1)+x(2)-1 )^2 )  ) + ...
k*(  ( max( 0, -x(1)-x(2)+1 )  )^2 ) + ...
k*( ( max( 0, x(1)+x(2)-2 )^2 )  ) +  k*(  ( max( 0, -x(1)-x(2)+2 )^2 ) ) ;
\end{lstlisting}

 

Wyniki wraz z poziomicami przedstawia Rysunek \ref{fig:zkara_3}
\begin{figure}[H]
\centerline{\includegraphics[scale=0.6]{zkara_3.jpg}}
\centering
\caption{Znalezione przez metodę zewnętrznej funkcji kary punkty w kolejnych iteracjach.}
\label{fig:zkara_3}
\end{figure}

\newpage
Znalezione przez metodą rozwiązania wraz z numerem iteracji przedstawiono w poniższej tabeli. 

\begin{figure}[H]
\centerline{\includegraphics[scale=0.8]{zkara_4.jpg}}
\centering
\caption{Tabela z wynikami działania metody zewnętrznej funkcji kary.}
\label{fig:zkara_4}
\end{figure}


Metoda znajduje pewne optimum, ale nie spełnia ono założeń, co łatwo sprawdzić poprzez podstawienie otrzymanego minimum do ograniczeń. Metoda nie jest więc odporna na zadania w których obszar dopuszczalny jest zbiorem pustym.

Analityczne rozwiązanie tego zadania nie dałoby również wyniku zgodnego z ograniczeniami więc trudno w tym przypadku o porównanie z rzeczywistą wartością minimum.


\section{Wnioski końcowe}

Metoda zewnętrznej funkcji kary pozwala na rozwiązywanie zadań optymalizacji z ograniczeniami poprzez zastosowanie funkcji, która daje wartość niezerową gdy znalezione rozwiązanie nie znajduje się w obszarze dopuszczalnym. Pomimo to, otrzymany wynik należy zawsze zweryfikować, ponieważ nie zawsze metoda daje wynik zgodny z ograniczeniami, co można zauważyć w zadaniu 1 ( potrzebna jest odpowiednia liczba iteracji aby osiągnąć wynik zwierający się w zbiorze dopuszczalnym ) oraz w zadaniu 2 ( metoda znajduje wynik, który nie zawiera się w zbiorze dopuszczalnym dla dowolnej liczby iteracji ). 

Nie bez znaczenia dla jakości wykonywania metody są parametry algorytmu, najbardziej znaczącym wydaje się być parametr k, który decyduje o tym jak duża kara jest nałożona na daną funkcję za wyjście poza obszar dopuszczalny. Z moich obserwacji wynika, że wartość tą najlepiej dobierać eksperymentalnie, ponieważ zasada, że im większa wartość tym lepiej nie zawsze działa dla wszystkich funkcji.
  
\end{document}