\documentclass[a4paper,15pt]{article}
\usepackage{amssymb}
\usepackage{amsmath}
\usepackage[english, polish]{babel}
\usepackage[utf8]{inputenc}   % lub utf8
\usepackage[T1]{fontenc}
\usepackage{graphicx}
\usepackage{anysize}
\usepackage{enumerate}
\usepackage{times}
\usepackage{titlesec}
\usepackage{float}
\usepackage{titleps,kantlipsum}
\usepackage{listings}
\usepackage{color}
\lstloadlanguages{Matlab}
 
\usepackage[justification=centering]{caption}
\titlelabel{\thetitle.\quad}


% Definicja nowego stylu strony
\newpagestyle{mypage}
{
  \headrule
  
  \sethead
  { \MakeUppercase{\thesection\quad \sectiontitle} } 
  {}
  {\thesubsection\quad \subsectiontitle}
  
  \setfoot
  {}
  {\thepage}
  {}
}

\newpagestyle{mypage_1}
{
	\headrule
	
	\sethead
	{  }
	{\MakeUppercase{Laboratorium metod optymalizacji}}
	{}
	
	\setfoot
	{}
	{}
	{}
}

\settitlemarks{section,subsection,subsubsection}

\pagestyle{mypage_1}

%\marginsize{left}{right}{top}{bottom}
\marginsize{3cm}{3cm}{3cm}{3cm}
\sloppy
\titleformat{\section}
  {\normalfont\Large\bfseries}{\thesection}{1em}{}[{\titlerule[0.8pt]}]
 
 \definecolor{darkred}{rgb}{0.9,0,0}
\definecolor{grey}{rgb}{0.4,0.4,0.4}
\definecolor{orange}{rgb}{1,0.6,0.05}
\definecolor{darkgreen}{rgb}{0.2,0.5,0.05}
 
\lstset{
language=Matlab,
basicstyle=\ttfamily\small,
keywordstyle=\color{darkgreen}\ttfamily\bfseries\small,
stringstyle=\color{red}\ttfamily\small,
commentstyle=\color{grey}\ttfamily\small,
numbers=left,
numberstyle=\color{darkred}\ttfamily\scriptsize,
identifierstyle=\color{blue}\ttfamily\small,
showstringspaces=false,
morekeywords={}
}


\begin{document}



\tableofcontents

%\thispagestyle{mypage_1}

\begin{table}
\begin{center}
\begin{tabular}{|l|l|l|}
\hline
\multicolumn{3}{|c|}{\textbf{Optymalizacja wielokryterialna}} \\ \hline Dominik Wróbel & \textbf{21 V 2018} & \textbf{Pon 08:00, s. 111} \\ \hline
\multicolumn{2}{|l|}{Numery zadań} & 1, 3 \\ \hline 

\end{tabular}
\end{center}
\end{table}

\section{Cel ćwiczenia}
Celem ćwiczenia jest rozwiązanie zadań optymalizacyjnych z wykorzystaniem podejścia opartego o optymalizację wielokryterialną. Podobnie jak inne zadania optymalizacji, zadania te polegać będą na znalezieniu rozwiązania optymalnego dla problemu, jednak w tym przypadku będą analizowane różne kryteria zależne od tych samych zmiennych decyzyjnych. Celem optymalizacji wielokryterialnej jest taki dobór zmiennych decyzyjnych aby w idealnym przypadku znaleźć optymalną wartość wszystkich wskaźników, nie zawsze jest to jednak możliwe i wówczas w zadaniu wyznaczany jest tzw. zbiór rozwiązań kompromisowych. 
\newpage
\section{Przebieg ćwiczenia}
\subsection{Zadanie 1 - Dwójnik elektryczny }
\subsubsection{Opis problemu}
W zadaniu rozważany jest układ dwójnika elektrycznego przedstawiony na Rysunku \ref{fig:opt_wiel_1}.
\begin{figure}[H]
\centerline{\includegraphics[scale=0.8]{opt_wiel_1.jpg}}
\centering
\caption{Analizowany układ dwójnika elektrycznego}
\label{fig:opt_wiel_1}
\end{figure}
Zadanie polega na wyznaczeniu optymalnej wartości zmiennej decyzyjnej \(x\) przy przyjętych kryteriach. Zmienna \(x\) dana jest wzorem:
\begin{equation*}
x=\frac{R_{a}}{R_{i}}, \quad x \in [0, \infty )
\end{equation*} 
Na zmienną \(R_{i}\) nałożone jest ograniczenie równościowe: \\
\begin{equation*}
R_{i} = n = 16
\end{equation*}
Przyjęte kryteria jakości to maksymalna wartość współczynnika sprawności \( \eta \) oraz maksymalna moc \( P \) wydzielana na obciążeniu.
\pagestyle{mypage}
\subsubsection{Analityczne wyznaczenie kryteriów jakości}
\begin{itemize}
\item Sprawność \( \eta \) jest to stosunek mocy wydzielanej na rezystorze \(R_{a}\) do mocy wydzielanej na rezystorach \(R_{a}\) oraz \(R_{i}\): \\
\begin{equation*}
\eta = \frac{P_{a}}{P_{i}+P_{a}}, \quad P_{a} = U_{a}I = I^{2}R_{a}, \quad P_{i} = U_{i}I = I^{2}R_{i}
\end{equation*}
\begin{equation*}
\eta = \frac{R_{a}}{R_{a}+R_{i}} = \frac{ \frac{R_{a}}{R_{i}} } {  \frac{R_{a}}{ R_{i} }  + 1 } = \frac{x}{x+1} 
\end{equation*}
\item Moc wydzielana na rezystorze 
\begin{equation*}
P_{a} = R_{a}I^{2} = \frac{R_{a}}{(R_{a}+R_{i})^{2}}E^{2} = \frac{R_{a}}{(R_{a}+R_{i})}\frac{1}{(R_{a}+R_{i})}E^{2}=\frac{x}{(x+1)}\frac{1}{(\frac{R_{a}}{R_{i}}+1)}\frac{E^{2}}{R_{i}}=\frac{x}{(x+1)^{2}}\frac{E^{2}}{R_{i}}
\end{equation*}
\end{itemize}
Ostatecznie więc wskaźnikami jakości są funkcje:
\begin{equation*}
Q_{1}(x) = \frac{x}{x+1}, \quad Q_{2}(x) = \frac{E^{2}}{R_{i}}\frac{x}{(x+1)^{2}}
\end{equation*}
\subsubsection{Wyznaczenie wykresów kryteriów jakości}
Wykresy wskaźników zostały wyznaczone przyjmując wartość napięcia \( E = 1 \). W przypadku wskaźnika \(Q_{1}\) nie ma to wpływu na przebieg wykresu. Również wartość opornika \(R_{i}\) nie ma większego wpływu na kształt wykresu.\\
Z kolei dla wyznaczenia wskaźnika \(Q_{2}\) wartości opornika \(R_{i}\) ma już duże znaczenie, przyjęto wartość zgodną z ograniczeniem \(R_{i}=16\).

Wykresy przedstawiają Rysunki \ref{fig:opt_wiel_2} i \ref{fig:opt_wiel_3}.

\begin{figure}[H]

\minipage{0.5\textwidth}

  \includegraphics[width=\linewidth]{opt_wiel_2.jpg}
  \caption{\(\quad Q_{1}(x), \quad E = 1, \quad R_{i}=16 \)}
  \label{fig:opt_wiel_2}

\endminipage\hfill
\minipage{0.5\textwidth}%

  \includegraphics[width=\linewidth]{opt_wiel_3.jpg}
  \caption{\( \quad Q_{2}(x), \quad E = 1, \quad R_{i}=16\)}
  \label{fig:opt_wiel_3}

\endminipage

\end{figure}

Kod wykorzystany do rysowania wykresów przedstawia Listing \ref{lis1}.
\begin{lstlisting}[caption=Kryteria jakości, captionpos=b,label=lis1, firstnumber=1,frame=single]
clear all;
close all;

max_x_axes = 500;
Ri = 16;

Ra = 0:0.1:max_x_axes;
x_1 = Ra/Ri;
fun_1 = x_1./(x_1+1);

plot(x_1,fun_1);

axis([0 20 0 1 ]);

xlabel('x') % x-axis label
ylabel('Q1(x)') % y-axis label
title('Wskaznik Q1(x)')
coef = 1/Ri;
x_3 = Ra/Ri;
fun_3 = coef .* ( x_3 ./ ( (x_3+1).^2 ) );

% hold on;
figure()
plot(x_3,fun_3);

xlabel('x') % x-axis label
ylabel('Q2(x)') % y-axis label
title('Wskaznik Q2(x)')

axis([0 5 0 0.03 ]);
\end{lstlisting}

\subsubsection{Zależność pomiędzy kryteriami}
Następnie w celu wyznaczenia zbioru rozwiązań kompromisowych wyznaczono zależność pomiędzy wskaźnikiem \( Q_{1} \) i \( Q_{2} \). \\
Wychodząc z równania na \( Q_{1} \) : \\
\begin{equation*}
Q_{1} = \frac{x}{x+1} \implies x = \frac{Q_{1}}{1-Q_{1}}
\end{equation*}
Wyrażenie to podstawiono do wyrażenia na drugi ze wskaźników: \\
\begin{equation*}
Q_{2} = \frac{E^{2}}{R_{i}}\frac{ \frac{Q_{1}}{1-Q_{1}}  }{ ( \frac{Q_{1}}{1-Q_{1}} + 1 ) ^{2} } = -\frac{E^{2}}{R_{i}}(Q_{1}-1)Q_{1}
\end{equation*}
\subsubsection{Wyznaczenie zbioru rozwiązań kompromisowych}
W celu wyznaczenie zbioru rozwiązań kompromisowych narysowano zależność pomiędzy wskaźnikami obliczoną w poprzednim punkcie. Podobnie jak w poprzednich punktach przyjęto, że napięcie E ma wartość 1. \\
Z Rysunku \ref{fig:opt_wiel_2} widać, że dla \( x \in [0, \infty) \), wskaźnik \(Q_{1}\) przyjmuje wartości \( Q_{1} \in [0,1) \) dlatego też zależność pomiędzy wskaźnikami analizowana jest tylko w tej dziedzinie. 
Otrzymany wykres wraz z zaznaczonym zbiorem kompromisowym przedstawia Rysunek \ref{fig:opt_wiel_4}. 
\begin{figure}[H]
\centerline{\includegraphics[scale=0.3]{opt_wiel_4.jpg}}
\centering
\caption{Zależność pomiędzy \(Q_{1}\) i \(Q_{2}\), czerwone punkty to znaleziony obszar rozwiązań kompromisowych. }
\label{fig:opt_wiel_4}
\end{figure}

Kod użyty do narysowania wykresu. przedstawia Listing \ref{lis2}.
\begin{lstlisting}[caption=Wykres ze zbiorem punktów kompromisowych, captionpos=b,label=lis2, firstnumber=1,frame=single]
clear all;
close all;

Q1 = 0:0.01:1;
Ri = 16;
coef = 1/Ri;
coef = -coef;
fun_4 = coef .* (( Q1-1 ).*Q1);
hold on;
plot(Q1,fun_4);

hold on;
Q1_half = 0.5:0.01:1 ;
fun_4 = coef .* (( Q1_half-1 ).*Q1_half);
plot(Q1_half,fun_4,'*');


axis([0 1 0 0.02 ]);
xlabel('Q2(x)');
ylabel('Q1(x)');


\end{lstlisting}

\subsection{Zadanie 3 - Preparat medyczny}
\subsubsection{Opis problemu}
W zadaniu dane są dane o preparacie medycznym, który w różnych wariantach ma różne efekty leczenia ( \(Q_{1}\)) oraz różną nietolerancje ( nieskuteczność \( Q_{2}\) ). Dane są przedstawione w tabeli poniżej. Zgodnie z numerem na liście ćwiczeniowej przyjęto \( n = 16 \). \\
\begin{table}[H]
\begin{center}
\begin{tabular}{|l|l|l|c|}
\hline
\multicolumn{4}{|c|}{\textbf{Preparat medyczny}} \\ \hline 
Wariant & \(Q_{1}\) & \(Q_{2}\) & \(Q_{1}-Q_{2}\) \\ \hline
\(1^{*}\) & \( 40 \) & \( 10+2n=42 \) & \(-2\)\\ \hline 
2 & \( 60 \) & \( 30+2n=62 \) & \(-2\)\\ \hline 
\(3^{*}\) & \( 60 \) & \( 20+2n=52 \) & \(8\)\\ \hline 
4 & \( 10-2n=-22\) & \(30 \) & \(-52\) \\ \hline 
\(5^{*}\) & \( 20 \) & \( 5+2n=37\) & \( -17\) \\ \hline 
6 & \(30\) & \(20+2n=52\) & \( -22 \)\\ \hline 
7 & 40 & \(25+2n=57\) & \(-17\)\\ \hline 
\end{tabular}
\end{center}
\end{table}
Gwizdką zaznaczono znalezione warianty podczas działania algorytmu. Jak widać prawie wszystkie ze znalezionych wariantów, to te dla których wartość różnicy \( Q_{1}-Q_{2}\) ma największą wartość.

\subsubsection{Sposób rozwiązania}
Problem zostanie rozwiązany przy użyciu punktów ze zbioru Pareto, czyli punktów, dla których znalezione rozwiązania dają najlepsze wyniki dla przyjętych kryteriów. Algorytm postępowania jest następujący :
\begin{itemize}
\item Jeżeli wariant ma ujemny wskaźnik \(Q_{1}\) to jest odrzucany ze względu na to, że nie ma sensu stosować leku o ujemnej skuteczności bez względu na jego nieskuteczność.
\item Jeśli wariant jest lepszy od wariantu ze zbioru Pareto, to 
zamień go z wariantem ze zbioru Pareto,
\item Jeśli wariant jest gorszy od wariantu ze zbioru Pareto, to
przejdź do analizy następnego wariantu,
\item Jeśli nie zachodzi żadne z powyższych, to dodaj wariant do
zbioru Pareto
\end{itemize}

\subsubsection{Rozwiązanie programowe}
Kod użyty do znalezienie punktów Pareto przedstawia Listing \ref{lis3}.
\begin{lstlisting}[caption=Wyznaczanie zbioru pareto, captionpos=b,label=lis3, firstnumber=1,frame=single]
clear all;
close all;
n = 16;
% Wartosci Q1 i Q2 z tabelki
Q1 = [ 40, 60, 60, 10-2*n, 20, 30, 40];
Q2 = [ 10+2*n, 30+2*n, 20+2*n, 30, 5+2*n, 20+2*n, 25+2*n];

wsk_1 = Q1;
wsk_2 = Q2;

% Numery wariantow
war = 1:1:10;

% Na poczatku zbior Pareto zawiera 1 wariant
pareto = [ 1 ];

flag_dodanie = 0;
flag = 0;
sizer = size(Q1);
sizer = sizer(2);
for i = 2 : sizer
   sizer = size(pareto);
   sizer = sizer(2);
   
   for j = 1 : sizer
    
    % Jesli wariant jest lepszy od wariantu ze zbioru Pareto, to 
    % zamien go z wariantem ze zbioru Pareto
    if (  wsk_1(i) >= wsk_1(pareto(j)) && wsk_2(i) <= wsk_2(pareto(j)) )
        
        pareto(j) = i;
        flag = 1;
        break;
    
    % Jesli wariant jest gorszy od wariantu ze zbioru pareto, to
    % przejdz do analizy nastepnego wariantu
    elseif(  ( wsk_1(i) <= wsk_1(pareto(j)) && wsk_2(i) >= ...
    wsk_2(pareto(j)) ) || wsk_1(i)<=0 )
        flag = 1;
        break;
    end
    % Jesli nie zachodzi zadne z powyzszych, to dodaj wariant do
    % zbioru pareto
   end
   
   if( flag == 0 )
        sizer = size(pareto);
        sizer = sizer(2);
        pareto(sizer+1) = i;     
   end
    flag = 0;
end

x = size(pareto);
x = x(2);
for k = 1:x
   Q_par_1(k) = wsk_1(pareto(k));
   Q_par_2(k) = wsk_2(pareto(k));
end
figure();
plot(wsk_1,wsk_2,'*');
hold on;
plot(Q_par_1,Q_par_2,'o');

xlabel('Wskaznik Q1');
ylabel('Wskaznik Q2');


\end{lstlisting}

\begin{figure}[H]
\centerline{\includegraphics[scale=0.6]{opt_wiel_5.jpg}}
\centering
\caption{Punkty Pareto zaznaczono czerwonym okręgiem. }
\label{fig:opt_wiel_5}
\end{figure}

\section{Wnioski końcowe}
Zastosowanie optymalizacji wielokryterialnej pozwala na rozważaniem problemów optymalizacji z uwzględnieniem różnych kryteriów. Ważne jest aby podczas rozwiązywania zadania wykorzystać wszystkie relacje pomiędzy kryteriami jakości i zmiennymi decyzyjnymi. \\
Dzięki takiemu podejściu możliwe jest wyznaczenie zbioru kompromisów, co ogranicza uniwersum poszukiwań wartości minimalnych. 

W zadaniu 3 optymalizacja wielokryterialna została wykorzystana do badania określonych punktów. Takie podejście pozwala na zastosowanie optymalizacji wielokryterialnej do skończonego zbioru przeszukiwań i pozwala wybrać z punktów dopuszczalnych te, które najlepiej pasują do przyjętych kryteriów. 
\end{document}
