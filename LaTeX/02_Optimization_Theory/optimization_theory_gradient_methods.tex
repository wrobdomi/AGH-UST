\documentclass[a4paper,15pt]{article}
\usepackage{amssymb}
\usepackage{amsmath}
\usepackage[english, polish]{babel}
\usepackage[utf8]{inputenc}   % lub utf8
\usepackage[T1]{fontenc}
\usepackage{graphicx}
\usepackage{anysize}
\usepackage{enumerate}
\usepackage{times}
\usepackage{titlesec}
\usepackage{float}
\usepackage{titlesec}
\usepackage{titleps,kantlipsum}
 
\usepackage[justification=centering]{caption}
\titlelabel{\thetitle.\quad}



\newpagestyle{mypage}{%
  \headrule
  \sethead{\MakeUppercase{\thesection\quad \sectiontitle}}{}{\thesubsection\quad \subsectiontitle}
  \setfoot{}{}{\thesubsubsection\quad \subsubsectiontitle}
}
\settitlemarks{section,subsection,subsubsection}
\pagestyle{mypage}

%\marginsize{left}{right}{top}{bottom}
\marginsize{3cm}{3cm}{3cm}{3cm}
\sloppy
\titleformat{\section}
  {\normalfont\Large\bfseries}{\thesection}{1em}{}[{\titlerule[0.8pt]}]
 
\begin{document}

\begin{table}
\begin{center}
\begin{tabular}{|l|l|l|}
\hline
\multicolumn{3}{|c|}{\textbf{Metody gradientowe}} \\ \hline Dominik Wróbel & \textbf{09 IV 2018} & \textbf{Pon 08:00, s. 111} \\ \hline
\multicolumn{2}{|l|}{Numery zadań} & 1, 2, 5 \\ \hline 

\end{tabular}
\end{center}
\end{table}

\section{Cel ćwiczenia}
Celem ćwiczenia jest zapoznanie się z metodami gradientowymi poszukiwania minimum funkcji. Analiza działania różnych metod  dla różnych funkcji pozwoli na określenie słabych i mocnych stron tych metod. Porównanie metod pozwoli na wybranie najlepszej dla danego problemu. 
\section{Przebieg ćwiczenia}
\subsection{Zadanie 1 - obserwacja zjawiska zygzakowania}
W zadaniu poszukiwane jest minimum funkcji celu danej wzorem:
\begin{equation*}
Q(x_{1},x_{2})=x_{1}^{2}+ax_{2}^{2}
\end{equation*}
W tym celu wykorzystywana jest metoda najszybszego spadku. Rozważane są różne postacie funkcji celu dla parametru \(a\):
\begin{itemize}
\item \(a=1\),
\item \(a=0.5\),
\item \(a=0.3\)
\end{itemize}
Zmiana parametru \(a\) powoduje zmianę kształtu poziomic funkcji celu. Dzięki temu możliwe jest obserwowanie wpływu wydłużania zbiorów poziomicowych i punktów startowych na działanie metody najszybszego spadku. \par
W funkcjach wprowadzono modyfikacje w celu dostosowania działania metody do zadania. Zmieniono funkcję kosztu tak aby obliczała wartość funkcji celu zgodnie ze wzorem podanym na początku zadania: 
\begin{figure}[H]
\centerline{\includegraphics[scale=1]{KOSZT.jpg}}
\centering
\caption{Funkcja obliczająca wartość funkcji celu w danym punkcie}
\label{fig:KOSZT}
\end{figure}
Zmodyfikowano także plik obliczający gradient podstawiając wzór na gradient funkcji z zadania:
\begin{figure}[H]
\centerline{\includegraphics[scale=1]{grad_2.jpg}}
\centering
\caption{Funkcja obliczająca gradient w danym punkcie}
\label{fig:grad_2}
\end{figure}
Napisano skrypt umożliwiający wywoływanie funkcji granad oraz rysowanie znalezionych rozwiązań wraz z kierunkami gradientów:
\begin{figure}[H]
\centerline{\includegraphics[scale=1]{grad_1.jpg}}
\centering
\caption{Funkcja rysująca działanie metody}
\label{fig:grad_1}
\end{figure}
\newpage

\subsubsection{Zadanie 1 - Parametr a = 1}
Dla parametru a = 1 poziomice są okręgami. Nie trudno zauważyć, że minimum funkcji celu jest punkt (0,0). Punkt startowy przyjęto dla różnych wartości parametru a taki sam zgodnie z numerem na liście ćwiczeniowej : 
\begin{equation*}
x_{0}=\left[n,\frac{n}{2}\right]=\left[16,8\right]
\end{equation*}
Uzyskane wyniki przedstawia Rysunek \ref{fig:grad_3}
\begin{figure}[H]
\centerline{\includegraphics[scale=0.5]{grad_3.jpg}}
\centering
\caption{a = 1, poszukiwanie minimum dla różnych punktów początkowych}
\label{fig:grad_3}
\end{figure}

\subsubsection{Zadanie 1 - Parametr a = 0.3}
Dla parametru a = 0.3 poziomice są elipsami. Nie trudno zauważyć, że minimum funkcji celu jest punkt (0,0). Punkt startowy przyjęto dla różnych wartości parametru a taki sam zgodnie z numerem na liście ćwiczeniowej : 
\begin{equation*}
x_{0}=\left[n,\frac{n}{2}\right]=\left[16,8\right]
\end{equation*}
Uzyskane wyniki przedstawia Rysunek \ref{fig:grad_4}
\begin{figure}[H]
\centerline{\includegraphics[scale=0.5]{grad_4.jpg}}
\centering
\caption{a = 0,3, poszukiwanie minimum dla różnych punktów początkowych}
\label{fig:grad_4}
\end{figure}

\subsubsection{Zadanie 1 - Parametr a = 0.5}
Dla parametru a = 0.5 poziomice są elipsami. Nie trudno zauważyć, że minimum funkcji celu jest punkt (0,0). Punkt startowy przyjęto dla różnych wartości parametru a taki sam zgodnie z numerem na liście ćwiczeniowej : 
\begin{equation*}
x_{0}=\left[n,\frac{n}{2}\right]=\left[16,8\right]
\end{equation*}
Uzyskane wyniki przedstawia Rysunek \ref{fig:grad_5}
\begin{figure}[H]
\centerline{\includegraphics[scale=0.5]{grad_5.jpg}}
\centering
\caption{a = 0.5, poszukiwanie minimum dla różnych punktów początkowych}
\label{fig:grad_5}
\end{figure}

\subsection{Zadanie 2 - Porównanie gradientowych metod optymalizacji}
Zadanie to polega na porównaniu gradientowych metod optymalizacji. Badana funkcja celu to :
\begin{equation*}
Q(x_{1},x_{2})=6x_{1}^{2}+6x_{1}x_{2}+x_{2}^{2}+4.5(e^{x_{1}}-x_{1}-1)+1.5(e^{x_{2}}-x_{2}-1)
\end{equation*}
W zadaniu tym ponownie zmodyfikowano funkcje koszt.m oraz gradie.m aby dostosować je do funkcji celu z zadania.
\begin{figure}[H]
\centerline{\includegraphics[scale=0.8]{grad_6.jpg}}
\centering
\caption{Funkcja koszt.m dla zadania 2}
\label{fig:grad_6}
\end{figure}
\begin{figure}[H]
\centerline{\includegraphics[scale=0.8]{grad_7.jpg}}
\centering
\caption{Funkcja gradie.m dla zadania 2}
\label{fig:grad_7}
\end{figure}
W niewielkim zakresie konieczna była także modyfikacja skryptu granad, tak aby odpowiednio dopasować rozmiary mnożonych macierzy dla poprawnego działania metody. Do wykonania wykresów została wykorzystana funkcja z zadania poprzedniego.

Przeanalizowane zostaną metody:
\begin{itemize}
\item Metoda najszybszego spadku,
\item Metoda Fletchera - Reevesa
\item Metoda Polaka - Ribiere'a,
\item Metoda z pełną formułą na współczynnik \( \beta \)
\end{itemize}
W zadaniu zostały przyjęte następujące założenia:
\begin{itemize}
\item Punkt startowy zgodnie z listą ćwiczeniową ( n = 16 ) to \( (-48,16) \)
\item Max. liczba iteracji to 20
\item \( x^{apr} \) - aktualne przybliżenie rozwiązania optymalnego,
\item \( Q^{apr} = Q(x^{apr}) \)
\item \( x^{*} \) - rozwiązanie optymalne ( wyznaczone analitycznie )
\item \( Q^{*}=Q(x^{*})\)
\item Norma gradientu nie mniejsza od \( 0.001 \) 
\end{itemize}
Na rysunkach wynikowych w celu lepszej widoczności działania metody kolejne kierunki zaznaczono strzałkami.
\newpage
\subsubsection{Zadanie 2 - Metoda najszybszego spadku}
Dla punktu startowego (-48,16) działanie metody przedstawia Rysunek \ref{fig:grad_8}
\begin{figure}[H]
\centerline{\includegraphics[scale=0.4]{grad_8.jpg}}
\centering
\caption{Metoda najszybszego spadku, punkt początkowy ( -48 , 16 )}
\label{fig:grad_8}
\end{figure}
\subsubsection{Zadanie 2 - Metoda Fletchera - Reevesa }
Dla punktu startowego (-48,16) działanie metody przedstawia Rysunek \ref{fig:grad_10}.
\begin{figure}[H]
\centerline{\includegraphics[scale=0.4]{grad_10.jpg}}
\centering
\caption{Metoda Fletchera - Reevesa, punkt początkowy (-48,16) }
\label{fig:grad_10}
\end{figure}
\subsubsection{Zadanie 2 - Metoda Polaka - Ribiere'a }
Dla punktu startowego (-48,16) działanie metody przedstawia Rysunek \ref{fig:grad_11}.
\begin{figure}[H]
\centerline{\includegraphics[scale=0.4]{grad_11.jpg}}
\centering
\caption{Metoda Polaka - Ribiere'a, punkt początkowy (-48,16) }
\label{fig:grad_11}
\end{figure}
\subsubsection{Zadanie 2 - Metoda z pełną formułą na współczynnik beta }
Dla punktu startowego (-48,16) działanie metody przedstawia Rysunek \ref{fig:grad_12}.
\begin{figure}[H]
\centerline{\includegraphics[scale=0.4]{grad_12.jpg}}
\centering
\caption{Metoda z pełną formułą na współczynnik beta, punkt początkowy (-48,16) }
\label{fig:grad_12}
\end{figure}
\subsubsection{Zadanie 2 - Tabelaryczne porównanie metod}
Zestawienie porównawcze metod przedstawiają Tabele 1, 2, 3, 4 i 5.  

\begin{figure}[H]
\centerline{\includegraphics[scale=0.9]{met_1.jpg}}
\centering
\caption{Metoda najszybszego spadku }
\label{fig:met_1}
\end{figure}

\begin{figure}[H]
\centerline{\includegraphics[scale=0.9]{met_2.jpg}}
\centering
\caption{Metoda Fletchera - Reevesa }
\label{fig:met_2}
\end{figure}

\begin{figure}[H]
\centerline{\includegraphics[scale=0.9]{met_3.jpg}}
\centering
\caption{Metoda Polaka - Ribiere'a }
\label{fig:met_3}
\end{figure}

\begin{figure}[H]
\centerline{\includegraphics[scale=0.9]{met_4.jpg}}
\centering
\caption{Z pełną formułą na współczynnik beta }
\label{fig:met_4}
\end{figure}

\subsection{ Zadanie 5 - Dolina bananowa Rossenbrocka }
W tym zadaniu wyznaczane jest minimum funkcji celu:
\begin{equation*}
Q(x_{1},x_{2})=100(x_{2}-x_{1}^{2})^{2}+(1-x_{1})^2
\end{equation*}
Zgodnie z numerem na liście ćwiczeniowej przyjęto, że punkt startowy to
\begin{equation*}
x_{0}=\left[n,\frac{n}{2}\right]=\left[16,8\right]
\end{equation*}
Zmodyfikowaną funkcję koszt.m przedstawia Rysunek \ref{fig:grad_13}. Zmodyfikowaną funkcję gradie.m przedstawia Rysunek \ref{fig:grad_14}.
\begin{figure}[H]
\centerline{\includegraphics[scale=0.7]{grad_13.jpg}}
\centering
\caption{koszt.m dla zadania 5}
\label{fig:grad_14}
\end{figure}
\begin{figure}[H]
\centerline{\includegraphics[scale=0.7]{grad_14.jpg}}
\centering
\caption{gradie.m dla zadania 5}
\label{fig:grad_13}
\end{figure}
Zbadano metody:
\begin{itemize}
\item Metoda najszybszego spadku,
\item Metoda Fletchera - Reevesa
\item Metoda Polaka - Ribiere'a,
\item Metoda z pełną formułą na współczynnik \( \beta \)
\end{itemize}
\newpage
\subsubsection{Zadanie 5 - Metoda najszybszego spadku}
Dla punktu startowego (32,16) działanie metody przedstawia Rysunek \ref{fig:gradd_8}
\begin{figure}[H]
\centerline{\includegraphics[scale=0.4]{gradd_8.jpg}}
\centering
\caption{Metoda najszybszego spadku, punkt początkowy ( 32 , 16 )}
\label{fig:gradd_8}
\end{figure}
\subsubsection{Zadanie 5 - Metoda Fletchera - Reevesa }
Dla punktu startowego (32,16) działanie metody przedstawia Rysunek \ref{fig:gradd_10}.
\begin{figure}[H]
\centerline{\includegraphics[scale=0.4]{gradd_10.jpg}}
\centering
\caption{Metoda Fletchera - Reevesa, punkt początkowy (32, 16) }
\label{fig:gradd_10}
\end{figure}
\subsubsection{Zadanie 5 - Metoda Polaka - Ribiere'a }
Dla punktu startowego (32,16) działanie metody przedstawia Rysunek \ref{fig:gradd_11}.
\begin{figure}[H]
\centerline{\includegraphics[scale=0.4]{gradd_11.jpg}}
\centering
\caption{Metoda Polaka - Ribiere'a, punkt początkowy (32,16) }
\label{fig:gradd_11}
\end{figure}
\subsubsection{Zadanie 5 - Metoda z pełną formułą na współczynnik beta }
Dla punktu startowego (32,16) działanie metody przedstawia Rysunek \ref{fig:gradd_12}.
\begin{figure}[H]
\centerline{\includegraphics[scale=0.4]{gradd_12.jpg}}
\centering
\caption{Metoda z pełną formułą na współczynnik beta, punkt początkowy (32,16) }
\label{fig:gradd_12}
\end{figure}

Zestawienie porównawcze metod przedstawia Tabela 5.  

\begin{figure}[H]
\centerline{\includegraphics[scale=0.9]{met_5.jpg}}
\centering
\caption{Porównanie metod - Zadanie 5 }
\label{fig:met_5}
\end{figure}

\section{Wnioski końcowe}
Zadanie 1 pozwala na zaobserwowanie wad metody najszybszego spadku, dla poziomic będących okręgami metoda ta bardzo szybko znajduje minimum, wystarczy jedna iteracja, natomiast dla poziomic będących elipsami kierunki gradientów zmieniają się w sposób 'zygzakowaty', co powoduje, że metoda ta jest dość wolna w użyciu.

W zadaniu drugim zostało wykonane porównanie czterech metod gradientowych. Metoda najszybszego spadku nie znalazła minimum z określoną dokładnością przed osiągnięciem limitu iteracji ( max. 20 ). Inne metody znacznie dokładniej i szybciej były w stanie znaleźć minimum, wynika to z innego sposobu obliczania przez te metody gradientu. Zadanie 2 pokazuje również różnice pomiędzy gradientami obliczanymi przez różne metody. Zależnie od wybory metody kierunki te są inne, co prowadzi do różnych rezultatów i różnych szybkości zbieżności do punktu optymalnego.

W zadaniu 5 żadna z metod nie dała zadowalających rezultatów, pomimo, że metoda najszybszego spadku poradziła sobie najgorzej, a inne metody lepiej, to każda z nich wykorzystała maksymalną liczbę 20 iteracji. 
\end{document}