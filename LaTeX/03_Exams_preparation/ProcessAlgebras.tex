\documentclass[a4paper,15pt]{article}
\usepackage{amssymb}
\usepackage{amsmath}
\usepackage[english, polish]{babel}
\usepackage[utf8]{inputenc}   % lub utf8
\usepackage[T1]{fontenc}
\usepackage{graphicx}
\usepackage{anysize}
\usepackage{enumerate}
\usepackage{times}
\usepackage{caption}
\usepackage{titlesec}
\usepackage{float}
\usepackage{titleps,kantlipsum}
\usepackage{listings}
\usepackage{xcolor}
\usepackage{hyperref}
\usepackage{framed}
\usepackage{tcolorbox}
\usepackage{mathtools}
\lstloadlanguages{Matlab}
 
\usepackage[justification=centering]{caption}
\titlelabel{\thetitle.\quad}

\pagenumbering{arabic}

\DeclareCaptionFont{white}{\color{white}}
\DeclareCaptionFormat{listing}{%
  \parbox{\textwidth}{\colorbox{darkgreen}{\parbox{\textwidth}{#1#2#3}}\vskip-4pt}}
\captionsetup[lstlisting]{format=listing,labelfont=white,textfont=white}
\lstset{frame=lrb,xleftmargin=\fboxsep,xrightmargin=-\fboxsep}

% Definicja nowego stylu strony
\newpagestyle{mypage}
{
  \headrule
  
  \sethead
  { \MakeUppercase{\thesection\quad \sectiontitle} } 
  {}
  {\thesubsection\quad \subsectiontitle}
  
  \setfoot
  {}
  {}
  {\thepage}
}

\newpagestyle{mypage_1}
{
	\headrule
	
	\sethead
	{  }
	{\MakeUppercase{Metody Formalne - Algebry Procesów}}
	{}
	
	\setfoot
	{}
	{\thepage}
	{}
}

\settitlemarks{section,subsection,subsubsection}

\pagestyle{mypage_1}

\definecolor{mGreen}{rgb}{0,0.6,0}
\definecolor{mGray}{rgb}{0.5,0.5,0.5}
\definecolor{mPurple}{rgb}{0.58,0,0.82}
\definecolor{mRed}{RGB}{234,67,53}
\definecolor{mKeyword}{RGB}{0,0,242}
\definecolor{mBlue}{RGB}{0,153,242}
\definecolor{backgroundColour}{RGB}{242,242,242}
\definecolor{issueColor}{RGB}{0,51,102}

\newcommand{\definition}[2]{
    \begin{tcolorbox}[colback=green!5!white,colframe=mGreen,title={Definicja -  #1}]
        #2
    \end{tcolorbox}
}

\newcommand{\question}[2]{
    \begin{tcolorbox}[colback=black!5!white,colframe=black,title={Zagadnienie #1}]
        #2
    \end{tcolorbox}
}

\newcommand{\kol}[2]{
    \begin{tcolorbox}[colback=mRed!5!white,colframe=mRed,title={Kolokwium 2018 #1}]
        #2
    \end{tcolorbox}
}

\newcommand{\example}[2]{
    \begin{tcolorbox}[colback=blue!5!white,colframe=blue,title={Przykład #1}]
        #2
    \end{tcolorbox}
}

\newcommand{\issue}[2]{
    \begin{tcolorbox}[colback=issueColor!5!white,colframe=issueColor,title={Zagadnienie #1}]
        #2
    \end{tcolorbox}
}


%\marginsize{left}{right}{top}{bottom}
\marginsize{3cm}{3cm}{3cm}{3cm}
\sloppy
\titleformat{\section}
  {\normalfont\Large\bfseries}{\thesection}{1em}{}[{\titlerule[0.8pt]}]
 
 \definecolor{darkred}{rgb}{0.9,0,0}
\definecolor{grey}{rgb}{0.4,0.4,0.4}
\definecolor{orange}{rgb}{1,0.6,0.05}
\definecolor{darkgreen}{rgb}{0.2,0.5,0.05}
 


\lstdefinestyle{CStyle}{
    backgroundcolor=\color{backgroundColour},   
    commentstyle=\color{mGreen},
    keywordstyle=\color{mKeyword},
    numberstyle=\tiny\color{mGray},
    stringstyle=\color{mPurple},
    basicstyle=\footnotesize,
    breakatwhitespace=false,         
    breaklines=true,                 
    %captionpos=b,                    
    keepspaces=true,                 
    numbers=left,                    
    numbersep=5pt,                  
    showspaces=false,                
    showstringspaces=false,
    showtabs=false,                  
    tabsize=2,
    language=C
}


\newcommand{\Hilight}{\makebox[0pt][l]{\color{cyan}\rule[-4pt]{0.65\linewidth}{14pt}}}


\begin{document}

\begin{table}
\begin{center}
\begin{tabular}{|c|c|c|}
\hline
\multicolumn{3}{|c|}{\textbf{Opracowanie nt. algebr procesów}} \\ \hline \multicolumn{3}{|c|}{Metody formalne - Kolokwium 2}  \\ \hline
\end{tabular}
\end{center}
\end{table}

\tableofcontents


\newpage

\section{Zagadnienia od Szymkata}

\begin{enumerate}

\item porównanie LOTOS-a z CCS (wykład)

\item porównanie CCS z CSP (wykład)

\item analogiczne pojęcia w CCS, CSP i LOTOS-ie (wykład)

\item LOTOS: etykietowany graf przejść (LTS) dla zdefiniowanej
specyfikacji

\item LOTOS: złożenie równoległe procesów (współbieżne) z synchronizacją
pełną, częściową lub bez synchronizacji, wybór procesu aktywnego z wielu
procesów możliwych, ukrywanie jednego procesu w drugim, rekursywna
(nieskończona) iteracja procesu

\item LOTOS: przekazywanie/uzgadnianie danych miedzy procesami za
pośrednictwem określonych bram (akcji synchronizujących), postaci ofert
komunikacyjnych

\item redukcja etykietowanych systemów przejść (LTS) w pakiecie CADP,
różnego typu równoważności bisymulacyjne grafów LTS

\item LOTOS: akcje wewnętrzne (nieobserwowalne), proces bezczynności,
zakończenie procesu, złożenie sekwencyjne, wyzwalanie (aktywowanie) i
blokowanie (przerwanie) procesu, ukrywanie procesu, zmianę nazwy procesu
(reetykietowanie), wykonanie jednego procesu pod warunkiem wykonania
drugiego procesu

\item LOTOS: ogólna składnia specyfikacji behawioralnej (podstawowe słowa
kluczowe i format definicji poszczególnych obiektów)

\end{enumerate}

\section{Źródła}
Opracowanie na podstawie:
\begin{itemize}
\item Szmuc, Szpyrka - Metody formalne w inżynierii oprogramowania systemów czasu rzeczywistego
\item Wykłady Szmuca
\item L. Logrippo, M. Faci, M. Haj-Hussein - An Introduction to LOTOS:  Learning by Examples
\item Kenneth J. Turner - The Formal Specification Language LOTOS A Course for Users
\item Calculi an Automata for Modelling Untimed and Timed Concurrent Systems - Bowman, Howard, Gomez, Rodolfo 
\end{itemize}



\newpage
\section{Pojęcia wstępne}

\begin{description}

\item[System] \hfill \\ 
Szerokie pojęcie, np. oprogramowanie, organizacja, firma, osoba. W kontekście algebr procesów przez system rozumiane są jednak tylko zachowania (oprogramowania, organizacji, itd.) oraz informacje dotyczące tych zachowań (kolejność, czas wykonania, priorytety, wzajemne zależności). System składa się z procesów.

\item[Proces] \hfill \\ 
Proces to pewien logicznie spójny aspekt zachowania systemu, podzbiór zachowań systemu.

\item[Algebry procesów] \hfill \\
Algebry procesów należą do metod formalnych, służą do opisu systemów współbieżnych, opierają się na rachunku algebraicznym.

\end{description}


\begin{description}

\item[Communicating Sequential Processes (CSP)] \hfill \\ 
Pierwsza z algebr procesów (1978), opiera się na modelu komunikacji międzyprocesowej opartym o przekazywanie wiadomości (message passing). Umożliwia modelowanie komunikacji synchronicznej.

\item[Calculus of Communicating Systems (CCS)] \hfill \\ 
Kolejna z algebr procesów (1980) z innymi definicjami i twierdzeniami. W tej algebrze dużą rolę odgrywa możliwość dowodzenia równoważności pomiędzy modelami.

\item[Język LOTOS] \hfill \\
Kolejna algebra procesów, która powstała na bazie CSP i CCS. LOTOS został stworzony (1989) przez ISO w celu standaryzacji specyfikacji usług i protokołów sieciowych.  

\end{description}


\newpage
\section{Zagadnienie 3}

\issue{3}{
Analogiczne pojęcia w CCS, CSP i LOTOS-ie (wykład)
}

Algebry procesów korzystają z różnych nazw dla tych samych pojęć. Zestawienie w tabeli poniżej.

\begin{figure}[H]
\centerline{\includegraphics[scale=1]{nazwy.png}}
\caption{Różne nazwy dla tych samych pojęć w każdej z algebr.}
\label{fig:twophilo}
\end{figure}


Chcąc wyrazić proces w algebrze procesów posługujemy się akcjami i operatorami, które mają go opisywać. Każda algebra definiuje swoje nazewnictwo jak widać w tabeli powyżej, jednak poszczególne pojęcia znaczą zazwyczaj to samo w każdej z algebr: 

\begin{description}

\item[Akcje] \hfill \\ 
Odpowiadają niepodzielnym (atomowym) czynnościom wykonywanym przez proces (w CCS akcje, w CSP zdarzenia, w LOTOS akcje lub bramy). Akcje oznacza się małymi literami.

\item[Operatory] \hfill \\ 
Właściwe dla danej algebry operacje algebraiczne

\item[Wyrażenie algebraiczne] \hfill \\ 
Połączenie akcji i operatorów tworzy wyrażenie algebraiczne. 

\item[Proces] \hfill \\ 
Wyrażenia algebraiczne po spełnieniu odpowiednich warunków syntaktycznych nazywane są procesami. Procesy oznacza się dużymi literami. (w CCS agenci, w CSP procesy, w LOTOS wyrażenie behawioralne)

\item[Alfabet procesu (rodzaj procesu)]
Danemu procesowi przyporządkowane są akcje, które może on wykonywać. Pełny zbiór akcji, które może wykonać dany proces nazywany jest alfabetem tego procesu (w CSP, natomiast w CCS rodzajem, a w LOTOS to po prostu zdarzenia lub akcje).

\item[Komunikacja pomiędzy procesami]
Istnieje możliwość komunikacji pomiędzy procesami (w CCS przez porty, w CSP przez kanały, w LOTOS przez bramy).

\end{description}


\newpage
\section{Zagadnienia 1 i 2}

\issue{1}{
Porównanie LOTOS-a z CCS (wykład)
}

Najważniejsze to:
\begin{enumerate}
\item Sposób synchronizacji procesów - LOTOS używa multi-synchronizacji, a CSS używa synchronizacji poprzez akcje komplementarne
\item Działania operatora ukrywania (w LOTOS) i jego odpowiednika Restriction w CSS jest inne - w LOTOS operator ten generuje akcje wewnętrzne, w CCS usuwa akcje
\item Akcja wewnętrzna w CCS ($\tau$), oprócz tego, że wprowadza niedeterminizm jak w LOTOS, jest także używana do synchronizacji 
\item Poza tym wiele operatorów z CCS ma prawie identyczne znaczenie jak w LOTOS
\item w LOTOS istnieje więcej operatorów niż w CCS 
\end{enumerate}


\issue{2}{
Porównanie CCS z CSP (wykład)
}

\begin{enumerate}

\item (Najważniejsze) Inne podejście do problemu równoważności procesów
\begin{itemize}
\item W CSP problem równoważności jest definiowany z użyciem pojęcia śladu lub przy użyciu modelu niepowodzeń
\item W CCS równoważność procesów jest zdefiniowana w kontekście relacji bisymulacji
\end{itemize}
Bisymulacja jest semantyką równoważności procesów, która bardziej precyzyjnie pozwala porównywać procesy.  

\item Inne podejście do synchronizacji procesów, w CCS synchronizacja z użyciem akcji komplementarnych, w CSP synchronizacja bliższa tej z LOTOS-a

\item Zostały pomyślane jako narzędzia do innych celów
\begin{itemize}
\item CSP została pomyślana jako narzędzie do modelowania i analizy systemów współbieżnych.
\item CCS została pomyślana jako narzędzie do analizy obserwowanego z zewnątrz systemu oraz dowodzenia równoważności między modelami 
\end{itemize}

\item Inna terminologia dla tych samych pojęć
\begin{itemize}
\item Przykładowo w CSP atomowe czynności wykonywane przez proces to akcje
\item Atomowe czynności wykonywane przez proces w CCS nazywane są zdarzeniami
\end{itemize}

\item Inne operatory w każdej z algebr
\begin{itemize}
\item W CSP istnieją inne operatory, których nie ma w CCS, a także w CCS istnieją operatory, których nie ma w CSP.
\end{itemize}

\end{enumerate}



\issue{Bonus}{
Porównanie LOTOS-a z CSP 
}

Najważniejsze to
\begin{enumerate}
\item Operator wyboru - CSP posiada wiele operatorów wyboru, które stosowane są zależnie od sytuacji, LOTOS ma jeden taki operator
\item Akcje wewnętrzne - w CSP nie można się jawnie odwołać do akcji wewnętrznej, a w LOTOS można
\end{enumerate}

\newpage
\section{Zagadnienie 4}

\issue{4}{
LOTOS: etykietowany graf przejść (LTS) dla zdefiniowanej
specyfikacji
}

Chyba chodzi o to żeby narysować LTS-a dla zadanego programu w LOTOS. Przykłady:

\example{}{
\begin{figure}[H]
\centerline{\includegraphics[scale=0.8]{LTS1.png}}
\end{figure}\begin{figure}[H]
\centerline{\includegraphics[scale=0.8]{LTS2.png}}
\end{figure}
}

\example{}{
\begin{figure}[H]
\centerline{\includegraphics[scale=0.95]{LTS3.png}}
\end{figure}
}

\begin{framed}
Przypomnienie definicji:

LTS (Labelled Transition System) to krotka LTS = (S, Act, $\rightarrow$, $s_0$), gdzie:
\begin{itemize}
\item S to skończony zbiór stanów
\item Act to skończony zbiór akcji
\item $\rightarrow$ to przejścia pomiędzy stanami 
\item $s_0 \in S$ to stan początkowy 
\end{itemize}
\end{framed}


\newpage
\section{Zagadnienia 5 i 8}


\issue{5}{
LOTOS: złożenie równoległe procesów (współbieżne) z synchronizacją
pełną, częściową lub bez synchronizacji, wybór procesu aktywnego z wielu
procesów możliwych, ukrywanie jednego procesu w drugim, rekursywna
(nieskończona) iteracja procesu
}


\issue{8}{
LOTOS: akcje wewnętrzne (nieobserwowalne), proces bezczynności,
zakończenie procesu, złożenie sekwencyjne, wyzwalanie (aktywowanie) i
blokowanie (przerwanie) procesu, ukrywanie procesu, zmiana nazwy procesu
(reetykietowanie), wykonanie jednego procesu pod warunkiem wykonania
drugiego procesu
}

Czym się różnią: złożenie sekwencyjne, wyzwalanie (aktywowanie) i wykonanie jednego procesu pod warunkiem wykonania
drugiego procesu - tego nie wiem. 

Przypominamy, że w LOTOS używamy pojęć:
\begin{figure}[H]
\centerline{\includegraphics[scale=0.8]{LOTOS.png}}
\caption{Nazwy stosowane w LOTOS.}
\label{fig:nazwyLOTOS}
\end{figure}


\subsection{Operatory w LOTOS}
Operatory będą przedstawione na przykładzie problemu Producent-Konsument(P-K). \\


\subsubsection{Proces bezczynności stop i zakończenia exit}
W LOTOS istnieją dwa predefiniowane wyrażenie behawioralne - stop i exit. \\
\textbf{stop} - reprezentuje proces zupełnie nieaktywny, oznacza deadlock lub brak możliwości wykonania akcji \\
\textbf{exit} - oznacza normalne (pozytywne) zakończenie procesu



\example{Producent-Konsument}{

\begin{figure}[H]
\centerline{\includegraphics[scale=0.8]{pc.png}}
\caption{Problem producent-konsument.}
\label{fig:pc}
\end{figure}
\begin{itemize}
\item Producent i konsument komunikują się przez kanał. 
\item Kanał działa tak jak FIFO, nie może zmieniać kolejności wiadomości ani ich tracić. Kanał synchronizuje się z konsumentem dopiero gdy zakończy synchronizacje z producentem.  
\item Producent musi utworzyć dwa elementy, a później kończy działanie
\item Konsument musi odebrać oba elementy i kończy działanie
\end{itemize}
W przykładzie łatwo jest wyróżnić trzy procesy (producent, kanał, konsument).
}



\newpage
\subsubsection{Operator prefix ; (operator następstwa sekwencyjnego)}
Operator prefix, czyli średnik \textbf{;} służy do sekwencyjnego złożenia \underline{akcji i wyrażenia behawioralnego}.
\example{}{
\begin{align*}
a; B
\end{align*}

\begin{figure}[H]
\centerline{\includegraphics[scale=0.6]{daily_example.png}}
\caption{Po naciśnięciu guzika, dzwoni dzwonek.}
\label{fig:daily_example}
\end{figure}

To wyrażenie behawioralne, które oznacza, że po wykonaniu akcji a proces przechodzi w stan B (reprezentowany przez wyrażenie behawioralne B).
}

Producent jako proces może być widziany jak czarna skrzynka komunikująca się ze światem przez port pc1(tworzy jeden element) oraz pc2(tworzy drugi element). Proces producenta można więc zapisać jako:
\begin{figure}[H]
\centerline{\includegraphics[scale=1]{prod1_proc.png}}
\label{fig:prod1_proc}
\end{figure}
Właściwe bramy z których korzysta proces są jego parametrami i muszą być określone na etapie tworzenia instancji procesu. (podrozdział psanie programów z użyciem operatorów). \\
Analogicznie można opisać proces konsumenta i kanału:
\begin{figure}[H]
\centerline{\includegraphics[scale=1]{cons1_proc.png}}
\label{fig:cons1_proc}
\end{figure}
\begin{figure}[H]
\centerline{\includegraphics[scale=1]{chan1_proc.png}}
\label{fig:chan1_proc}
\end{figure}
Taki opis procesów jest dobry jeśli analizujemy każdy proces oddzielnie, jednak aby złożyć trzy procesy razem potrzebne będzie jeszcze użycie operatorów składania równoległego (parallel composition operators), które będą omówione później. \\ \\
Zauważmy w tym przykładzie, że wszystkie wykonywane akcje są obserwowalne (zewnętrzne). Przykładowo wykonanie akcji pc1 przez producenta powoduje również wykonanie w tym samym momencie wykonanie tej samej akcji w procesie Kanału. Analogicznie dla pozostałych akcji.


Warto jeszcze zwrócić na notację 
\begin{figure}[H]
\centerline{\includegraphics[scale=1]{notacja.png}}
\label{fig:chan1_proc}
\end{figure}
Konkretnie na końcówkę ":\textbf{ exit}". \\
Taka końcówka oznacza, że proces jest w stanie wykonać zachowanie \textbf{exit} na końcu działania. Końcówka ta może być ustawiona na jedną z dwóch wartości: \textit{exit} jeśli proces może pomyślnie zakończyć działanie albo wartość \textit{noexit} w innym wypadku. 



\newpage
\subsubsection{Operator wyboru $\lbrack \rbrack$ (choice operator)} 

Aby zilustrować działanie operatora wyboru modyfikujemy rozważany problem producent-konsument w ten sposób, że:
\begin{itemize}
\item Kanał może dostarczyć konsumentowi pierwszy element od producenta, zanim producent wyprodukuje drugi.
\end{itemize}
Aby zamodelować to zachowanie potrzebny jest nam operator wyboru, który jest oznaczany \textbf{[]}. Operator ten oznacza wybór pomiędzy dwoma lub większą ilością zachowań. \\
W przypadku problemu P-K Kanał musi najpierw się synchronizować z Producentem na bramce pc1, a później ma dwie możliwości: 
\begin{enumerate}
\item Może się synchronizować z Producentem na bramce pc2 
\item Może się synchronizować z Konsumentem na bramce cc1
\end{enumerate}
Wybór jednego z zachowań eliminuje możliwość wykonania drugiego. Można więc zapisać: 


\begin{figure}[H]
\centerline{\includegraphics[scale=1]{chan2_proc.png}}
\label{fig:chan2_proc}
\end{figure}

Operator wyboru umożliwia modelowanie niedeterministycznych zachowań, np. tak jak poniżej dla automatu sprzedającego kawę i mleko w linii numer 2.

\begin{figure}[H]
\centerline{\includegraphics[scale=1]{auto_proc.png}}
\label{fig:auto_proc}
\end{figure}
Zwróćmy uwagę na ważną różnicę pomiędzy 2 i 3. W przypadku drugim po wrzuceniu monety (insert\_quarter) nie mamy już wyboru (środowisko procesu nie ma wyboru) czy chcemy kawę czy mleko, bo automat niedeterministycznie sam wybiera kawę lub mleko. W przykładzie 3 po wrzuceniu monety możemy wybrać (środowisko procesu może wybrać) czy chcemy kawę czy mleko.


\begin{figure}[H]
\centerline{\includegraphics[scale=0.6]{nondeter.png}}
\label{fig:auto_proc}
\end{figure}

Jaka jest różnica pomiędzy dwoma rysunkami ? \\ Chodzi o to, że środowisko ma wybór albo go nie ma (tak jak przy kupowaniu kawy lub mleka w powyższym przykładzie). \\ \\ Na Rysunku z prawej środowisko może wybrać a, ale nie ma już wyboru co będzie następne (niedeterminizm). Na rysunku z lewej środowisko może wybrać a, a później może jeszcze wybrać czy chce b czy c (po wrzuceniu monety do automatu możemy jeszcze wybrać czy chcemy kawę czy mleko).



\newpage
\subsubsection{Operator aktywowania >> (enable) }
Operator aktywowania jest podobny do \textbf{;}, ale służy do sekwencyjnego łączenia \underline{dwóch wyrażeń behawioralnych} (a nie akcji i wyrażenia behawioralnego jak w przypadku operatora \textbf{;}).
\begin{figure}[H]
\centerline{\includegraphics[scale=0.8]{enables_1.png}}
\label{fig:enables_1}
\end{figure}

\begin{figure}[H]
\centerline{\includegraphics[scale=0.8]{enables_2.png}}
\label{fig:enables_2}
\end{figure}

\newpage
Wracając do przykładu problemu P-K, moglibyśmy zapisać, że:
\begin{figure}[H]
\centerline{\includegraphics[scale=0.8]{chan3_proc.png}}
\label{fig:chan3_proc}
\end{figure}





\newpage
\subsubsection{Zdarzenie (akcje) wewnętrzne \textit{i}}


W LOTOS wyrażenie behawioralne może wykonywać dwa rodzaje akcji.
\begin{itemize}
\item Akcje nieobserwowalne (wewnętrzne) - to akcje, które dany proces może wykonywać niezależnie od innych procesów lub jakichś innych zdarzeń, w LOTOS są oznaczane przez \textbf{i}.
\item Akcje obserwowalne (zewnętrzne) - to akcje, które wymagają synchronizacji ze środowiskiem działania procesu, te akcje mogą być wykonywane w punktach synchronizacji, czyli bramkach (gates). 

\end{itemize}



Znów dla celów pokazania działania zdarzenia wewnętrznego modyfikujemy rozważany problem P-K. 
\begin{itemize}
\item  Teraz chcemy zamodelować sytuację w której kanał może samoistnie stracić pierwszy oraz drugi element (np. jakiś błąd wewnętrzny). Zauważmy, że taka akcja nie jest zależna od żadnego innego procesu, ale jest akcją wewnątrz kanału (nieobserwowalną).
\end{itemize}

Do tej pory w procesie używaliśmy tylko akcji obserwowalnych, przykładowo pc1, oznaczało w procesie producenta wyprodukowanie pierwszego elementu, a w procesie kanału odebranie pierwszego elementu. \\ \\
Teraz chcemy dodać do procesu kanału akcję, która dzieje się tylko w tym procesie (wewnętrzną). Akcję wewnętrzną w procesie oznaczamy przez \textbf{i}. \\
Chcemy więc mieć sytuację opisaną nieformalnie jako:


\begin{figure}[H]
\centerline{\includegraphics[scale=0.8]{chan4_proc.png}}
\label{fig:chan4_proc}
\end{figure}

Zauważamy, że 2 i 4 są równoważne, bo występuje w nich akcja wewnętrzna, co prowadzi do formalnej implementacji:

\begin{figure}[H]
\centerline{\includegraphics[scale=0.8]{chan5_proc.png}}
\label{fig:chan5_proc}
\end{figure}

Użycie akcji wewnętrznej wprowadza niedeterminizm, ponieważ nie wiadomo czy wykona się synchronizacja pc2, cc1 czy też zdarzenie wewnętrzne i. Gdyby nie było w wyborze trzeciego przypadku z i, to mielibyśmy determinizm, bo środowisko mogłoby wybrać pomiędzy pc2 lub cc1.  \\ \\
Oczywiście trzeba teraz dostosować działanie konsumenta:

\begin{figure}[H]
\centerline{\includegraphics[scale=0.8]{cons2_proc.png}}
\label{fig:cons2_proc}
\end{figure}

\subsubsection{Kilka ciekawszych przykładów z wydarzeniami wewnętrznymi}
\example{}{
\begin{align*}
& \text{coffe; exit } [] \text{ milk; exit}
\end{align*}
To proces, który może synchronizować się zarówno na akcji coffe jak i milk.
}

\example{}{
\begin{align*}
& \text{i; coffe; exit } [] \text{ milk; exit}
\end{align*}
To proces, który może nie być w stanie synchronizować się na milk, ale zawsze będzie w stanie synchronizować się na coffe.
\begin{itemize}
\item Jeśli środowisko proponuje milk, a w procesie zaszło już wydarzenie i, to nie może się synchronizować przez milk
\item Jeśli środowisko proponuje coffe, a w procesie nie zaszło jeszcze wydarzenie i, to można założyć, że wydarzenie i kiedyś zajdzie.
\end{itemize}
}

\example{}{
\begin{align*}
& \text{i; coffe; exit } [] \text{ i; milk; exit}
\end{align*}
Przez analogię do poprzedniego przypadku, to proces, który może nie być w stanie synchronizować się na coffee lub na milk.
}





\newpage
\subsubsection{Operator przerwania, deaktywowania (disable) [>}
\begin{figure}[H]
\centerline{\includegraphics[scale=0.8]{disable_1.png}}
\label{fig:disable1}
\end{figure}
\begin{figure}[H]
\centerline{\includegraphics[scale=0.8]{disable_2.png}}
\label{fig:disable2}
\end{figure}

\newpage
Wracając do przykładu P-K:
\begin{figure}[H]
\centerline{\includegraphics[scale=0.8]{disable_pc.png}}
\label{fig:disable_pc}
\end{figure}

\begin{framed}
\begin{itemize}
\item Jak działa i jak wygląda operator przerwania ? 
\end{itemize}
\end{framed}



\newpage
\subsubsection{Operatory złożenia równoległego}
Są trzy operatory złożenia równoległego w LOTOS:
\begin{itemize}
\item \textbf{Operator złożenia równoległego bez synchronizacji (interleaving) |||} \\
Operator ten służy do opisu równoległości pomiędzy zachowaniami gdy żadna synchronizacja nie jest wymagana. 
\example{}{
\begin{align*}
\text{(out1; out2; exit) ||| (in1; in2; exit)}
\end{align*}
}
Taki zapis nie oznacza jednak, że akcje out1 i in1 mogą wykonać się jednocześnie. W LOTOS wszystkie operacje są atomiczne i wszystkie procesy wykonywane są jakby na jednym procesorze (tzw. interleaving concurrency). \\ Jeśli wykonywane są więc dwa zdarzenia a i b równolegle, to oznacza to, że a może wykonać się przed b lub b przed a. Konsekwencją takiego działania jest to, że każda formuła wykorzystująca równoległość może być przepisana przy pomocy operatora wyboru. Ten przykład można równoważnie przepisać jako:
\example{}{
\begin{figure}[H]
\centerline{\includegraphics[scale=0.8]{oralt.png}}
\label{fig:oralt}
\end{figure}
Czyli wszystkie możliwe kombinacje kolejności wykonać zdarzeń z obu zachowań.
}


\newpage
\item \textbf{Operator złożenia równoległego z częściową synchronizacją (selective) |[L]| } \\
Tutaj procesy wykonują wybrane akcje równocześnie, akcje które oba procesy mają wykonać równolegle w tym samym czasie ujmowane są w miejsce litery L.
\example{}{
\begin{figure}[H]
\centerline{\includegraphics[scale=0.7]{partsynch.png}}
\label{fig:partsynch}
\end{figure}
Proces 1 musi czekać aż proces drugi wykona d, ponieważ zdarzenie a musi być wykonane równocześnie przez dwa procesy. Pozostałe zdarzenia mogą wykonać się w dowolnej kolejności i nie wykonują się w jednej chwili. 
}



\item \textbf{Operator złożenia równoległego z pełną synchronizacją (full) || } \\
Wszystkie zdarzenia w procesach się synchronizują, a nie tylko niektóre wybrane, czyli tak jakby lista w operatorze częściowej synchronizacji zawierała wszystkie bramki.
\end{itemize} 


\newpage
\subsubsection{Operator ukrywania (hiding)}
W operatorze ukrywania chodzi o to, że chcemy zamodelować zdarzenie, które się synchronizuje, ale tylko wewnątrz pewnego procesu. \\ \\
Przykładowo mamy proces Producenta, który składa się z dwóch procesów:
\begin{figure}[H]
\centerline{\includegraphics[scale=0.7]{producent.png}}
\label{fig:producent}
\end{figure}

Chcemy aby procesy z których składa się Producent synchronizowały się między sobą na zdarzenia mem\_val, ale nie chcemy aby synchronizacja ta wymagała synchronizacji z otoczeniem procesu Producenta tak jak jest to dla zdarzeń pc1 i pc2. Wystarczy zatem dodać operator hide tak jak w przykładzie powyżej, wówczas zdarzenie mem\_val nie musi się synchronizować z otoczeniem, a powoduje synchronizację procesów Compute. 



\newpage
\subsubsection{Rekurencja procesów}
Proces może odwoływać się w swojej definicji do samego siebie. 
\example{}{
\begin{figure}[H]
\centerline{\includegraphics[scale=0.7]{recur.png}}
\label{fig:recur}
\end{figure}
Warto zwrócić uwagę na \textbf{noexit} w deklaracji procesu.
}

Rekurencja jest w LOTOS osiągana przez tworzenie instancji procesu - proces tworzy instancję samego siebie. \\


Pamiętać jednak należy, że o rekurencji w LOTOS nie należy myśleć jak o rekurencji w języku programowania. Nie ma czegoś takiego jak stos wywołania procesu itp. Rekurencja w LOTOS odnosi się do iteracji, a więc tak jakbyśmy kilkukrotnie wykonywali ten sam kod. 


\subsubsection{Reetykietowanie}

Operator reetykietowania nie służy do specyfikowania zachowania procesu, jest to operator, który jest używany do formalnej definicji tego jak bramy formalne są zastępowane przez właściwe bramy przy tworzeniu instancji procesu. Operator ten jest zapisywany jako:

\begin{figure}[H]
\centerline{\includegraphics[scale=0.7]{labels.png}}
\end{figure}

\newpage
\section{Zagadnienie 9}

\issue{9}{
LOTOS: ogólna składnia specyfikacji behawioralnej (podstawowe słowa
kluczowe i format definicji poszczególnych obiektów)
}


W języku lotos aby zdefiniować proces i wyrażenia behawioralne korzystamy z kilku słów kluczowych. 
\begin{figure}[H]
\centerline{\includegraphics[scale=0.6]{pdef.png}}
\caption{Definiowanie procesu w LOTOS.}
\label{fig:pdef}
\end{figure}


W LOTOS tworzone są instancje procesów, najpierw definiowany jest proces (co odpowiada napisaniu metody w języku programowania), a następnie tworzona jest jego instancja (co odpowiada wywołaniu danej metody w języku programowania). \\ \\

Wracając do przykładu P-K, zdefiniowaliśmy już procesy osobno, teraz chcemy stworzyć system przy pomocy operatorów składania równoległego. 
\example{}{
\begin{figure}[H]
\centerline{\includegraphics[scale=0.7]{program.png}}
\label{fig:program}
\end{figure}
}  

\begin{figure}[H]
\centerline{\includegraphics[scale=0.7]{prog_expl.png}}
\label{fig:program_expl}
\end{figure}

\begin{figure}[H]
\centerline{\includegraphics[scale=0.7]{prog_expl2.png}}
\label{fig:program_expl2}
\end{figure}











\newpage
\section{Zgadanienie 6}

\issue{6}{
LOTOS: przekazywanie/uzgadnianie danych miedzy procesami za
pośrednictwem określonych bram (akcji synchronizujących), postaci ofert komunikacyjnych
}

Dane i przekazywanie danych są dostępne w pełnej wersji LOTOS, tzw. full LOTOS. \\ 
W podstawowym LOTOS akcja była synonimem bramy. W full LOTOS akcja składa się z trzech komponentów: 
\begin{itemize}
\item nazwa bramy
\item lista zdarzeń
\item opcjonalny predykat
\end{itemize}

Procesy się synchronizują pod warunkiem, że 
\begin{itemize}
\item Bramy nazywają się tak samo
\item Lista zdarzeń pasuje do siebie
\item Predykat jest spełniony
\end{itemize}

Przykład akcji w full LOTOS:\

\begin{figure}[H]
\centerline{\includegraphics[scale=0.75]{fullLotos.png}}
\end{figure}

Zdarzenie może oferować (!) lub akceptować wartość (?). 
Możliwe typy komunikacji: 

\begin{figure}[H]
\centerline{\includegraphics[scale=0.75]{fullLotos2.png}}
\end{figure}


\newpage
\section{Zagadnienie 7}

\issue{7}{
Redukcja etykietowanych systemów przejść (LTS) w pakiecie CADP, różnego typu równoważności bisymulacyjne grafów LTS
}

W pakiecie CADP dla danego systemu możemy wygenerować LTS, a następnie zredukować go według wskazanego typu równoważności przy użyciu narzędzia REDUCTOR. Grafy LTS mogą być porównywane przy użyciu wieli równoważności bisymulacyjnych, m.in.: 
\begin{itemize}
\item Strong equivalence - stany są równoważne gdy akcje możliwe do wykonania w jednym są możliwe do wykonania w drugim oraz stany wynikowe są również powiązane.

\begin{figure}[H]
\centerline{\includegraphics[scale=0.75]{bisim1.png}}
\end{figure}
 
\item Branching equivalnce
\begin{figure}[H]
\centerline{\includegraphics[scale=0.75]{bisim2.png}}
\end{figure}
 
 
\item Weak equivalence 
\begin{figure}[H]
\centerline{\includegraphics[scale=0.75]{bisim3.png}}
\end{figure}


\end{itemize}
Stosowanie tych oraz innych równoważności dostępne jest w pakiecie CADP.



\end{document}


