\documentclass[a4paper,15pt]{article}
\usepackage{amssymb}
\usepackage{amsmath}
\usepackage[english, polish]{babel}
\usepackage[utf8]{inputenc}   % lub utf8
\usepackage[T1]{fontenc}
\usepackage{graphicx}
\usepackage{anysize}
\usepackage{enumerate}
\usepackage{times}
\usepackage{caption}
\usepackage{titlesec}
\usepackage{float}
\usepackage{titleps,kantlipsum}
\usepackage{listings}
\usepackage{xcolor}
\usepackage{hyperref}
\usepackage{framed}
\usepackage{tcolorbox}
\usepackage{subfig}
\usepackage{tikz}

\lstloadlanguages{Matlab}
 
\usepackage[justification=centering]{caption}
\titlelabel{\thetitle.\quad}

\pagenumbering{arabic}

\DeclareCaptionFont{white}{\color{white}}
\DeclareCaptionFormat{listing}{%
  \parbox{\textwidth}{\colorbox{darkgreen}{\parbox{\textwidth}{#1#2#3}}\vskip-4pt}}
\captionsetup[lstlisting]{format=listing,labelfont=white,textfont=white}
\lstset{frame=lrb,xleftmargin=\fboxsep,xrightmargin=-\fboxsep}

% Definicja nowego stylu strony
\newpagestyle{mypage}
{
  \headrule
  
  \sethead
  { \MakeUppercase{\thesection\quad \sectiontitle} } 
  {}
  {\thesubsection\quad \subsectiontitle}
  
  \setfoot
  {}
  {}
  {\thepage}
}

\newpagestyle{mypage_1}
{
	\headrule
	
	\sethead
	{  }
	{\MakeUppercase{Metody pomiaru i szacowania oprogramowania}}
	{}
	
	\setfoot
	{}
	{\thepage}
	{}
}

\settitlemarks{section,subsection,subsubsection}

\pagestyle{mypage_1}

\definecolor{mGreen}{rgb}{0,0.6,0}
\definecolor{mGray}{rgb}{0.5,0.5,0.5}
\definecolor{mPurple}{rgb}{0.58,0,0.82}
\definecolor{mKeyword}{RGB}{0,0,242}
\definecolor{backgroundColour}{RGB}{242,242,242}

\newcommand{\definition}[2]{
    \begin{tcolorbox}[colback=gray!5!white,colframe=gray,title={Definicja -  #1}]
        #2
    \end{tcolorbox}
}

\newcommand{\question}[2]{
    \begin{tcolorbox}[colback=black!5!white,colframe=black,title={Zalety i wady}]
        #2
    \end{tcolorbox}
}

%\marginsize{left}{right}{top}{bottom}
\marginsize{3cm}{3cm}{3cm}{3cm}
\sloppy
\titleformat{\section}
  {\normalfont\Large\bfseries}{\thesection}{1em}{}[{\titlerule[0.8pt]}]
 
 \definecolor{darkred}{rgb}{0.9,0,0}
\definecolor{grey}{rgb}{0.4,0.4,0.4}
\definecolor{orange}{rgb}{1,0.6,0.05}
\definecolor{darkgreen}{rgb}{0.2,0.5,0.05}
 


\lstdefinestyle{CStyle}{
    backgroundcolor=\color{backgroundColour},   
    commentstyle=\color{mGreen},
    keywordstyle=\color{mKeyword},
    numberstyle=\tiny\color{mGray},
    stringstyle=\color{mPurple},
    basicstyle=\footnotesize,
    breakatwhitespace=false,         
    breaklines=true,                 
    %captionpos=b,                    
    keepspaces=true,                 
    numbers=left,                    
    numbersep=5pt,                  
    showspaces=false,                
    showstringspaces=false,
    showtabs=false,                  
    tabsize=2,
    language=C
}


\newcommand{\Hilight}{\makebox[0pt][l]{\color{cyan}\rule[-4pt]{0.65\linewidth}{14pt}}}


\begin{document}

\begin{table}
\begin{center}
\begin{tabular}{|c|c|c|}
\hline
\multicolumn{3}{|c|}{\textbf{Opracowanie}} \\ \hline MPiSO & 25 IV 2019 & Kolokwium 1 \\ \hline

\end{tabular}
\end{center}
\end{table}

\tableofcontents

\newpage
\section{Zagadnienia}
\begin{enumerate}

\item Typy relacji pomiędzy zadaniami w harmonogramie + przykłady z projektów informatycznych

\item Cykle życia oprogramowania - opis + różnice pomiędzy poszczególnymi cyklami

\end{enumerate}


\section{Źródła}
Opracowanie na podstawie:
\begin{itemize}
\item Zarządzanie projektem informatycznym, Kazimierz Frączkowski  -> doku wiki
\item \url{https://www.pmbypm.com/finish-to-start-relationship/}
\item \url{https://www.pmbypm.com/start-to-start-relationship/}
\item \url{https://www.pmbypm.com/finish-to-finish-relationship/}
\item \url{https://www.pmbypm.com/start-to-finish-relationship/}
\item \url{https://existek.com/blog/sdlc-models/}
\end{itemize}


\newpage
\section{Opracowanie zagadnień}

\subsection{Typy relacji}
\subsubsection{Finish to start}
Koniec – Start (ang. Finish-to-Start FS) – zadanie B nie może rozpocząć się przed ukończeniem zadania A. \\
Oznaczenie na harmonogramie:


\begin{figure}[H]
    \centering
    \subfloat[]{{\includegraphics[width=5cm]{FS.png} }}%
    \qquad
    \subfloat[]{{\includegraphics[width=5cm]{FSh.png} }}%
    \caption{Finish to start}%
    \label{fig:example}%
\end{figure}


Przykłady z projektu informatycznego:
\begin{itemize}
\item Projekt interfejsu / prototyp interfejsu
\item Napisanie user guide / wydrukowanie user guide
\item Zebranie wymagań / podpisanie umowy dotyczącej wymagań
\end{itemize}


\begin{framed}
\begin{itemize}
\item Podaj definicję, graficzną reprezentację oraz 3 przykłady relacji Finish-to-Start.
\end{itemize}
\end{framed}



\newpage
\subsubsection{Start to start}
Start – Start (ang. Start-to-Start SS) – zadanie B nie może rozpocząć się przed rozpoczęciem zadania A. \\
Oznaczenie na harmonogramie:

\begin{figure}[H]
    \centering
    \subfloat[]{{\includegraphics[width=5cm]{SS.png} }}%
    \qquad
    \subfloat[]{{\includegraphics[width=5cm]{SSh.png} }}%
    \caption{Start to start}%
    \label{fig:example}%
\end{figure}


Przykłady z projektu informatycznego:
\begin{itemize}
\item Pisanie dokumentu HTML / pisanie pliku CSS
\item Uzyskanie danych z REST API / obróbka danych z REST API
\item Pisanie kodu / Pisanie dokumentacji 
\end{itemize}


\begin{framed}
\begin{itemize}
\item Podaj definicję, graficzną reprezentację oraz 3 przykłady relacji Start-to-Start.
\end{itemize}
\end{framed}

\newpage
\subsubsection{Finish to finish}
Koniec – Koniec (ang. Finish-to-Finish FF) – zadanie B nie może zakończyć się dopóki nie zakończy się zadanie A. \\
Oznaczenie na harmonogramie:

\begin{figure}[H]
    \centering
    \subfloat[]{{\includegraphics[width=5cm]{FF.png} }}%
    \qquad
    \subfloat[]{{\includegraphics[width=5cm]{FFh.png} }}%
    \caption{Finish to finish}%
    \label{fig:example}%
\end{figure}


Przykłady z projektu informatycznego:
\begin{itemize}
\item Napisanie kodu dla modułu X / testowanie jednostkowe modułu X
\item Lutowanie połączeń / Testowanie połączeń
\item Wdrożenie systemu / Zarządzanie projektem
\end{itemize}

\begin{framed}
\begin{itemize}
\item Podaj definicję, graficzną reprezentację oraz 3 przykłady relacji Finish-to-finish.
\end{itemize}
\end{framed}


\newpage
\subsubsection{Start to finish}
Start – Koniec (ang. Start-to-Finish SF) – zadanie B nie może zakończyć się dopóki nie rozpocznie się zadanie A. 

\begin{figure}[H]
    \centering
    \subfloat[]{{\includegraphics[width=5cm]{SF.png} }}%
    \qquad
    \subfloat[]{{\includegraphics[width=5cm]{SFh.png} }}%
    \caption{Start to finish}%
    \label{fig:example}%
\end{figure}


Przykłady z projektu informatycznego:
\begin{itemize}
\item Wypuszczenie nowej wersji systemu / Likwidacja starego systemu
\item Publiczna reklama systemu  / Opracowanie strategii marketingowej    
\item Dostosowanie do potrzeb niepełnosprawnych / Otrzymanie certyfikatu bezpieczeństwa 
\end{itemize}

\begin{framed}
\begin{itemize}
\item Podaj definicję, graficzną reprezentację oraz 3 przykłady relacji Start-to-finish.
\end{itemize}
\end{framed}






\newpage
\subsection{Cykle życia oprogramowania}

Software Development Life Cycle (SDLC), czyli cykl życia oprogramowania to model określający fazy tworzenia i funkcjonowania oprogramowania, a także ich przebieg. SDLC powinno być odpowiednio dobrane do wykonywanego projektu. \\
Do najbardziej znanych SDLC należą:
\begin{itemize}
\item Waterfall model (kaskadowy)
\item V-shaped model
\item Evolutionary model (ewolucyjny)
\item Spiral Method (spiralny)
\item Iterative and Incremental (Iteracyjny i przyrostowy)
\item Agile development
\end{itemize}

\subsubsection{Fazy tworzenia oprogramowania}
Niezależnie od wyboru SDLC w procesie tworzenia oprogramowania obecnych jest kilka faz. Znajomość tych faz pozwala na lepsze zrozumienie każdego SDLC oraz różnic pomiędzy nimi.
\begin{itemize}
\item Określenie wymagań
\item Projektowanie
\item Implementacja
\item Testowanie
\item Wdrożenie
\end{itemize}



\newpage
\subsubsection{Waterfall model}


\definition{Waterfall model}{
W tym modelu każda z faz tworzenia oprogramowania zaczyna się dopiero wtedy gdy poprzednia została zakończona.
\begin{figure}[H]
\centerline{\includegraphics[scale=0.5]{waterfall.png}}
\caption{Model kaskadowy}
\label{fig:ANTIinhibitor}
\end{figure}

}

\begin{tcolorbox}[colback=red!5!white,colframe=red!75!black]
  Wady:
  \begin{itemize}
  	\item Trudność wprowadzania zmian w projekcie
  	\item Kosztowny i wymagający dużo czasu
  \end{itemize}
\end{tcolorbox}

\begin{tcolorbox}[colback=green!5!white,colframe=green!75!black]
  Zalety:
  \begin{itemize}
  	\item Łatwy do wyjaśnienia
  	\item Pozwala dokładnie zaplanować projekt
  \end{itemize}
\end{tcolorbox}


\begin{framed}
\begin{itemize}
\item Opisz model kaskadowy, podaj dwie wady i zalety.
\end{itemize}
\end{framed}




\newpage
\subsubsection{Evolutionary model}


\definition{Evolutionary model}{
Celem modelu ewolucyjnego jest poprawienie modelu kaskadowego poprzez rezygnację ze ścisłego, liniowego następstwa faz

Pozostawia się te same czynności, ale pozwala na powroty, z pewnych faz do innych faz poprzedzających

Tym samym umożliwia się adaptowanie do zmian w wymaganiach i korygowanie popełnionych błędów (oba zjawiska występują w niemal wszystkich praktycznie wykonywanych projektach – stąd model ewolucyjny jest bardziej realistyczny od kaskadowego) 

}

\begin{framed}
\begin{itemize}
\item Czym różni się model ewolucyjny od kaskadowego ?
\end{itemize}
\end{framed}




\newpage
\subsubsection{Spiral model}


\definition{Spiral model}{
W tym modelu oprogramowanie przechodzi iteracyjnie przez cztery fazy: rozwój oprogramowania, przewidywanie ryzyka, planowanie, weryfikacja. W odróżnieniu od innych modeli duży nacisk kładziony jest na szacowanie ryzyka.
\begin{figure}[H]
\centerline{\includegraphics[scale=0.5]{spiral.png}}
\caption{Model spiralny}
\label{fig:ANTIinhibitor}
\end{figure}

}

\begin{tcolorbox}[colback=red!5!white,colframe=red!75!black]
  Wady:
  \begin{itemize}
  	\item Wymagana umiejętność szacowania ryzyka
  	\item Kosztowny i wymagający dużo czasu
  \end{itemize}
\end{tcolorbox}

\begin{tcolorbox}[colback=green!5!white,colframe=green!75!black]
  Zalety:
  \begin{itemize}
  	\item Szansa realnej estymacji (czasu, kosztów itp.) ze względu na wczesną eliminację ryzyka
  \end{itemize}
\end{tcolorbox}


\begin{framed}
\begin{itemize}
\item Opisz model spiralny, podaj wady i zalety.
\end{itemize}
\end{framed}







\newpage
\subsubsection{Incremental model}


\definition{Model przyrostowy}{
Jest to model iteracyjny, który polega na realizacji środkowych (planowanie raz na początku i deployment raz na końcu) faz modelu kaskadowego, ale w krótszych odstępach czasowych.
\begin{figure}[H]
\centerline{\includegraphics[scale=0.5]{incremental.png}}
\caption{Model przyrostowy}
\label{fig:ANTIinhibitor}
\end{figure}

}

\begin{tcolorbox}[colback=red!5!white,colframe=red!75!black]
  Wady:
  \begin{itemize}
  	\item Wymaga dużego zaangażowania klienta
  	\item Sztywno zdefiniowana kolej wykonywania procesów
  \end{itemize}
\end{tcolorbox}

\begin{tcolorbox}[colback=green!5!white,colframe=green!75!black]
  Zalety:
  \begin{itemize}
  	\item Możliwość wprowadzania zmian w projekcie
  	\item Łatwa identyfikacja błędów i ich naprawa
  \end{itemize}
\end{tcolorbox}


\begin{framed}
\begin{itemize}
\item Opisz model przyrostowy, podaj dwie wady i zalety.
\end{itemize}
\end{framed}


\end{document}
