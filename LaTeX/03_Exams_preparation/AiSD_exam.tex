\documentclass[a4paper,11pt]{article}
\usepackage{algorithm}
\usepackage{algpseudocode}
\usepackage{amssymb}
\usepackage{amsthm}
\usepackage{amsmath}
\usepackage[english, polish]{babel}
\usepackage[utf8]{inputenc}   % lub utf8
\usepackage[T1]{fontenc}
\usepackage{graphicx}
\usepackage{anysize}
\usepackage{enumerate}
\usepackage{times}
\usepackage{xcolor}
\usepackage{titlesec}
\usepackage{float}
\usepackage[justification=centering]{caption}
\titlelabel{\thetitle.\quad}
\usepackage{titlesec}
\usepackage{titleps,kantlipsum}
\usepackage{tikz}
\usepackage{color}
\usepackage{listings}
\usepackage{caption}
\lstloadlanguages{Matlab,C++}
\usepackage{hyperref}
\usepackage{framed}
\usepackage{siunitx}
\usepackage{mathrsfs}
\usepackage{cancel}

\usetikzlibrary{calc,through,backgrounds,positioning,fit}
\usetikzlibrary{shapes,arrows,shadows,patterns}

\tikzstyle{place}=[shape=circle, draw, minimum height=10mm]
\tikzstyle{place_1}=[shape=circle, draw, minimum height=5mm]
\tikzstyle{trig}=[shape=circle, draw, dashed, minimum height=10mm]
\tikzstyle{trans}=[shape=rectangle, draw, minimum height=15mm, minimum width=16mm]

\newdimen\LineSpace
\tikzset{
    line space/.code={\LineSpace=#1},
    line space=3pt
}

\pgfdeclarepatternformonly[\LineSpace]{my north east lines}{\pgfqpoint{-1pt}{-1pt}}{\pgfqpoint{\LineSpace}{\LineSpace}}{\pgfqpoint{\LineSpace}{\LineSpace}}%
{
    \pgfsetlinewidth{0.4pt}
    \pgfpathmoveto{\pgfqpoint{0pt}{0pt}}
    \pgfpathlineto{\pgfqpoint{\LineSpace + 0.1pt}{\LineSpace + 0.1pt}}
    \pgfusepath{stroke}
}

\newpagestyle{mypage}
{
  \headrule
  
  \sethead
  { \MakeUppercase{\thesection\quad \sectiontitle} } 
  {}
  {\thesubsection\quad \subsectiontitle}
  
  \setfoot
  {}
  {\thepage}
  {}
}

\newpagestyle{mypage_1}
{
	\headrule
	
	\sethead
	{  }
	{\MakeUppercase{Algorytmy i struktury danych - Egzamin}}
	{}
	
	\setfoot
	{}
	{}
	{}
}

\pagestyle{mypage_1}
%\marginsize{left}{right}{top}{bottom}
\marginsize{3cm}{3cm}{3cm}{3cm}
\sloppy
 
\begin{document}
\tableofcontents

\newpage
\section{Pytania na egzamin z algorytmów}
\subsection{Ogólne}
\begin{enumerate}
\item Co to jest algorytm ? \\ \colorbox{green}{Plusy:+} \colorbox{red}{Minusy: }

\item Co to jest struktura danych ? \\ \colorbox{green}{Plusy:+} \colorbox{red}{Minusy: }

\item Czym jest rozumowanie indukcyjne ? Przykład na ciągu arytmetycznym. \\ \colorbox{green}{Plusy+:} \colorbox{red}{
Minusy: }

\item Czym jest złożoność algorytmu ? \\ \colorbox{green}{Plusy+:} \colorbox{red}{Minusy: }

\end{enumerate}
\subsection{Sortowanie przez wstawianie, notacje asymptotyczne}
\begin{enumerate}
\item Algorytm sortowania przez wstawianie - napisać algorytm. \\ \colorbox{green}{Plusy:+} \colorbox{red}{Minusy: }

\item Algorytm sortowania przez wstawianie - działanie na danych: 5, 2, 4, 6, 1, 3 \\ \colorbox{green}{Plusy:+} \colorbox{red}{Minusy: }

\item Algorytm sortowania przez wstawianie - obliczyć czas działania w przypadku pesymistycznym i optymistycznym.\\ \colorbox{green}{Plusy:+} \colorbox{red}{Minusy: }

\item Algorytm sortowania przez wstawianie - wykazać złożoność w przypadku pesymistycznym i optymistycznym. \\ \colorbox{green}{Plusy:+} \colorbox{red}{Minusy: }

\item Podać definicje notacji asymptotycznych wraz z przykładowymi rysunkami.\\ \colorbox{green}{Plusy:+} \colorbox{red}{Minusy: }

\end{enumerate}
\subsection{Dziel i zwyciężaj, MergeSort, złożoność rekurencyjna}
\begin{enumerate}
\item Na czym polega podejście dziel i zwyciężaj w pisaniu algorytmów ?\\ \colorbox{green}{Plusy:+} \colorbox{red}{Minusy: }

\item Napisać funkcję Merge oraz MergeSort z algorytmu sortowania przez scalanie.\\ \colorbox{green}{Plusy:+} \colorbox{red}{Minusy: }

\item Sortowanie przez scalanie - działanie na danych: 2, 8, 0, 2, 2, 0, 1, 8 . Kolejność wykonywania.\\  \colorbox{green}{ Plusy:+} \colorbox{red}{Minusy: }

\item Jaka jest złożoność MergeSort ? Jaka jest złożoność Merge ? \\ \colorbox{green}{Plusy:+} \colorbox{red}{Minusy: }

\item Podać równanie określające złożoność algorytmów rekurencyjnych. \\ 
\colorbox{green}{Plusy:+} \colorbox{red}{Minusy: }

\item Podaj 3 sposoby rozwiązywania równań rekurencyjnych. \\ \colorbox{green}{Plusy:+} \colorbox{red}{Minusy: }

\item Czym są drzewa rekursji ? \\ 
\colorbox{green}{Plusy:+} \colorbox{red}{Minusy:}

\item Podaj twierdzenie o rekurencji uniwersalnej. \\ \colorbox{green}{Plusy:+} \colorbox{red}{Minusy: }

\item Rozwiąż metodą rekurencji uniwersalnej : \( T(n)=9T(\frac{n}{3})+n\), \( T(n)=T(\frac{2n}{3})+1\)  , \( T(n)=3T(\frac{n}{4})+n \log n\)  .  \\ \colorbox{green}{Plusy:} \colorbox{red}{Minusy:-}

\item Rozwiązać drzewem rekursji \( T(n)=2T(\frac{n}{2})+n^2 \) , \( T(n)=T(\frac{n}{3})+T(\frac{2n}{3})+n \)\\ \colorbox{green}{Plusy:+} \colorbox{red}{Minusy: }

\item Obliczyć złożoność MergeSort przy pomocy rekurencji uniwersalnej. \\ \colorbox{green}{Plusy:+} \colorbox{red}{Minusy: }

\item Obliczyć złożoność MergeSort przy pomocy drzewa rekursji.\\  \colorbox{green}{Plusy:+} \colorbox{red}{Minusy: }
\end{enumerate}

\subsection{Kopiec}
\begin{enumerate}
\item Czym jest kopiec ? Podaj przykład dwóch reprezentacji kopca. 
\\  \colorbox{green}{Plusy:+} \colorbox{red}{Minusy: }

\item Jak są numerowane elementy kopca w drzewie z tablicy ?\\  \colorbox{green}{Plusy:+} \colorbox{red}{Minusy: }

\item Czym jest własność kopca ?\\  \colorbox{green}{Plusy:+} \colorbox{red}{Minusy: }

\item Co robi procedura Heapify ? Na jakich danych działa ? \\  \colorbox{green}{Plusy:+} \colorbox{red}{Minusy: }

\item Napisz algorytm heapify.\\  
\colorbox{green}{Plusy:+} \colorbox{red}{Minusy: }

\item Podaj przykład działania heapify na danych z tablicy: 16, 4, 10, 14, 7, 9, 3, 2, 8, 1 \\  
\colorbox{green}{Plusy:+} \colorbox{red}{Minusy: }

\item Podaj złożoność obliczeniową heapify.\\  
\colorbox{green}{Plusy:+} \colorbox{red}{Minusy: }

\item Co robi procedura Build-Heap ? \\  
\colorbox{green}{Plusy:+} \colorbox{red}{Minusy: }

\item Działanie Build-Heap na danych 4, 1, 3, 2, 16, 9, 10, 14, 8, 7\\  
\colorbox{green}{Plusy:+} \colorbox{red}{Minusy: }

\item Napisz algorytm Build-Heap.\\  
\colorbox{green}{Plusy:+} \colorbox{red}{Minusy: }

\item Podaj złożoność obliczeniową Build-Heap z krótkim uzasadnieniem. \\  
\colorbox{green}{Plusy:+} \colorbox{red}{Minusy: }

\item Heap-Sort - ogólnie na czym polega ?\\  
\colorbox{green}{Plusy:+} \colorbox{red}{Minusy: }

\item Heap-Sort - napisz algorytm.\\  
\colorbox{green}{Plusy:} \colorbox{red}{Minusy:- }

\item Heap-Sort działanie na przykładowych danych: 16, 14, 10, 8, 7, 9, 3, 2, 4, 1\\  
\colorbox{green}{Plusy:+} \colorbox{red}{Minusy: }

\item Podaj złożoność obliczeniową Heap-Sort i krótkie uzasadnienie. \\  
\colorbox{green}{Plusy:+} \colorbox{red}{Minusy: }
\end{enumerate}
\subsection{Kolejka priorytetowa}
\begin{enumerate}
\item Jak działają / do czego służą kolejki priorytetowe ? \\  \colorbox{green}{Plusy:+} \colorbox{red}{Minusy: }

\item Przy pomocy czego / jak są implementowane kolejki priorytetowe ?\\  
\colorbox{green}{Plusy:+} \colorbox{red}{Minusy: }

\item Podaj 4 metody działające na kolejkach priorytetowych.\\  \colorbox{green}{Plusy:+} \colorbox{red}{Minusy: }

\item Napisz algorytmy i podaj złożoność algorytmów : Heap-Maximum, Heap-Extract-Max, Heap-Increase-Key, Max-Heap-Insert. \\  \colorbox{green}{Plusy:+} \colorbox{red}{Minusy: }

\end{enumerate}
\subsection{Sortowanie szybkie}
\begin{enumerate}
\item Quick-Sort - jak ogólnie działa ? Czy sortuje w miejscu ? \\ \colorbox{green}{Plusy:+} \colorbox{red}{Minusy: }

\item Procedura Partition - jak ogólnie działa ? \\ 
\colorbox{green}{Plusy:+} \colorbox{red}{Minusy: }

\item Napisz algorytm Quick-Sort  \\ 
\colorbox{green}{Plusy:+} \colorbox{red}{Minusy: }

\item Napisz algorytm Partition  \\ 
\colorbox{green}{Plusy:+} \colorbox{red}{Minusy: }

\item Podaj złożoność obliczeniową Partition  \\
\colorbox{green}{Plusy:+} \colorbox{red}{Minusy: }
 
\item Podaj złożoność obliczeniową Quick-Sort w przypadku optymistycznym oraz w przypadku pesymistycznym. Czym jest ona uwarunkowana ?  \\ 
\colorbox{green}{Plusy:+} \colorbox{red}{Minusy: }

\item Oblicz złożoność obliczeniową Quick-Sort w przypadku optymistycznym przy pomocy twierdzenia o rekurencji uniwersalnej / drzewa rekursji.  \\ 
\colorbox{green}{Plusy:+} \colorbox{red}{Minusy: }

\item Partition - działanie na danych : 2, 9, 7, 1, 3, 5, 6, 4
\\ 
\colorbox{green}{Plusy:+} \colorbox{red}{Minusy: }

\item Quick-Sort - działanie na danych: 2, 9, 7, 3, 5, 6, 4
\\ \colorbox{green}{Plusy:} \colorbox{red}{Minusy:- }

\end{enumerate}

\subsection{Drzewa decyzyjne}
\begin{enumerate}

\item Czym są drzewa decyzyjne ? Co reprezentują ? \\ \colorbox{green}{Plusy:+} \colorbox{red}{Minusy: }

\item Czym są węzły w drzewach decyzyjnych ? \\ 
\colorbox{green}{Plusy:+} \colorbox{red}{Minusy: }

\item Czym są liście w drzewach decyzyjnych ? Ile ich jest ? \\ \colorbox{green}{Plusy:+} \colorbox{red}{Minusy: }

\item Jakie jest dolne ograniczenie pesymistycznego czasu sortowania dla sortowań wykorzystujących porównania ? \\ 
\colorbox{green}{Plusy:+} \colorbox{red}{Minusy: }

\item Narysuj drzewo decyzyjne dla tablicy o 3 elementach. \\ 
\colorbox{green}{Plusy:+} \colorbox{red}{Minusy: }
 
\end{enumerate}

\subsection{Sortowania liniowe}
\begin{enumerate}
\item Co oznacza, że algorytm jest stabilny dla sortowania ? \\ \colorbox{green}{Plusy:+} \colorbox{red}{Minusy: }

\item Czym się różni sortowanie w czasie liniowym od innych sortowań ? \\ 
\colorbox{green}{Plusy:+} \colorbox{red}{Minusy: }

\item Jakie znasz sortowania liniowe ? \\ 
\colorbox{green}{Plusy:+} \colorbox{red}{Minusy: }

\item Counting-Sort - ogólna idea algorytmu, cechy algorytmu, stosowanie, złożoność \\ 
\colorbox{green}{Plusy:+} \colorbox{red}{Minusy: }

\item Counting-Sort przykład na danych : 3, 6, 4, 1, 3, 4, 1, 4\\ \colorbox{green}{Plusy:+} \colorbox{red}{Minusy: }

\item Sortowanie pozycyjne - ogólna idea, cechy algorytmu, stosowanie, złożoność \\ 
\colorbox{green}{Plusy:+} \colorbox{red}{Minusy: }

\item Sortowanie pozycyjny - przykład na danych: 329, 457, 657, 839, 436, 720, 355 
\\ \colorbox{green}{Plusy:+} \colorbox{red}{Minusy: }

\item Sortowanie kubełkowe - ogólna, idea, cechy algorytmu, stosowanie, złożoność \\ 
\colorbox{green}{Plusy:+} \colorbox{red}{Minusy: }

\item Sortowanie kubełkowe - przykład na danych: 0,78 , 0,17 , 0,39 , 0,26 , 0,72 , 0,94 , 0,21 , 0,12 , 0,23 , 0,68 
\\ \colorbox{green}{Plusy:+} \colorbox{red}{Minusy: }

\end{enumerate}
\subsection{Stos, lista, kolejka}


\subsection{Drzewa binarne}
\begin{enumerate}
\item Czym są drzewa binarne ? Jaką mają własność ? \\ \colorbox{green}{Plusy:+} \colorbox{red}{Minusy: }

\item Napisz kod procedury przeglądania drzewa binarnego w kolejności inorder ? Jaką ten algorytm ma złożoność ? \\ 
\colorbox{green}{Plusy:+} \colorbox{red}{Minusy: }

\item Napisz kod wyszukiwania w drzewie binarnym ? Jaką ma złożoność ? \\ 
\colorbox{green}{Plusy:+} \colorbox{red}{Minusy: }

\item Napisz kod dodawania do drzewa binarnego ? Czym jest węzeł x, a czym węzeł  y ? Jaką ma złożoność ? 
\\ \colorbox{green}{Plusy:+} \colorbox{red}{Minusy: }

\item Jakie 3 przypadki są możliwe w procedurze usuwania węzła z drzewa binarnego ? Zrób rysunki. \\ 
\colorbox{green}{Plusy:+} \colorbox{red}{Minusy: }

\item Napisz kod usuwania z drzewa binarnego ? Jaką ma złożoność ? \\ \colorbox{green}{Plusy:+} \colorbox{red}{Minusy: } 

\end{enumerate}

\subsection{Drzewa czerwono-czarne}
\begin{enumerate}
\item W jakim czasie są wykonywane operacje na drzewach binarnych, a w jakim na drzewach czerwono-czarnych, podaj złożoności oraz powód.\\ \colorbox{green}{Plusy:+} \colorbox{red}{Minusy: } 

\item Jakie 5 atrybutów ma węzeł drzewa czerwono-czarnego ? \\ \colorbox{green}{Plusy:+} \colorbox{red}{Minusy: } 

\item Jakie 5 własności ma drzewo czerwono-czarne ? \\ \colorbox{green}{Plusy:+} \colorbox{red}{Minusy: } 

\item Czym jest wysokość drzewa i jaka jest maksymalna dla drzewa czerwono-czarnego o n węzłach wewnętrznych ( wszystkie oprócz NIL ) ?
\\ \colorbox{green}{Plusy:+} \colorbox{red}{Minusy: } 

\item Na czym polega operacja rotacji ? Do czego jest wykorzystywana ? Przedstaw to za pomocą grafu. \\ 
\colorbox{green}{Plusy:+} \colorbox{red}{Minusy: }  

\item Czym jest wartownik ? Jak jest oznaczany ? Jakie ma atrybuty ? 
\\ \colorbox{green}{Plusy:+} \colorbox{red}{Minusy: } 

\item Kiedy nie można wykonać lewej/prawej rotacji ? \\ \colorbox{green}{Plusy:+} \colorbox{red}{Minusy: } 

\item Napisz kod Left-Rotate, w jakim czasie działa ta procedura ? 
\\ \colorbox{green}{Plusy:+} \colorbox{red}{Minusy: } 

\item Do czego służy RB-Insert(T,z) ? W jakim czasie działa ? Jaki kolor ma wstawiany węzeł. Napisz kod. 
\\ \colorbox{green}{Plusy:+} \colorbox{red}{Minusy: } 

\item Jakie możliwe naruszenia własności drzewa czerwono-czarnego mogą wystąpić po operacji wstawiania ?
\\ \colorbox{green}{Plusy:+} \colorbox{red}{Minusy: } 

\item Czym jest węzeł z w RB-Fixup ? Jaki jest warunek stopu w tej procedurze ? 
\\ \colorbox{green}{Plusy:+} \colorbox{red}{Minusy: } 

\item Jakie trzy przypadki są rozważane w RB-Fixup i jak są obsługiwane ? Zrób rysunki. 
\\ \colorbox{green}{Plusy:+} \colorbox{red}{Minusy: } 

\item Napisz procedurę RB-Fixup. 
\\ \colorbox{green}{Plusy:+} \colorbox{red}{Minusy: }  
\end{enumerate}


\subsection{Skip-listy}
\begin{enumerate}
\item Jakiego typu strukturą jest Skip-Lista ? Dla jakiej innej struktury jest alternatywą ? Jakie są jej zalety ?
\\ \colorbox{green}{Plusy:+} \colorbox{red}{Minusy: }  

\item Jaki jest czas działania procedur delete, insert, search dla skip-listy ?
\\ \colorbox{green}{Plusy:+} \colorbox{red}{Minusy: }  

\item Jakie atrybuty ma węzeł w skip-liście ?
\\ \colorbox{green}{Plusy:+} \colorbox{red}{Minusy: }  

\item Jak jest zbudowana Skip-Lista ?
\\ \colorbox{green}{Plusy:+} \colorbox{red}{Minusy: }  

\item Jak przebiega przeszukiwanie Skip-Listy ? Napisz kod + złożoność.
\\ \colorbox{green}{Plusy:+} \colorbox{red}{Minusy: }  

\item Jak przebiega wstawianie do Skip-Listy ?
\\ \colorbox{green}{Plusy:+} \colorbox{red}{Minusy: } 

\item Jak przebiega usuwanie ze Skip-Listy ?
\\ \colorbox{green}{Plusy:+} \colorbox{red}{Minusy: } 

\end{enumerate}
\subsection{Tablice haszujące}
\begin{enumerate}
\item Na czym polega adresowanie bezpośrednie ? Kiedy jest stosowane ? Jakie ma wady ? 
\\ \colorbox{green}{Plusy:+} \colorbox{red}{Minusy: } 

\item Zaimplementuj metody insert, search, delete dla adresowania bezpośredniego.
\\ \colorbox{green}{Plusy:+} \colorbox{red}{Minusy: } 

\item Czym są tablice z haszowaniem ? Kiedy są stosowane ? 
\\ \colorbox{green}{Plusy:+} \colorbox{red}{Minusy: } 

\item Jaka jest złożoność w przypadku optymistycznym i pesymistycznym dla działań na tablicach z haszowaniem ?
\\ \colorbox{green}{Plusy:+} \colorbox{red}{Minusy: } 

\item  Czym jest funkcja haszująca ? Czym jest kolizja ?
\\ \colorbox{green}{Plusy:+} \colorbox{red}{Minusy: } 

\item Jakie znasz sposoby rozwiązywania kolizji ? 
\\ \colorbox{green}{Plusy:+} \colorbox{red}{Minusy: } 

\item Jakie są cechy dobrej funkcji haszującej ? 
\\ \colorbox{green}{Plusy:+} \colorbox{red}{Minusy: } 

\item Zaimplementuj operacje słownikowe dla tablicy z haszowaniem z rozwiązywaniem kolizji
\\ \colorbox{green}{Plusy:+} \colorbox{red}{Minusy: } 

\item Czym jest haszowanie modularne ? Jakie powinno być m ? 
\\ \colorbox{green}{Plusy:+} \colorbox{red}{Minusy: } 

\item Czym jest haszowanie przez mnożenie ? Jakie powinno być m ? 
\\ \colorbox{green}{Plusy:+} \colorbox{red}{Minusy: } 

\item Czym jest haszowanie uniwersalne ?
\\ \colorbox{green}{Plusy:+} \colorbox{red}{Minusy: } 

\item Na czym polega adresowanie otwarte ? 
\\ \colorbox{green}{Plusy:+} \colorbox{red}{Minusy: } 

\item Jaką postać ma funkcja haszująca w adresowaniu otwartym ?
\\ \colorbox{green}{Plusy:+} \colorbox{red}{Minusy: } 

\item Napisz algorytm wstawiania w adresowaniu otwartym.
\\ \colorbox{green}{Plusy:+} \colorbox{red}{Minusy: } 

\item Jak przebiega wyszukiwanie w adresowaniu otwartym ?
\\ \colorbox{green}{Plusy:+} \colorbox{red}{Minusy: } 

\item Jak jest realizowanie usuwanie w adresowaniu otwartym ?
\\ \colorbox{green}{Plusy:+} \colorbox{red}{Minusy: } 

\item Czym jest adresowanie liniowe / kwadratowe / dwukrotne ? Podaj postać funkcji oraz wady/zalety. 
\\ \colorbox{green}{Plusy:} \colorbox{red}{Minusy:-} 

\item Jaka jest złożoność operacji słownikowych w adresowaniu otwartym ?
\\ \colorbox{green}{Plusy:+} \colorbox{red}{Minusy: } 



\end{enumerate}
 \subsection{Statystyki pozycyjne, drzewa przedziałowe}
\begin{enumerate}
\item Czym jest i-ta statystyka pozycyjna ? 
\\ \colorbox{green}{Plusy:+} \colorbox{red}{Minusy: } 

\item Ile przynajmniej porównań trzeba wykonać aby znaleźć minimum/maksimum zbioru n elementów ?
\\ \colorbox{green}{Plusy:+} \colorbox{red}{Minusy: } 

\item Na czym polega problem wyboru i-tej statystyki pozycyjnej ? Co jest wejściem, a co jest wyjściem ?
\\ \colorbox{green}{Plusy:+} \colorbox{red}{Minusy: } 

\item Jaki czas jest potrzebny do rozwiązania problemu wyboru i-tej statystyki pozycyjnej i jakim algorytmem jest to wykonywane ?
\\ \colorbox{green}{Plusy:+} \colorbox{red}{Minusy: } 

\item Jaka jest różnica pomiędzy Quick-Sort, a Randomized-Select ? 
\\ \colorbox{green}{Plusy:+} \colorbox{red}{Minusy: } 

\item Podaj algorytm Randomized-Select oraz Randomized-Partition.
\\ \colorbox{green}{Plusy:+} \colorbox{red}{Minusy: } 

\item Czym są wzbogacane struktury danych ? 
\\ \colorbox{green}{Plusy:+} \colorbox{red}{Minusy: } 

\item Jak wzbogacamy drzewa RB aby uzyskać drzewa statystyk pozycyjnych ? Jaki atrybut dodajemy i czym on jest ? Czym jest drzewo statystyk pozycyjnych / czym są dynamiczne statystyki pozycyjne ? 
\\ \colorbox{green}{Plusy:+} \colorbox{red}{Minusy: } 

\item W jakim czasie możliwe jest wyznaczenie dowolnej statystyki pozycyjnej w drzewie statystyk pozycyjnych ?
\\ \colorbox{green}{Plusy:+} \colorbox{red}{Minusy: } 

\item Czym jest ranga w drzewie RB ? 
\\ \colorbox{green}{Plusy:+} \colorbox{red}{Minusy: } 

\item Co robi procedura OS-Select ? W jakim czasie działa ? 
\\ \colorbox{green}{Plusy:+} \colorbox{red}{Minusy: } 

\item Napisz procedurę OS-Select, omów działanie.
\\ \colorbox{green}{Plusy:} \colorbox{red}{Minusy:- } 

\item Jak odbywa się wstawianie do drzew statystyk pozycyjnych ? Jakie modyfikacje są potrzebne w algorytmie ? Czy zmienia to czas działania wstawiania ? Zrób rysunki rotacji. 
\\ \colorbox{green}{Plusy:+} \colorbox{red}{Minusy: } 

\item Jak wzbogacamy drzewa RB aby uzyskać drzewa przedziałowe ? Jakie dwa atrybuty dodajemy ? 
\\ \colorbox{green}{Plusy:+} \colorbox{red}{Minusy: } 

\item Jakie przedziały zawierają węzły w drzewie przedziałowym ? Domknięte / niedomknięte ? Jak są uporządkowane ? 
\\ \colorbox{green}{Plusy:+} \colorbox{red}{Minusy: } 

\item Narysuj przykładowe drzewo przedziałowe.
\\ \colorbox{green}{Plusy:+} \colorbox{red}{Minusy: } 

\item Napisz algorytm Interval-Search znajdujący przedział zachodzący na i w drzewie przedziałowym. W jakim czasie działa ?
\\ \colorbox{green}{Plusy:+} \colorbox{red}{Minusy: } 

\end{enumerate}

\subsection{Grafy - podstawowe informacje}
\begin{enumerate}
\item Czym jest zbiór E, a czym jest zbiór V dla grafów ? 
\\ \colorbox{green}{Plusy:} \colorbox{red}{Minusy: } 

\item W zależności od czego wyrażana jest złożoność obliczeniowa algorytmów grafowych ?
\\ \colorbox{green}{Plusy:} \colorbox{red}{Minusy: } 

\item Jakie są dwie reprezentacje grafów ?
\\ \colorbox{green}{Plusy:} \colorbox{red}{Minusy: } 

\item Czym jest tablica adj ?
\\ \colorbox{green}{Plusy:} \colorbox{red}{Minusy: } 

\item Czym jest graf spójny ?
\\ \colorbox{green}{Plusy:} \colorbox{red}{Minusy: } 

\item Czym jest graf pełny ? Czym różni się od spójnego ?
\\ \colorbox{green}{Plusy:} \colorbox{red}{Minusy: } 

\end{enumerate}

\subsection{BFS, DFS}
\begin{enumerate}

\item Jakie atrybuty ma węzeł w przeszukiwaniu wszerz (BFS) ?
 \\ \colorbox{green}{Plusy:} \colorbox{red}{Minusy: } 
 
\item Jaka pomocnicza struktura danych jest wykorzystywana w BFS i jak jest obsługiwana ?
\\ \colorbox{green}{Plusy:} \colorbox{red}{Minusy: } 

\item Jak zmieniają się kolory wierzchołków w BFS ?
\\ \colorbox{green}{Plusy:} \colorbox{red}{Minusy: } 

\item Jak działa BFS ? 
\\ \colorbox{green}{Plusy:} \colorbox{red}{Minusy: } 

\item Jaki jest czas działania BFS ?
\\ \colorbox{green}{Plusy:} \colorbox{red}{Minusy: } 

\item Zrób BFS dla grafu 
\begin{figure}[H]
\includegraphics[scale=0.3]{egz_1.jpg}
\end{figure}
\colorbox{green}{Plusy:} \colorbox{red}{Minusy: } 

\item Jakie atrybuty ma węzeł w przeszukiwaniu w głąb (DFS) ?
 \\ \colorbox{green}{Plusy:} \colorbox{red}{Minusy: }

\item Jak zmieniają się kolory wierzchołków w DFS ?
\\ \colorbox{green}{Plusy:} \colorbox{red}{Minusy: } 

\item Jak działa DFS ? 
\\ \colorbox{green}{Plusy:} \colorbox{red}{Minusy: } 

\item Jaki jest czas działania DFS ?
\\ \colorbox{green}{Plusy:} \colorbox{red}{Minusy: } 

\item Zrób DFS dla grafu 
\begin{figure}[H]
\includegraphics[scale=0.4]{egz_2.jpg}
\end{figure}
\colorbox{green}{Plusy:} \colorbox{red}{Minusy: } 

\item Krawędź drzewowa, powrotna, w przód, poprzeczna (str. 621) \\
\colorbox{green}{Plusy:} \colorbox{red}{Minusy: } 

\end{enumerate}
\subsection{Sortowanie topologiczne}
\begin{enumerate}

\item Czym jest DAG ? \\
\colorbox{green}{Plusy:} \colorbox{red}{Minusy: } 

\item Na jakich grafach działa sortowanie topologiczne ? \\
\colorbox{green}{Plusy:} \colorbox{red}{Minusy: } 

\item Do czego służy sortowanie topologiczne ?  \\
\colorbox{green}{Plusy:} \colorbox{red}{Minusy: } 

\item Z jakiego informacji z jakiego algorytmu korzysta sortowanie topologiczne ?  \\
\colorbox{green}{Plusy:} \colorbox{red}{Minusy: } 

\item Jaki jest czas sortowania topologicznego ?  \\
\colorbox{green}{Plusy:} \colorbox{red}{Minusy: } 

\item Podaj algorytm sortowania topologicznego.  \\
\colorbox{green}{Plusy:} \colorbox{red}{Minusy: } 
 
\end{enumerate}


\subsection{MTS}

\begin{enumerate}


\item Jak działa algorytm Kruskala ?  \\
\colorbox{green}{Plusy:} \colorbox{red}{Minusy: } 
 
\item Czym są zbiory ( set ) w algorytmie Kruskala ? Czemu zapobiegają ?  \\
\colorbox{green}{Plusy:} \colorbox{red}{Minusy: } 
 
\item Działanie algorytmu Kruskala na grafie: 
\begin{figure}[H]
\includegraphics[scale=0.7]{egz_3.jpg}
\end{figure}
\colorbox{green}{Plusy:} \colorbox{red}{Minusy: } 
  
\item Złożoność algorytmu Kruskala \\
\colorbox{green}{Plusy:} \colorbox{red}{Minusy: } 

\item Jak działa algorytm Prima ?  \\
\colorbox{green}{Plusy:} \colorbox{red}{Minusy: } 
 
\item Jaką dodatkową strukturę wykorzystuje algorytm Prima ?  \\
\colorbox{green}{Plusy:} \colorbox{red}{Minusy: } 
 
\item Działanie algorytmu Prima na grafie: 
\begin{figure}[H]
\includegraphics[scale=0.7]{egz_4.jpg}
\end{figure}
\colorbox{green}{Plusy:} \colorbox{red}{Minusy: } 
  
\item Złożoność algorytmu Prima w dwóch przypadkach \\
\colorbox{green}{Plusy:} \colorbox{red}{Minusy: }

\end{enumerate}

 
\subsection{Najkrótsze ścieżki w grafie}
\begin{enumerate}
\item Co oblicza algorytm Belmana-Forda ? \\
\colorbox{green}{Plusy:} \colorbox{red}{Minusy: }

\item Czy jest jakieś ograniczenie na wartości wag krawędzi w algorytmie B-F ? \\
\colorbox{green}{Plusy:} \colorbox{red}{Minusy: }

\item Co robi procedura Relax(u,v,w) ? \\
\colorbox{green}{Plusy:} \colorbox{red}{Minusy: }

\item Jakie dwa atrybuty ma węzeł w algorytmie Belmana-Forda ? \\
\colorbox{green}{Plusy:} \colorbox{red}{Minusy: }

\item Jak działa algorytm Belmana-Forda ? \\
\colorbox{green}{Plusy:} \colorbox{red}{Minusy: }

\item Jaki jest czas działania algorytmu Belmana-Forda ? \\
\colorbox{green}{Plusy:} \colorbox{red}{Minusy: }

\item Działanie algorytmu Belmana-Forda na grafie: 
\begin{figure}[H]
\includegraphics[scale=0.7]{egz_5.jpg}
\end{figure}
\colorbox{green}{Plusy:} \colorbox{red}{Minusy: } 

\item Co oblicza algorytm Dijkstry ? \\
\colorbox{green}{Plusy:} \colorbox{red}{Minusy: }

\item Czy jest jakieś ograniczenie na wartości wag krawędzi w algorytmie Dijkstry ? \\
\colorbox{green}{Plusy:} \colorbox{red}{Minusy: }

\item Jakie dwa atrybuty ma węzeł w algorytmie Dijkstry ? \\
\colorbox{green}{Plusy:} \colorbox{red}{Minusy: }

\item Jak działa algorytm Dijkstry ? \\
\colorbox{green}{Plusy:} \colorbox{red}{Minusy: }

\item Jaki jest czas działania algorytmu Dijkstry w dwóch przypadkach ? \\
\colorbox{green}{Plusy:} \colorbox{red}{Minusy: }

\item Działanie algorytmu Dijkstry na grafie: 
\begin{figure}[H]
\includegraphics[scale=0.7]{egz_6.jpg}
\end{figure}
\colorbox{green}{Plusy:} \colorbox{red}{Minusy: } 

\item Co wyznacza algorytm Floyda - Warshalla ? \\
\colorbox{green}{Plusy:} \colorbox{red}{Minusy: }

\item Jaką ma złożoność algorytm Floyda-Warshalla ? \\
\colorbox{green}{Plusy:} \colorbox{red}{Minusy: }

\item Działanie na danych -> opracowanie \\
\colorbox{green}{Plusy:} \colorbox{red}{Minusy: }


\end{enumerate}

\subsection{Koszt zamortyzowany}
\begin{enumerate}
\item Trzy metody obliczania kosztu operacji przy pomocy kosztu  \\
\colorbox{green}{Plusy:} \colorbox{red}{Minusy: } zamortyzowanego
\end{enumerate}
\subsection{Wyszukiwanie wzorca}
\begin{enumerate}
\item Algorytm naiwny \\
\colorbox{green}{Plusy:} \colorbox{red}{Minusy: }

\item Algorytm Rabina-Karpa \\
\colorbox{green}{Plusy:} \colorbox{red}{Minusy: }

\item Wyszukiwanie wzorca przy pomocy automatów skończonych \\
\colorbox{green}{Plusy:} \colorbox{red}{Minusy: }
\end{enumerate}
\end{document}
 