\documentclass[a4paper,11pt]{article}
\usepackage{algorithm}
\usepackage{algpseudocode}
\usepackage{amssymb}
\usepackage{amsthm}
\usepackage{amsmath}
\usepackage[english, polish]{babel}
\usepackage[utf8]{inputenc}   % lub utf8
\usepackage[T1]{fontenc}
\usepackage{graphicx}
\usepackage{anysize}
\usepackage{enumerate}
\usepackage{times}
\usepackage{xcolor}
\usepackage{titlesec}
\usepackage{float}
\usepackage[justification=centering]{caption}
\titlelabel{\thetitle.\quad}
\usepackage{titlesec}
\usepackage{titleps,kantlipsum}
\usepackage{tikz}
\usepackage{color}
\usepackage{listings}
\usepackage{caption}
\lstloadlanguages{Matlab,C++}
\usepackage{hyperref}
\usepackage{framed}
\usepackage{siunitx}
\usepackage{mathrsfs}
\usepackage{cancel}

\usetikzlibrary{calc,through,backgrounds,positioning,fit}
\usetikzlibrary{shapes,arrows,shadows,patterns}

\tikzstyle{place}=[shape=circle, draw, minimum height=10mm]
\tikzstyle{place_1}=[shape=circle, draw, minimum height=5mm]
\tikzstyle{trig}=[shape=circle, draw, dashed, minimum height=10mm]
\tikzstyle{trans}=[shape=rectangle, draw, minimum height=15mm, minimum width=16mm]

\newdimen\LineSpace
\tikzset{
    line space/.code={\LineSpace=#1},
    line space=3pt
}

\pgfdeclarepatternformonly[\LineSpace]{my north east lines}{\pgfqpoint{-1pt}{-1pt}}{\pgfqpoint{\LineSpace}{\LineSpace}}{\pgfqpoint{\LineSpace}{\LineSpace}}%
{
    \pgfsetlinewidth{0.4pt}
    \pgfpathmoveto{\pgfqpoint{0pt}{0pt}}
    \pgfpathlineto{\pgfqpoint{\LineSpace + 0.1pt}{\LineSpace + 0.1pt}}
    \pgfusepath{stroke}
}

\newpagestyle{mypage}
{
  \headrule
  
  \sethead
  { \MakeUppercase{\thesection\quad \sectiontitle} } 
  {}
  {\thesubsection\quad \subsectiontitle}
  
  \setfoot
  {}
  {\thepage}
  {}
}

\newpagestyle{mypage_1}
{
	\headrule
	
	\sethead
	{  }
	{\textit{Metody i algorytmy optymalizacji - Zbiór zadań 2017/18} }
	{}
	
	\setfoot
	{}
	{}
	{}
}

\pagestyle{mypage_1}
%\marginsize{left}{right}{top}{bottom}
\marginsize{3cm}{3cm}{3cm}{3cm}
\sloppy
 
\begin{document}

\tableofcontents

\newpage
\section{Lista opracowanych tematów}
\begin{enumerate}

\item Wypukłość, wklęsłość, twierdzenie o punktach optymalnych


\item Poszukiwanie ekstremum na kierunku


\item Metoda Lagrange'a + regularność punktów 


\item Kierunki sprzężone - ogólnie


\item Kierunki sprzężone - Metoda gradientów sprzężonych


\item Metoda Newtona-Rapsona


\item Wewnętrzna funkcja kary

\item Dualność

\item Kierunki poprawy

\item Kuhn - Tucker

\item Programowanie liniowe

\item \textbf{Do zrobienia}:
\item Kierunki sprzężone - Metoda Powella



\end{enumerate}


\newpage
\section{Wypukłość, wklęsłość, warunki optymalności}

\subsection{Pytania własne - teoria do opracowania}
\begin{enumerate}
\item Definicja wypukłości 
\item Definicja ścisłej wypukłości 
\item Definicja wklęsłości
\item Funkcja wklęsła, a wypukła ( zależność )
\item Wypukłość - liniowa kombinacja
\item Wypukłość - funkcja złożona 
\item Czym jest hesjan ?
\item Dodatnia i ściśle dodatnia określoność macierzy - jak stwierdzić ?
\item Ujemna i ściśle ujemna określoność macierzy - jak stwierdzić ?
\item Określoność hesjanu, a wypukłość.
\item Twierdzenie Fermata, część I = WK optymalności rozwiązania
\item WD istnienia minimum
\item Kiedy punkt jest minimum, maksimum, punktem siodłowym ?
\item Kiedy minimum lokalne jest globalne ? Kiedy ekstremum jest globalne ?
\item Kiedy zadanie optymalizacji jest wypukłe ?
\end{enumerate}

\subsection{Ćwiczenia}

\begin{framed}
\textbf{Zadanie 1 - Notatki z ćwiczeń od Gregi} \\
Udowodnić, że jeśli \( F(x) \) jest liniowa, a \( G(x) \) jest wypukła to \( H(x) = G(F(x)) \) jest wypukła.
\end{framed}

\begin{framed}
\textbf{Zadanie 2 - Notatki z ćwiczeń od Gregi} \\
Dowieść, że jeśli \( F(x) \) i \( G(x) \) są wypukłe, to stwierdzenie, że \( H(x)=G(F(x)) \) jest wklęsła jest nie zawsze prawdziwe. 
\end{framed}

\begin{framed}
\textbf{Zadanie 3 - Notatki z ćwiczeń od Gregi} \\
Udowodnić, że jeśli \( F_i(x) : R^N \rightarrow R \) jest wypukła oraz \( \forall i \; c_i \geq 0 \), to \( F(x) = \sum_{i=1}^n c_i F_i(x) \) jest także funkcją wypukłą. 
\end{framed}

\begin{framed}
\textbf{Zadanie 4 - Notatki z ćwiczeń od Gregi} \\
Udowodnić, że funkcja \( e^{x_1^2+x_2^2} + e^{x_1+x_2} \) jest wypukła. 
\end{framed}

\begin{framed}
\textbf{Zadanie 5 - Notatki z ćwiczeń od Gregi} \\
Wyznaczyć gradient i hesjan funkcji \( F(x)=x_1^2+2x_2^2+x_1x_2 \). Przyjąć punkt początkowy \( x_0 = [ x_{01} \; x_{02} ] \). 
\end{framed}

\begin{framed}
\textbf{Zadanie 6 - Notatki z ćwiczeń od Gregi} \\
Czy funkcja \( F(x)=12x_1^{\frac{1}{3}} \cdot x_2^{\frac{1}{2}} \) osiąga maksimum dla pewnego \( x_1 \; , x_2 \), gdzie \( x_1, \; x_2 \geq 0 \).
\end{framed}

\begin{framed}
\textbf{Zadanie 7 - Notatki z ćwiczeń od Gregi} \\
Wykazać ścisłą wklęsłość funkcji \( F(x,y) = x+y(-e^x)(-e^{x+y}) \).
\end{framed}

\subsection{Egzamin}

\begin{framed}
\textbf{\colorbox{orange}{Zadanie 8 - Egzamin +}} \\
Znaleźć obszar wypukłości funkcji \( F(x)=\frac{1}{x_1}+2x_2 \).
\end{framed}

\begin{framed}
\textbf{\colorbox{orange}{Zadanie 9 - Egzamin +}} \\
Czy funkcja \( F(x_1,x_2)=\frac{1}{x_1}+\frac{1}{x_2} \) jest \\ a) wypukła, b) wklęsła c) wypukła na podzbiorach ( jakich ? ).
\end{framed}

\begin{framed}
\textbf{\colorbox{orange}{Zadanie 10 - Egzamin +}} \\
Czy rozwiązanie zadanie \( \text{min} \sum_{i=1}^N \frac{1}{x_i} \) na zbiorze \( X_0 = \{ x: Ax \leq b , \; b \geq 0 , \; x_i \geq 0 \} \) jest globalne ? 
\end{framed}

\begin{framed}
\textbf{\colorbox{orange}{Zadanie 11 - Egzamin +}} \\
Czy zadanie \( \text{min} \{ (x-5)^2+(y-1)^2+xy\} \), przy ograniczeniach \( x+y \leq 0 \), \( x \geq 0 \) , \( y \geq 0 \) jest wypukłym zadaniem optymalizacji w \( R^2 \)? Odpowiedź uzasadnij.
\end{framed}

\begin{framed}
\textbf{\colorbox{orange}{Zadanie 12 - Egzamin + }} \\
Dla zadania \( \text{min} \{ 2x_1^3+-3x_1^2-6x_1x_2(x_1-x_2-2)\} \), gdzie \( (x_1,x_2) \in R^2 \) : \\
a) Znaleźć punkty spełniające warunki konieczne optymalności. \\
b) Na podstawie warunków wystarczających, określić charakter tych punktów.
\end{framed}





\newpage
\section{Optymalizacja na kierunku}

\subsection{Pytania własne - teoria do opracowania}
\begin{enumerate}

\item Jak wyznaczamy minimum na określonym kierunku i przy danym punkcie początkowym ? Podaj algorytm. Czym jest krok ? 


\end{enumerate}

\subsection{Egzamin}

\begin{framed}
\textbf{\colorbox{orange}{Zadanie 1 - Egzamin +}} \\
Obliczyć: \\
a) gradient \\
b) hesjan \\
c) określić wypukłość \\
funkcji \( F(x_1,x_2) = x_1^2+2x_2^2+x_1x_2 \). \\
d) Jakiej długości należy wykonać krok z punktu \( (1,0) \) na kierunku \( d=(1,1) \) aby osiągnąć ekstremum? 
\end{framed}

\begin{framed}
\textbf{\colorbox{orange}{Zadanie 2 - Egzamin +}} \\
Dla zadania: \\
\( F(x)=x_1^2+x_1x_2+(x_1-x_2)^4 \) \\
wykonać dokładne poszukiwanie minimum na kierunku \( d = \begin{bmatrix}
1 \\ -3 
\end{bmatrix} \) 
z punktu \( x_0 = (0,0)^T \).
\end{framed}

\begin{framed}
\textbf{\colorbox{orange}{Zadanie 3 - Egzamin +}} \\
\begin{align*}
&\text{min} F(x) = x_1^2 -2x_1x_2+x_2^2+x_2^3+cx_2 \\
&x \in R^2, \; c - \text{parametr} \\
\end{align*}
Dla \( c = -3 \) znaleźć wszystkie wektory d, które w punkcie \( x^0 = (0,1)^T n \) są ani kierunkami wzrostu ani kierunkami spadku.
\end{framed}

\begin{framed}
\textbf{\colorbox{orange}{Zadanie 4 - Egzamin +}} \\
Wyznacz minimum na kierunku  antygradientu z punktu \( x_0 \)  dla formy kwadratowej
\begin{align*}
&F(x)=\frac{1}{2}x^TAx \\
& A = \begin{bmatrix}
1 & 1 \\
1 & 2
\end{bmatrix}
\quad 
x_0 = \begin{bmatrix}
10 \\
-5
\end{bmatrix}
\end{align*}
\end{framed}

\begin{framed}
\textbf{\colorbox{orange}{Zadanie 5 - Egzamin +}} \\
Dla zadania
\begin{align*}
\text{min} \; x_1^2 + 2x_2^2 + 4x_1 + 4x_2 
\end{align*}
a) Znajdź punkty w \( R^2 \) spełniające WK optymalności \\
b) Metodą analityczną (dokładną) dokonaj jednokrotnej optymalizacji kierunkowej z punktu startowego \( x = [ 0 \; 0 ]^T \).
\end{framed}








\newpage
\section{Metoda Lagrange'a + regularność}

\subsection{Pytania własne - teoria do opracowania}
\begin{enumerate}

\item Do jakiego problemu jest używana metoda Lagrange'a ?
\item Jak przekształcamy zadanie z ograniczeniami na zadanie bez ograniczeń ? Jaki jest WK ?
\item Jak stwierdzić na podstawie tej metody czy obliczony punkt to minimum czy maksimum ?
\item Czym jest punkt regularny w zadaniu z ograniczeniami równościowymi ?


\end{enumerate}

\subsection{Ćwiczenia}

\begin{framed}
\textbf{Zadanie 1 - Notatki z ćwiczeń od Gregi } \\
Dana jest hiperpłaszczyzna \( x+y-z=3 \) . Na tej płaszczyźnie znaleźć punkt najbliżej punktu środka układu współrzędnych.
\end{framed}

\subsection{Egzamin}

\begin{framed}
\textbf{\colorbox{orange}{Zadanie 2 - Egzamin +}} \\
Wykorzystując metodę mnożników Lagrange'a znaleźć punkty ekstremalne funkcji \\ \( F(x) = x_1x_2 \) na zbiorze dopuszczalnym \( x_1^2+x_2^2  = 1\).
\end{framed}

\begin{framed}
\textbf{\colorbox{orange}{Zadanie 3 - Egzamin ( podpunkt b) ? ) }} \\
Zadanie
\begin{align*}
&\text{min} \; x_1^2 + x_2^2 + \frac{1}{4}x_3^2 \\
&-x_1+x_3-1=0
\end{align*}
a) Rozwiąż metodą mnożników Lagrangea \\
\colorbox{yellow}{b) Sprawdź warunki regularności w punkcie rozwiązania.}
\end{framed}









\newpage
\section{Kierunki sprzężone - ogólne }

\subsection{Pytania własne - teoria do opracowania}

\begin{enumerate}
\item Jaki problem jest rozpatrywany w metodach kierunków sprzężonych ?
\item Jaki jest WK dla zastosowania tych metod ?
\item Jakie znasz metody kierunków sprzężonych ?
\item Kiedy dwa kierunki są sprzężone ?
\item Kiedy dwa kierunki są sprzężone względem ściśle dodatnio określonej macierzy A ?
\item Czy ściśle dodatnio określona macierz A ma zawsze kierunki sprzężone ?
\item Ile potrzeba kroków do znalezienia minimum w metodach kierunków sprzężonych ? Podaj twierdzenie.
\end{enumerate}


\subsection{Egzamin}

\begin{framed}
\textbf{\colorbox{orange}{Zadanie 1 - Egzamin +}} \\
Zbadać sprzężoność podanych kierunków d w stosunku do podanej macierzy A:
\begin{align*}
A = 
\begin{bmatrix}
2 & 1 \\
1 & 3 
\end{bmatrix}
\end{align*}
a) \( d^1 = \begin{bmatrix}
1 \\ 0 
\end{bmatrix} \quad d^2 = \begin{bmatrix}
1 \\ -2
\end{bmatrix} \) \\
b)  \( d^1 = \begin{bmatrix}
1 \\ 0 
\end{bmatrix} \quad d^2 = \begin{bmatrix}
3 \\ -1
\end{bmatrix} \)

\end{framed}

\begin{framed}
\textbf{\colorbox{orange}{Zadanie 2 - Egzamin +}} \\
Udowodnić, że jeśli \( u, v \) są wektorami własnymi macierzy \( A \), to \( <u,Av>=0 \), gdzie \( A > 0 \). 
\end{framed}

\begin{framed}
\textbf{\colorbox{orange}{Zadanie 3 - Egzamin +}} \\
Dla jakiego \( a \) kierunki \( d^1 = \begin{bmatrix}
a \\ 0 
\end{bmatrix} \quad d^2=\begin{bmatrix}
1 \\ -2 
\end{bmatrix} \) są sprzężone względem \( A = \begin{bmatrix}
2 & 1 \\ 1 & 3 
\end{bmatrix}
\)
\end{framed}


\begin{framed}
\textbf{\colorbox{orange}{Zadanie 4 - Egzamin +}} \\
Metoda kierunków  sprzężonych zastosowana dla znalezienia minimum  funkcji: 
\begin{align*}
&F(x)=\frac{1}{2}x^TAx+b^Tx+c \\
&A > 0, \; x \in R^3
\end{align*}
zakończy poszukiwania w co najwyżej n krokach. Podać n  i uzasadnić. 
\end{framed}










\newpage
\section{Kierunki sprzężone - Metoda Powella}










\newpage
\section{Kierunki sprzężone - Metoda gradientów sprzężonych}

\subsection{Pytania własne - teoria do opracowania}

\begin{enumerate}
\item Podaj algorytm metody gradientów sprzężonych
\item Ile kroków potrzebnych jest do wykonania co najwyżej aby metoda ta znalazła minimum przy ściśle dodatnio określonej macierzy A ?
\end{enumerate}

\subsection{Ćwiczenia}

\begin{framed}
\textbf{Zadanie 1 - Notatki z ćwiczeń od Gregi} \\
Metodą gradientów sprzężonych znaleźć minimum funkcji \( F(x_1,x_2)=2x_1^2+x_2^2-x_1x_2 \).
\end{framed}

\subsection{Egzamin}
\begin{framed}
\textbf{\colorbox{orange}{Zadanie 2 - Egzamin}} \\
a) Uzupełnić krok algorytmu gradientu sprzężonego \\
b) Jeśli F jest formą kwadratową ściśle dodatnio określoną to w ilu krokach algorytm osiąga minimum  
\end{framed}










\newpage
\section{Metoda Newtona - Rapsona}

\subsection{Pytania własne - teoria do opracowania}

\begin{enumerate}

\item Podaj WKiD zbieżności metody Newtona 
\item Dla formy kwadratowej, w ilu krokach jest osiągane minimum przez algorytm i jaki jest konieczny do tego warunek ?
\item Podaj algorytm metody Newtona-Rapsona
\item Jaki jest związek pomiędzy WK optymalności ( Fermat ), a działaniem metody dla macierzy ściśle dodatnio określonej ?
\end{enumerate}


\subsection{Egzamin}

\begin{framed}
\textbf{\colorbox{orange}{Zadanie 1 - Egzamin +}} \\
Wykazać, że algorytm Newtona-Rapsona zastosowany do funkcji \\ \( F(x) = (x_1-2)^2 +4(x_2-2)^2 \) pozwoli osiągnąć jej minimum w jednym kroku.
\end{framed}

\begin{framed}
\textbf{\colorbox{orange}{Zadanie 2 - Egzamin +}} \\
Minimum formy kwadratowej \( F(x) = x^TAx \) , gdzie \( x \in \mathbb{R}^N \) będzie osiągnięte w jednym kroku przez algorytm Newtona-Rapsona pod warunkiem, że...
\end{framed}

\begin{framed}
\textbf{\colorbox{orange}{Zadanie 3 - Egzamin +}} \\
\( min F(x) = x_1^2-2x_1x_2+x_2^2+x_2^3+cx_2 \) \\
a)  dla \( c = 3 \) znaleźć punkty stacjonarne funkcji celu i określić charakter \\
b) dla \( c = -1 \) i \( x0 = [2 \; 1]^T \) wyznaczyć kierunek poszukiwań metody Newtona 
\end{framed}














\newpage
\section{Wewnętrzna funkcja kary}

\subsection{Pytania własne - teoria do opracowania}
\begin{enumerate}
\item Podaj postać problemu w zadaniu z wewnętrzną funkcją kary. ( UWAGA: Jakie są tam ograniczenia ? )
\item Jak zamieniany jest problem wyjściowy na problem bez ograniczeń ? Jaką postać ma nowa funkcja celu ?
\item Podaj postać dwóch najczęściej wykorzystywanych funkcji kary.
\item Jak wyznaczamy rozwiązanie ? Co należy obliczyć ?
\end{enumerate}




\subsection{Egzamin}

\begin{framed}
\textbf{\colorbox{orange}{Zadanie 1 - Egzamin +}} \\
Rozwiązać zadanie metodą wewnętrznej funkcji kary
\begin{align*}
& \text{min} \{ (x_1+1)^3 + x_2 \} \\
& g_1(x) = x_1-1 \geq 0 \\
& g_2(x) = x_2 \geq 0 
\end{align*}
Podać postać nowej funkcji celu.
\end{framed}

\begin{framed}
\textbf{\colorbox{orange}{Zadanie 2 - Egzamin +}} \\
Rozwiązać zadanie metodą wewnętrznej funkcji kary
\begin{align*}
& \text{min} \{ (x_1+1)^3 + x_2 \} \\
& g_1(x) = x_1 \geq 1 \\
& g_2(x) = x_2 \geq 10 
\end{align*}
\end{framed}

\begin{framed}
\textbf{\colorbox{orange}{Zadanie 3 - Egzamin + }} \\
Rozwiązać zadanie przy użyciu logarytmicznej wewnętrznej funkcji kary
\begin{align*}
& \text{min} \{ \frac{1}{2}x_1^2 + \frac{1}{2}x_2^2 \} \\
&  x_1 + x_2 \geq 0 \\
\end{align*}
a) wykorzystać logarytmiczną wewnętrzną funkcję kary \\ 
b) czy rozwiązanie jest globalne + uzasadnienie
\end{framed}






\newpage
\section{Dualność}

\subsection{Pytania własne - teoria do opracowania}

\begin{enumerate}

\item Jaki problem jest rozważany w problemie dualności ? Jaką postać mają ograniczenia ?
\item Czym różni się ten problem od problemu KKT ?
\item Jaką postać ma funkcja Lagrange'a dla tego problemu ?
\item Czym jest punkt siodłowy ? Dlaczego jest ważny dla problemu wyjściowego ? Jak go wyznaczyć ?
\item Jak wyznaczyć funkcję dualną ?
\item Jak wyznaczyć punkt siodłowy ?
\item Czym jest odstęp dualności ?
\item Podaj treść silnego twierdzenia o dualności.


\end{enumerate}


\subsection{Egzamin}

\begin{framed}
\textbf{\colorbox{orange}{Zadanie 1 - Egzamin + }} \\
Co to jest odstęp  dualności? Kiedy odstęp dualności   jest równy zero? 
\end{framed}

\begin{framed}
\textbf{\colorbox{orange}{Zadanie 2 - Egzamin + }} \\
Znaleźć punkt położony najbliżej początku układu współrzędnych i spełniający:
\begin{align*}
& x_1+x_2 \geq 1 \\
& x_1 \geq 0 \\
& x_2 \geq 0
\end{align*}
Zadanie rozwiązać metodą dekompozycji dualnej. 
\end{framed}

\begin{framed}
\textbf{\colorbox{orange}{Zadanie 3 - Egzamin +}} \\
Zaleźć funkcję dualną dla zadania:
\begin{align*}
&\text{min} \; x_1^2+x_2^2 \\
&x_1+x_2 \geq 0 
\end{align*}
Następnie zadanie rozwiązać metodą dekompozycji dualnej. 
\end{framed}

\begin{framed}
\textbf{\colorbox{orange}{Zadanie 4 - Egzamin + }} \\
Obliczyć funkcję dualną i składową \( \lambda \) punktu siodłowego dla problemu:
\begin{align*}
&\text{min} \{ 2x_1^2+x_2^2-4x_1-6x_2 \} \\
&-x_1+2x_1 \leq 4
\end{align*}
\end{framed}





\newpage
\section{Kierunki poprawy}

\subsection{Pytania własne - teoria do opracowania}

\begin{enumerate}

\item Czym są kierunki dopuszczalne w danym punkcie ?
\item Czym są kierunki spadku w danym punkcie ?
\item Czym są kierunki poprawy w danym punkcie ?

\end{enumerate}

\subsection{Egzamin}

\begin{framed}
\textbf{\colorbox{orange}{Zadanie 1 - Egzamin +}} \\
Wyznaczyć analitycznie kierunek dopuszczalny poprawy. Optymalizować długość kroku na tym kierunku. 
\begin{align*}
& d = (-1,-1)^T \\
& F(x)=x_1+x_2 \\
& h(x) = x_1^2+x_2^2-1 \\
& x_0 = ( \frac{\sqrt{2}}{2} ,\; \frac{\sqrt{2}}{2} )
\end{align*}
a) Czy d jest kierunkiem poprawy ?
b) Czy d jest kierunkiem dopuszczalnym ?
\end{framed}






\newpage
\section{Kuhn - Tucker}

\subsection{Pytania własne - teoria do opracowania}

\begin{enumerate}

\item Jaki jest problem w zadaniu KT ?
\item Podaj warunki konieczne KT
\item Czym są ograniczenia aktywne ? Jak mają się do lambd ?
\item Podaj WKiD KT 
\item Co się dodaje do WK KT w przypadku ograniczeń równościowych ?
\item Jak wyglądają WK KT w przypadku maksymalizacji ?
\item Jak wykazać regularność punktów wyznaczonych przez warunki konieczne KT ?

\end{enumerate}


\subsection{Egzamin}

\begin{framed}
\textbf{\colorbox{orange}{Zadanie 1 - Egzamin + }} \\
Dla zadania:
\begin{align*}
& \text{min} -x^2-x^3 \\
& x^2 \leq 1 
\end{align*}
Wyznaczyć pary \( (x,\lambda ) \) spełniające warunki konieczne Kuhna-Tuckera.
\end{framed}

\begin{framed}
\textbf{\colorbox{orange}{Zadanie 2 - Egzamin + }} \\
Czy w punkcie \( x = 6 \) spełnione są warunki konieczne Khuna-Tuckera ?
\begin{align*}
& \text{max} (x-4)^2 \\
& 1 \leq x \leq 6 
\end{align*}
\end{framed}

\begin{framed}
\textbf{\colorbox{orange}{Zadanie 3 - Egzamin +}} \\
Sformułować warunki Kuhna Tuckera dla zadania: 
\begin{align*}
& \text{max} F(x) \\
& h_i(x) \leq 0 
\end{align*}
\end{framed}

\begin{framed}
\textbf{\colorbox{orange}{Zadanie 4 - Egzamin + }} \\
\begin{align*}
& \text{min} ( e^{-x_1}-x_2^2 ) \\
& x_1^2+x_2^2 \leq 1 \\
& x_1^2-1 \leq x_2 \\
& x \in R^2 
\end{align*}
a) Czy jest to wypukły problem optymalizacji ? \\
b) Czy punkt \( (1,0) \) spełnia WK Kuhna-Tuckera ? \\
c) Jeśli tak, to czy jest to minimum czy maksimum ?
\end{framed}

\begin{framed}
\textbf{\colorbox{orange}{Zadanie 5 - Egzamin +}} \\
\begin{align*}
& \text{min} ( e^{-x_1}+2x_2^2 ) \\
& x_1^2+x_2^2 \leq 1 \\
& x_1^2-1 \leq x_2 \\
& x \in R^2 
\end{align*}
Czy w punkcie \( (2,1) \) są spełniona warunki KKT ?
\end{framed}

\begin{framed}
\textbf{\colorbox{orange}{Zadanie 5 - Egzamin }} \\
\begin{align*}
& \text{min} ( e^{-x_1}+2x_2^2 ) \\
& x_1^2+x_2^2 \leq 1 \\
& x_1^2-1 \leq x_2 \\
& x \in R^2 
\end{align*}
Czy w punkcie \( (2,1) \) są spełniona warunki KKT ?
\end{framed}

\begin{framed}
\textbf{\colorbox{orange}{Zadanie 6 - Egzamin}} \\
Dla zadania
\begin{align*}
&\text{min} \; x_1^3-x_1^2x_2+2x_2^2 \\
&x_1 \geq 0 \; x_2 \geq 0
\end{align*}
a) Podaj warunki konieczne optymalności ( Karusha - Tuckera ) \\
b) Udowodnij, że istnieją dwa punkty spełniające te warunki
\end{framed}

\begin{framed}
\textbf{\colorbox{orange}{Zadanie 7 - Egzamin}} \\
Dla zadania 
\begin{align*}
&\text{min} \; -x_1-x_2 \\
&x_1+x_2^2-5\leq 0 \\
&x_1 - 2 \leq 0 \\
&x_1 , \; x_2 \in R^2
\end{align*}
a) Podać warunki konieczne optymalności i znaleźć punkty, które je spełniają. Który z tych punktów jest rozwiązaniem? \\
b) Pokazać, że w tym punkcie są spełnione warunki regularności. 
\end{framed}




\newpage
\section{Programowanie liniowe}

\subsection{Pytania własne - teoria do opracowania}

\begin{enumerate}

\item Jak dodawane są zmienne przy przekształcaniu ograniczeń w programowaniu liniowym ?
\item Które zmienna są zmiennymi bazowymi w pierwszej iteracji ?
\item Jak wyznaczyć pierwsze rozwiązanie bazowe ?


\end{enumerate}


\subsection{Egzamin}

\begin{framed}
\textbf{\colorbox{orange}{Zadanie 1 - Egzamin +}} \\
Podać pierwsze rozwiązanie bazowe: 
\begin{align*}
& \text{max} \; 2x_1 + x_2 \\
& 3x_1+x_2 \leq 7 \\
& 2x_1+3x_2 \geq 10
\end{align*}
\end{framed}

\begin{framed}
\textbf{\colorbox{orange}{Zadanie 2 - Egzamin }} \\
Wypisać ograniczenia oraz funkcję celu dla budynku o zadanych parametrach

\end{framed}






\end{document}
 