\documentclass[a4paper,15pt]{article}
\usepackage{amssymb}
\usepackage{amsmath}
\usepackage[english, polish]{babel}
\usepackage[utf8]{inputenc}   % lub utf8
\usepackage[T1]{fontenc}
\usepackage{graphicx}
\usepackage{anysize}
\usepackage{enumerate}
\usepackage{times}
\usepackage{titlesec}
\usepackage{float}
\usepackage{titlesec}
\usepackage{titleps,kantlipsum}
 
\usepackage[justification=centering]{caption}
\titlelabel{\thetitle.\quad}



\settitlemarks{section,subsection,subsubsection}


%\marginsize{left}{right}{top}{bottom}
\marginsize{3cm}{3cm}{3cm}{3cm}
\sloppy
\titleformat{\section}
  {\normalfont\Large\bfseries}{\thesection}{1em}{}[{\titlerule[0.8pt]}]
 
\begin{document}

\center{\textbf{\textit{Logika - Pytania do mojego opracowania}}}


\begin{itemize}
\item Czym jest zmienna zdaniowa ? Jak się ja oznacza ? Jakie może mieć wartości ?

\item Czym jest zdanie ? Co do niego przypisujemy ? 

\item Wymień 5 podstawowych spójników logicznych i podaj dla nich tabelę prawdy.

\item Podaj spójniki według priorytetów od największego.

\item Co wchodzi w skład alfabetu logiki rachunku zdań ?

\item Czym jest formuła atomiczna ?

\item Czym jest formuła ?

\item Wyjaśnij pojęcie interpretacji, podaj przykład dla dowolnej formuły.

\item Wyjaśnij pojęcia : formuła spełnialna, niespełnialna, tautologia. 

\item Czym jest system funkcjonalnie pełny ? 

\item Czy AND, NOT tworzą system funkcjonalnie pełny ? Czy tworzą go: (AND, OR, NOT) , (OR, NOT), (NOT, IMPLIKACJA), (NAND), (NOR) ? 

\item Czym jest system funkcyjny funkcjonalnie pełny minimalny, a czym redundantny ?

\item Czym jest literał ?

\item Czym jest minterm ?

\item Czym jest maxterm ?

\item Czym jest CNF ? Co łatwo sprawdzić na podstawie CNF ?

\item Czym jest DNF ? Co łatwo sprawdzić na podstawie DNF ?

\item Czym jest NNF ?

\item Wymień najważniejsze równoważności logiczne ? ( 9 praw )

\item Co oznacza \( \varphi \vDash \psi \) ? Jak nazywamy ten symbol ?

\item Jaką postać ma problem dowodzenia twierdzeń w logice ?

\item Jakie znasz metody dowodzenia twierdzeń ? Podaj nazwy i wzory ( 5 )

\item O czym należy pamiętać przy dowodzeniu równoważności ?

\item Podaj treść I twierdzenia o dedukcji.

\item Podaj treść II twierdzenia o dedukcji.

\item Na czym polega dowód liniowy ? Od czego się go rozpoczyna ? Czym może być każda z linii w dowodzie liniowym ?

\item Podaj 3 reguły wnioskowania związane z implikacją.

\item Podaj po dwie reguły wnioskowania dla: koniunkcji, dysjunkcji, negacji, równoważności ( system Fitcha ).

\item Do czego służy metoda rezolucji ?

\item Na jakich wyrażeniach tylko i wyłącznie działa metoda rezolucji ?

\item Podaj algorytm zamiany dowolnego wyrażenia na CNF ( 4 kroki ).

\item Podaj zasadę rezolucji.

\item Czy można wyciągnąć więcej niż jeden wniosek stosując zasadę rezolucji do dwóch wyrażeń i o czym w związku z tym należy pamiętać ?  

\item Z którego twierdzenia korzystamy wykonując dowody przy użyciu metody rezolucji. 

\item Do czego służy rezolucja dualna ?

\item Kiedy jest łatwiej zastosować rezolucję dualną niż zwykłą ( w jakiej postaci wyrażenia do dowodzenia ) ? Do czego głównie służy rezolucja dualna ?

\item Jak zapisywane są wyrażenia w rezolucji dualnej ( wiersze i co one oznaczają ) ?

\item Podaj zasadę rezolucji dualnej

\item Jaki schemat wnioskowania stosuje się często w rezolucji dualnej ( związany z wprowadzeniem koniunkcji ) ?

\item Do czego dążymy ( co chcemy uzyskać aby stwierdzić, że tautologia jest prawdziwa ) podczas dowodzenia tautologii przy użyciu rezolucji dualnej ?

\item Do czego służy metoda tablic semantycznych ? Jak przy jej użyciu dowieść twierdzenie ?

\item Czy są jakieś ograniczenia co do formuły wyjściowej w metodzie tablic semantycznych ?

\item Jak rysujemy metodę tablic semantycznych ?

\item Jakie dwa typy wyrażeń są w metodzie tablic semantycznych ? Który z nich jest zawsze realizowany jako pierwszy ?

\item Co robimy w drzewie z formułą \( \alpha \), a co z formułą \( \beta \) ?

\item Kiedy zamykamy daną gałąź drzewa ?

\item Co oznacza zamknięcie wszystkich gałęzi drzewa ? Co oznacza gdy przynajmniej jedna pozostaje nie zamknięta ?

\item Wymień wszystkie 5 formuł \( \alpha \) oraz na co się one rozkładają.

\item Wymień wszystkie 4 formuły \( \beta \) oraz na co się one rozkładają. 
 
\item Czym różnią się dowody strukturalne od dowodów liniowych jeśli chodzi o budowę ?

\item Co rozpoczyna, a co kończy dowód zagnieżdżony ?

\item Jaki schemat wnioskowania jest wykorzystywany przy wyjściu z zagnieżdżonego dowodu ?

\item Jakie założenia uznajemy za prawdziwe w dowodzie zagnieżdżonym, a jakie prawdziwe nie są ?

\item Do jakich wyrażeń można stosować klasyczne reguły wnioskowania w dowodzie zagnieżdżonym ( gdzie się mogą znajdować ) ?

\item Czym jest system Fitch'a ?

\item Jakie są ograniczenia logiki zdaniowej ? W związku z tym do czego służy rachunek predykatów ?

\item Czym jest predykat ? Jak go oznaczamy ?

\item Czym jest symbol stały ? Jak go oznaczamy ?

\item Czym jest zmienna ? Jak ją oznaczamy ?

\item Czym są kwantyfikatory ? Jakie dwa rodzaje istnieją ?

\item Czym są symbole funkcyjne ?

\item Jaki priorytet mają kwantyfikatory ?

\item Czym jest term ?

\item Czym jest formuła atomiczna w rachunku predykatów ?

\item Czym są formuły rachunku predykatów ?

\item Czym są zmienne wolne, a czym zmienne związane ?

\item Czym jest baza Herbranda ? Czym jest interpretacja Herbranda ? 

\item Jak badamy wartość logiczną wyrażenia w logice predykatów na podstawie bazy Herbranda ?

\item Wymień reguły bazowe dla kwantyfikatorów

\item Na czym polega dystrybucja kwantyfikatorów ?

\item Na czym polega przemianowanie zmiennych w przekształceniach z kwantyfikatorami ?

\item Na czym polega włączenie wyłączenie negacji w kwantyfikatorach ?

\item Czym jest standardowa postać Skolema ( inaczej postać klauzulowa ) ?

\item Podaj kolejne kroki sprowadzania formuły do postaci klauzulowej ( 7 ).

\item Czym różni się rezolucja w logice predykatów od rezolucji w logice zdaniowej ?

\item Na czym polega identyczność dwóch literałów w logice predykatów ?

\item Kiedy mówimy, że dwa wyrażenia mają łącznik ?

\item Jak przebiega sprawdzanie identyczności dwóch formuł ?

\item Podaj zasadę rezolucji dla logiki predykatów.


\end{itemize}




\end{document}