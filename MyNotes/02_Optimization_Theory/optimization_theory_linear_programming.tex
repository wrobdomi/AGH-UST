\documentclass[a4paper,15pt]{article}
\usepackage{amssymb}
\usepackage{amsmath}
\usepackage[english, polish]{babel}
\usepackage[utf8]{inputenc}   % lub utf8
\usepackage[T1]{fontenc}
\usepackage{graphicx}
\usepackage{anysize}
\usepackage{enumerate}
\usepackage{times}
\usepackage{titlesec}
\usepackage{float}
\usepackage{titlesec}
\usepackage{titleps,kantlipsum}
\usepackage{listings}
\usepackage{color}
\lstloadlanguages{Matlab}
 
\usepackage[justification=centering]{caption}
\titlelabel{\thetitle.\quad}



\newpagestyle{mypage}{%
  \headrule
  \sethead{\MakeUppercase{\thesection\quad \sectiontitle}}{}{\thesubsection\quad \subsectiontitle}
  \setfoot{}{}{\thesubsubsection\quad \subsubsectiontitle}
}
\settitlemarks{section,subsection,subsubsection}
\pagestyle{mypage}

%\marginsize{left}{right}{top}{bottom}
\marginsize{3cm}{3cm}{3cm}{3cm}
\sloppy
\titleformat{\section}
  {\normalfont\Large\bfseries}{\thesection}{1em}{}[{\titlerule[0.8pt]}]
 
 \definecolor{darkred}{rgb}{0.9,0,0}
\definecolor{grey}{rgb}{0.4,0.4,0.4}
\definecolor{orange}{rgb}{1,0.6,0.05}
\definecolor{darkgreen}{rgb}{0.2,0.5,0.05}
 
\lstset{
language=Matlab,
basicstyle=\ttfamily\small,
keywordstyle=\color{darkgreen}\ttfamily\bfseries\small,
stringstyle=\color{red}\ttfamily\small,
commentstyle=\color{grey}\ttfamily\small,
numbers=left,
numberstyle=\color{darkred}\ttfamily\scriptsize,
identifierstyle=\color{blue}\ttfamily\small,
showstringspaces=false,
morekeywords={}
}

\begin{document}

\begin{table}
\begin{center}
\begin{tabular}{|l|l|l|}
\hline
\multicolumn{3}{|c|}{\textbf{Programowanie liniowe}} \\ \hline Dominik Wróbel & \textbf{23 IV 2018} & \textbf{Pon 08:00, s. 111} \\ \hline
\multicolumn{2}{|l|}{Numery zadań} & 2, 3 \\ \hline 

\end{tabular}
\end{center}
\end{table}

\section{Cel ćwiczenia}
Celem ćwiczenia jest zapoznanie się z optymalizacją praktycznych zagadnień przy użyciu programowania liniowego. Rozwiązane przykłady pozwolą na nabycie umiejętności budowy poprawnego modelu matematycznego dla danego zagadnienia. 
\section{Przebieg ćwiczenia}
\subsection{Zadanie 2 - Optymalizacja mocy wydzielanej na mostku }
W zadaniu rozważany jest układ mostka z 5 opornikami. Układ z polecenia przedstawia Rysunek  \ref{fig:pl_1}
\begin{figure}[H]
\centerline{\includegraphics[scale=1]{pl_1.jpg}}
\centering
\caption{Mostek analizowany w zadaniu}
\label{fig:pl_1}
\end{figure}
W zadaniu dane są napięcia odkładające się na każdym z oporników oraz zakresy wartości prądów jakie muszą przepływać przez dany opornik. 
Każdy prąd \( I_{i} \) musi spełniać warunek 
\begin{equation*}
\overline{I_{i}}-\Delta_{i} \leq I_{i} \leq \overline{I_{i}}+\Delta_{i}
\end{equation*}  
przy danych założeniach 
\begin{figure}[H]
\centerline{\includegraphics[scale=1]{pl_2.jpg}}
\centering
\caption{Założenia zadania}
\label{fig:pl_2}
\end{figure}
\newpage
Zadanie rozpoczęto od budowy modelu matematycznego na podstawie I prawa Kirchhoffa dla węzłów mostka :
\begin{equation*}
\begin{cases}
I_{1}-I_{3}-I_{4}=0 \\
I_{2}+I_{3}-I_{5}=0 \\
I_{1}+I_{2}-I_{4}-I_{5}=0
\end{cases}
\end{equation*}
Powyższe równania odpowiadają modelowi matematycznemu postaci \( Ax = b \), w którym 
\begin{equation*}
A=
\begin{bmatrix} 
1 & 0 & -1 & -1 & 0 \\ 
0 & 1 & 1 & 0 & -1 \\
1 & 1 & 0 & -1 & -1  
\end{bmatrix}
\quad 
b=
\begin{bmatrix} 
0 \\ 
0 \\
0
\end{bmatrix}
\end{equation*}
Celem zadania jest minimalizacja całkowitej mocy wydzielanej na mostku, a zatem funkcja celu jest postaci: 
\begin{equation*}
f_{c}(x)=U_{1}I_{1}+U_{2}I_{2}+U_{3}I_{3}+U_{4}I_{4}+U_{5}I_{5} \rightarrow min
\end{equation*}
Funkcja celu odpowiada modelowi \( f_{c}(x)=cx \), zmienne decyzyjne w tym modelu to prądy płynące przez i-te oporniki, współczynniki funkcji celu, to napięcia na i-tych opornikach, stąd 
\begin{equation*}
c = 
\begin{bmatrix}
U_{1} \quad U_{2} \quad  U_{3} \quad  U_{4}  \quad  U_{5}
\end{bmatrix}
\end{equation*}
Nierówność 
\begin{equation*}
\overline{I_{i}}-\Delta_{i} \leq I_{i} \leq \overline{I_{i}}+\Delta_{i}
\end{equation*}  
określa górne i dolne ograniczenie dla zmiennych decyzyjnych, dlatego
\begin{equation*}
l = 
\overline{I_{i}}-\Delta_{i} \quad
u = 
\overline{I_{i}}+\Delta_{i}
\end{equation*}
W celu implementacji powyższego modelu do pliku zrodpl.m dopisano wszystkie potrzebne dane oraz na ich podstawie stworzono odpowiednie macierze. Skrypt przedstawia listing \ref{lis1}.
\begin{lstlisting}[caption=zrodpl.m, captionpos=b,
label=lis1, firstnumber=12,frame=single]

Diff = [];
Sum = [];
U_i = [ 6 10 4 7 3 ];
I_i = [ 4 2 2 2 4 ];
Delta_i = [ 1 1 1 1 0 ];
for i = 1:5
Diff = [ Diff I_i(i)-Delta_i(i) ];
end
for i = 1:5
Sum = [ Sum I_i(i)+Delta_i(i) ];
end
Diff=Diff';
Sum=Sum';
A = [ 1, 0, -1, -1,  0;
      0, 1 , 1, 0 , -1;
      1, 1, 0,  -1, -1 ];
b = [ 0 ; 0 ; 0 ];  % Ograniczenia Ax = b
t = [ 0 ;0 ; 0 ];   % typy ograniczen
c = U_i;            % wspolczynniki funkcji celu
l = Diff;           % I_i - delta_i, dolne ograniczenie
u = Sum;            % I_i + delta_i, gorne ograniczenie
zadan ='mini';
\end{lstlisting}

Stworzono także plik main.n wywołujący zrodpl.m oraz obliczający wartości optymalne mocy i oporników. Skrypt przedstawia listing \ref{lis2}.
\begin{lstlisting}[caption=main.m, captionpos=b,
label=lis2, firstnumber=12,frame=single]

plnad;
xopt 
M = c * xopt ; % zuzyta moc optymalna
Moc = M / 1000
Oporniki = c ./ xopt'*1000; % opory optymalne R = u / i
Oporniki
Moc
\end{lstlisting}

Otrzymano następujące wyniki: 
\begin{equation*}
R_{opt} = 
\begin{bmatrix}
2000 \\ 
5000 \\
2000 \\ 
7000 \\
750
\end{bmatrix}
\Omega
\quad
P_{opt} = 0.0650 W
\quad
I_{opt} =
\begin{bmatrix}
3 \\
2 \\
2 \\
1 \\
4
\end{bmatrix}
mA
\end{equation*}
\subsection{Zadanie 3 - Optymalizacja procesu wytwarzania}
W zadaniu rozważany zakład produkcyjny, który wytwarza trzy wyroby. Dana jest liczba roboczogodzin potrzebna do wykonania sztuki produktu: \\
\begin{figure}[H]
\centerline{\includegraphics[scale=1]{pl_3.jpg}}
\centering
\caption{Założenia zadania}
\label{fig:pl_3}
\end{figure}
Dane jest też liczba możliwych do wykorzystania roboczogodzin w ciągu tygodnia w każdym z działów zakładu produkcyjnego:
\begin{itemize}
\item Montaż 1800
\item Kontrola 800
\item Pakowanie 700
\end{itemize}
oraz zysk ze sprzedaży każdego z produktu:
\begin{itemize}
\item Wyrób A 400 zł/szt
\item Wyrób B 300 zł/szt
\item Wyrób C 200 zł/szt
\end{itemize}
Powyższe ograniczenia odpowiadają modelowi matematycznemu postaci \( Ax \leq b \), w którym wektor b zawiera całkowitą możliwą liczbę roboczogodzin w danym dziale. x to wektor zmiennych decyzyjnych, który opisuje liczbę wyprodukowanych produktów danego typu.
\begin{equation*}
A=
\begin{bmatrix} 
0.3 & 0.5 & 0.4 \\ 
0.1 & 0.08 & 0.12 \\
0.06 & 0.04 & 0.05  
\end{bmatrix}
\quad 
b=
\begin{bmatrix} 
1800 \\ 
800 \\
700
\end{bmatrix}
\end{equation*}
Celem zadania jest maksymalizacja zysku z produkowanych produktów.
\begin{equation*}
f_{c}(x)= 400x_{a} + 300x_{b} + 200x_{c} \rightarrow max
\end{equation*}
Funkcja celu odpowiada modelowi \( f_{c}(x)=cx \), zmienne decyzyjne w tym modelu to ilość wyprodukowanych sztuk danego wyrobu
\begin{equation*}
c = 
\begin{bmatrix}
400 \quad 300 \quad  200 
\end{bmatrix}
\end{equation*}
Ilość produkowanych produktów powinna być większa lub równa zero oraz mniejsza od nieskończoności, są to ograniczenia dolne górne na wartość znalezionego rozwiązania :
\begin{equation*}
l = 
\begin{bmatrix}
0 \\
0 \\
0
\end{bmatrix}
=
0 \leq x_{i} < \inf
=
\begin{bmatrix}
inf \\ 
inf \\
inf
\end{bmatrix} = 
u
\end{equation*}  
W celu implementacji powyższego modelu do pliku zrodpl.m dopisano wszystkie potrzebne dane oraz na ich podstawie stworzono odpowiednie macierze. Skrypt przedstawia listing \ref{lis3}.
\begin{lstlisting}[caption=zrodpl.m, captionpos=b,
label=lis3, firstnumber=12,frame=single]

product_A = [ 0.3 0.1 0.06 ]; % roboczogodziny potrzebne na sztuke 
product_B = [ 0.5 0.08 0.04 ]; % danego produktu
product_C = [ 0.4 0.12 0.05 ];
max_mon = 1800; % max. dostepna liczba roboczogodzin
max_kont = 800; % w tygodniu
max_pakow = 700;
zysk_A = 400;
zysk_B = 300;
zysk_C = 200;
A = [ product_A(1) product_B(1) product_C(1);
      product_A(2) product_B(2) product_C(2);
      product_A(3) product_B(3) product_C(3);
      ];
b = [ max_mon ; max_kont ; max_pakow ];  % Ograniczenia Ax = b
t = [ -1 ;-1 ; -1 ];   % -1 to <=
c = [ zysk_A zysk_B zysk_C ];            % wspolczynniki funkcji celu
l = [ 0; 0; 0] ;           % , dolne ograniczenie
u = [ inf; inf; inf];      % , gorne ograniczenie
zadan ='maks';

\end{lstlisting}

Stworzono także plik main.n wywołujący zrodpl.m oraz obliczający wartości optymalne liczy sztuk produkowanych produktów oraz całkowity zysk. Skrypt przedstawia listing \ref{lis4}.
\newpage
\begin{lstlisting}[caption=main.m, captionpos=b,
label=lis4, firstnumber=12,frame=single]

plnad;
xopt 
profit = xopt(1)*zysk_A+xopt(2)*zysk_B+xopt(3)*zysk_C;
profit
\end{lstlisting}

Otrzymano następujące wyniki: 
\begin{equation*}
x_{opt} = 
\begin{bmatrix}
6000 \\
0 \\ 
0
\end{bmatrix}
= 
\begin{bmatrix}
x_{a} \\
x_{b} \\
x_{c}
\end{bmatrix}
\quad
Zysk_{opt} = 2400000
\end{equation*}
Ilość produkowanego produktu \( x_{a} \) odpowiada optymalnemu tygodniowemu planowi produkcji z zyskiem 2400000 zł. Uzyskana odpowiedź spełnia wszystkie założenia. 

\section{Wnioski końcowe}
Programowanie liniowe pozwala na łatwe rozwiązywanie wielu praktycznych problemów. Kluczowe przy tego typu zadaniach jest zbudowanie prawidłowego modelu matematycznego z uwzględnieniem wszystkich ograniczeń. Następnie rozwiązanie przy pomocy metody Simpleks wymaga dbałości o zapisane ograniczenia. 
  
\end{document}