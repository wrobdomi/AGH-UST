\documentclass[a4paper,15pt]{article}
\usepackage{amssymb}
\usepackage{amsmath}
\usepackage[english, polish]{babel}
\usepackage[utf8]{inputenc}   % lub utf8
\usepackage[T1]{fontenc}
\usepackage{graphicx}
\usepackage{anysize}
\usepackage{enumerate}
\usepackage{times}
\usepackage{caption}
\usepackage{titlesec}
\usepackage{float}
\usepackage{titleps,kantlipsum}
\usepackage{listings}
\usepackage{xcolor}
\usepackage{hyperref}
\usepackage{framed}
\usepackage{tcolorbox}
\lstloadlanguages{Matlab}
 
\usepackage[justification=centering]{caption}
\titlelabel{\thetitle.\quad}

\pagenumbering{arabic}

\DeclareCaptionFont{white}{\color{white}}
\DeclareCaptionFormat{listing}{%
  \parbox{\textwidth}{\colorbox{darkgreen}{\parbox{\textwidth}{#1#2#3}}\vskip-4pt}}
\captionsetup[lstlisting]{format=listing,labelfont=white,textfont=white}
\lstset{frame=lrb,xleftmargin=\fboxsep,xrightmargin=-\fboxsep}

% Definicja nowego stylu strony
\newpagestyle{mypage}
{
  \headrule
  
  \sethead
  { \MakeUppercase{\thesection\quad \sectiontitle} } 
  {}
  {\thesubsection\quad \subsectiontitle}
  
  \setfoot
  {}
  {}
  {\thepage}
}

\newpagestyle{mypage_1}
{
	\headrule
	
	\sethead
	{  }
	{\MakeUppercase{Metody Formalne - Sieci Petriego}}
	{}
	
	\setfoot
	{}
	{\thepage}
	{}
}

\settitlemarks{section,subsection,subsubsection}

\pagestyle{mypage_1}

\definecolor{mGreen}{rgb}{0,0.6,0}
\definecolor{mGray}{rgb}{0.5,0.5,0.5}
\definecolor{mPurple}{rgb}{0.58,0,0.82}
\definecolor{mKeyword}{RGB}{0,0,242}
\definecolor{backgroundColour}{RGB}{242,242,242}

\newcommand{\definition}[2]{
    \begin{tcolorbox}[colback=green!5!white,colframe=mGreen,title={Definicja -  #1}]
        #2
    \end{tcolorbox}
}

\newcommand{\question}[2]{
    \begin{tcolorbox}[colback=black!5!white,colframe=black,title={Zagadnienie #1}]
        #2
    \end{tcolorbox}
}

%\marginsize{left}{right}{top}{bottom}
\marginsize{3cm}{3cm}{3cm}{3cm}
\sloppy
\titleformat{\section}
  {\normalfont\Large\bfseries}{\thesection}{1em}{}[{\titlerule[0.8pt]}]
 
 \definecolor{darkred}{rgb}{0.9,0,0}
\definecolor{grey}{rgb}{0.4,0.4,0.4}
\definecolor{orange}{rgb}{1,0.6,0.05}
\definecolor{darkgreen}{rgb}{0.2,0.5,0.05}
 


\lstdefinestyle{CStyle}{
    backgroundcolor=\color{backgroundColour},   
    commentstyle=\color{mGreen},
    keywordstyle=\color{mKeyword},
    numberstyle=\tiny\color{mGray},
    stringstyle=\color{mPurple},
    basicstyle=\footnotesize,
    breakatwhitespace=false,         
    breaklines=true,                 
    %captionpos=b,                    
    keepspaces=true,                 
    numbers=left,                    
    numbersep=5pt,                  
    showspaces=false,                
    showstringspaces=false,
    showtabs=false,                  
    tabsize=2,
    language=C
}


\newcommand{\Hilight}{\makebox[0pt][l]{\color{cyan}\rule[-4pt]{0.65\linewidth}{14pt}}}


\begin{document}

\begin{table}
\begin{center}
\begin{tabular}{|c|c|c|}
\hline
\multicolumn{3}{|c|}{\textbf{Opracowanie do kolokwium nr 1}} \\ \hline Metody formalne & 16 IV 2019 & s. 429 08:00 \\ \hline

\end{tabular}
\end{center}
\end{table}

\tableofcontents

\newpage
\section{Zagadnienia od Szymkata}
\begin{enumerate}
\item Definicja sieci Petriego, miejsca, przejścia, znaczniki, warunki
wykonania przejścia, stany (znakowania)

\item Własności sieci Petriego: osiągalność, ograniczoność, zachowawczość,
żywotność, sprawiedliwość

\item Definicja kolorowanej sieci Petriego, miejsca, przejścia, znaczniki,
kolory, opisy łuków, dozory, warunki wykonania przejścia (wiązania),
stany (znakowania)

\item Podstawowe konstrukcje w sieci Petriego, licznik, limiter, inhibitor

\item Graf przestrzeni stanów kolorowanej sieci Petriego, elementy,
poprzedniki, następniki, stany zakleszczone, znakowanie początkowe,
znakowania docelowe (home)

\item Elementy raportu o wygenerowanym grafie stanów w CPNtools, statystyki
(węzły, łuki, status, graf komponentów spójnych), ograniczoność,
osiągalność, żywotność, sprawiedlwość

\item Modelowanie współdzielenia zasobów w kolorowanej sieci Petriego

\item Indeterminizm w sieciach Petriego

\item Synchronizacja procesów współbieżnych w sieciach Petriego
\end{enumerate}

\section{Uwagi}
\begin{itemize}
\item Podobno definicje formalne są korzystniejsze od ogólnych
\end{itemize}

\section{Źródła}
Opracowanie na podstawie:
\begin{itemize}
\item Wykłady Szmuca
\item Szpyrka, Szmuc - Metody formalne w inżynierii oprogramowania systemów czasu rzeczywistego
\item Szpyrka - sieci Petriego w modelowaniu i analizie systemów współbieżnych
\item cpntools.org - Anti places/limit places
\item cpntools.org - Inhibitor arcs
\item \url{http://home.hib.no/ansatte/lmkr/talks/lmk_statespaces_pn2014.pdf} - całkiem fajny pdf
\end{itemize}


\newpage
\section{Opracowanie zagadnień}

\subsection{Zagadnienie 1}
Wszystkie opisy i definicje w tym podrozdziale (Zagadnienie 1) dotyczą \textit{Sieci Petriego miejsc i przejść}. (są to sieci różne od sieci kolorowanych).
\question{1}{
Definicja sieci Petriego, miejsca, przejścia, znaczniki, warunki
wykonania przejścia, stany (znakowania).
}

\subsubsection{Definicja sieci Petriego}
\definition{Sieć Petriego (miejsc i przejść)}{
Siecią Petriego nazywamy piątkę, $PM = (P, T, A, W, s_0)$, gdzie:
\begin{itemize}
\item P - zbiór (skończony) miejsc
\item T - zbiór (skończony) przejść
\item A $ \subseteq P \times T \cup T \times P$ - zbiór łuków
\item W: $A \rightarrow N$ - funkcja wag przypisująca etykiety (liczby naturalne) do każdego łuku
\item $s_0: P \rightarrow N^*$ - funkcja opisująca oznakowanie początkowe, gdzie $N^*$ oznacza zbiór liczb całkowitych nieujemnych
\end{itemize} 
Ponadto dla każdej sieci Petriego spełniony jest warunek:
\begin{align*}
& P \cap T = \emptyset \wedge P \cup T \neq \emptyset
\end{align*}
}

\subsubsection{Miejsca}
Miejsca to niepusty zbiór $P = \{ p_1, p_2, \dots , p_n \}$. Miejsca są graficznie reprezentowane jako elipsy. Miejsca mogą być połączone łukami jedynie z tranzycjami. Miejscom przyporządkowane są żetony.


\subsubsection{Przejścia (tranzycje)}
Przejścia to niepusty zbiór $T = \{ t_1, t_2, \dots , t_n \}$. Przejścia są graficznie reprezentowane jako prostokąty. Przejścia mogą być połączone łukami jedynie z miejscami. Przejście może być aktywne lub nieaktywne.

\subsubsection{Znaczniki}
Znaczniki to liczby całkowite przyporządkowane miejscom, które określają stan sieci i oznaczają liczbę żetonów znajdującą się w danym miejscu.

\newpage
\subsubsection{Warunki wykonania przejścia (w sieciach miejsc i przejść)}
Przejście może być wykonane gdy jest aktywne
\definition{Przejście aktywne - Sieć miejsc i przejść}{
Przejście $t \in T$ jest aktywne przy znakowaniu M, jeżeli każde z jego miejsc wejściowych zawiera co najmniej tyle znaczników, ile wynosi waga łuku prowadzącego od tego miejsca do przejścia t oraz każde z jego miejsc wyjściowych ma wystarczającą pojemność, by przyjąć tyle znaczników, ile wynosi waga łuku $(t,p)$ tzn.: 
\begin{align*}
& \forall p \in In(t): M(p) \geq W(p,t) \wedge \\
& \forall p \in Out(t): M(p) + W(t,p) \leq K(p)
\end{align*}
}

\subsubsection{Znakowanie}
\definition{Znakowanie}{
Znakowaniem sieci miejsc i przejść nazywamy dowolną funkcję 
\begin{align*}
M: P \rightarrow Z^+ \text{ , taką że } \forall p \in P: M(p) \leq K(p)
\end{align*}
$K(p)$ to funkcja określająca 'pojemność' maksymalną miejsc.
}

\begin{framed}
\begin{itemize}
\item Podaj definicję sieci Petriego (miejsc i przejść).
\item Podaj warunki wykonania przejścia w sieciach Petriego
\item Podaj definicję znakowania
\end{itemize}
\end{framed}

\newpage
\subsection{Zagadnienie 2}
\question{2}{
Własności sieci Petriego: osiągalność, ograniczoność, zachowawczość,
żywotność, sprawiedliwość
}

\subsubsection{Osiągalność}

\definition{Osiągalność}{
Niech będzie dana sieć $(PN, s_0)$. Mówimy, że stan s jest osiągalny ze stanu $s_0$ wtedy i tylko wtedy jeśli $s \in R(PN, s_0)$. \\
gdzie $R(PN,s_0)$ oznacza zbiór wszystkich stanów osiągalnych ze stanu początkowego $s_0$. \\ \\
$R(PN,s_0) = Ran(\rho ^*) = Ran (\cup_{n=0}\rho ^n)$ - to przeciwdziedzina iteracji relacji bezpośredniego następstwa stanów $\rho$ w sieci Petriego.
}


\subsubsection{Ograniczoność}
\definition{Ograniczoność}{
Znakowana sieć $(PN, s_0)$ jest k-ograniczona, jeśli przy dowolnym osiągalnym oznakowaniu w każdym miejscu ilość znaczników nie przekracza skończonej liczby k, \\ 
tzn, dla dowolnego $s \in R(PN, s_0)$ oraz dla każdego miejsca $p\in P$ spełniony jest warunek $s(p)\leq k$. \\ \\
gdzie $R(PN,s_0)$ oznacza zbiór wszystkich stanów osiągalnych ze stanu początkowego $s_0$. \\ \\
$R(PN,s_0) = Ran(\rho ^*) = Ran (\cup_{n=0}\rho ^n)$ - to przeciwdziedzina iteracji relacji bezpośredniego następstwa stanów $\rho$ w sieci Petriego. \\ \\
Znakowana sieć $PM = (PN, s_0)$ jest ograniczona jeśli jest k-ograniczona dla pewnego skończonego k
}

\subsubsection{Zachowawczość}
\definition{Zachowawczość}{
Znakowana sieć $(PN, s_0)$ jest (bezwzględnie) zachowawcza jeśli dla łączna liczba znaczników w sieci pozostaje stała dla każdego znakowania osiągalnego ze znakowania początkowego, \\ tzn. dla dowolnego stanu $s \in R(PN, s_0)$ spełniony jest warunek 
\begin{align*}
\sum_{p\in P}s(p)=\sum_{p\in P}s_0(p)
\end{align*}
\\ \\
gdzie $R(PN,s_0)$ oznacza zbiór wszystkich stanów osiągalnych ze stanu początkowego $s_0$. \\ \\
$R(PN,s_0) = Ran(\rho ^*) = Ran (\cup_{n=0}\rho ^n)$ - to przeciwdziedzina iteracji relacji bezpośredniego następstwa stanów $\rho$ w sieci Petriego. \\ \\

Znakowana sieć $(PN, s_0)$ jest zachowawcza względem wektora wag $(w_1, w_2, \dots , w_n)$ jeżeli łączna ważona liczba znaczników w sieci pozostaje stała dla każdego znakowania osiągalnego ze znakowania początkowego \\ tzn. dla dowolnego stanu $s \in R(PN, s_0)$ spełniony jest warunek 
\begin{align*}
\sum_{i}^{} w_is(p_i)=\sum_{i}w_is_0(p_i)
\end{align*}
}

\newpage
\subsubsection{Żywotność}
Żywotność określa wykonywalność pewnych tranzycji w czasie 'wykonywania' sieci. Własności te są różne w zależności od tego, które z przejść i jak często są wykonywane dlatego konieczne jest rozbicie definicji na 5 przypadków.
\definition{Żywotność}{
Niech dana będzie sieć Petriego $(PN, s_0)$. Przez $L(s_0)$ oznaczamy zbiór wszystkich ciągów przejść, które można wykonać rozpoczynając od stanu $s_0$. \\
Mówimy, że dowolne przejście $t \in T$ jest:
\begin{itemize}
\item L0-żywotne (martwe) -  jeśli to przejście nie może być wykonane dla każdej sekwencji odpaleń $L(s_0)$
\item L1-żywotne (potencjalnie odpalalne) jeśli może być wykonane przynajmniej raz dla pewnej sekwencji odpaleń $L(s_0)$
\item L2-żywotne jeśli to przejście może być wykonane przynajmniej  k razy  dla pewnej sekwencji odpaleń $L(s_0)$
\item L3-żywotne jeśli t występuje nieskończenie wiele razy w pewnej sekwencji odpaleń $L(s_0)$
\item L4-żywotne (żywotne) jeśli t jest L1-żywotne w każdym oznakowaniu osiągalnym  $s \in R(PN, s_0)$
\end{itemize}
gdzie $R(PN,s_0)$ oznacza zbiór wszystkich stanów osiągalnych ze stanu początkowego $s_0$. \\ \\
}

\subsubsection{Sprawiedliwość}
\definition{Sprawiedliwość}{
Niech dana będzie sieć Petriego $(PN, s_0)$. Niech $\sigma$ oznacz sekwencję wykonań przejść.
\begin{itemize}
\item Sekwencja wykonań $\sigma$ sieci PM jest (globalnie) sprawiedliwa jeśli jest skończona lub każde przejście występuje nieskończenie wiele razy w $\sigma$
\item Sieć PM jest (globalnie) sprawiedliwa jeśli każda sekwencja wykonań tej sieci jest globalnie sprawiedliwa.
\end{itemize}
}


\begin{framed}
\begin{itemize}
\item Podaj definicję osiągalności w sieciach Petriego.
\item Podaj definicję ograniczoności w sieciach Petriego.
\item Podaj definicję zachowawczości w sieciach Petriego.
\item Podaj definicję żywotności w sieciach Petriego.
\item Podaj definicję sprawiedliwości w sieciach Petriego.
\end{itemize}
\end{framed}

\newpage
\subsection{Zagadnienie 3}
Wszystkie opisy i definicje w tym podrozdziale (Zagadnienie 3) dotyczą \textit{Kolorowanych Sieci Petriego (CPN)}. (są to sieci różne od sieci miejsc i przejść - Zagadnienie 1).
\question{3}{
Definicja kolorowanej sieci Petriego, miejsca, przejścia, znaczniki,
kolory, opisy łuków, dozory, warunki wykonania przejścia (wiązania),
stany (znakowania)
}

\subsubsection{Definicja kolorowanej sieci Petriego (CPN)}
\definition{Sieć kolorowana}{
Siecią kolorowaną nazywamy krotkę 
\begin{align*}
CPN = (\Sigma , P, T, A, N, C, G, E, I)
\end{align*}
gdzie:
\begin{itemize}
\item $\Sigma$ - niepusty, skończony, zbiór typów zwany zbiorem kolorów
\item P - skończony zbiór miejsc
\item T - skończony zbiór przejść
\item A - skończony zbiór łuków, taki, że $P \cap T = P \cap A = T \cap A = \emptyset$
\item $N: A \rightarrow (P\times T)\cup (T \times P)$ - funkcja zaczepienia przypisująca każdemu łukowi uporządkowaną parę węzłów
\item $C: P \rightarrow \Sigma $ - jest funkcją typów (kolorów), która określa, jakiego typu znaczniki każde z miejsc może zawierać
\item $G$ - funkcja dozorów (zastrzeżeń), każdemu z przejść przypisuje wyrażenie (tzw. dozór), które może zawierać zmienne typów należących do $\Sigma$ i którego dowolne wartościowanie daje w wyniku wartość logiczną
\item E - funkcja wag łuków, przypisuje każdemu z łuków wyrażenie, które może zawierać zmienne należące do $\Sigma$ i którego dowolne wartościowanie daje w wyniku wielozbiór nad typem przypisanym do danego miejsca
\item I - funkcja inicjalizacji, która każdemu miejscu przyporządkowuje wielozbiór nad typem przypisanym do tego miejsca.
\end{itemize}
}

\subsubsection{Miejsca}
W CPN każde miejsce musi mieć przypisany kolor, który określa jakiego typu znaczniki to miejsce może posiadać. 

\subsubsection{Przejścia}
W CPN przejściom przypisywane są wyrażenia, które zawierają zmienne należące do $\Sigma$ i których dowolne wartościowanie daje w wyniku wartość logiczną


\subsubsection{Parę pojęć pomocniczych dla dalszych definicji}
\begin{itemize}
\item $Var(expr)$ - zbiór zmiennych występujących w wyrażeniu
\item $Type(v)$ - typ zmiennej $v$
\item $expr<b>$ - wartość wyrażenia expr przy wartościowaniu b, np. $G<b>$ to wartość dozoru przy wartościowaniu b
\item $b(v)$ - wiązanie zmiennej $v$
\end{itemize}

\subsubsection{Wiązanie przejścia}
Wiązaniem przejścia $t \in T$ nazywamy odwzorowanie $b$ takie, że każdej zmiennej przypisano wartość należącą do typu tej zmiennej i jednocześnie jest spełnione zastrzeżenie(dozór) przejścia t. \\ 
tzn.
\begin{align*}
\forall v \in Var(t): b(v) \in Type(v) \wedge G(t)<b>
\end{align*}
Zbiór wszystkich wiązań dla t oznaczamy $B(t)$.

\subsubsection{Element znacznikowy}
Jest to dowolna para $(p,c)$ składająca się z koloru i miejsca.  

\subsubsection{Element wiązania}
To para $(t,b)$, składająca się z przejścia i wiązania przejścia, gdzie $t \in T$ oraz $b \in B(t)$

\subsubsection{Zbiory TE oraz BE}
Zbiór wszystkich elementów znacznikowych oznaczamy TE. \\
Zbiór wszystkich elementów wiązań oznaczamy BE. 

\subsubsection{Oznakowanie}
W sieciach kolorowanych oznakowaniem jest odwzorowanie, które każdemu miejscu przypisuje wielozbiór nad TE (TE to zbiór wszystkich elementów znacznikowych).
\begin{align*}
\forall p \in P: M(p) \in 2^{C(p)^*}
\end{align*}

\subsubsection{Warunki wykonania przejścia (w sieciach CPN) }
W sieciach kolorowanych krokiem jest niepusty i skończony wielozbiór nad BE (BE to zbiór wszystkich elementów wiązań), tj. wielozbiór nad wartościami spełniającymi warunek dozoru. \\
Przejście $t \in T$ jest aktywne (czyli dowolny krok jest wzbudzony) przy oznakowaniu M wtedy i tylko wtedy gdy spełniony jest warunek:
\begin{align*}
\forall p \in In(t): E(p,t)<b> \text{ } \leq  M(p)
\end{align*}
tzn. jeżeli każde miejsce wejściowe przejścia t zawiera przy wiązaniu b odpowiednią liczbę i odpowiednie wartości znaczników.


\begin{framed}
\begin{itemize}
\item Podaj definicję CPN.
\item Podaj definicję oznakowania w CPN.
\item Podaj warunki wykonania przejścia w CPN. 
\end{itemize}
\end{framed}

\newpage
\subsection{Zagadnienie 4}
\question{4}{
Podstawowe konstrukcje w sieci Petriego, licznik, limiter, inhibitor
}

\subsubsection{Inhibitor}
Łukiem \textit{inhibior} w sieciach Petriego nazywany jest łuk, który łączy miejsce z przejściem, ale którego logika jest odwrotna do normalnego działania sieci Petriego. Przejście jest aktywne wtedy gdy w miejscu wejściowym \underline{nie ma} znacznika, a nie wtedy gdy tam jest. Przykład
\begin{figure}[H]
\centerline{\includegraphics[scale=0.8]{inhibitor.png}}
\caption{Przykładowy inhibitor}
\label{fig:inhibitor}
\end{figure}
Na Rysunku \ref{fig:inhibitor} przejście T1 jest aktywne wtedy gdy w miejscu P2 \underline{nie ma} żetonu, a nie gdy tam jest. \\
Przykładowo w sytuacji pokazanej na Rysunku poniżej chcielibyśmy mieć inhibtor w miejscu łuku różowego, po to aby konsumer mógł iść spać jeśli nic nie ma do odebrania z sieci. (Przejście sleep powinno być aktywne gdy w miejscu Network \underline{nie ma} żadnego żetonu.
\begin{figure}[H]
\centerline{\includegraphics[scale=0.6]{CPNinhibitor.png}}
\caption{Przykład gdzie przydatne jest wykorzystanie łuku Inhibitor}
\label{fig:CPNinhibitor}
\end{figure}
W CPN Tools nie ma takiego łuku jak Inhibitor dlatego trzeba radzić sobie pewnymi konstrukcjami zastępczymi, które działają tak jak łuk inhibitor. \\
Inhibitor można zaimplementować przy pomocy tzw. anty-miejsca. Polega to na dodaniu miejsca dla którego liczba żetonów zwiększa się wraz z przybywaniem ich wiadomości do sieci, a gdy wszystkie wiadomości zostały przesłane to wówczas aktywne jest przejście Sleep. Przykład:
\begin{figure}[H]
\centerline{\includegraphics[scale=0.5]{ANTIinhibitor.png}}
\caption{Implementacja inhibitora z wykorzystaniem anty-miejsca}
\label{fig:ANTIinhibitor}
\end{figure}

Inhibitor można także zaimplementować z wykorzystaniem licznika.

\begin{framed}
\begin{itemize}
\item Czym jest inhibitor ?
\item Jak zamodelować inhibitor przy użyciu anty-miejsca ?
\end{itemize}
\end{framed}


\subsubsection{Licznik}
Jest to osobne miejsce, które oblicza liczbę zajść pewnego zdarzenia w sieci. Charakterystyczne dla licznika jest to, że z każdą tranzycją związane są dwa łuki
\begin{itemize}
\item Wchodzący do tranzycji od licznika - daje informację o aktualnej wartości licznika, zmienna n
\item Wychodzący od tranzycji do licznika - zmniejsza/zwiększa licznik w wyniku odpalenie jakiegoś przejścia, np. n+1 lub n-1
\end{itemize}
Przykład 
\begin{figure}[H]
\centerline{\includegraphics[scale=0.7]{counter.png}}
\caption{Implementacja inhibitora przy użyciu licznika}
\label{fig:counter}
\end{figure}


\begin{framed}
\begin{itemize}
\item Czym jest licznik?
\item Jak zamodelować licznik w sieciach Petriego ?
\end{itemize}
\end{framed}


\subsubsection{Limiter}
Limiter do konstrukcja, która ma umożliwić ograniczenie liczb przechowywanych żetonów w danym miejscu. Np. na Rysunku poniżej mamy sieć i chcemy aby w sieci znajdywały się maksymalnie dwa pakiety (dwa żetony).
\begin{figure}[H]
\centerline{\includegraphics[scale=0.8]{net.png}}
\caption{Sieć gdzie potrzebny jest limiter}
\label{fig:net}
\end{figure}
Można to osiągnąć przez wprowadzenie anty-miejsca, które będzie działać na tranzycję send w ten sposób, że dopiero po odebraniu dwóch pakietów możliwe będzie dalsze wysyłanie.
\begin{figure}[H]
\centerline{\includegraphics[scale=0.8]{limiter.png}}
\caption{Przykład implementacji limitera z wykorzystaniem anty-miejsca.}
\label{fig:limiter}
\end{figure}

\begin{framed}
\begin{itemize}
\item Czym jest limiter ?
\item Jak zamodelować limiter w sieciach Petriego ?
\end{itemize}
\end{framed}



\newpage
\subsection{Zagadnienie 5}
\question{5}{
Graf przestrzeni stanów kolorowanej sieci Petriego, elementy,
poprzedniki, następniki, stany zakleszczone, znakowanie początkowe,
znakowania docelowe (home)
}

To zagadnienie omówione zostanie na przykładzie 5 filizofów z CPN Tools.
\begin{figure}[H]
\centerline{\includegraphics[scale=0.8]{philo.png}}
\caption{Pięciu filozofów}
\label{fig:philo}
\end{figure}

Dla tego grafu generujemy sobie raport i dostajemy jako pierwsze:
\begin{figure}[H]
\centerline{\includegraphics[scale=0.8]{philostatic.png}}
\caption{Pięciu filozofów - statystyki z raportu}
\label{fig:philostatic}
\end{figure}

\subsubsection{Graf przestrzeni stanów}
W tym zagadnieniu interesuje nas graf przestrzeni stanów (State space), SCC omówione będzie w następnym zagadnieniu. \\
Jak widać mamy 11 węzłów, 30 łuków, czas(jakiś bardzo krótki) i status Full. Dane te odpowiadają strukturze grafu przestrzeni stanów, który dla 5 filozofów wygląda tak:
\begin{figure}[H]
\centerline{\includegraphics[scale=0.8]{ssgraph.png}}
\caption{Graf przestrzeni stanów dla 5 filozofów}
\label{fig:sccgraph}
\end{figure}
\subsubsection{Węzły, łuki, status}
Jak widać jest na nim rzeczywiście 11 węzłów i 30 łuków (liczymy łuk dwustronny jako 2), o tym właśnie mówi statistics. Status full oznacza, że wszystkie możliwe znakowania dla tej sieci zostały osiągnięte w symulacji. \\
Pojedynczy węzeł reprezentuje unikalne znakowanie sieci, czyli np. węzeł o numerze 1 reprezentuje stan w którym:
\begin{itemize}
\item w miejscu Think jest 5 rzetonów
\item w miejscu Unused chopsticks jest 5 rzetonów
\item w miejscu PutDown nie ma ani jednego żetona
\end{itemize}
Pojedynczy węzeł zawiera u góry liczbę porządkową węzła (\underline{Uwaga: znakowanie początkowe to węzeł o numerze 1} ), w lewym dolnym rogu liczbę poprzedników, a prawym dolnym liczbę następników. 
Skąd wiadomo, że ten węzeł to właśnie takie znakowanie ? Wystarczy najechać na strzałkę w lewym dolnym rogu i dostaniemy dokładną informację:
\begin{figure}[H]
\centerline{\includegraphics[scale=0.6]{marking.png}}
\caption{Każdy węzeł odpowiada innemu, unikalnemu znakowaniu w grafie przestrzeni stanów}
\label{fig:marking}
\end{figure}

\begin{framed}
\begin{itemize}
\item Czym jest State Space (graf przestrzeni stanów) ?
\item Czym jest pojedynczy węzeł w grafie przestrzeni stanów ?
\item Co oznaczają nodes, arcs, status w raporcie dla State Space ?
\end{itemize}
\end{framed}



\subsubsection{Poprzednicy, następnicy}
Liczba następników to liczba łuków wychodzących i oznacza ona liczbę znakowań(a więc i węzłów), które możliwe są po wykonaniu tranzycji (którejkolwiek z aktywnych). W tym przypadku można np. przejść z węzła 1 do węzła 2, co oznacza, że jeden z filozofów zaczął jeść. Następnicy to zatem wszystkie możliwe stany które mogą nastąpić po wykonaniu którejkolwiek aktywnej tranzycji w danym stanie, a \textit{na grafie następnicy węzła to węzły, do których prowadzi łuk od tego węzła}.
\begin{figure}[H]
\centerline{\includegraphics[scale=0.6]{ssgraph2.png}}
\caption{Przejście z węzła 1 do 2, to zmiana znakowania po wykonaniu tranzycji.}
\label{fig:sccgraph2}
\end{figure}
Poprzednicy danego węzła oznaczają wszystkie znakowania z których osiągalny jest dany stan, a \textit{na grafie to wszystkie węzły z których prowadzi łuk do danego węzła.}
\begin{framed}
\begin{itemize}
\item Czym są następnicy ? Jak rozpoznać ich na grafie stanów ?
\item Czym są poprzednicy ? Jak rozpoznać ich na grafie stanów ?
\end{itemize}
\end{framed}


\subsubsection{Stany zakleszczone}
W raporcie uzyskujemy też, że 
\begin{figure}[H]
\centerline{\includegraphics[scale=0.6]{dead.png}}
\caption{Dead markings w raporcie dla 5 filozofów.}
\label{fig:dead}
\end{figure}

Dead markings są to oznakowania z których nie ma przejścia do innego znakowania. \textit{Na grafie stanów można je rozpoznać jako węzły, które nie mają następników.} Z jednej strony mogą to być zakleszczenia w systemie. Z drugiej jeśli nasza symulacja 'nie kręci się w kółko' to może być po prostu stan w którym się zakończyła i nie można iść już do żadnego następnego stanu. \\
Dead transition to przejścia, które nie pojawiają się w grafie stanów.

\begin{framed}
\begin{itemize}
\item Czym są dead markings (zakleszczenia) ? Jak rozpoznać je na grafie stanów ?
\end{itemize}
\end{framed}


\subsubsection{Znakowanie docelowe (home), znakowanie początkowe}
Home marking to \underline{nie jest} znakowanie początkowe ! \\
Home marking oznacza znakowanie, które jest osiągalne z jakiegokolwiek osiągalnego znakowania. Tzn. będąc w danym stanie, zawsze znajdzie się sekwencja przejść prowadząca do home marking. Dla pięciu filozofów mamy po prostu home markings all, ponieważ z dowolnego znakowania można przejść do dowolnego innego po pewnej liczbie wykonanych tranzycji. \textit{Na grafie widać to w taki sposób, że biorąc dowolny węzeł zawsze znajdziemy drogę po łukach do home marking}.
\begin{figure}[H]
\centerline{\includegraphics[scale=0.6]{home.png}}
\caption{Home markings w raporcie dla 5 filozofów.}
\label{fig:home}
\end{figure}
W raporcie nie ma czegoś takiego jak znakowanie początkowe, ale chyba chodzi po prostu o... znakowanie początkowe... i o to, że na grafie ma węzeł numer 1.

\begin{framed}
\begin{itemize}
\item Czym jest znakowanie docelowe (home marking) ? Jak rozpoznać takie znakowanie na grafie stanów ?
\item Jak rozpoznać węzeł oznakowania początkowego na grafie stanów ?
\end{itemize}
\end{framed}


\newpage
\subsection{Zagadnienie 6}
\question{6}{
Elementy raportu o wygenerowanym grafie stanów w CPNtools, statystyki
(węzły, łuki, status, graf komponentów spójnych), ograniczoność,
osiągalność, żywotność, sprawiedlwość
}

Węzły i łuki dla grafu stanów oraz status omówione zostały w zagadnieniu 5. 

\subsubsection{Graf komponentów spójnych}
Jest to swego rodzaju 'nakładka' na graf stanów taki jak na Rysunku \ref{fig:sccgraph}. Najpierw trzeba wiedzieć czym jest komponent spójny. \\
Komponentem spójnym jest maksymalny pograf w którym dowolny stan może być osiągnięty z dowolnego innego stanu. Przykład na rysunku poniżej: 
\begin{figure}[H]
\centerline{\includegraphics[scale=0.5]{scc.png}}
\caption{Przykładowy graf scc.}
\label{fig:scc}
\end{figure}
Kolorem czerwonym zaznaczone są węzły oraz łuki grafu SCC. Dla 5 filozofów otrzymano w raporcie:
\begin{figure}[H]
\centerline{\includegraphics[scale=0.5]{sccphilo.png}}
\caption{SCC raport dla philo.}
\label{fig:sccphilo}
\end{figure}

\begin{framed}
\begin{itemize}
\item Czym jest graf komponentów spójnych ? Co oznaczają węzły w tym grafie ?
\item Co oznaczają arcs, nodes dla grafu SCC w raporcie ?
\end{itemize}
\end{framed}


\subsubsection{Ograniczoność}
\textbf{Integer Bounds} to wskaźnik określający liczbę rzetonów w danym miejscu. Wskaźnik ten wyrażany jest liczbą całkowitą.
\begin{itemize}
\item Best upper integer bound - to maksymalna liczba rzetonów w danym miejscu w pewnym osiągalnym znakowaniu
\item Best lower integer bound - to minimalna liczba rzetonów w danym miejscu w pewnym osiągalnym znakowaniu
\end{itemize} 

\textbf{Multiset Bounds} to wskaźnik określający liczbę wystąpień danego (konkretnego) rzetona w danym miejscu. Wskaźnik ten wyrażany jest przez wielozbiór.
\begin{itemize}
\item Best upper multiset bounds - określa ile maksymalnie konkretnych rzetonów może znaleźć się w danym miejscu podczas działania systemu 
\item Best lower multiset bounds - określa ile minimalnie konkretnych rzetonów może znaleźć się w danym miejscu podczas działania systemu 
\end{itemize} 

W przykładzie poniżej mamy np. że jeśli weźmiemy rzeton ph(1) (filozof numer 1), to może być maksymalnie jeden taki w każdym z miejsc podczas symulacji (informuje o tym liczba 1 przed ph(1)) - ważne są współczynniki przed elementami wielozbioru. \\
Również biorąc ten sam żeton widzimy, że minimalna liczba jego wystąpień w danym miejscu podczas symulacji wynosi 0 (empty).


\begin{figure}[H]
\centerline{\includegraphics[scale=0.5]{raport_bound.png}}
\caption{Właściwości ograniczoności dla 5 filozofów w raporcie.}
\label{fig:raport_bound}
\end{figure}

\begin{framed}
\begin{itemize}
\item Czym są integer bounds ? Co oznaca Best upper integer bound, a co best lower integer bound ?
\item Czym są multiset bounds ? Co oznacza best upper multiset bounds, a co lower multiset bounds ?
\end{itemize}
\end{framed}



\subsubsection{Osiągalność}
Raport \underline{nie zawiera} informacji nt. osiągalności (czyli czy z danego węzła można dojść po łukach do innego), ale można ją łatwo sprawdzić korzystając z grafu komponentów spójnych sprawdzając czy istnieje ścieżka pomiędzy węzłem zawierającym znakowanie startowe i końcowe. Jeśli natomiast w statystykach wygeneruje się, że graf SCC ma więcej niż jeden węzeł oznacza to, że nie wszystkie stany są wzajemnie osiągalne, czyli nie prawdą jest, że biorąc dowolne dwa stany można dojść z jednego do drugiego i na odwrót. \\
Jeśli graf SCC ma 1 węzeł (jak w przypadku filozofów), to wtedy wszystkie znakowania są wzajemnie osiągalne.

\begin{framed}
\begin{itemize}
\item Czym jest osiągalność ? Jak sprawdzić czy wszystkie stany są wzajemnie osiągalne ?
\end{itemize}
\end{framed}


\subsubsection{Żywotność}
\textbf{Dead marking} to znakowanie przy którym nie ma aktywnego ani jednego przejścia. W raporcie jest ono oznaczane liczbami np: [4356], wtedy oznacza to, że dead marking jest reprezentowane w grafie stanów przez węzeł numer 4356.\\
\textbf{Dead transitions} to tranzycja, która nigdy nie jest aktywna \\
\textbf{Live tansition} to tranzycja dla której wybierając dowolne osiągalne oznakowanie można znaleźć taki ciąg przejść, że zawarta w nim będzie ta tranzycja. Jeśli mamy choćby jedno dead marking, to na pewno live transition będzie miało wartość None. 

\begin{figure}[H]
\centerline{\includegraphics[scale=0.5]{raport_live.png}}
\caption{Właściwości żywotności dla 5 filozofów w raporcie.}
\label{fig:raport_bound}
\end{figure}

\begin{framed}
\begin{itemize}
\item Czym jest dead marking ? Jak jest oznaczane ?
\item Czym jest dead transition ?
\item Czym jest live transition ?
\end{itemize}
\end{framed}

\subsubsection{Sprawiedliwość}
\textbf{Impartial transition} to tranzycja, która pojawia się w \underline{każdej} nieskończonej sekwencji wykonywania tranzycji nieskończenie wiele razy. \\
\textbf{Fair Transtion Instances} to tranzycje, które pojawiają się nieskończenie wiele razy, ale tylko w niektórych sekwencjach wykonywania tranzycji (...czyli tylko w tych w których występuje nieskończenie wiele razy) \\
\textbf{Just Transition Instances} to tranzycje, które pojawiają się nieskończenie wiele razy, ale tylko w tych sekwencjach wykonywania tranzycji w których są ciągle aktywne od pewnego momentu \\
\textbf{No fairness} przeciwieństwo just, czyli tranzycja dla której istnieje taka sekwencja wykonywania przejść, że tranzycja t jest aktywna od pewnego momentu w czasie, ale nie jest nigdy więcej wykonywana \\


\begin{framed}
\begin{itemize}
\item Czym jest impartial transition ?
\item Czym jest fair transition instances ?
\item Czym jest just tranistion instances ?
\item Czym jest no fairness ?
\end{itemize}
\end{framed}

 
Przykłady:
\begin{figure}[H]
\centerline{\includegraphics[scale=0.5]{raport_live.png}}
\caption{Właściwości żywotności dla 5 filozofów w raporcie.}
\label{fig:raport_bound}
\end{figure}


\begin{figure}[H]
\centerline{\includegraphics[scale=0.5]{ex_1fair.png}}
\caption{Przykłady własności sprawiedliwości.}
\label{fig:ex_1fair}
\end{figure}

\begin{figure}[H]
\centerline{\includegraphics[scale=0.5]{ex_2fair.png}}
\caption{Przykłady własności sprawiedliwości.}
\label{fig:ex_2fair}
\end{figure}

\newpage
\subsection{Zagadnienie 7}
\question{7}{
Modelowanie współdzielenia zasobów w kolorowanej sieci Petriego
}

(?) Np. coś jak resource allocation, miejsca mają żetony, które reprezentują zasoby. (?)
\begin{figure}[H]
\centerline{\includegraphics[scale=0.8]{resource.png}}
\caption{Przykład współdzielenie zasobów przez kilka wątków w CPN.}
\label{fig:resource}
\end{figure}

\newpage
\subsection{Zagadnienie 8}
\question{8}{
Indeterminizm w sieciach Petriego
}
(?) Jeśli mamy kilka aktywnych przejść, to nie wiemy, które z nich się wykona. (?)
\begin{figure}[H]
\centerline{\includegraphics[scale=0.5]{inder.png}}
\caption{Przykład indeterminizmu w sieciach Petriego.}
\label{fig:inder}
\end{figure}

\newpage
\subsection{Zagadnienie 9}
\question{9}{
Synchronizacja procesów współbieżnych w sieciach Petriego
}

\begin{figure}[H]
\centerline{\includegraphics[scale=0.5]{synch.png}}
\caption{Przykład synchronizacji przez miejsce konfliktowe.}
\label{fig:synch}
\end{figure}

\begin{framed}
\begin{itemize}
\item Zamodelować dwa procesy z synchronizacją.
\end{itemize}
\end{framed}



\newpage
\subsection{Różne inne}

\subsection{Sygnalizacja świetlna bez skrętu}
\begin{figure}[H]
\centerline{\includegraphics[scale=0.8]{tf0.png}}
\caption{Sygnalizacja świetlna bez skrętu (np. na zwężeniu mostu)}
\label{fig:tf1}
\end{figure}


\begin{framed}
\begin{itemize}
\item Zamodelować sygnalizację bez skrętu.
\end{itemize}
\end{framed}

\newpage
\subsection{Sygnalizacja świetlna ze skrętem}
\begin{figure}[H]
\centerline{\includegraphics[scale=0.65]{tf1.png}}
\caption{Zjawisko do zamodelowania.}
\label{fig:tf1}
\end{figure}
\begin{figure}[H]
\centerline{\includegraphics[scale=0.8]{tf2.png}}
\caption{Model skrzyżowania ze skrętem.}
\label{fig:tf2}
\end{figure}


\begin{framed}
\begin{itemize}
\item Zamodelować sygnalizację ze skrętem.
\end{itemize}
\end{framed}


\newpage
\subsubsection{Zadanka z kolosa 2018}
\begin{framed}
\begin{itemize}
\item Zaproponować sieć Petriego, która po odpaleniu kilku przejść wchodzi w stan zakleszczenia.
\end{itemize}
\end{framed}

Np. dwóch filozofów:
\begin{figure}[H]
\centerline{\includegraphics[scale=0.8]{twophilo.png}}
\caption{Przykład sieci Petriego, gdzie będzie deadlock.}
\label{fig:twophilo}
\end{figure}

\begin{framed}
\begin{itemize}
\item Zaprojektować sieć Petriego, w której zachodzi wykluczanie jednoczesnej aktywności wybranej pary przejść.
\end{itemize}
\end{framed}

Np. tak jak w zagadnieniu 9, gdzie wykluczają się t2 i t4.

\end{document}
