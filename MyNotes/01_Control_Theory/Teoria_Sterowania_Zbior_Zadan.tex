\documentclass[a4paper,11pt]{article}
\usepackage{algorithm}
\usepackage{algpseudocode}
\usepackage{amssymb}
\usepackage{amsthm}
\usepackage{amsmath}
\usepackage[english, polish]{babel}
\usepackage[utf8]{inputenc}   % lub utf8
\usepackage[T1]{fontenc}
\usepackage{graphicx}
\usepackage{anysize}
\usepackage{enumerate}
\usepackage{times}
\usepackage{xcolor}
\usepackage{titlesec}
\usepackage{float}
\usepackage[justification=centering]{caption}
\titlelabel{\thetitle.\quad}
\usepackage{titlesec}
\usepackage{titleps,kantlipsum}
\usepackage{tikz}
\usepackage{color}
\usepackage{listings}
\usepackage{caption}
\lstloadlanguages{Matlab,C++}
\usepackage{hyperref}
\usepackage{framed}
\usepackage{siunitx}
\usepackage{mathrsfs}
\usepackage{cancel}

\usetikzlibrary{calc,through,backgrounds,positioning,fit}
\usetikzlibrary{shapes,arrows,shadows,patterns}

\tikzstyle{place}=[shape=circle, draw, minimum height=10mm]
\tikzstyle{place_1}=[shape=circle, draw, minimum height=5mm]
\tikzstyle{trig}=[shape=circle, draw, dashed, minimum height=10mm]
\tikzstyle{trans}=[shape=rectangle, draw, minimum height=15mm, minimum width=16mm]

\newdimen\LineSpace
\tikzset{
    line space/.code={\LineSpace=#1},
    line space=3pt
}

\pgfdeclarepatternformonly[\LineSpace]{my north east lines}{\pgfqpoint{-1pt}{-1pt}}{\pgfqpoint{\LineSpace}{\LineSpace}}{\pgfqpoint{\LineSpace}{\LineSpace}}%
{
    \pgfsetlinewidth{0.4pt}
    \pgfpathmoveto{\pgfqpoint{0pt}{0pt}}
    \pgfpathlineto{\pgfqpoint{\LineSpace + 0.1pt}{\LineSpace + 0.1pt}}
    \pgfusepath{stroke}
}

\newpagestyle{mypage}
{
  \headrule
  
  \sethead
  { \MakeUppercase{\thesection\quad \sectiontitle} } 
  {}
  {\thesubsection\quad \subsectiontitle}
  
  \setfoot
  {}
  {\thepage}
  {}
}

\newpagestyle{mypage_1}
{
	\headrule
	
	\sethead
	{  }
	{\MakeUppercase{Teoria Sterowania - Zbiór Zadań}}
	{}
	
	\setfoot
	{}
	{}
	{}
}

\pagestyle{mypage_1}
%\marginsize{left}{right}{top}{bottom}
\marginsize{3cm}{3cm}{3cm}{3cm}
\sloppy
 
\begin{document}

\tableofcontents

\newpage

\textbf{Zadania pochodzą z czterech źródeł}:
\begin{itemize}
\item Zadania wysłane do AiR przed kolokwium I i II przez Bauera
\item Zadania z egzaminów z doku
\item Zadania z niebieskiego skryptu
\item Zadania z przedmiotu MSD od informatyków
\end{itemize}

Nie ma żadnej gwarancji, że rozwiązania są poprawne, ale wiele z nich zostało już przedyskutowane i poprawione, niemniej jednak, należy zachować krytyczne spojrzenie na rozwiązania zadań. \\
Wersja rozwojowa w \LaTeX . W celu utrzymania porządku edycja zgodnie z przyjętym w opracowaniu schematem. Zadania pasujące do więcej niż jednego działu zostały dodane tylko w jednym z nich w celu utrzymania porządku więc czasem trzeba poszukać. \\
\textbf{Legenda}:
\begin{itemize}
\item \colorbox{green}{Kolor zielony w zbiorze zadań} oznacza, że zadanie ma rozwiązanie w sekcji z rozwiązaniami
\item \colorbox{yellow}{Kolor żółty w zbiorze zadań} oznacza, że zadanie ma niepełne rozwiązanie lub jest pozostawione na pewnym etapie w celu dorobienia czegoś albo z powodu braku pomysłu na dalsze rozwiązanie. 
\end{itemize}

\newpage
\pagestyle{mypage}
\section{Portrety fazowe}
\subsection{Zbiór zadań}
\begin{framed}
\textbf{\colorbox{green}{Zadanie 1 - Egzamin} } \\ 
Naszkicuj portret fazowy układu
\begin{align*}
&\dot{x}(t)=A_{i}x(t) \quad
i = 0,1 \\
&A_{0} = 
\begin{bmatrix}
0 & 0 \\
1 & 0
\end{bmatrix}
\quad
A_{1} = 
\begin{bmatrix}
0 & 1 \\
-1 & 0
\end{bmatrix}
\end{align*}
\end{framed}

\begin{framed}
\textbf{\colorbox{green}{Zadanie 2 - Egzamin} } \\ 
Narysować portret fazowy układu
\begin{align*}
&\dot{x}(t)=A_{i}x(t) \quad
i = 0,1 \\
&A_{0} = 
\begin{bmatrix}
0 & 1 \\
-1 & 0
\end{bmatrix}
\quad
A_{1} = 
\begin{bmatrix}
0 & 10 \\
-10 & 0
\end{bmatrix}
\end{align*}
Podać macierze \( e^{A_{i}}t \), napisać czym różnią się portrety fazowe powyższych systemów.
\end{framed}

\begin{framed}
\textbf{\colorbox{green}{Zadanie 3 - Egzamin} } \\ 
Naszkicować portret fazowy układu
\begin{align*}
&\dot{x}_{1}(t)=x_{2}(t) \\
&\dot{x}_{2}(t)=ax_{1}(t)
\end{align*}
\end{framed}

\begin{framed}
\textbf{\colorbox{green}{ 4 - Zadania od Bauera do kolokwium I }} \\ 
Naszkicować portrety fazowe systemów dynamicznych i opisać czym się różnią
\begin{align*}
&\dot{x}_{1}(t)=-x_{1}(t)+x_{2}(t) \quad &\dot{x}_{1}(t)=x_{1}(t)+x_{2}(t) \\
&\dot{x}_{2}(t)=-x_{2}(t) \quad &\dot{x}_{2}(t)=-x_{2}(t) 
\end{align*}
i opisać czym się różnią.
\end{framed}


\begin{framed}
\textbf{\colorbox{yellow}{Zadanie 5 - Egzamin}}  \\ 
Dany jest układ:
\begin{align*}
\dot{x}_{1}(t)=-x_2+x_1x_2 \\
\dot{x}_{2}(t)=x_1+\frac{1}{2}(x_1^2-x_2^2)
\end{align*}
Narysować portrety fazowe systemu zlinearyzowanego w punktach równowagi.
\end{framed}

\newpage
\subsection{Rozwiązania zadań ze zbioru}
\begin{framed}
\textbf{Zadanie 1 - Egzamin } \\ 
Naszkicuj portret fazowy układu
\begin{align*}
&\dot{x}(t)=A_{i}x(t) \quad
i = 0,1 \\
&A_{0} = 
\begin{bmatrix}
0 & 0 \\
1 & 0
\end{bmatrix}
\quad
A_{1} = 
\begin{bmatrix}
0 & 1 \\
-1 & 0
\end{bmatrix}
\end{align*}
\end{framed}
Macierz \( A_{0} \) jest w postaci normalnej, obliczam jej wartości własne:
\begin{align*}
&det (\lambda I-A)=
det \begin{bmatrix}
\lambda & 0 \\
-1 & \lambda
\end{bmatrix} =
\lambda^{2} \\
&\lambda ^{2} = 0 \\
&\lambda _{1} = \lambda _{2} = 0
\end{align*}
Każdy punkt płaszczyzny jest punktem równowagi oraz trajektorią fazową. \\
Macierz \( A_{1} \) jest w postaci kanonicznej Jordana i ma wartości własne \( \lambda = \pm i \). Z tabelki odczytuję, że dla takich wartości własnych portret fazowy to 'środek'. Kierunek strzałek obliczam przez pomnożenia dowolnego punktu przez macierz A.
 
\begin{figure}[!h]
\begin{center}
\begin{tikzpicture}
	\draw[color=blue] (0,0) circle(.5);
	\draw[color=blue] (0,0) circle(1);
	\draw[color=blue] (0,0) circle(1.5);
	\draw[color=blue] (0,0) circle(2);
	\draw[color=blue] (0,0) circle(2.5);

	\draw[color=blue,very thick][<-](-.3,-.45)--(-.1,-.5);
	\draw[color=blue,very thick][<-](-.3,-.95)--(-.1,-1);
	\draw[color=blue,very thick][<-](-.3,-1.45)--(-.1,-1.5);
	\draw[color=blue,very thick][<-](-.3,-1.95)--(-.1,-2);
	\draw[color=blue,very thick][<-](-.3,-2.45)--(-.1,-2.5);

	\draw[very thick][->](-3,0)--(3,0) node[right=.2] {$x_1$};
	\draw[very thick][->](0,-2.625)--(0,2.75) node[above=.2] {$x_2$};
\end{tikzpicture}
\end{center}
\end{figure}

\newpage
\begin{framed}
\textbf{Zadanie 2 - Egzamin } \\ 
Narysować portret fazowy układu
\begin{align*}
&\dot{x}(t)=A_{i}x(t) \quad
i = 0,1 \\
&A_{0} = 
\begin{bmatrix}
0 & 1 \\
-1 & 0
\end{bmatrix}
\quad
A_{1} = 
\begin{bmatrix}
0 & 10 \\
-10 & 0
\end{bmatrix}
\end{align*}
Podać macierze \( e^{A_{i}}t \), napisać czym różnią się portrety fazowe powyższych systemów.
\end{framed}
Obie macierze są w kanonicznych postaciach Jordana z zespolonymi wartościami własnymi. Dla macierzy \( A_{0} \) są to \( \lambda = \pm i \), dla macierzy \( A_{1} \) są to \( \lambda = \pm 10i \). Z tabelki odczytuję, że w tym przypadku portretem fazowym jest środek ( pierwiastki urojone, sprzężone, o zerowych częściach rzeczywistych ). 
Kierunek strzałek wyznaczam mnożąc dowolny punkt przez macierz A.
\begin{figure}[!h]
\begin{tikzpicture}
	\draw[color=blue] (0,0) circle(.5);
	\draw[color=blue] (0,0) circle(1);
	\draw[color=blue] (0,0) circle(1.5);
	\draw[color=blue] (0,0) circle(2);
	\draw[color=blue] (0,0) circle(2.5);

	\draw[color=blue,very thick][<-](-.3,-.45)--(-.1,-.5);
	\draw[color=blue,very thick][<-](-.3,-.95)--(-.1,-1);
	\draw[color=blue,very thick][<-](-.3,-1.45)--(-.1,-1.5);
	\draw[color=blue,very thick][<-](-.3,-1.95)--(-.1,-2);
	\draw[color=blue,very thick][<-](-.3,-2.45)--(-.1,-2.5);

	\draw[very thick][->](-3,0)--(3,0) node[right=.2] {$x_1$};
	\draw[very thick][->](0,-2.625)--(0,2.75) node[above=.2] {$x_2$};
\end{tikzpicture}
\hspace*{2cm}
\begin{tikzpicture}
	\draw[color=blue] (0,0) circle(.5);
	\draw[color=blue] (0,0) circle(1);
	\draw[color=blue] (0,0) circle(1.5);
	\draw[color=blue] (0,0) circle(2);
	\draw[color=blue] (0,0) circle(2.5);

	\draw[color=blue,very thick][<-](-.3,-.45)--(-.1,-.5);
	\draw[color=blue,very thick][<-](-.3,-.95)--(-.1,-1);
	\draw[color=blue,very thick][<-](-.3,-1.45)--(-.1,-1.5);
	\draw[color=blue,very thick][<-](-.3,-1.95)--(-.1,-2);
	\draw[color=blue,very thick][<-](-.3,-2.45)--(-.1,-2.5);

	\draw[very thick][->](-3,0)--(3,0) node[right=.2] {$x_1$};
	\draw[very thick][->](0,-2.625)--(0,2.75) node[above=.2] {$x_2$};
\end{tikzpicture}
\end{figure}
Macierze \( e^{A_{i}t} \) wyznaczam ze wzoru na \( e^{Jt} \) dla klatki zespolonej.
\begin{align*}
e^{A_{0}t}=e^{at}
\begin{bmatrix}
\cos bt & \sin bt \\
-\sin bt & \cos bt
\end{bmatrix}
=
e^{0}
\begin{bmatrix}
\cos t & \sin t \\
-\sin t & \cos t
\end{bmatrix}
=
\begin{bmatrix}
\cos t & \sin t \\
-\sin t & \cos t
\end{bmatrix} \\
e^{A_{1}t}=e^{at}
\begin{bmatrix}
\cos bt & \sin bt \\
-\sin bt & \cos bt
\end{bmatrix}
=
e^{0}
\begin{bmatrix}
\cos 10t & \sin 10t \\
-\sin 10t & \cos 10t
\end{bmatrix}
=
\begin{bmatrix}
\cos 10t & \sin 10t \\
-\sin 10t & \cos 10t
\end{bmatrix}
\end{align*}
Portrety fazowe są takie same, różnicą jest szybkość zmian trajektorii w dziedzinie czasu, co widać na podstawie macierzy \( e^{At} \). 

\newpage
\begin{framed}
\textbf{Zadanie 3 - Egzamin } \\ 
Naszkicować portret fazowy układu
\begin{align*}
&\dot{x}_{1}(t)=x_{2}(t) \\
&\dot{x}_{2}(t)=ax_{1}(t)
\end{align*}
\end{framed}
Macierz A:
\begin{align*}
A = 
\begin{bmatrix}
0 & 1 \\
a & 0
\end{bmatrix}
\end{align*}
Wartości własne:
\begin{align*}
det ( \lambda I - A ) = det 
\begin{bmatrix}
\lambda & -1 \\
-a & \lambda
\end{bmatrix}
=
\lambda ^{2} - a
\end{align*}
I przypadek a = 0
\begin{align*}
\lambda _{1} = \lambda _{2} = 0
\end{align*}
Wtedy każdy punkt płaszczyzny jest punktem równowagi oraz trajektorią fazową. 
\\ 
II przypadek a > 0
\begin{align*}
det ( \lambda I - A ) = ( \lambda - \sqrt{a} )( \lambda + \sqrt{a} ) 
\end{align*}
Dwie wartości własne rzeczywiste o przeciwnych znakach - siodło, osie zależą od wektorów własnych, które zależą od wartości a. Strzałki idą do nieskończoności przy wektorze związanym z dodatnią wartością \( \lambda \). 
\begin{figure}[H]
\begin{center}
\begin{tikzpicture}
\draw [color=blue](-3.0,2.96)--(-2.93,2.9)--(-2.87,2.84)--(-2.81,2.77)--(-2.75,2.71)--(-2.68,2.65)--(-2.62,2.58)--(-2.56,2.52)--(-2.5,2.45)--(-2.43,2.39)--(-2.37,2.33)--(-2.31,2.26)--(-2.25,2.2)--(-2.18,2.14)--(-2.12,2.07)--(-2.06,2.01)--(-2.0,1.94)--(-1.93,1.88)--(-1.87,1.82)--(-1.81,1.75)--(-1.75,1.69)--(-1.68,1.62)--(-1.62,1.56)--(-1.56,1.49)--(-1.5,1.43)--(-1.43,1.36)--(-1.37,1.3)--(-1.31,1.23)--(-1.25,1.16)--(-1.18,1.1)--(-1.12,1.03)--(-1.06,0.96)--(-1.0,0.89)--(-0.93,0.82)--(-0.87,0.75)--(-0.81,0.67)--(-0.75,0.6)--(-0.68,0.52)--(-0.62,0.43)--(-0.56,0.34)--(-0.5,0.22)--(-0.5,-0.22)--(-0.56,-0.34)--(-0.62,-0.43)--(-0.68,-0.52)--(-0.75,-0.6)--(-0.81,-0.67)--(-0.87,-0.75)--(-0.93,-0.82)--(-1.0,-0.89)--(-1.06,-0.96)--(-1.12,-1.03)--(-1.18,-1.1)--(-1.25,-1.16)--(-1.31,-1.23)--(-1.37,-1.3)--(-1.43,-1.36)--(-1.5,-1.43)--(-1.56,-1.49)--(-1.62,-1.56)--(-1.68,-1.62)--(-1.75,-1.69)--(-1.81,-1.75)--(-1.87,-1.82)--(-1.93,-1.88)--(-2.0,-1.94)--(-2.06,-2.01)--(-2.12,-2.07)--(-2.18,-2.14)--(-2.25,-2.2)--(-2.31,-2.26)--(-2.37,-2.33)--(-2.43,-2.39)--(-2.5,-2.45)--(-2.56,-2.52)--(-2.62,-2.58)--(-2.68,-2.65)--(-2.75,-2.71)--(-2.81,-2.77)--(-2.87,-2.84)--(-2.93,-2.9)--(-3.0,-2.96);
\draw [color=blue](3.0,2.96)--(2.93,2.9)--(2.87,2.84)--(2.81,2.77)--(2.75,2.71)--(2.68,2.65)--(2.62,2.58)--(2.56,2.52)--(2.5,2.45)--(2.43,2.39)--(2.37,2.33)--(2.31,2.26)--(2.25,2.2)--(2.18,2.14)--(2.12,2.07)--(2.06,2.01)--(2.0,1.94)--(1.93,1.88)--(1.87,1.82)--(1.81,1.75)--(1.75,1.69)--(1.68,1.62)--(1.62,1.56)--(1.56,1.49)--(1.5,1.43)--(1.43,1.36)--(1.37,1.3)--(1.31,1.23)--(1.25,1.16)--(1.18,1.1)--(1.12,1.03)--(1.06,0.96)--(1.0,0.89)--(0.93,0.82)--(0.87,0.75)--(0.81,0.67)--(0.75,0.6)--(0.68,0.52)--(0.62,0.43)--(0.56,0.34)--(0.5,0.22)--(0.5,-0.22)--(0.56,-0.34)--(0.62,-0.43)--(0.68,-0.52)--(0.75,-0.6)--(0.81,-0.67)--(0.87,-0.75)--(0.93,-0.82)--(1.0,-0.89)--(1.06,-0.96)--(1.12,-1.03)--(1.18,-1.1)--(1.25,-1.16)--(1.31,-1.23)--(1.37,-1.3)--(1.43,-1.36)--(1.5,-1.43)--(1.56,-1.49)--(1.62,-1.56)--(1.68,-1.62)--(1.75,-1.69)--(1.81,-1.75)--(1.87,-1.82)--(1.93,-1.88)--(2.0,-1.94)--(2.06,-2.01)--(2.12,-2.07)--(2.18,-2.14)--(2.25,-2.2)--(2.31,-2.26)--(2.37,-2.33)--(2.43,-2.39)--(2.5,-2.45)--(2.56,-2.52)--(2.62,-2.58)--(2.68,-2.65)--(2.75,-2.71)--(2.81,-2.77)--(2.87,-2.84)--(2.93,-2.9)--(3.0,-2.96);
\draw [color=blue](-2.93,2.97)--(-2.87,2.9)--(-2.81,2.84)--(-2.75,2.78)--(-2.68,2.72)--(-2.62,2.66)--(-2.56,2.6)--(-2.5,2.53)--(-2.43,2.47)--(-2.37,2.41)--(-2.31,2.35)--(-2.25,2.29)--(-2.18,2.23)--(-2.12,2.17)--(-2.06,2.11)--(-2.0,2.04)--(-1.93,1.98)--(-1.87,1.92)--(-1.81,1.86)--(-1.75,1.8)--(-1.68,1.74)--(-1.62,1.68)--(-1.56,1.62)--(-1.5,1.56)--(-1.43,1.5)--(-1.37,1.44)--(-1.31,1.38)--(-1.25,1.32)--(-1.18,1.26)--(-1.12,1.21)--(-1.06,1.15)--(-1.0,1.09)--(-0.93,1.03)--(-0.87,0.98)--(-0.81,0.92)--(-0.75,0.87)--(-0.68,0.82)--(-0.62,0.76)--(-0.56,0.71)--(-0.5,0.67)--(-0.43,0.62)--(-0.37,0.58)--(-0.31,0.54)--(-0.25,0.51)--(-0.18,0.48)--(-0.12,0.46)--(-0.06,0.45)--(0.0,0.44)--(0.06,0.45)--(0.12,0.46)--(0.18,0.48)--(0.25,0.51)--(0.31,0.54)--(0.37,0.58)--(0.43,0.62)--(0.5,0.67)--(0.56,0.71)--(0.62,0.76)--(0.68,0.82)--(0.75,0.87)--(0.81,0.92)--(0.87,0.98)--(0.93,1.03)--(1.0,1.09)--(1.06,1.15)--(1.12,1.21)--(1.18,1.26)--(1.25,1.32)--(1.31,1.38)--(1.37,1.44)--(1.43,1.5)--(1.5,1.56)--(1.56,1.62)--(1.62,1.68)--(1.68,1.74)--(1.75,1.8)--(1.81,1.86)--(1.87,1.92)--(1.93,1.98)--(2.0,2.04)--(2.06,2.11)--(2.12,2.17)--(2.18,2.23)--(2.25,2.29)--(2.31,2.35)--(2.37,2.41)--(2.43,2.47)--(2.5,2.53)--(2.56,2.6)--(2.62,2.66)--(2.68,2.72)--(2.75,2.78)--(2.81,2.84)--(2.87,2.9)--(2.93,2.97);
\draw [color=blue](-2.93,-2.97)--(-2.87,-2.9)--(-2.81,-2.84)--(-2.75,-2.78)--(-2.68,-2.72)--(-2.62,-2.66)--(-2.56,-2.6)--(-2.5,-2.53)--(-2.43,-2.47)--(-2.37,-2.41)--(-2.31,-2.35)--(-2.25,-2.29)--(-2.18,-2.23)--(-2.12,-2.17)--(-2.06,-2.11)--(-2.0,-2.04)--(-1.93,-1.98)--(-1.87,-1.92)--(-1.81,-1.86)--(-1.75,-1.8)--(-1.68,-1.74)--(-1.62,-1.68)--(-1.56,-1.62)--(-1.5,-1.56)--(-1.43,-1.5)--(-1.37,-1.44)--(-1.31,-1.38)--(-1.25,-1.32)--(-1.18,-1.26)--(-1.12,-1.21)--(-1.06,-1.15)--(-1.0,-1.09)--(-0.93,-1.03)--(-0.87,-0.98)--(-0.81,-0.92)--(-0.75,-0.87)--(-0.68,-0.82)--(-0.62,-0.76)--(-0.56,-0.71)--(-0.5,-0.67)--(-0.43,-0.62)--(-0.37,-0.58)--(-0.31,-0.54)--(-0.25,-0.51)--(-0.18,-0.48)--(-0.12,-0.46)--(-0.06,-0.45)--(0.0,-0.44)--(0.06,-0.45)--(0.12,-0.46)--(0.18,-0.48)--(0.25,-0.51)--(0.31,-0.54)--(0.37,-0.58)--(0.43,-0.62)--(0.5,-0.67)--(0.56,-0.71)--(0.62,-0.76)--(0.68,-0.82)--(0.75,-0.87)--(0.81,-0.92)--(0.87,-0.98)--(0.93,-1.03)--(1.0,-1.09)--(1.06,-1.15)--(1.12,-1.21)--(1.18,-1.26)--(1.25,-1.32)--(1.31,-1.38)--(1.37,-1.44)--(1.43,-1.5)--(1.5,-1.56)--(1.56,-1.62)--(1.62,-1.68)--(1.68,-1.74)--(1.75,-1.8)--(1.81,-1.86)--(1.87,-1.92)--(1.93,-1.98)--(2.0,-2.04)--(2.06,-2.11)--(2.12,-2.17)--(2.18,-2.23)--(2.25,-2.29)--(2.31,-2.35)--(2.37,-2.41)--(2.43,-2.47)--(2.5,-2.53)--(2.56,-2.6)--(2.62,-2.66)--(2.68,-2.72)--(2.75,-2.78)--(2.81,-2.84)--(2.87,-2.9)--(2.93,-2.97);

\draw [color=blue](-3.0,2.82)--(-2.93,2.76)--(-2.87,2.69)--(-2.81,2.62)--(-2.75,2.56)--(-2.68,2.49)--(-2.62,2.42)--(-2.56,2.35)--(-2.5,2.29)--(-2.43,2.22)--(-2.37,2.15)--(-2.31,2.08)--(-2.25,2.01)--(-2.18,1.94)--(-2.12,1.87)--(-2.06,1.8)--(-2.0,1.73)--(-1.93,1.65)--(-1.87,1.58)--(-1.81,1.51)--(-1.75,1.43)--(-1.68,1.35)--(-1.62,1.28)--(-1.56,1.2)--(-1.5,1.11)--(-1.43,1.03)--(-1.37,0.94)--(-1.31,0.85)--(-1.25,0.75)--(-1.18,0.64)--(-1.12,0.51)--(-1.06,0.35)--(-1.0,0.0)--(-0.93,0.0)--(-0.87,0.0)--(-0.81,0.0)--(-0.75,0.0)--(-0.68,0.0)--(-0.62,0.0)--(-0.56,0.0)--(-0.5,0.0)--(-0.43,0.0)--(-0.37,0.0)--(-0.31,0.0)--(-0.25,0.0)--(-0.18,0.0)--(-0.12,0.0)--(-0.06,0.0)--(0.0,0.0)--(0.06,0.0)--(0.12,0.0)--(0.18,0.0)--(0.25,0.0)--(0.31,0.0)--(0.37,0.0)--(0.43,0.0)--(0.5,0.0)--(0.56,0.0)--(0.62,0.0)--(0.68,0.0)--(0.75,0.0)--(0.81,0.0)--(0.87,0.0)--(0.93,0.0)--(1.0,0.0)--(1.06,0.35)--(1.12,0.51)--(1.18,0.64)--(1.25,0.75)--(1.31,0.85)--(1.37,0.94)--(1.43,1.03)--(1.5,1.11)--(1.56,1.2)--(1.62,1.28)--(1.68,1.35)--(1.75,1.43)--(1.81,1.51)--(1.87,1.58)--(1.93,1.65)--(2.0,1.73)--(2.06,1.8)--(2.12,1.87)--(2.18,1.94)--(2.25,2.01)--(2.31,2.08)--(2.37,2.15)--(2.43,2.22)--(2.5,2.29)--(2.56,2.35)--(2.62,2.42)--(2.68,2.49)--(2.75,2.56)--(2.81,2.62)--(2.87,2.69)--(2.93,2.76)--(3.0,2.82);
\draw [color=blue](-3.0,-2.82)--(-2.93,-2.76)--(-2.87,-2.69)--(-2.81,-2.62)--(-2.75,-2.56)--(-2.68,-2.49)--(-2.62,-2.42)--(-2.56,-2.35)--(-2.5,-2.29)--(-2.43,-2.22)--(-2.37,-2.15)--(-2.31,-2.08)--(-2.25,-2.01)--(-2.18,-1.94)--(-2.12,-1.87)--(-2.06,-1.8)--(-2.0,-1.73)--(-1.93,-1.65)--(-1.87,-1.58)--(-1.81,-1.51)--(-1.75,-1.43)--(-1.68,-1.35)--(-1.62,-1.28)--(-1.56,-1.2)--(-1.5,-1.11)--(-1.43,-1.03)--(-1.37,-0.94)--(-1.31,-0.85)--(-1.25,-0.75)--(-1.18,-0.64)--(-1.12,-0.51)--(-1.06,-0.35)--(-1.0,0.0)--(-0.93,0.0)--(-0.87,0.0)--(-0.81,0.0)--(-0.75,0.0)--(-0.68,0.0)--(-0.62,0.0)--(-0.56,0.0)--(-0.5,0.0)--(-0.43,0.0)--(-0.37,0.0)--(-0.31,0.0)--(-0.25,0.0)--(-0.18,0.0)--(-0.12,0.0)--(-0.06,0.0)--(0.0,0.0)--(0.06,0.0)--(0.12,0.0)--(0.18,0.0)--(0.25,0.0)--(0.31,0.0)--(0.37,0.0)--(0.43,0.0)--(0.5,0.0)--(0.56,0.0)--(0.62,0.0)--(0.68,0.0)--(0.75,0.0)--(0.81,0.0)--(0.87,0.0)--(0.93,0.0)--(1.0,0.0)--(1.06,-0.35)--(1.12,-0.51)--(1.18,-0.64)--(1.25,-0.75)--(1.31,-0.85)--(1.37,-0.94)--(1.43,-1.03)--(1.5,-1.11)--(1.56,-1.2)--(1.62,-1.28)--(1.68,-1.35)--(1.75,-1.43)--(1.81,-1.51)--(1.87,-1.58)--(1.93,-1.65)--(2.0,-1.73)--(2.06,-1.8)--(2.12,-1.87)--(2.18,-1.94)--(2.25,-2.01)--(2.31,-2.08)--(2.37,-2.15)--(2.43,-2.22)--(2.5,-2.29)--(2.56,-2.35)--(2.62,-2.42)--(2.68,-2.49)--(2.75,-2.56)--(2.81,-2.62)--(2.87,-2.69)--(2.93,-2.76)--(3.0,-2.82);
\draw [color=blue](-3.0,3.16)--(-2.93,3.1)--(-2.87,3.04)--(-2.81,2.98)--(-2.75,2.92)--(-2.68,2.86)--(-2.62,2.8)--(-2.56,2.75)--(-2.5,2.69)--(-2.43,2.63)--(-2.37,2.57)--(-2.31,2.51)--(-2.25,2.46)--(-2.18,2.4)--(-2.12,2.34)--(-2.06,2.29)--(-2.0,2.23)--(-1.93,2.18)--(-1.87,2.12)--(-1.81,2.07)--(-1.75,2.01)--(-1.68,1.96)--(-1.62,1.9)--(-1.56,1.85)--(-1.5,1.8)--(-1.43,1.75)--(-1.37,1.7)--(-1.31,1.65)--(-1.25,1.6)--(-1.18,1.55)--(-1.12,1.5)--(-1.06,1.45)--(-1.0,1.41)--(-0.93,1.37)--(-0.87,1.32)--(-0.81,1.28)--(-0.75,1.25)--(-0.68,1.21)--(-0.62,1.17)--(-0.56,1.14)--(-0.5,1.11)--(-0.43,1.09)--(-0.37,1.06)--(-0.31,1.04)--(-0.25,1.03)--(-0.18,1.01)--(-0.12,1.0)--(-0.06,1.0)--(0.0,1.0)--(0.06,1.0)--(0.12,1.0)--(0.18,1.01)--(0.25,1.03)--(0.31,1.04)--(0.37,1.06)--(0.43,1.09)--(0.5,1.11)--(0.56,1.14)--(0.62,1.17)--(0.68,1.21)--(0.75,1.25)--(0.81,1.28)--(0.87,1.32)--(0.93,1.37)--(1.0,1.41)--(1.06,1.45)--(1.12,1.5)--(1.18,1.55)--(1.25,1.6)--(1.31,1.65)--(1.37,1.7)--(1.43,1.75)--(1.5,1.8)--(1.56,1.85)--(1.62,1.9)--(1.68,1.96)--(1.75,2.01)--(1.81,2.07)--(1.87,2.12)--(1.93,2.18)--(2.0,2.23)--(2.06,2.29)--(2.12,2.34)--(2.18,2.4)--(2.25,2.46)--(2.31,2.51)--(2.37,2.57)--(2.43,2.63)--(2.5,2.69)--(2.56,2.75)--(2.62,2.8)--(2.68,2.86)--(2.75,2.92)--(2.81,2.98)--(2.87,3.04)--(2.93,3.1)--(3.0,3.16);
\draw [color=blue](-3.0,-3.16)--(-2.93,-3.1)--(-2.87,-3.04)--(-2.81,-2.98)--(-2.75,-2.92)--(-2.68,-2.86)--(-2.62,-2.8)--(-2.56,-2.75)--(-2.5,-2.69)--(-2.43,-2.63)--(-2.37,-2.57)--(-2.31,-2.51)--(-2.25,-2.46)--(-2.18,-2.4)--(-2.12,-2.34)--(-2.06,-2.29)--(-2.0,-2.23)--(-1.93,-2.18)--(-1.87,-2.12)--(-1.81,-2.07)--(-1.75,-2.01)--(-1.68,-1.96)--(-1.62,-1.9)--(-1.56,-1.85)--(-1.5,-1.8)--(-1.43,-1.75)--(-1.37,-1.7)--(-1.31,-1.65)--(-1.25,-1.6)--(-1.18,-1.55)--(-1.12,-1.5)--(-1.06,-1.45)--(-1.0,-1.41)--(-0.93,-1.37)--(-0.87,-1.32)--(-0.81,-1.28)--(-0.75,-1.25)--(-0.68,-1.21)--(-0.62,-1.17)--(-0.56,-1.14)--(-0.5,-1.11)--(-0.43,-1.09)--(-0.37,-1.06)--(-0.31,-1.04)--(-0.25,-1.03)--(-0.18,-1.01)--(-0.12,-1.0)--(-0.06,-1.0)--(0.0,-1.0)--(0.06,-1.0)--(0.12,-1.0)--(0.18,-1.01)--(0.25,-1.03)--(0.31,-1.04)--(0.37,-1.06)--(0.43,-1.09)--(0.5,-1.11)--(0.56,-1.14)--(0.62,-1.17)--(0.68,-1.21)--(0.75,-1.25)--(0.81,-1.28)--(0.87,-1.32)--(0.93,-1.37)--(1.0,-1.41)--(1.06,-1.45)--(1.12,-1.5)--(1.18,-1.55)--(1.25,-1.6)--(1.31,-1.65)--(1.37,-1.7)--(1.43,-1.75)--(1.5,-1.8)--(1.56,-1.85)--(1.62,-1.9)--(1.68,-1.96)--(1.75,-2.01)--(1.81,-2.07)--(1.87,-2.12)--(1.93,-2.18)--(2.0,-2.23)--(2.06,-2.29)--(2.12,-2.34)--(2.18,-2.4)--(2.25,-2.46)--(2.31,-2.51)--(2.37,-2.57)--(2.43,-2.63)--(2.5,-2.69)--(2.56,-2.75)--(2.62,-2.8)--(2.68,-2.86)--(2.75,-2.92)--(2.81,-2.98)--(2.87,-3.04)--(2.93,-3.1)--(3.0,-3.16);

\draw [color=blue](-3.0,2.44)--(-2.93,2.37)--(-2.87,2.29)--(-2.81,2.21)--(-2.75,2.13)--(-2.68,2.05)--(-2.62,1.97)--(-2.56,1.88)--(-2.5,1.8)--(-2.43,1.71)--(-2.37,1.62)--(-2.31,1.53)--(-2.25,1.43)--(-2.18,1.33)--(-2.12,1.23)--(-2.06,1.11)--(-2.0,1.0)--(-1.93,0.86)--(-1.87,0.71)--(-1.81,0.53)--(-1.75,0.25)--(-1.75,-0.25)--(-1.81,-0.53)--(-1.87,-0.71)--(-1.93,-0.86)--(-2.0,-1.0)--(-2.06,-1.11)--(-2.12,-1.23)--(-2.18,-1.33)--(-2.25,-1.43)--(-2.31,-1.53)--(-2.37,-1.62)--(-2.43,-1.71)--(-2.5,-1.8)--(-2.56,-1.88)--(-2.62,-1.97)--(-2.68,-2.05)--(-2.75,-2.13)--(-2.81,-2.21)--(-2.87,-2.29)--(-2.93,-2.37)--(-3.0,-2.44);
\draw [color=blue](3.0,2.44)--(2.93,2.37)--(2.87,2.29)--(2.81,2.21)--(2.75,2.13)--(2.68,2.05)--(2.62,1.97)--(2.56,1.88)--(2.5,1.8)--(2.43,1.71)--(2.37,1.62)--(2.31,1.53)--(2.25,1.43)--(2.18,1.33)--(2.12,1.23)--(2.06,1.11)--(2.0,1.0)--(1.93,0.86)--(1.87,0.71)--(1.81,0.53)--(1.75,0.25)--(1.75,-0.25)--(1.81,-0.53)--(1.87,-0.71)--(1.93,-0.86)--(2.0,-1.0)--(2.06,-1.11)--(2.12,-1.23)--(2.18,-1.33)--(2.25,-1.43)--(2.31,-1.53)--(2.37,-1.62)--(2.43,-1.71)--(2.5,-1.8)--(2.56,-1.88)--(2.62,-1.97)--(2.68,-2.05)--(2.75,-2.13)--(2.81,-2.21)--(2.87,-2.29)--(2.93,-2.37)--(3.0,-2.44);
\draw [color=blue](-2.43,2.99)--(-2.37,2.93)--(-2.31,2.88)--(-2.25,2.83)--(-2.18,2.79)--(-2.12,2.74)--(-2.06,2.69)--(-2.0,2.64)--(-1.93,2.59)--(-1.87,2.55)--(-1.81,2.5)--(-1.75,2.46)--(-1.68,2.41)--(-1.62,2.37)--(-1.56,2.33)--(-1.5,2.29)--(-1.43,2.25)--(-1.37,2.21)--(-1.31,2.17)--(-1.25,2.13)--(-1.18,2.1)--(-1.12,2.06)--(-1.06,2.03)--(-1.0,2.0)--(-0.93,1.96)--(-0.87,1.94)--(-0.81,1.91)--(-0.75,1.88)--(-0.68,1.86)--(-0.62,1.84)--(-0.56,1.82)--(-0.5,1.8)--(-0.43,1.78)--(-0.37,1.77)--(-0.31,1.76)--(-0.25,1.75)--(-0.18,1.74)--(-0.12,1.73)--(-0.06,1.73)--(0.0,1.73)--(0.06,1.73)--(0.12,1.73)--(0.18,1.74)--(0.25,1.75)--(0.31,1.76)--(0.37,1.77)--(0.43,1.78)--(0.5,1.8)--(0.56,1.82)--(0.62,1.84)--(0.68,1.86)--(0.75,1.88)--(0.81,1.91)--(0.87,1.94)--(0.93,1.96)--(1.0,2.0)--(1.06,2.03)--(1.12,2.06)--(1.18,2.1)--(1.25,2.13)--(1.31,2.17)--(1.37,2.21)--(1.43,2.25)--(1.5,2.29)--(1.56,2.33)--(1.62,2.37)--(1.68,2.41)--(1.75,2.46)--(1.81,2.5)--(1.87,2.55)--(1.93,2.59)--(2.0,2.64)--(2.06,2.69)--(2.12,2.74)--(2.18,2.79)--(2.25,2.83)--(2.31,2.88)--(2.37,2.93)--(2.43,2.99);
\draw [color=blue](-2.43,-2.99)--(-2.37,-2.93)--(-2.31,-2.88)--(-2.25,-2.83)--(-2.18,-2.79)--(-2.12,-2.74)--(-2.06,-2.69)--(-2.0,-2.64)--(-1.93,-2.59)--(-1.87,-2.55)--(-1.81,-2.5)--(-1.75,-2.46)--(-1.68,-2.41)--(-1.62,-2.37)--(-1.56,-2.33)--(-1.5,-2.29)--(-1.43,-2.25)--(-1.37,-2.21)--(-1.31,-2.17)--(-1.25,-2.13)--(-1.18,-2.1)--(-1.12,-2.06)--(-1.06,-2.03)--(-1.0,-2.0)--(-0.93,-1.96)--(-0.87,-1.94)--(-0.81,-1.91)--(-0.75,-1.88)--(-0.68,-1.86)--(-0.62,-1.84)--(-0.56,-1.82)--(-0.5,-1.8)--(-0.43,-1.78)--(-0.37,-1.77)--(-0.31,-1.76)--(-0.25,-1.75)--(-0.18,-1.74)--(-0.12,-1.73)--(-0.06,-1.73)--(0.0,-1.73)--(0.06,-1.73)--(0.12,-1.73)--(0.18,-1.74)--(0.25,-1.75)--(0.31,-1.76)--(0.37,-1.77)--(0.43,-1.78)--(0.5,-1.8)--(0.56,-1.82)--(0.62,-1.84)--(0.68,-1.86)--(0.75,-1.88)--(0.81,-1.91)--(0.87,-1.94)--(0.93,-1.96)--(1.0,-2.0)--(1.06,-2.03)--(1.12,-2.06)--(1.18,-2.1)--(1.25,-2.13)--(1.31,-2.17)--(1.37,-2.21)--(1.43,-2.25)--(1.5,-2.29)--(1.56,-2.33)--(1.62,-2.37)--(1.68,-2.41)--(1.75,-2.46)--(1.81,-2.5)--(1.87,-2.55)--(1.93,-2.59)--(2.0,-2.64)--(2.06,-2.69)--(2.12,-2.74)--(2.18,-2.79)--(2.25,-2.83)--(2.31,-2.88)--(2.37,-2.93)--(2.43,-2.99);


\draw [color=blue](-3.0,2.0)--(-2.93,1.9)--(-2.87,1.8)--(-2.81,1.7)--(-2.75,1.6)--(-2.68,1.49)--(-2.62,1.37)--(-2.56,1.25)--(-2.5,1.11)--(-2.43,0.97)--(-2.37,0.8)--(-2.31,0.58)--(-2.25,0.25)--(-2.25,-0.25)--(-2.31,-0.58)--(-2.37,-0.8)--(-2.43,-0.97)--(-2.5,-1.11)--(-2.56,-1.25)--(-2.62,-1.37)--(-2.68,-1.49)--(-2.75,-1.6)--(-2.81,-1.7)--(-2.87,-1.8)--(-2.93,-1.9)--(-3.0,-2.0);
\draw [color=blue](3.0,2.0)--(2.93,1.9)--(2.87,1.8)--(2.81,1.7)--(2.75,1.6)--(2.68,1.49)--(2.62,1.37)--(2.56,1.25)--(2.5,1.11)--(2.43,0.97)--(2.37,0.8)--(2.31,0.58)--(2.25,0.25)--(2.25,-0.25)--(2.31,-0.58)--(2.37,-0.8)--(2.43,-0.97)--(2.5,-1.11)--(2.56,-1.25)--(2.62,-1.37)--(2.68,-1.49)--(2.75,-1.6)--(2.81,-1.7)--(2.87,-1.8)--(2.93,-1.9)--(3.0,-2.0);
\draw [color=blue](-1.93,2.95)--(-1.87,2.91)--(-1.81,2.87)--(-1.75,2.83)--(-1.68,2.8)--(-1.62,2.76)--(-1.56,2.72)--(-1.5,2.69)--(-1.43,2.65)--(-1.37,2.62)--(-1.31,2.59)--(-1.25,2.56)--(-1.18,2.53)--(-1.12,2.5)--(-1.06,2.47)--(-1.0,2.44)--(-0.93,2.42)--(-0.87,2.4)--(-0.81,2.37)--(-0.75,2.35)--(-0.68,2.33)--(-0.62,2.32)--(-0.56,2.3)--(-0.5,2.29)--(-0.43,2.27)--(-0.37,2.26)--(-0.31,2.25)--(-0.25,2.25)--(-0.18,2.24)--(-0.12,2.23)--(-0.06,2.23)--(0.0,2.23)--(0.06,2.23)--(0.12,2.23)--(0.18,2.24)--(0.25,2.25)--(0.31,2.25)--(0.37,2.26)--(0.43,2.27)--(0.5,2.29)--(0.56,2.3)--(0.62,2.32)--(0.68,2.33)--(0.75,2.35)--(0.81,2.37)--(0.87,2.4)--(0.93,2.42)--(1.0,2.44)--(1.06,2.47)--(1.12,2.5)--(1.18,2.53)--(1.25,2.56)--(1.31,2.59)--(1.37,2.62)--(1.43,2.65)--(1.5,2.69)--(1.56,2.72)--(1.62,2.76)--(1.68,2.8)--(1.75,2.83)--(1.81,2.87)--(1.87,2.91)--(1.93,2.95);
\draw [color=blue](-1.93,-2.95)--(-1.87,-2.91)--(-1.81,-2.87)--(-1.75,-2.83)--(-1.68,-2.8)--(-1.62,-2.76)--(-1.56,-2.72)--(-1.5,-2.69)--(-1.43,-2.65)--(-1.37,-2.62)--(-1.31,-2.59)--(-1.25,-2.56)--(-1.18,-2.53)--(-1.12,-2.5)--(-1.06,-2.47)--(-1.0,-2.44)--(-0.93,-2.42)--(-0.87,-2.4)--(-0.81,-2.37)--(-0.75,-2.35)--(-0.68,-2.33)--(-0.62,-2.32)--(-0.56,-2.3)--(-0.5,-2.29)--(-0.43,-2.27)--(-0.37,-2.26)--(-0.31,-2.25)--(-0.25,-2.25)--(-0.18,-2.24)--(-0.12,-2.23)--(-0.06,-2.23)--(0.0,-2.23)--(0.06,-2.23)--(0.12,-2.23)--(0.18,-2.24)--(0.25,-2.25)--(0.31,-2.25)--(0.37,-2.26)--(0.43,-2.27)--(0.5,-2.29)--(0.56,-2.3)--(0.62,-2.32)--(0.68,-2.33)--(0.75,-2.35)--(0.81,-2.37)--(0.87,-2.4)--(0.93,-2.42)--(1.0,-2.44)--(1.06,-2.47)--(1.12,-2.5)--(1.18,-2.53)--(1.25,-2.56)--(1.31,-2.59)--(1.37,-2.62)--(1.43,-2.65)--(1.5,-2.69)--(1.56,-2.72)--(1.62,-2.76)--(1.68,-2.8)--(1.75,-2.83)--(1.81,-2.87)--(1.87,-2.91)--(1.93,-2.95);

\draw [color=blue](-3.0,1.41)--(-2.93,1.27)--(-2.87,1.12)--(-2.81,0.95)--(-2.75,0.75)--(-2.68,0.47)--(-2.68,-0.47)--(-2.75,-0.75)--(-2.81,-0.95)--(-2.87,-1.12)--(-2.93,-1.27)--(-3.0,-1.41);
\draw [color=blue](3.0,1.41)--(2.93,1.27)--(2.87,1.12)--(2.81,0.95)--(2.75,0.75)--(2.68,0.47)--(2.68,-0.47)--(2.75,-0.75)--(2.81,-0.95)--(2.87,-1.12)--(2.93,-1.27)--(3.0,-1.41);
\draw [color=blue](-1.37,2.98)--(-1.31,2.95)--(-1.25,2.92)--(-1.18,2.9)--(-1.12,2.87)--(-1.06,2.85)--(-1.0,2.82)--(-0.93,2.8)--(-0.87,2.78)--(-0.81,2.76)--(-0.75,2.75)--(-0.68,2.73)--(-0.62,2.71)--(-0.56,2.7)--(-0.5,2.69)--(-0.43,2.68)--(-0.37,2.67)--(-0.31,2.66)--(-0.25,2.65)--(-0.18,2.65)--(-0.12,2.64)--(-0.06,2.64)--(0.0,2.64)--(0.06,2.64)--(0.12,2.64)--(0.18,2.65)--(0.25,2.65)--(0.31,2.66)--(0.37,2.67)--(0.43,2.68)--(0.5,2.69)--(0.56,2.7)--(0.62,2.71)--(0.68,2.73)--(0.75,2.75)--(0.81,2.76)--(0.87,2.78)--(0.93,2.8)--(1.0,2.82)--(1.06,2.85)--(1.12,2.87)--(1.18,2.9)--(1.25,2.92)--(1.31,2.95)--(1.37,2.98);
\draw [color=blue](-1.37,-2.98)--(-1.31,-2.95)--(-1.25,-2.92)--(-1.18,-2.9)--(-1.12,-2.87)--(-1.06,-2.85)--(-1.0,-2.82)--(-0.93,-2.8)--(-0.87,-2.78)--(-0.81,-2.76)--(-0.75,-2.75)--(-0.68,-2.73)--(-0.62,-2.71)--(-0.56,-2.7)--(-0.5,-2.69)--(-0.43,-2.68)--(-0.37,-2.67)--(-0.31,-2.66)--(-0.25,-2.65)--(-0.18,-2.65)--(-0.12,-2.64)--(-0.06,-2.64)--(0.0,-2.64)--(0.06,-2.64)--(0.12,-2.64)--(0.18,-2.65)--(0.25,-2.65)--(0.31,-2.66)--(0.37,-2.67)--(0.43,-2.68)--(0.5,-2.69)--(0.56,-2.7)--(0.62,-2.71)--(0.68,-2.73)--(0.75,-2.75)--(0.81,-2.76)--(0.87,-2.78)--(0.93,-2.8)--(1.0,-2.82)--(1.06,-2.85)--(1.12,-2.87)--(1.18,-2.9)--(1.25,-2.92)--(1.31,-2.95)--(1.37,-2.98);


	\draw[color=blue,very thick][<-](-.7,-1.2)--(-.5,-1.1);
	\draw[color=blue,very thick][<-](.7,1.2)--(.5,1.1);
	\draw[color=blue,very thick][<-](1.2,.7)--(1.1,.5);
	\draw[color=blue,very thick][<-](-1.2,-.7)--(-1.1,-.5);

	\draw[dashed](-3,-3)--(3,3);
	\draw[dashed](-3,3)--(3,-3);

	\draw[very thick][->](-3,0)--(3,0) node[right=.2] {$x_1$};
	\draw[very thick][->](0,-2.625)--(0,2.75) node[above=.2] {$x_2$};
\end{tikzpicture}
\end{center}
\end{figure}
III przypadek a < 0
\begin{align*}
det(\lambda I - A ) = ( \lambda - i\sqrt{-a} ) ( \lambda + i\sqrt{-a} ) 
\end{align*}
Dwie wartości własne urojone sprzężone, o zerowych częściach rzeczywistych, portretem jest środek, którego kształt zależy od wartości a.
\begin{figure}[!h]
\begin{center}
\begin{tikzpicture}
	\draw[color=blue] (0,0) circle(.5);
	\draw[color=blue] (0,0) circle(1);
	\draw[color=blue] (0,0) circle(1.5);
	\draw[color=blue] (0,0) circle(2);
	\draw[color=blue] (0,0) circle(2.5);

	\draw[color=blue,very thick][<-](-.3,-.45)--(-.1,-.5);
	\draw[color=blue,very thick][<-](-.3,-.95)--(-.1,-1);
	\draw[color=blue,very thick][<-](-.3,-1.45)--(-.1,-1.5);
	\draw[color=blue,very thick][<-](-.3,-1.95)--(-.1,-2);
	\draw[color=blue,very thick][<-](-.3,-2.45)--(-.1,-2.5);

	\draw[very thick][->](-3,0)--(3,0) node[right=.2] {$x_1$};
	\draw[very thick][->](0,-2.625)--(0,2.75) node[above=.2] {$x_2$};
\end{tikzpicture}
\end{center}
\end{figure}

\newpage
\begin{framed}
\textbf{Zadanie 4 - Zadania od Bauera do kolokwium I } \\ 
Naszkicować portrety fazowe systemów dynamicznych i opisać czym się różnią
\begin{align*}
&\dot{x}_{1}(t)=-x_{1}(t)+x_{2}(t) \quad &\dot{x}_{1}(t)=x_{1}(t)+x_{2}(t) \\
&\dot{x}_{2}(t)=-x_{2}(t) \quad &\dot{x}_{2}(t)=-x_{2}(t) 
\end{align*}
i opisać czym się różnią.
\end{framed}
Macierz A:
\begin{align*}
A =
\begin{bmatrix}
-1 & 1 \\
0 & -1
\end{bmatrix} = J
\end{align*}
Macierz A jest w postaci Jordana, jedna wartość własna podwójna, rzeczywista, różna od zera \( \lambda = -1 \). Suma stopni klatek = krotność wartości własnej, liczba klatek = liczba wektorów własnych odpowiadających wartości własnej \( \implies \) Jeden wektor własny. \\
Wartość własna jest ujemna więc wykresem jest węzeł asymptotycznie stabilny. 
\begin{figure}[H]
\begin{center}
\begin{tikzpicture}
\draw [color=blue]
(	3.00	,	3.00		)	--
(	2.95	,	2.46		)	--
(	2.82	,	2.01		)	--
(	2.64	,	1.65		)	--
(	2.43	,	1.35		)	--
(	2.21	,	1.10		)	--
(	1.99	,	0.90		)	--
(	1.78	,	0.74		)	--
(	1.58	,	0.61		)	--
(	1.39	,	0.50		)	--
(	1.22	,	0.41		)	--
(	1.06	,	0.33		)	--
(	0.93	,	0.27		)	--
(	0.80	,	0.22		)	--
(	0.69	,	0.18		)	--
(	0.60	,	0.15		)	--
(	0.51	,	0.12		)	--
(	0.44	,	0.10		)	--
(	0.38	,	0.08		)	--
(	0.32	,	0.07		)	--
(	0.27	,	0.05		)	--
(	0.23	,	0.04		)	--
(	0.20	,	0.04		)	--
(	0.17	,	0.03		)	--
(	0.14	,	0.02		)	--
(	0.12	,	0.02		)	--
(	0.10	,	0.02		)	--
(	0.09	,	0.01		)	--
(	0.07	,	0.01		)	--
(	0.06	,	0.01		)	--
(	0.05	,	0.01		);
\draw [color=blue]
(	0.00	,	3.00	)	--
(	0.49	,	2.46	)	--
(	0.81	,	2.01	)	--
(	0.99	,	1.65	)	--
(	1.08	,	1.35	)	--
(	1.11	,	1.10	)	--
(	1.09	,	0.90	)	--
(	1.04	,	0.74	)	--
(	0.97	,	0.61	)	--
(	0.89	,	0.50	)	--
(	0.81	,	0.41	)	--
(	0.73	,	0.33	)	--
(	0.65	,	0.27	)	--
(	0.58	,	0.22	)	--
(	0.51	,	0.18	)	--
(	0.45	,	0.15	)	--
(	0.39	,	0.12	)	--
(	0.34	,	0.10	)	--
(	0.30	,	0.08	)	--
(	0.26	,	0.07	)	--
(	0.22	,	0.05	)	--
(	0.19	,	0.04	)	--
(	0.16	,	0.04	)	--
(	0.14	,	0.03	)	--
(	0.12	,	0.02	)	--
(	0.10	,	0.02	)	--
(	0.09	,	0.02	)	--
(	0.07	,	0.01	)	--
(	0.06	,	0.01	)	--
(	0.05	,	0.01	)	--
(	0.04	,	0.01	)	;
\draw [color=blue]
(	-3.00	,	0.00	)	--
(	-2.46	,	0.00	)	--
(	-2.01	,	0.00	)	--
(	-1.65	,	0.00	)	--
(	-1.35	,	0.00	)	--
(	-1.10	,	0.00	)	--
(	-0.90	,	0.00	)	--
(	-0.74	,	0.00	)	--
(	-0.61	,	0.00	)	--
(	-0.50	,	0.00	)	--
(	-0.41	,	0.00	)	--
(	-0.33	,	0.00	)	--
(	-0.27	,	0.00	)	--
(	-0.22	,	0.00	)	--
(	-0.18	,	0.00	)	--
(	-0.15	,	0.00	)	--
(	-0.12	,	0.00	)	--
(	-0.10	,	0.00	)	--
(	-0.08	,	0.00	)	--
(	-0.07	,	0.00	)	--
(	-0.05	,	0.00	)	--
(	-0.04	,	0.00	)	--
(	-0.04	,	0.00	)	--
(	-0.03	,	0.00	)	--
(	-0.02	,	0.00	)	--
(	-0.02	,	0.00	)	--
(	-0.02	,	0.00	)	--
(	-0.01	,	0.00	)	--
(	-0.01	,	0.00	)	--
(	-0.01	,	0.00	)	--
(	-0.01	,	0.00	)	;
		\draw [color=blue]			
(	-3.00	,	-3.00	)	--
(	-2.95	,	-2.46	)	--
(	-2.82	,	-2.01	)	--
(	-2.64	,	-1.65	)	--
(	-2.43	,	-1.35	)	--
(	-2.21	,	-1.10	)	--
(	-1.99	,	-0.90	)	--
(	-1.78	,	-0.74	)	--
(	-1.58	,	-0.61	)	--
(	-1.39	,	-0.50	)	--
(	-1.22	,	-0.41	)	--
(	-1.06	,	-0.33	)	--
(	-0.93	,	-0.27	)	--
(	-0.80	,	-0.22	)	--
(	-0.69	,	-0.18	)	--
(	-0.60	,	-0.15	)	--
(	-0.51	,	-0.12	)	--
(	-0.44	,	-0.10	)	--
(	-0.38	,	-0.08	)	--
(	-0.32	,	-0.07	)	--
(	-0.27	,	-0.05	)	--
(	-0.23	,	-0.04	)	--
(	-0.20	,	-0.04	)	--
(	-0.17	,	-0.03	)	--
(	-0.14	,	-0.02	)	--
(	-0.12	,	-0.02	)	--
(	-0.10	,	-0.02	)	--
(	-0.09	,	-0.01	)	--
(	-0.07	,	-0.01	)	--
(	-0.06	,	-0.01	)	--
(	-0.05	,	-0.01	)	;
		\draw [color=blue]			
(	0.00	,	-3.00	)	--
(	-0.49	,	-2.46	)	--
(	-0.81	,	-2.01	)	--
(	-0.99	,	-1.65	)	--
(	-1.08	,	-1.35	)	--
(	-1.11	,	-1.10	)	--
(	-1.09	,	-0.90	)	--
(	-1.04	,	-0.74	)	--
(	-0.97	,	-0.61	)	--
(	-0.89	,	-0.50	)	--
(	-0.81	,	-0.41	)	--
(	-0.73	,	-0.33	)	--
(	-0.65	,	-0.27	)	--
(	-0.58	,	-0.22	)	--
(	-0.51	,	-0.18	)	--
(	-0.45	,	-0.15	)	--
(	-0.39	,	-0.12	)	--
(	-0.34	,	-0.10	)	--
(	-0.30	,	-0.08	)	--
(	-0.26	,	-0.07	)	--
(	-0.22	,	-0.05	)	--
(	-0.19	,	-0.04	)	--
(	-0.16	,	-0.04	)	--
(	-0.14	,	-0.03	)	--
(	-0.12	,	-0.02	)	--
(	-0.10	,	-0.02	)	--
(	-0.09	,	-0.02	)	--
(	-0.07	,	-0.01	)	--
(	-0.06	,	-0.01	)	--
(	-0.05	,	-0.01	)	--
(	-0.04	,	-0.01	)	;
		\draw [color=blue]			
(	3.00	,	-3.00	)	--
(	1.96	,	-2.46	)	--
(	1.21	,	-2.01	)	--
(	0.66	,	-1.65	)	--
(	0.27	,	-1.35	)	--
(	0.00	,	-1.10	)	--
(	-0.18	,	-0.90	)	--
(	-0.30	,	-0.74	)	--
(	-0.36	,	-0.61	)	--
(	-0.40	,	-0.50	)	--
(	-0.41	,	-0.41	)	--
(	-0.40	,	-0.33	)	--
(	-0.38	,	-0.27	)	--
(	-0.36	,	-0.22	)	--
(	-0.33	,	-0.18	)	--
(	-0.30	,	-0.15	)	--
(	-0.27	,	-0.12	)	--
(	-0.24	,	-0.10	)	--
(	-0.21	,	-0.08	)	--
(	-0.19	,	-0.07	)	--
(	-0.16	,	-0.05	)	--
(	-0.14	,	-0.04	)	--
(	-0.13	,	-0.04	)	--
(	-0.11	,	-0.03	)	--
(	-0.09	,	-0.02	)	--
(	-0.08	,	-0.02	)	--
(	-0.07	,	-0.02	)	--
(	-0.06	,	-0.01	)	--
(	-0.05	,	-0.01	)	--
(	-0.04	,	-0.01	)	--
(	-0.04	,	-0.01	)	;
			\draw [color=blue]		
(	3.00	,	0.00	)	--
(	2.46	,	0.00	)	--
(	2.01	,	0.00	)	--
(	1.65	,	0.00	)	--
(	1.35	,	0.00	)	--
(	1.10	,	0.00	)	--
(	0.90	,	0.00	)	--
(	0.74	,	0.00	)	--
(	0.61	,	0.00	)	--
(	0.50	,	0.00	)	--
(	0.41	,	0.00	)	--
(	0.33	,	0.00	)	--
(	0.27	,	0.00	)	--
(	0.22	,	0.00	)	--
(	0.18	,	0.00	)	--
(	0.15	,	0.00	)	--
(	0.12	,	0.00	)	--
(	0.10	,	0.00	)	--
(	0.08	,	0.00	)	--
(	0.07	,	0.00	)	--
(	0.05	,	0.00	)	--
(	0.04	,	0.00	)	--
(	0.04	,	0.00	)	--
(	0.03	,	0.00	)	--
(	0.02	,	0.00	)	--
(	0.02	,	0.00	)	--
(	0.02	,	0.00	)	--
(	0.01	,	0.00	)	--
(	0.01	,	0.00	)	--
(	0.01	,	0.00	)	--
(	0.01	,	0.00	)	;
\draw [color=blue]	
(	-3.00	,	3.00	)	--
(	-1.96	,	2.46	)	--
(	-1.21	,	2.01	)	--
(	-0.66	,	1.65	)	--
(	-0.27	,	1.35	)	--
(	0.00	,	1.10	)	--
(	0.18	,	0.90	)	--
(	0.30	,	0.74	)	--
(	0.36	,	0.61	)	--
(	0.40	,	0.50	)	--
(	0.41	,	0.41	)	--
(	0.40	,	0.33	)	--
(	0.38	,	0.27	)	--
(	0.36	,	0.22	)	--
(	0.33	,	0.18	)	--
(	0.30	,	0.15	)	--
(	0.27	,	0.12	)	--
(	0.24	,	0.10	)	--
(	0.21	,	0.08	)	--
(	0.19	,	0.07	)	--
(	0.16	,	0.05	)	--
(	0.14	,	0.04	)	--
(	0.13	,	0.04	)	--
(	0.11	,	0.03	)	--
(	0.09	,	0.02	)	--
(	0.08	,	0.02	)	--
(	0.07	,	0.02	)	--
(	0.06	,	0.01	)	--
(	0.05	,	0.01	)	--
(	0.04	,	0.01	)	--
(	0.04	,	0.01	)	;

	\draw[color=blue,very thick][->](0.00	,	1.10)--(0.30	,	0.74);
	\draw[color=blue,very thick][->](-0.36	,	-0.61)--(-0.36	,	-0.4	);
	\draw[color=blue,very thick][->](-0.89	,	-0.50)--(-0.73	,	-0.33);
	\draw[color=blue,very thick][->](-1.39	,	-0.50)--(-1.25	,	-0.4);
	\draw[color=blue,very thick][->](1.04	,	0.74)--(0.9	,	0.6);
	\draw[color=blue,very thick][->](1.58	,	0.61)--(1.5	,	0.5);
	



	\draw[very thick][->](-3,0)--(3,0) node[right=.2] {$x_1$};
	\draw[very thick][->](0,-3)--(0,3) node[above=.2] {$x_2$};
\end{tikzpicture}
\end{center}
\end{figure}
Dla drugiego systemu macierz A:
\begin{align*}
A = 
\begin{bmatrix}
1 & 1 \\
0 & -1
\end{bmatrix}
\end{align*}
Macierz ta nie jest w postaci Jordana. Wartości własne:
\begin{align*}
&det( \lambda I - A ) =
det
\begin{bmatrix}
\lambda -1  & -1 \\
0 & \lambda + 1
\end{bmatrix} =
( \lambda - 1 )( \lambda + 1 )\\
&\lambda _{1} = 1, \lambda _{2} = -1
\end{align*}
Dwie wartości własne rzeczywiste różnych znaków \( \implies \) portret fazowy to siodło. Obliczam wektory własne aby wyznaczyć kierunki osi ( macierz nie jest w postaci kanonicznej ). \\ Wektory własne dla \( \lambda _{1} = 1 \) : 
\begin{align*}
\begin{bmatrix}
0 & -1 \\
0 & 1 
\end{bmatrix}
\begin{bmatrix}
x \\
y
\end{bmatrix}
=
\begin{bmatrix}
0 \\
0
\end{bmatrix}
\implies
y = 0
\end{align*}
Np. \( ( 1,0) \) \\
Wartości własne dla \( \lambda _{2} = -1 \) :
\begin{align*}
\begin{bmatrix}
-2 & -1 \\
0 & 0 
\end{bmatrix}
\begin{bmatrix}
x \\
y
\end{bmatrix}
=
\begin{bmatrix}
0 \\
0
\end{bmatrix}
\implies
y = -2x
\end{align*}
Np. \( (1,-2) \) \\
Portret fazowy:


Jeden z portretów to węzeł, a drugi to siodło, węzeł jest stabilny asymptotycznie, siodło nie. 
\begin{figure}[H]
\begin{center}
\begin{tikzpicture}

\draw [color=blue] (-1.3,3) -- (-0.6,1.5) -- (0,0.8) -- (0.5,0.5) --(3,0.2);
\draw [color=blue] (1.3,-3) -- (0.6,-1.5) -- (0,-0.8) -- (-0.5,-0.5) --(-3,-0.2);
\draw [color=blue] (-1.7,3)  -- (-1.6,2.3)-- (-1.5,1.2) -- (-1.8,0.5) -- (-2.2,0.3) -- (-3,0.2);
\draw [color=blue] (1.7,-3)  -- (1.6,-2.3)-- (1.5,-1.2) -- (1.8,-0.5) -- (2.2,-0.3) -- (3,-0.2);

	\draw[color=blue,very thick][->](0,0.8)--(0.2,0.7);
	\draw[color=blue,very thick][->](0,-0.8)--(-0.2,-0.7);
	\draw[color=blue,very thick][->](1.5,-1.2)--(1.6,-1);
	\draw[color=blue,very thick][->](-1.5,1.2)--(-1.6,1);

	\draw[dashed](-1.5,3)--(1.5,-3);

	\draw[very thick][->](-3,0)--(3,0) node[right=.2] {$x_1$};
	\draw[very thick][->](0,-2.625)--(0,2.75) node[above=.2] {$x_2$};
\end{tikzpicture}
\end{center}
\end{figure}


\newpage
\begin{framed}
\textbf{Zadanie 5 - Egzamin}  \\ 
Dany jest układ:
\begin{align*}
\dot{x}_{1}(t)=-x_2+x_1x_2 \\
\dot{x}_{2}(t)=x_1+\frac{1}{2}(x_1^2-x_2^2)
\end{align*}
Narysować portrety fazowe systemu zlinearyzowanego w punktach równowagi.
\end{framed}
\colorbox{red}{nie ma sprzeczności w równaniu, ma tam być x1 zamiast 1 - poprawić } \\
Wyznaczam punkty równowagi:
\begin{align*}
&\begin{cases}
-x_2+x_1x_2=0 \\
x_1+\frac{1}{2}(x_1^2-x_2^2) = 0 
\end{cases} \\
&\begin{cases}
x_2(x_1-1)=0 \\
x_1+\frac{1}{2}(x_1-x_2)(x_1+x_2) = 0 
\end{cases}
\implies
x_2 = 0 \lor x_1 = 1 \\
&x_2 = 0 \\
&1+\frac{1}{2}(x_1-0)(x_1+0)=0 \\
&1+\frac{1}{2}x_1^2=0 \\
&-\frac{1}{2}x_1^2 = 1 \\
&x_1^2=-2\\
&\text{Sprzeczność} \\
&x_1=1 \\
&1+\frac{1}{2}(1-x_2)(1+x_2)=0 \\
&1+\frac{1}{2}(1-x_2^2)=0 \\
&-1 = \frac{1}{2}-\frac{1}{2}x_2^2 \\
&\frac{1}{2}x_2^2=\frac{3}{2} \\
&x_2^2=3 \\
&x_2=\sqrt{3} \; \lor \; x_2=-\sqrt{3}
\end{align*}
Punkty równowagi: \( (1,\sqrt{3}) \) \; , \( (1, -\sqrt{3} ) \). \\
Macierz Jacobiego:
\begin{align*}
&J = 
\begin{bmatrix}
\frac{\partial f_1}{x_1} & \frac{\partial f_1}{x_2} \\
\frac{\partial f_2}{x_1} &  \frac{\partial f_2}{x_2}
\end{bmatrix}
=
\begin{bmatrix}
x_2 & -1+x_1 \\
1+x_1 & -x_2
\end{bmatrix}
\end{align*}
Dla punktu \( x^* = (1, \sqrt{3}) \):
\begin{align*}
J = 
\begin{bmatrix}
\sqrt{3} & 0 \\
2 & - \sqrt{3} 
\end{bmatrix}
\end{align*}
Wartości własne:
\begin{align*}
&\lambda I - A = 
\begin{bmatrix}
\lambda - \sqrt{3} & 0 \\
-2 & \lambda + \sqrt{3}
\end{bmatrix} \\
&\text{det} ( \lambda I - A ) = (\lambda - \sqrt{3})(\lambda + \sqrt{3}) = \lambda ^2 -3 \\
& \lambda _1 = -\sqrt{3} , \; \lambda _2 = \sqrt{3}
\end{align*}
Dla punktu \( x^* = ( 1, -\sqrt{3}) \):
\begin{align*}
J = 
\begin{bmatrix}
-\sqrt{3} & 0 \\
2 & \sqrt{2}
\end{bmatrix}
\end{align*}
Wartości własne:
\begin{align*}
&\lambda I - A = 
\begin{bmatrix}
\lambda + \sqrt{3} & 0 \\
-2 & \lambda - \sqrt{3}
\end{bmatrix} \\
&\text{det} ( \lambda I - A ) = (\lambda - \sqrt{3})(\lambda + \sqrt{3}) = \lambda ^2 -3 \\
& \lambda _1 = -\sqrt{3} , \; \lambda _2 = \sqrt{3}
\end{align*}
W obu przypadkach istnieją dwie wartości własne różnych znaków rzeczywiste, a więc portretem fazowym jest siodło. 
Następnie obliczam wektory własne: \\
Dla punktu \( x^* = (1, \sqrt{3}) \) wektory własne to: \\
Dla \( \lambda _1  = \sqrt{3} \; \; (\sqrt{3}, 1) \), \( \lambda _2 = -\sqrt{3} \; \; (1,0) \). \\
Dla punktu \( x^* = (1, -\sqrt{3}) \) wektory własne to: \\
Dla \( \lambda _1  = -\sqrt{3} \; \; (-\sqrt{3},1) \), \( \lambda _2 = \sqrt{3} \; \; (0,1) \). 
Portrety fazowe będą więc różne, ponieważ kierunki wektorów własnych są różne. \\
( Zrobić rysunki ... )




\newpage
\section{Kryteria częstotliwościowe}
\subsection{Zbiór zadań}
\begin{framed}
\textbf{\colorbox{yellow}{Zadanie 1 - Egzamin }} \\ 
Jaka będzie postać rozwiązania w stanie ustalonym jeśli na układ  
\begin{align*}
&\dot{x}_{1}(t)=\frac{-1}{10}x_{1}(t)+u(t) \\
&\dot{x}_{2}(t)=\frac{-1}{10}x_{2}(t)+x_{1}(t) \\
&\dot{x}_{3}(t)=\frac{-1}{11}x_{3}(t)+x_{2}(t) \\
&\dot{x}_{4}(t)=\frac{-1}{9}x_{4}(t)+x_{3}(t) \\
&\text{gdzie}\\
&y(t) = x_{2}(t)\\
&\text{podano sygnał} \\
& u(t)=\sin (0,1t)
\end{align*}
\end{framed}


\begin{framed}
\textbf{\colorbox{green}{Zadanie 2 - Zadania od Bauera do kolokwium I} } \\ 
Rozwiązanie równania różniczkowego: 
\begin{align*}
&\dot{x}(t)=-10x(t)+5\sin (2t+\frac{3\pi}{4}) \\
&\text{gdzie} \\
&x(0)=0, \; t \geq 0 \; \text{ma postać} \\
&x(t)=ae^{-2t}+A\sin (2t+\varphi)
\end{align*}
Obliczyć \( A \) i \( \varphi \).
\end{framed}

\begin{framed}
\textbf{\colorbox{green}{ 3 - Zadania od Bauera do kolokwium I }} \\ 
Mając daną transmitancję \( G(s) = \frac{100}{s+20} \) określić amplitudę sygnału wyjściowego jeśli na wejście podano:

\begin{itemize}
\item \( -3 \sin (5t+\frac{\pi}{2}) \)
\item \( 10 \sin (2t+\pi) \)
\item \( 5 \cos (2t+\frac{2\pi}{3}) \)
\end{itemize}
\end{framed}


\begin{framed}
\textbf{\colorbox{green}{Zadanie 4 - Zadania od Bauera do kolokwium I }} \\ 
Narysować charakterystyki Nyquista dla układu opisanego transmitancją operatorową:
\begin{align*}
G(s)=\frac{s}{s^{2}+s+1}
\end{align*}
Podać wzór na transmitancję widmową tego układu ( w postaci rozbicia na część urojoną i rzeczywistą )
\end{framed}


\begin{framed}
\textbf{\colorbox{green}{Zadanie 5 - Zadania od Bauera do kolokwium I }} \\ 
Niech dany będzie układ opisany transmitancją
\begin{align*}
G(s)=\frac{s+4}{2s^{2}+3s+1}
\end{align*}
Korzystając z kryterium Nyquista sprawdzić czy układ zamknięty przedstawiony na rysunku będzie asymptotycznie stabilny.
\begin{figure}[H]
\centering
\begin{tikzpicture}[scale=1,inner sep=0.4mm]

\node (pid) [trans] at (2,0) {$G(s)$};
% \node (env) [trans] at (3.75,1.5) {$env_{1}$};

% \draw [->] (sum) to (pid);
\path[->] (-2,0) edge node [above ] {$u(s)$} (pid);
\path[->] (pid) edge node [above ] {$y(s)$} (5,0);
\draw[->] (4,0)  -- (4,-1.5) -- (-0.5,-1.5) -- (-0.5,0);
\end{tikzpicture}
\label{fig:opt_par_1}
\end{figure}

\end{framed}

\begin{framed}
\textbf{\colorbox{green}{Zadanie 6 - Zadania od Bauera do kolokwium I }} \\ 
Korzystając z kryterium Michajłowa zbadać asymptotyczną stabilność układu opisanego transmitancją
\begin{align*}
G(s)=\frac{s^{2}+1}{s^{3}+s^{2}+9s+4}
\end{align*}
\end{framed}


\begin{framed}
\textbf{\colorbox{green}{Zadanie 7 - Informatyka Modelowanie }} \\ 
Rozwiązanie równania różniczkowego\\
	$\ddot{x}(t)+\dot{x}(t)=-x(t)+12\sin({\omega t})$\\
	gdzie $x(0)=0$, ($\dot{x}(0)$ - w domysle), $t \geq 0$ ma postać\\
	$x(t)=f(t)+A\sin({\omega t})$\\
znaleźć takie $\omega$, dla którego $A$ jest największe\\
\end{framed}

\begin{framed}
\textbf{Zadanie 8 - Egzamin } \\ 
Wykorzystując kryterium Nyquista sprawdzić czy układ zamknięty będzie asymptotycznie stabilny, jeśli układ otwarty opisany transmitancją:
\begin{align*}
G(s)=\frac{6}{s^3+2s^2+2s+40}
\end{align*}
połączono szeregowo z regulatorem proporcjonalnym \( G_K = 100 \). 
\end{framed}


\newpage
\subsection{Rozwiązania zadań ze zbioru}
\begin{framed}
\textbf{Zadanie 1 - Egzamin } \\ 
Jaka będzie postać rozwiązania w stanie ustalonym jeśli na układ  
\begin{align*}
&\dot{x}_{1}(t)=\frac{-1}{10}x_{1}(t)+u(t) \\
&\dot{x}_{2}(t)=\frac{-1}{10}x_{2}(t)+x_{1}(t) \\
&\dot{x}_{3}(t)=\frac{-1}{11}x_{3}(t)+x_{2}(t) \\
&\dot{x}_{4}(t)=\frac{-1}{9}x_{4}(t)+x_{3}(t) \\
&\text{gdzie}\\
&y(t) = x_{2}(t)\\
&\text{podano sygnał} \\
& u(t)=\sin (0,1t)
\end{align*}
\end{framed}
Na wejście układu podany jest sinus, a więc na wyjściu również otrzymamy sinus, o innej amplitudzie i fazie i o tej samej częstotliwości. \\
Aby wyznaczyć sygnał wyjściowy rozważam tylko równania wpływające na wyjście, czyli na pewno równanie na \( x_2 \) oraz równanie na \( x_1 \), bo \( x_1 \) zawiera się w równaniu na \( x_2 \) więc też od niego zależy. \( x_3 \) i \( x_4 \) nie wpływają na \( x_2 \) ani też na \( x_1 \), jedynie \( x_3 \) zależy od \( x_2 \), ale nie wpływa na nie. \\
W takim przypadku macierze są postaci: \colorbox{red}{błąd w macierzy A - do poprawy}
\begin{align*}
A = 
\begin{bmatrix}
-\frac{1}{10} & 0 \\
1 & \frac{1}{10}
\end{bmatrix}
\; 
B = 
\begin{bmatrix}
1 \\
0
\end{bmatrix}
C^T = 
\begin{bmatrix}
0 & 1
\end{bmatrix}
\end{align*}
Transmitancja:
\begin{align*}
&G(s)=C(sI-A)^{-1}B=
\begin{bmatrix}
0 & 1 
\end{bmatrix}
\begin{bmatrix}
s+\frac{1}{10} & 0 \\
-1 & s-\frac{1}{10}
\end{bmatrix}^{-1}
\begin{bmatrix}
1 \\
0
\end{bmatrix} = \\
&\frac{1}{s^2-\frac{1}{100}}\cdot 
\begin{bmatrix}
0 & 1 
\end{bmatrix}
\begin{bmatrix}
s-\frac{1}{10} & 0 \\
1 & s+\frac{1}{10}
\end{bmatrix}
\begin{bmatrix}
1 \\
0
\end{bmatrix} = \\
&\frac{1}{s^2-\frac{1}{100}}\cdot 
\begin{bmatrix}
1 & s+\frac{1}{10}
\end{bmatrix}
\begin{bmatrix}
1 \\
0
\end{bmatrix} = \\
&\frac{1}{s^2-\frac{1}{100}} \\
\end{align*}
\begin{align*}
&G(j\omega ) = \underbrace{\frac{1}{-\omega ^{2}-\frac{1}{100}}}_{P(\omega)} \\
&\omega = 0,1 \; A_u=1 \; \varphi _u = 0 \\
&|Y(j\omega)|=|G(j\omega)||U(j\omega)|=1 \cdot \sqrt{\frac{1}{(-0,02)^2}}=50 \\
&\varphi _y = \varphi _u + \text{arg}G(j\omega) \\
&\cos \varphi = \frac{P(\omega)}{|G(j\omega)|} = \frac{-50}{50}=-1 \\
&\sin \varphi = \frac{Q(\omega)}{|G(j\omega)|} = \frac{0}{50} = 0
\end{align*}
Sprzeczność ? 
\newpage
\begin{framed}
\textbf{Zadanie 2 - Zadania od Bauera do kolokwium I } \\ 
Rozwiązanie równania różniczkowego: 
\begin{align*}
&\dot{x}(t)=-10x(t)+5\sin (2t+\frac{3\pi}{4}) \\
&\text{gdzie} \\
&x(0)=0, \; t \geq 0 \; \text{ma postać} \\
&x(t)=ae^{-2t}+A\sin (2t+\varphi)
\end{align*}
Obliczyć \( A \) i \( \varphi \).
\end{framed}
\begin{align*}
&G(s)=\frac{Y(s)}{U(s)} \\
&G(j \omega)=\frac{Y(j\omega)}{U(j\omega)} \\
&A_y=|Y(j\omega)|=|G(j\omega)|\cdot |U(j\omega)|=|G(j\omega)|\cdot A_u \\
&\varphi _{y} = \text{arg} Y(j \omega) = \text{arg} G(j \omega) + \text{arg} U(j \omega )= \text{arg} G(j \omega) + \varphi _u
\end{align*}
Zakładam, że \( y(t) = x(t) \implies C = 1 \), wtedy:
\begin{align*}
&u(t)=\sin (2t+\frac{3\pi}{4} )\\
&A_{u}=1 \; \text{bo} \; B=5, \; \omega _{u} = \omega _y = 2 \\
&G(s)=C(sI-A)^{-1}B=1(s+10)^{-1}\cdot 5=\frac{5}{s+10} \\
&G(j \omega)=\frac{5}{j\omega + 10} \\
&|G(j\omega|=\frac{5}{\sqrt{\omega ^{2}+100}}=\frac{5}{\sqrt{104}}=0,49 \\
&A_y=0,49\cdot A_u = 0,49 \cdot 1 = 0,49\\
&\text{arg} G(j \omega)=\text{arg}5-\text{arg}(j\omega+10)= 0 - \text{arg}(j\omega +10)=-\text{arg}(2j+10) \\
& \sin \varphi = \frac{2}{\sqrt{4+100}} \\
& \cos \varphi = \frac{10}{\sqrt{4+100}} \\
& \tan \varphi = \frac{2}{10} \\
& \varphi = \arctan \frac{2}{10} \\
&\varphi _y = \arctan \frac{2}{10}+\frac{3\pi}{4}
\end{align*}

\newpage
\begin{framed}
\textbf{Zadanie 3 - Zadania od Bauera do kolokwium I } \\ 
Mając daną transmitancję \( G(s) = \frac{100}{s+20} \) określić amplitudę sygnału wyjściowego jeśli na wejście podano:
\begin{itemize}
\item \( -3 \sin (5t+\frac{\pi}{2}) \)
\item \( 10 \sin (2t+\pi) \)
\item \( 5 \cos (2t+\frac{2\pi}{3}) \)
\end{itemize}
\end{framed}
\begin{gather*}
u(t) = A_u \cdot \sin{(\omega t+\phi_u)}\text{ - wejście} \\
A_u \text{ - amplituda wejścia} \\
\phi_u \text{ - faza wejścia} \\
y(t) = A_y \cdot \sin{(\omega t+\phi_y)}\text{ - wyjście} \\
A_y = A(\omega) \cdot A_u \text{ - amplituda wyjścia} \\
\phi_y \text{ - faza wyjścia} \\
A(\omega) = |G(j\omega)| \\
G(s) = \frac{100}{s+20}
\end{gather*}

a)
\begin{align*}
u(t) &= -3\sin{(5t)} \\
A_u &= -3 \\
\omega &= 5 \\
A(\omega) &= A(5) = |G(5j)| = \left| \frac{100}{5j+20} \right| 
= \left| \frac{20}{4+j} \cdot \frac{4-j}{4-j} \right|
= \left| \frac{20}{17} \cdot (4-j) \right| = \\
&= \frac{20}{17} \cdot |4-j| = \frac{20}{17} \cdot \sqrt{4^2+(-1)^2} 
= \frac{20 \cdot \sqrt{17}}{17} \\
A_y &= A(\omega) \cdot A_u = \frac{20 \cdot \sqrt{17}}{17} \cdot -3
= \boxed{ \frac{-60 \cdot \sqrt{17}}{17} }
\end{align*}

b)
\begin{align*}
u(t) &= 10\sin{\left( 2t+\pi \right)} \\
A_u &= 10 \\
\omega &= 2 \\
A(\omega) &= A(2) = |G(2j)| = \left| \frac{100}{2j+20} \right| 
= \left| \frac{50}{10+j} \cdot \frac{10-j}{10-j} \right|
= \left| \frac{50}{101} \cdot (10-j) \right| = \\
&= \frac{50}{101} \cdot |10-j| = \frac{50}{101} \cdot \sqrt{10^2+(-1)^2} 
= \frac{50 \cdot \sqrt{101}}{101} \\
A_y &= A(\omega) \cdot A_u = \frac{50\sqrt{101}}{101} \cdot 5
= \frac{250\sqrt{101}}{101}
\end{align*}

c)
\begin{align*}
u(t) &= 5\cos{\left( 2t+\frac{2\pi}{3} \right)} = 5\sin{\left( 2t+\frac{2\pi}{3}+\frac{\pi}{2} \right)}
= \sin{\left( 3t+\frac{5\pi}{6} \right)} \\
A_u &= 5 \\
\omega &= 2 \\
A(\omega) &= A(2) = |G(2j)| = \left| \frac{100}{2j+20} \right| 
= \left| \frac{50}{10+j} \cdot \frac{10-j}{10-j} \right|
= \left| \frac{50}{101} \cdot (10-j) \right| = \\
&= \frac{50}{101} \cdot |10-j| = \frac{50}{101} \cdot \sqrt{10^2+(-1)^2} 
= \frac{50 \cdot \sqrt{101}}{101} \\
A_y &= A(\omega) \cdot A_u = \frac{50 \cdot \sqrt{101}}{101} \cdot 5
= \boxed{ \frac{250 \cdot \sqrt{101}}{101} }
\end{align*}

\newpage
\begin{framed}
\textbf{Zadanie 4 - Zadania od Bauera do kolokwium I } \\ 
Narysować charakterystyki Nyquista dla układu opisanego transmitancją operatorową:
\begin{align*}
G(s)=\frac{s}{s^{2}+s+1}
\end{align*}
Podać wzór na transmitancję widmową tego układu ( w postaci rozbicia na część urojoną i rzeczywistą )
\end{framed}
\begin{align*}
&G(j\omega)=\frac{j\omega}{(j\omega )^{2}+j\omega + 1} = \frac{j\omega}{-\omega ^{2}+j\omega + 1}=\frac{j\omega}{-\omega ^{2}+1+j\omega}=\frac{j\omega(1-\omega ^{2}-j\omega)}{(1-\omega ^{2})^{2}-(j\omega )^{2}}= \\
&\frac{j\omega -\omega ^{3}j-j^{2}\omega ^{2}}{1-2\omega ^{2}+\omega ^{4}-j^{2}\omega ^{2}}=\frac{\omega ^{2}+j(\omega-\omega ^{3})}{\omega ^{4}- \omega^{2} + 1}=\frac{\omega ^{2}}{\omega ^{4}-\omega ^{2}+1}+j \frac{\omega - \omega ^{3}}{\omega ^{4} - \omega ^{2} + 1}
\end{align*}
Tebelka
\begin{table}[H]
\begin{center}
\begin{tabular}{|p{2cm}|p{2cm}|p{2cm}|p{2cm}|p{2cm}|p{2cm}|}
\hline
\( \omega \) & 0 & \( \frac{1}{2} \) & 1 & \(\frac{3}{2}\) & \( \infty\)  \\ \hline
\( P(\omega) \) & 0 & \( \frac{4}{13} \) & 1 & \( \frac{36}{61} \) & 0  \\ \hline
\( Q(\omega) \) & 0 & \( \frac{6}{13} \) & 0 & \( \frac{30}{61} \) & 0  \\ \hline
\end{tabular}
\end{center}
\end{table}
Szkicuję charakterystykę:
\begin{figure}[H]
\centerline{\includegraphics[scale=0.2]{kol2_13.jpg}}
\caption{Szkic charakterystyki amp-faz}
\label{fig:kol2_13}
\end{figure}

\newpage
\begin{framed}
\textbf{Zadanie 5 - Zadania od Bauera do kolokwium I } \\ 
Niech dany będzie układ opisany transmitancją
\begin{align*}
G(s)=\frac{s+4}{2s^{2}+3s+1}
\end{align*}
Korzystając z kryterium Nyquista sprawdzić czy układ zamknięty przedstawiony na rysunku będzie asymptotycznie stabilny.
\begin{figure}[H]
\centering
\begin{tikzpicture}[scale=1,inner sep=0.4mm]
\node (pid) [trans] at (2,0) {$G(s)$};
% \node (env) [trans] at (3.75,1.5) {$env_{1}$};
% \draw [->] (sum) to (pid);
\path[->] (-2,0) edge node [above ] {$u(s)$} (pid);
\path[->] (pid) edge node [above ] {$y(s)$} (5,0);
\draw[->] (4,0)  -- (4,-1.5) -- (-0.5,-1.5) -- (-0.5,0);
\end{tikzpicture}
\label{fig:opt_par_1}
\end{figure}
\end{framed}
Sprawdzam ile pierwiastków ma w \( C^{+} \) wielomian \( M(s) = 2s^{2}+3s+1 \):
\begin{align*}
&2s^{2}+3s+1=0 \\
&\Delta = 9-4\cdot2\cdot1 \\
&\sqrt{\Delta}=1 \\
&s_{1}=\frac{-3+1}{4} \; , s_{2}=\frac{-3-1}{4} \\
&s_{1}=\frac{-1}{2} \; , s_{2}=-1
\end{align*}
Wielomian \( M(s) \) nie ma pierwiastków w \( C^{+} \). Wyznaczam charakterystykę amplitudowo-fazową:
\begin{align*}
&G(j \omega)=\frac{j\omega + 4}{2j^{2}\omega ^{2}+3j\omega + 1}= \frac{4+j\omega}{-2\omega ^{2}+3j\omega + 1}= \frac{4+j\omega}{1-2\omega ^{2}+3j\omega}\cdot\frac{1-2\omega ^{2}-3j\omega}{1-2\omega ^{2}-3j\omega} = \\
&\frac{(4+j\omega)(1-2\omega^{2}-3j\omega)}{(1-2\omega^{2})^{2})-(3j\omega ) ^{2}}= \frac{4-8\omega ^{2}-12j\omega+j\omega-2j\omega ^{3} - 3j^{2}\omega ^{2}}{1-4\omega ^{2}+4\omega^{4}+9\omega^{2}} = \\
&\frac{4-5\omega ^{2}}{4\omega ^{4}+5\omega^{2}+1}+j\frac{-2\omega^{3}-11\omega}{4\omega ^{4}+5\omega ^{2}+1}
\end{align*} 
Tebelka
\begin{table}[H]
\begin{center}
\begin{tabular}{|p{2cm}|p{2cm}|p{2cm}|p{2cm}|p{2cm}|p{2cm}|}
\hline
\( \omega \) & 0 & \( 0,1 \) & 1 & \( 0,5 \)& \( \infty\)  \\ \hline
\( P(\omega) \) & 4 & \(3,76 \) & 1 & \( 1,1 \) & 0  \\ \hline
\( Q(\omega) \) & 0 & \( -1,04 \) & 0 & \( -2,3 \) & 0  \\ \hline
\end{tabular}
\end{center}
\end{table}
Szkicuję charakterystykę:
\begin{figure}[H]
\centerline{\includegraphics[scale=0.2]{kol2_14.jpg}}
\caption{Szkic charakterystyki amp-faz}
\label{fig:kol2_14}
\end{figure}
Charakterystyka obejmuje sumarycznie 0 razy punkt \( ( -1, j0 ) \), układ zamknięty jest asymptotycznie stabilny, bo liczba ta jest równa liczbie pierwiastków wielomianu \( M(s) \) w \( C^+ \).

\newpage
\begin{framed}
\textbf{Zadanie 6 - Zadania od Bauera do kolokwium I } \\ 
Korzystając z kryterium Michajłowa zbadać asymptotyczną stabilność układu opisanego transmitancją
\begin{align*}
G(s)=\frac{s^{2}+1}{s^{3}+s^{2}+9s+4}
\end{align*}
\end{framed}

$M(s)=s^3+s^2+9s+4$\\
$M(j\omega)=-j\omega^3-\omega^2+9j\omega+4=\underbrace{4-\omega^2}_{Re}+j\underbrace{(9\omega -\omega ^{3})}_{Im}$\\
Tebelka
\begin{table}[H]
\begin{center}
\begin{tabular}{|p{2cm}|p{2cm}|p{2cm}|p{2cm}|p{2cm}|}
\hline
\( \omega \) & 0 & \( 2 \) & 3 & \( \frac{n\pi}{2}=\frac{3\pi}{2} \) \\ \hline
\( P(\omega) \) & 4 & \( 0  \) & \( (-) \) &  0  \\ \hline
\( Q(\omega) \) & 0 & \( (+) \) & 0 & \( -\infty \)  \\ \hline
\end{tabular}
\end{center}
\end{table}

\begin{figure}[!h]
\begin{tikzpicture}
	\draw [color=red, thick](2,0) arc (0:90:2 and 1.5);
	\draw [color=red, thick](0,1.5) arc (90:180:1 and 1.5);
	\draw [color=red, thick](-1,0) arc (180:220:1.5 and 3);

	\draw[thick][->](-3,0)--(3,0) node [right=3pt]{$Re$};
	\draw[thick][->](0,-3)--(0,3) node [right=3pt]{$Im$};

	\draw (-0.1,-2) -- (0.1,-2);
	\draw (2,-0.1) -- (2,0.1) node [below=4pt]{{4}};
\end{tikzpicture}
\end{figure}
Wektor zakreśla kąt \( \Delta \alpha = \frac{3\pi}{2} \) dla \( \omega \in [0, \infty ) \). Sprawdzam więc warunek:
\begin{align*}
&\Delta \alpha = \frac{n\pi}{2}-m\pi \\
&3\pi = n\pi-2m\pi \\
&3\pi =3\pi -2m\pi \\
& 0 = -2m\pi \\
& m = 0 \\
\end{align*}
m to liczba pierwiastków wielomianu \( M(s) \) w \( C^{+} \), a zatem układ jest asymptotycznie stabilny. 

\newpage
\begin{framed}
\textbf{Zadanie 7 - Informatyka Modelowanie } \\ 
Rozwiązanie równania różniczkowego\\
	$\ddot{x}(t)+\dot{x}(t)=-x(t)+12\sin({\omega t})$\\
	gdzie $x(0)=0$, ($\dot{x}(0)$ - w domysle), $t \geq 0$ ma postać\\
	$x(t)=f(t)+A\sin({\omega t})$\\
znaleźć takie $\omega$, dla którego $A$ jest największe\\
\end{framed}
$u(t)=\sin({\omega t})$\\
$y(t)=x(t)$\\\\
$\begin{cases} \ddot{x}(t)+\dot{x}(t)=-x(t)+12u(t)\\y(t)=x(t)\end{cases}$\\\\
$\left.\begin{cases} u(t)=\frac{\ddot{x}(t)+\dot{x}(t)+x(t)}{12}\\y(t)=x(t)\end{cases}\right|\mathscr{L}$\\\\
$\begin{cases} U(s)=\frac{s^2\cdot X(s)-s\cdot x(0)-\dot{x}(0)+sX(s)-x(0)+X(s)}{12}\\Y(s)=X(s)\end{cases}$\\\\
$\begin{cases} U(s)=\frac{s^2\cdot X(s)+sX(s)+X(s)}{12}\\Y(s)=X(s)\end{cases}$\\\\\\
$G(S)=\frac{Y(s)}{U(s)}=\frac{12\cdot \cancel{X(s)}}{\cancel{X(s)}\cdot (s^2+s+1)}=\frac{12}{s^2+s+1}$\\\\
$G(j\omega)=\frac{12}{-\omega^2+j\omega+1}=-\frac{12\cdot \omega^2-12}{\omega^4-\omega^2+1}-j\frac{12\cdot \omega}{\omega^4-\omega^2+1}$\\
$A_y=A_u\cdot ku(\omega)$\\
$ku(\omega)=|G(j\omega)|=\sqrt{\left(\frac{12\cdot \omega^2-12}{\omega^4-\omega^2+1}\right)^2+\left(\frac{12\cdot \omega}{\omega^4-\omega^2+1}\right)^2}=12\cdot \sqrt{\frac{1}{\omega^4-\omega^2+1}}$\\\\\\
$A_y$ będzie max., gdy $ku(\omega)$ będzie max., tj. $\sqrt{\omega^4-\omega^2+1}$ będzie min.
$\omega \geq 0$, $min(\sqrt{\omega^4-\omega^2+1})$ dla $\omega=\frac{1}{\sqrt{2}}$

\newpage
\begin{framed}
\textbf{Zadanie 8 - Egzamin } \\ 
Wykorzystując kryterium Nyquista sprawdzić czy układ zamknięty będzie asymptotycznie stabilny, jeśli układ otwarty opisany transmitancją:
\begin{align*}
G(s)=\frac{6}{s^3+2s^2+2s+40}
\end{align*}
połączono szeregowo z regulatorem proporcjonalnym \( G_K = 100 \). 
\end{framed}


\newpage
\section{I metoda Lapunowa}
\subsection{Zbiór zadań}
\begin{framed}
\textbf{\colorbox{green}{Zadanie 1 - Informatyka Modelowanie} } \\ 
Dany jest system dynamiczny 
\begin{align*}
\dot{x}(t)=\cos ( x(t) ) e^{-x(t)^{2}} \\
\end{align*}
Wyznaczyć jego punkty równowagi i za pomocą I metody Lapunowa zbadać ich stabilność.
\end{framed}

\begin{framed}
\textbf{\colorbox{green}{Zadanie 2 - Informatyka Modelowanie} } \\ 
Dany jest system dynamiczny 
\begin{align*}
&\dot{x}_{1}(t)=x_{2}(t) \\
&\dot{x}_{2}(t)=-2x_{1}(t)-3x_{1}(t)^{2}-x_{2}(t)
\end{align*}
Wyznaczyć jego punkty równowagi i za pomocą I metody Lapunowa zbadać ich stabilność.
\end{framed}

\begin{framed}
\textbf{\colorbox{green}{Zadanie 3 - Informatyka Modelowanie} } \\ 
Dla jakich wartości parametru \( \epsilon \) zerowy punkt równowagi układu zwanego oscylatorem Van der Pola będzie niestabilny.  
\begin{align*}
\ddot{x}(t)- \epsilon ( 1-x(t)^{2})\dot{x}(t)+x(t) = 0
\end{align*}
\end{framed}

\begin{framed}
\textbf{\colorbox{green}{Zadanie 4 - Informatyka Modelowanie} } \\ 
Dla jakich wartości parametru \( a \) linearyzacja przestaje spełniać warunki twierdzenia Grobmana-Hartmana dla układu opisanego równaniami:
\begin{align*}
\dot{x}_{1}(t)=-x_{2}(t)+(a-x_{1}(t)^{2}-x_{2}(t)^{2})x_{1}(t) \\
\dot{x}_{2}(t)=x_{1}(t)+(a-x_{1}(t)^{2}-x_{2}(t)^{2})x_{2}(t)
\end{align*}
\end{framed}

\newpage
\begin{framed}
\textbf{\colorbox{green}{Zadanie 5 - Informatyka Modelowanie} } \\ 
Dla jakich wartości parametru \( a \) zerowy punkt równowagi układu opisanego równaniami:
\begin{align*}
\dot{x}_{1}(t)=-x_{2}(t)+(a-x_{1}(t)^{2}-x_{2}(t)^{2})x_{1}(t) \\
\dot{x}_{2}(t)=x_{1}(t)+(a-x_{1}(t)^{2}-x_{2}(t)^{2})x_{2}(t)
\end{align*}
będzie niestabilny.
\end{framed}

\begin{framed}
\textbf{\colorbox{green}{Zadanie 6 - Informatyka Modelowanie} } \\ 
Dla jakich wartości parametru \( a \) zerowy punkt równowagi układu opisanego równaniami:
\begin{align*}
\dot{x}_{1}(t)=-x_{2}(t)+(a-x_{1}(t)^{2}-x_{2}(t)^{2})x_{1}(t) \\
\dot{x}_{2}(t)=x_{1}(t)+(a-x_{1}(t)^{2}-x_{2}(t)^{2})x_{2}(t)
\end{align*}
będzie asymptotycznie stabilny.
\end{framed}


\begin{framed}
\textbf{\colorbox{yellow}{Zadanie 7 - Egzamin}} \\ 
Do układu opisanego równaniami
\begin{align*}
&\dot{x}_{1}(t)= (u(t)-x_1(t))(1+x_2^2(t)) \\
&\dot{x}_{2}(t)= (x_1(t)-2x_2(t))(1+x_1^2(t)) \\
&y(t)=x_2(t)
\end{align*}
wpięto sprzężenie zwrotne postaci \( u(t)=-Ky(t) \). Znaleźć punkty równowagi układu zamkniętego w zależności od parametru \( K \). Dla jakich wartości parametru K zerowy punkt równowagi jest lokalnie asymptotycznie stabilny. 
\end{framed}


\newpage
\subsection{Rozwiązania zadań ze zbioru}

\begin{framed}
\textbf{Zadanie 1 - Informatyka Modelowanie } \\ 
Dany jest system dynamiczny 
\begin{align*}
\dot{x}(t)=\cos ( x(t) ) e^{-x(t)^{2}} \\
\end{align*}
Wyznaczyć jego punkty równowagi i za pomocą I metody Lapunowa zbadać ich stabilność.
\end{framed}

$\dot{x}(t)=\cos(x(t))e^{-x(t)^2}$\\
$\dot{x}(t)=f(x(t))$\\
$x_r$ jest punktem równowagi $ \Leftrightarrow f(x_r)=0$\\
$f(x_r)=\cos(x_r)\cdot \underbrace{e^{-x_r^2}}_{<0}=0 \Rightarrow cos(x_r)=0 \Rightarrow x_r=\frac{\pi}{2}+k\pi, k \in \mathbb{Z}$\\
\\
$J(x)=\frac{\partial f}{\partial x} = - \sin(x) \cdot e^{-x^2}+\cos(x) \cdot (-2xe^{-x^2})=-e^{-x^2}(\sin(x)+2x\cos(x))$\\ \\
$J(x_r)=\underbrace{-e^{-(\frac{\pi}{2}+k\pi)^2}}_{<0}(\underbrace{\sin(\frac{\pi}{2}+k\pi)}_{=1 \vee =-1})+\underbrace{2(\frac{\pi}{2}+k\pi)\cos(\frac{\pi}{2}+k\pi)}_{=0}=-e^{-(\frac{\pi}{2}+k\pi)^2} (\sin(\frac{\pi}{2}+k\pi))$\\

$\lambda = -e^{-(\frac{\pi}{2}+k\pi)^2}\cdot \sin(\frac{\pi}{2}+k\pi)$\\ \\
Niestabilny gdy \( \lambda > 0 \) :\\ \\
$ -e^{-(\frac{\pi}{2}+k\pi)^2}\cdot \sin(\frac{\pi}{2}+k\pi)>0 \Rightarrow \sin(\frac{\pi}{2}+k\pi)=-1 \Rightarrow x_r=\frac{\pi}{2}+(2k\pi+1)\pi,\ \  k \in \mathbb{Z}$\\ \\
Stabilny asymptotycznie gdy \( \lambda < 0 \) : \\ \\
$-e^{-(\frac{\pi}{2}+k\pi)^2}\cdot \sin(\frac{\pi}{2}+k\pi)>0 \Rightarrow \sin(\frac{\pi}{2}+k\pi)= 1 \Rightarrow x_r=\frac{\pi}{2}+2k\pi,\ \  k \in \mathbb{Z}$\\ 



\newpage
\begin{framed}
\textbf{Zadanie 2 - Informatyka Modelowanie } \\ 
Dany jest system dynamiczny 
\begin{align*}
&\dot{x}_{1}(t)=x_{2}(t) \\
&\dot{x}_{2}(t)=-2x_{1}(t)-3x_{1}(t)^{2}-x_{2}(t)
\end{align*}
Wyznaczyć jego punkty równowagi i za pomocą I metody Lapunowa zbadać ich stabilność.
\end{framed}


$J(x)=\left[ \begin{array}{cc}  0&1\\-2-6x_1 & -1   \end{array}\right]$\\
$\begin{cases}x_2=0\\-2x_1-3x_1^2-x_2=0\end{cases}$\\
$-2x_1-3x_1^2=0$\\
$x_1(2+3x_1)=0$\\
$x_1=0\ \  \vee \ \ x_1=-\frac23$\\
$\begin{array}{lll}
x_r= \left[ \begin{array}{c}   0\\0    \end{array}\right] &\ \ \ \vee \ \ \ & x_r= \left[ \begin{array}{c}   -\frac 23\\0    \end{array}\right] \\
J(x_r)=\left[ \begin{array}{cc}  0&1\\-2&-1    \end{array}\right] && J(x_r)=\left[ \begin{array}{cc}   0&1\\2&-1    \end{array}\right]\\
\left| \begin{array}{cc}  -\lambda & 1 \\-2&-1-\lambda    \end{array}\right|&&\left| \begin{array}{cc}  -\lambda & 1 \\2&-1-\lambda    \end{array}\right|\\
=(-\lambda)(-1-\lambda)+2=\lambda^2+\lambda+2&&=(-\lambda)(-1-\lambda)-2=\lambda^2+\lambda-2\\
\Delta=-7&&\Delta=9\\
\lambda=-\frac 12 \pm \frac{\sqrt 7}{2}i&&\lambda=\frac{-1 \pm 3}{2}\\
\lambda = -\frac 12 <0 \Rightarrow \text{Stabilny} && \lambda = 1 \ \ \vee \ \ \lambda=-2\\
 && \lambda = 1 >0 \Rightarrow \text{Niestabilny}
\end{array}$\\

\newpage
\begin{framed}
\textbf{Zadanie 3 - Informatyka Modelowanie } \\ 
Dla jakich wartości parametru \( \epsilon \) zerowy punkt równowagi układu zwanego oscylatorem Van der Pola będzie niestabilny.  
\begin{align*}
\ddot{x}(t)- \epsilon ( 1-x(t)^{2})\dot{x}(t)+x(t) = 0
\end{align*}
\end{framed}
$\begin{cases} x_1=x \\ x_2 =\dot{x} \end{cases} \begin{cases} \dot{x}_1=\dot{x}=x_2 \\ \dot{x}_2=\ddot{x}= \epsilon(1-x(t)^2) \cdot \dot{x}(t)-x(t)=\epsilon(1-x_1^2) \cdot x_2 - x_1 \end{cases}$\\
$f(x)=\left[ \begin{array}{c}   x_2  \\  \epsilon(1-x_1^2) \cdot x_2 - x_1  \end{array}\right] = \left[ \begin{array}{c}  f_1(x)   \\ f_2(x)   \end{array}\right]$\\
$\begin{cases} x_2=0 \\ \epsilon(1-x_1^2) \cdot x_2 - x_1 = 0\end {cases} \Rightarrow \begin{cases}x_2=0 \\ x_1 = 0 \end {cases} \ \ \ x_r = \left[ \begin{array}{c}     0\\0   \end{array}\right]$\\
$J(x)=\left[ \begin{array}{cc}   0 &1  \\ -2\epsilon x_1 x_2 -1 & \epsilon(1-x_1^2)   \end{array}\right] \ \ \ \ \ \ \ 
{\color{lightgray}\boxed{\left[ \begin{array}{cc}    \frac{\partial f_1}{\partial x_1} &\frac{\partial f_1}{\partial x_2} \\ \frac{\partial f_2}{\partial x_1} & \frac{\partial f_2}{\partial x_2}   \end{array}\right]}}$\\
$J(x_r)= \left[ \begin{array}{cc}    0&1 \\-1 & \epsilon    \end{array}\right]$\\
Wielomian charakterystyczny:\\
$(- \lambda)(\epsilon - \lambda)+1 = \lambda^2 - \lambda \epsilon +1 = 0$\\ \\ 
$\Delta = \epsilon^2-4 \Rightarrow \lambda =\frac{\epsilon \pm \sqrt{\epsilon^2-4}}{2}$\\
Niestabilny jeżeli część rzeczywista $>0 \Rightarrow \frac{\epsilon}{2}>0 \Rightarrow \boxed{\epsilon >0}$\\
Asymptotycznie stabilny gdy $\left[ \begin{array}{cc}    -\epsilon & 0 \\ 1 & 1   \end{array}\right]$ ( kr. Hurwitza ) $-\epsilon>0 \Rightarrow \boxed{\epsilon<0}$\\
Dla macierzy systemu zlinearyzowanego można już stosować kryteria stabilności dotyczące układów liniowych.


\newpage
\begin{framed}
\textbf{Zadanie 4 - Informatyka Modelowanie } \\ 
Dla jakich wartości parametru \( a \) linearyzacja przestaje spełniać warunki twierdzenia Grobmana-Hartmana dla układu opisanego równaniami:
\begin{align*}
\dot{x}_{1}(t)=-x_{2}(t)+(a-x_{1}(t)^{2}-x_{2}(t)^{2})x_{1}(t) \\
\dot{x}_{2}(t)=x_{1}(t)+(a-x_{1}(t)^{2}-x_{2}(t)^{2})x_{2}(t)
\end{align*}
\end{framed}
$\dot{x_1}(t)=-x_2(t)+(a-x_1(t)^2-x_2(t)^2)x_1(t)=f_1(x(t))$\\
$\dot{x_2}(t)=x_1(t)+(a-x_1(t)^2-x_2(t)^2)x_2(t)=f_2(x(t))$\\
$f(x)=\left[ \begin{array}{c}     f_1(x) \\ f_2(x)   \end{array}\right]$\\
$\begin{cases} -x_2+(a-x_1^2-x_2^2)x_1=0 \\ x_1+(a-x_1^2-x_2^2)x_2=0\end{cases}  $\\
Zauważamy, że albo $x_1=x_2=0$ albo dla $x_2 \neq 0 \wedge x_1 \neq 0$ :\\
$\begin{cases} -\frac{x_2}{x_1}+(a-x_1^2-x_2^2)=0 \\ \frac{x_1}{x_2}+(a-x_1^2-x_2^2)=0\end{cases}  \Rightarrow \frac{-x_2}{x_1} = \frac{x_1}{x_2} \Rightarrow -x_2^2=x_1^2 \Rightarrow x_1=x_2=0$ (sprzeczność)\\
więc $x_r=\left[ \begin{array}{c}     0\\0   \end{array}\right]$\\
$J(x)=\left[ \begin{array}{cc}   a-3x_1^2-x_2^2 & -1-2x_2x_1 \\ 1-2x_1x_2 & a-x_1^2-3x_2^2    \end{array}\right]$\\
$J(x_r)=\left[ \begin{array}{cc}    a & -1 \\ 1 & a    \end{array}\right]$\\
z tw. Grobmana-Hartmana:\\
$\begin{array}{ll}
\det(j\omega I-J(x_r)) \neq 0, \ \ \ \omega \in \mathbb{R} & J(x_r)\text{ nie ma wartości własnych na osi urojonej}\\
 \left| \begin{array}{cc}     j\omega-a& -1 \\ 1 & j\omega-a    \end{array}\right|=0 &  \left[ \begin{array}{cc}    a-\lambda & -1 \\ 1 & a- \lambda   \end{array}\right]\\
(j\omega-a)^2+1=0 & (a-\lambda)^2+1=0\\
j\omega-a= \pm j \Rightarrow \boxed{ a=0} & a^2-2a\lambda +\lambda^2+1 =0\\
&\lambda^2-2a\lambda+a^2+1 = 0\\
&\Delta=4a^2-4a^2-4\\
&\sqrt{\Delta}=2i\\
&\lambda=\frac{2a \pm 2i}{2} = a \pm i\\
&\text{dla } a=0 \text{ wartości własne są na osi urojonej}
\end{array}$\\



\newpage
\begin{framed}
\textbf{Zadanie 5 - Informatyka Modelowanie } \\ 
Dla jakich wartości parametru \( a \) zerowy punkt równowagi układu opisanego równaniami:
\begin{align*}
\dot{x}_{1}(t)=-x_{2}(t)+(a-x_{1}(t)^{2}-x_{2}(t)^{2})x_{1}(t) \\
\dot{x}_{2}(t)=x_{1}(t)+(a-x_{1}(t)^{2}-x_{2}(t)^{2})x_{2}(t)
\end{align*}
będzie niestabilny.
\end{framed}
$x_r=\left[ \begin{array}{c}     0\\0   \end{array}\right]$\\
$J(x)=\left[ \begin{array}{cc}   a-3x_1^2-x_2^2 & -1-2x_2x_1 \\ 1-2x_1x_2 & a-x_1^2-3x_2^2    \end{array}\right]$\\
$J(x_r)=\left[ \begin{array}{cc}    a & -1 \\ 1 & a    \end{array}\right]$\\
$\left| \begin{array}{cc}     a-\lambda & -1 \\ 1 & a-\lambda   \end{array}\right|=0$\\
$\lambda^2-2a\lambda+a^2+1=0$\\
$\Delta=-4$\\
$\lambda = \frac{2a \pm 2i}{2}=a \pm i$\\
niestabilny, gdy $Re(\lambda)>0$\\
$a>0$\\




\newpage
\begin{framed}
\textbf{Zadanie 6 - Informatyka Modelowanie } \\ 
Dla jakich wartości parametru \( a \) zerowy punkt równowagi układu opisanego równaniami:
\begin{align*}
\dot{x}_{1}(t)=-x_{2}(t)+(a-x_{1}(t)^{2}-x_{2}(t)^{2})x_{1}(t) \\
\dot{x}_{2}(t)=x_{1}(t)+(a-x_{1}(t)^{2}-x_{2}(t)^{2})x_{2}(t)
\end{align*}
będzie asymptotycznie stabilny.
\end{framed}
$x_r=\left[ \begin{array}{c}     0\\0   \end{array}\right]$\\
$J(x)=\left[ \begin{array}{cc}   a-3x_1^2-x_2^2 & -1-2x_2x_1 \\ 1-2x_1x_2 & a-x_1^2-3x_2^2    \end{array}\right]$\\
$J(x_r)=\left[ \begin{array}{cc}    a & -1 \\ 1 & a    \end{array}\right]$\\
$\left| \begin{array}{cc}     a-\lambda & -1 \\ 1 & a-\lambda   \end{array}\right|=0$\\
$\lambda^2-2a\lambda+a^2+1=0$\\
Kr. Hurwitza:\\
$-2a>0 \Rightarrow a<0$\\
$-2a(a^2+1)>0 \Rightarrow a<0$\\

\newpage
\begin{framed}
\textbf{Zadanie 7 - Egzamin} \\ 
Do układu opisanego równaniami
\begin{align*}
&\dot{x}_{1}(t)= (u(t)-x_1(t))(1+x_2^2(t)) \\
&\dot{x}_{2}(t)= (x_1(t)-2x_2(t))(1+x_1^2(t)) \\
&y(t)=x_2(t)
\end{align*}
wpięto sprzężenie zwrotne postaci \( u(t)=-Ky(t) \). Znaleźć punkty równowagi układu zamkniętego w zależności od parametru \( K \). Dla jakich wartości parametru K zerowy punkt równowagi jest lokalnie asymptotycznie stabilny. 
\end{framed}
\colorbox{red}{W niezerowym punkcie równowagi x1 nie jest równe 0 - do poprawy}
\begin{align*}
&u(t)=-ky(t)=-kx_2(t) \\
&\dot{x}_{1}(t)= (-kx_2(t)-x_1(t))(1+x_2^2(t)) \\
&\dot{x}_{2}(t)= (x_1(t)-2x_2(t))(1+x_1^2(t)) \\
&\text{Zapisuje dla wygody}: \\
&\dot{x}_{1}= (-kx_2-x_1)(1+x_2^2) \\
&\dot{x}_{2}= (x_1-2x_2)(1+x_1^2) \\
&\text{Szukam punktów równowagi, widać, że tylko jeden z czynników może być równy 0}: \\
&\begin{cases}
-kx_2-x_1 = 0 \\
x_1-2x_2 = 0
\end{cases} \implies
\begin{cases}
-kx_2 = x_1 \\
-kx_2-2x_2 = 0
\end{cases}
\implies
\begin{cases}
-kx_2 = x_1 \\
-kx_2-2x_2 = 0
\end{cases}
\implies
\begin{cases}
-kx_2 = x_1 \\
x_2(-k-2) = 0
\end{cases}
\end{align*}
\begin{align*}
&\begin{cases}
x_2 = 0 \\ 
x_1 = 0 \\
k \in \mathbb{R} - \{-2\}
\end{cases}
\lor
\begin{cases}
-k-2 = 0 \\
x_2 \in \mathbb{R} \\
x_1 = 0
\end{cases}
\implies
\begin{cases}
k = -2 \\
x_2 \in \mathbb{R} \\
x_1 = 0
\end{cases}
\end{align*}
Wymnażając prawe strony równań wyjściowych otrzymano:
\begin{align*}
\dot{x}_1(t)=-kx_2-kx_2^3-x_1-x_1x_2^3 \\
\dot{x}_2(t)=x_1+x_1^3-2x_2-2x_2x_1^2
\end{align*}
Macierz Jacobiego:
\begin{align*}
&J = 
\begin{bmatrix}
\frac{\partial f_1}{x_1} & \frac{\partial f_1}{x_2} \\
\frac{\partial f_2}{x_1} &  \frac{\partial f_2}{x_2}
\end{bmatrix}
=
\begin{bmatrix} 
-1-x_2^2 & -k-3kx_2^2-2x_1x_2 \\
1+3x_1^2-4x_2 & -2-2x_1^2 
\end{bmatrix}
\end{align*}
Po podstawieniu zerowego punktu równowagi:
\begin{align*}
\begin{bmatrix} 
-1 & -k \\
1 & -2 
\end{bmatrix}
\end{align*}
Wielomian charakterystyczny:
\begin{align*}
\text{det}(\lambda I - A ) = 
\text{det}
\begin{bmatrix} 
\lambda +1 & k \\
-1 & \lambda +2 
\end{bmatrix} = 
\lambda ^{2} + 3\lambda + 2 + k
\end{align*}
Równanie jest drugiego rzędu więc asymptotyczna stabilność będzie dla \( k > -2 \). 

\newpage
\section{II metoda Lapunowa}

\subsection{Zbiór zadań}
\begin{framed}
\textbf{\colorbox{green}{Zadanie 1 - Niebieski skrypt} } \\ 
Zbadać stabilność nieliniowego systemu dynamicznego
\begin{align*}
\dot{x}(t)&=-x^{3} \\
\end{align*}
\end{framed}

\begin{framed}
\textbf{\colorbox{green}{Zadanie 2 - Niebieski skrypt} } \\ 
Zbadać stabilność nieliniowego systemu dynamicznego
\begin{align*}
\dot{x}(t)&=-x+x^{3} \\
\end{align*}
\end{framed}

\begin{framed}
\textbf{\colorbox{green}{Zadanie 3 - Niebieski skrypt / Egzamin } } \\ 
Zbadać stabilność punktów równowagi nieliniowego systemu dynamicznego
\begin{align*}
\dot{x_{1}}(t)&=x_{2}(t) - \sin x_{1}(t) \\
\dot{x_{2}}(t)&=-x_{1}^{3}
\end{align*}
\end{framed}

\begin{framed}
\textbf{\colorbox{green}{Zadanie 4 - Zadania od Bauera do kolokwium II} } \\ 
Dany jest system dynamiczny 
\begin{align*}
\dot{x}(t)&=x(t)- \sin (x(t))
\end{align*}
Wyznaczyć jego punkty równowagi i za pomocą II metody Lapunowa zbadać ich stabilność. 
\end{framed}

\begin{framed}
\textbf{\colorbox{green}{Zadanie 5 - Zadania od Bauera do kolokwium II }} \\ 
Dany jest system dynamiczny 
\begin{align*}
\dot{x}(t)&=-x(t)+2x(t)^{2}
\end{align*}
Za pomocą metody La Salle'a określić przybliżony zbiór przyciągania zerowego punktu równowagi. 
\end{framed}

\begin{framed}
\textbf{\colorbox{green}{Zadanie 6 - Zadania od Bauera do kolokwium II }} \\ 
Dany jest system dynamiczny 
\begin{align*}
\dot{x_{1}}(t)&=x_{2}(t) \\
\dot{x_{2}}(t)&=-2x_{1}(t)-x_{2}(t)
\end{align*}
Za pomocą II metody Lapunowa pokazać asymptotyczną stabilność zerowego punktu równowagi ( wskazówka: funkcjonał Lapunowa można wyznaczyć za pomocą równania Lapunowa ). 
\end{framed}

\begin{framed}
\textbf{\colorbox{green}{Zadanie 7 - Zadania od Bauera do kolokwium II }} \\ 
Układ postaci
\begin{align*}
&\dot{x}(t)=Ax(t)+Bu(t) \\
&y(t)=Cx(t) 
\end{align*}
z macierzami:
\begin{align*}
A =
\begin{bmatrix}
0 & 1 \\
-1 & -1 
\end{bmatrix}
B =
\begin{bmatrix}
0  \\
-1  
\end{bmatrix}
C^{T} =
\begin{bmatrix}
1  \\
0 
\end{bmatrix}
\end{align*}
objęto ujemnym, nieliniowym, sprzężeniem zwrotnym od wyjścia \( f(y)=y^{3} \). Uzasadnić, że forma kwadratowa \( V(x)=x^{T}Hx \), gdzie \( H \) jest rozwiązaniem równania: 
\begin{align*}
A^{T}H+HA=-C^{T}C
\end{align*} 
jest funkcjonałem Lapunowa dla tego układu. Znaleźć maksymalny obszar atrakcji zerowego punktu równowagi.
\end{framed}

\newpage
\subsection{Rozwiązania zadań ze zbioru}

\begin{framed}
\textbf{Zadanie 1 - Niebieski skrypt } \\ 
Zbadać stabilność nieliniowego systemu dynamicznego
\begin{align*}
\dot{x}(t)&=-x^{3} \\
\end{align*}
\end{framed}

System jest nieliniowy, stacjonarny, skończenie wymiarowy, ciągły. Załóżmy, że istnieje potok rozwiązań układu w otoczeniu punktu równowagi. \\
Punkty równowagi:
\begin{align*}
-x^{3} = 0
\implies
x = 0 
\end{align*}
Istnieje tylko jeden punkt równowagi \( x^{*}=0 \). \\
Kandydat na funkcjonał Lapunowa:
\begin{equation*}
V(x)=\frac{1}{2}x^{2}
\end{equation*}
\(V(x)\) spełnia założenia:
\begin{itemize}
\item Istnieje otoczenie \( \Omega_{1} \) takie, że $ V(x) > 0 \quad \forall x \in \Omega_{1} - \{0\} $
\item $ V(x) $ jest ciągły
\item $ V(x) $ ma ciągłe pochodne cząstkowe pierwszego rzędu
\item $ V(0) = 0 $
\end{itemize}
Obliczam \( \dot{V}(x) \) : \\
\begin{equation*}
\dot{V} = 
\begin{bmatrix}
\frac{\partial V}{\partial x}
\end{bmatrix}
\begin{bmatrix}
f(x)
\end{bmatrix}
=
x \cdot (-x^{3}) =
-x^{4} 
\end{equation*}
Istnieje więc otoczenie \( \Omega_{2} \) punktu \( x^{*} = 0 \) takie, że spełnione są warunki: 
\begin{equation*}
V(x)>0 \land \dot{V}(x) < 0 \quad \forall x \in \Omega_{2} - \{0\}
\end{equation*}
Punkt równowagi \( x^{*}=0 \) jest więc stabilny asymptotycznie. Jako, że jest to jedyny punkt równowagi może być on także globalnie asymptotycznie stabilny. Zauważmy, że zachodzi dla:
\begin{equation*}
||x|| \rightarrow \infty \implies V(x) \rightarrow \infty
\end{equation*} 
więc punkt \( x^{*}=0 \) jest globalnie asymptotycznie stabilny. 

\newpage

\begin{framed}
\textbf{Zadanie 2 - Niebieski skrypt} \\ 
Zbadać stabilność nieliniowego systemu dynamicznego
\begin{align*}
\dot{x}(t)&=-x+x^{3} \\
\end{align*}
\end{framed}

System jest nieliniowy, stacjonarny, skończenie wymiarowy, ciągły. Załóżmy, że istnieje potok rozwiązań układu w otoczeniu punktu równowagi. \\
Punkty równowagi:
\begin{align*}
& -x+x^{3} = 0 \\
& x(-1+x^{2}) = 0 \\
& x(x-1)(x+1) = 0 \\
& x^{*} = 0 \lor x^{*} = -1 \lor x^{*} = 1 
\end{align*}
Istnieją trzy punkty równowagi. \\
\textbf{Przypadek I} \( x^{*} = 0 \) \\
Kandydat na funkcjonał Lapunowa:  
\begin{equation*}
V(x)=\frac{1}{2}x^{2}
\end{equation*}
\(V(x)\) spełnia założenia:
\begin{itemize}
\item Istnieje otoczenie \( \Omega_{1} \) takie, że $ V(x) > 0 \quad \forall x \in \Omega_{1} - \{0\} $
\item $ V(x) $ jest ciągły
\item $ V(x) $ ma ciągłe pochodne cząstkowe pierwszego rzędu
\item $ V(0) = 0 $
\end{itemize}
Obliczam \( \dot{V}(x) \) : \\
\begin{equation*}
\dot{V} = 
\begin{bmatrix}
\frac{\partial V}{\partial x}
\end{bmatrix}
\begin{bmatrix}
f(x)
\end{bmatrix}
=
x \cdot (-x+x^{3}) = -x^{2}+x^{4} = x^{2}(x^{2}-1) = x^{2}(x-1)(x+1)
\end{equation*}
Istnieje więc otoczenie \( \Omega_{2} \) punktu \( x^{*} = 0 \) takie, że spełnione są warunki: 
\begin{equation*}
V(x)>0 \land \dot{V}(x) < 0 \quad \forall x \in \Omega_{2} - \{0\}
\end{equation*}
\begin{figure}[H]
\centerline{\includegraphics[scale=0.5]{kol2_8.jpg}}
\caption{Pochodna}
\label{fig:kol2_8}
\end{figure}
Punkt równowagi \( x^{*}=0 \) jest więc stabilny asymptotycznie. Jako, że nie jest to jedyny punkt równowagi nie rozpatrujemy globalnej asymptotycznej stabilności.

\textbf{Przypadek II} \( x^{*} = 1 \) \\
Lapunow II jest sformułowany dla zerowego punktu równowagi, aby badać niezerowe trzeba wykonać przesunięcie układu współrzędnych podstawiając \( x = x - x^{*} ] \). W tym przypadku podstawiam \( x = x - 1 \). \\
Wtedy system ma postać:
\begin{align*}
& \frac{d(x(t)-1)}{dt} = -( x(t) - 1 ) + ( x(t) - 1 )^{3} \\
& \frac{dx}{dt} = -x+1+x^{3}-3x^{2}+3x-1=x^{3}-3x^{2}+2x \\
& \dot{x}=x^{3}-3x^{2}+2x 
\end{align*}
Przyjmuje funkcjonał Lapunowa jak w przypadku I i liczę pochodną:  
\begin{equation*}
\dot{V} = 
\begin{bmatrix}
\frac{\partial V}{\partial x}
\end{bmatrix}
\begin{bmatrix}
f(x)
\end{bmatrix}
=
x \cdot ( x^{3}-3x^{2}+2x ) = x^{2}(x-2)(x-1)
\end{equation*}
\begin{figure}[H]
\centerline{\includegraphics[scale=0.5]{kol2_9.jpg}}
\caption{Pochodna}
\label{fig:kol2_9}
\end{figure}
Pochodna w otoczeniu 0 ma więc wartość dodatnią, co oznacza, że punkt równowagi \( x^{*} = -1 \) jest niestabilny.

\textbf{Przypadek III} \( x^{*} = -1 \) \\
W tym przypadku podstawiam \( x = x + 1 \). \\
Wtedy system ma postać:
\begin{align*}
& \frac{d(x(t)+1)}{dt} = -( x(t) + 1 ) + ( x(t) + 1 )^{3} \\
& \frac{dx}{dt} = -x-1+x^{3}+3x^{2}+3x+1=x^{3}+3x^{2}+2x \\
& \dot{x}=x^{3}-3x^{2}+2x 
\end{align*}
Przyjmuje funkcjonał Lapunowa jak w przypadku I i liczę pochodną:  
\begin{equation*}
\dot{V} = 
\begin{bmatrix}
\frac{\partial V}{\partial x}
\end{bmatrix}
\begin{bmatrix}
f(x)
\end{bmatrix}
=
x \cdot ( x^{3}+3x^{2}+2x ) = x^{2}(x+2)(x+1)
\end{equation*}
\begin{figure}[H]
\centerline{\includegraphics[scale=0.5]{kol2_10.jpg}}
\caption{Pochodna}
\label{fig:kol2_10}
\end{figure}
Pochodna w otoczeniu 0 ma więc wartość dodatnią, co oznacza, że punkt równowagi \( x^{*} = 1 \) jest niestabilny.

\newpage
\begin{framed}
\textbf{Zadanie 3 - Niebieski skrypt / Egzamin } \\ 
Zbadać stabilność punktów równowagi nieliniowego systemu dynamicznego
\begin{align*}
\dot{x_{1}}(t)&=x_{2}(t) - \sin x_{1}(t) \\
\dot{x_{2}}(t)&=-x_{1}^{3}
\end{align*}
\end{framed}
System jest nieliniowy, stacjonarny, skończenie wymiarowy, ciągły. Załóżmy, że istnieje potok rozwiązań układu w otoczeniu punktu równowagi. \\
Punkty równowagi:
\begin{align*}
\begin{cases}
x_{2} - sinx_{1} = 0 \\
-x_{1}^{3} = 0
\end{cases}
\implies
\begin{cases}
x_{1}=0 \\
x_{2} = 0
\end{cases}
\end{align*}
System ma jeden punkt równowagi \(x^{*}=0\). \\
Wybieramy funkcjonał Lapunowa, gdybyśmy wzięli funkcjonał energetyczny, to w tym przypadku nie udałoby się nam określić jednoznacznie znaku pochodnej w otoczeniu 0 ( można przeliczyć i sprawdzić ), na podstawie czego nie można wnioskować o niczym. \\
System jest system postaci:
\begin{align*}
\dot{x_{1}}(t)&=x_{2}(t) - h(x_{1}) \\
\dot{x_{2}}(t)&=-g(x_{1})
\end{align*}
Dla systemu w takiej postaci często dobrym wyborem jest zastosowanie funkcjonały Lapunowa:
\begin{align*}
V(x)=\frac{1}{2}x_{2}^{2}+\int_{0}^{x_{1}}g(\xi ) d\xi
\end{align*}
którego pochodna jest dana wzorem:
\begin{align*}
\dot{V}(x)=\frac{\partial V(x)}{\partial x} = - g(x_{1})h(x_{1})
\end{align*}
Obliczam funkcjonał:
\begin{align*}
&V(x)=\frac{1}{2}x_{2}^{2}+\int_{0}^{x_{1}}-\xi ^{3} d\xi \\
&V(x)=\frac{1}{2}x_{2}^{2}+\frac{\xi ^{4}}{4}\Bigr|_{0}^{x_{1}} \\
&V(x)=\frac{1}{4}x_{1}^{4}+\frac{1}{2}x_{2}^{2}
\end{align*}
Spełnia on założenia:
\begin{itemize}
\item \( V(0) > 0 \) 
\item \( V(x) > 0 \) w pewnym otoczeniu 0
\item \( V(x) \) jest ciągły
\item \( V(x) \) ma ciągłe pierwsze pochodne
\end{itemize}
Pochodna:
\begin{align*}
\dot{V}(x)=-g(x_{1})h(x_{1})=-x_{1}^{3}\sin x_{1}
\end{align*}
\begin{figure}[H]
\centerline{\includegraphics[scale=0.5]{kol2_11.jpg}}
\caption{Pochodna}
\label{fig:kol2_11}
\end{figure}
Istnieje więc takie otocznie punktu równowagi, że pochodna jest niedodatnia \( \dot{V}(x) \leq 0 \), ale nie zachodzi nierówność ostra, bo dla punktów \( ( 0, x_{2} ) \) pochodna się zeruje dla dowolnego \( x_{2} \). \\
Zatem wykazano, że punkt równowagi \(x^{*}=0\) jest stabilnym punktem równowagi systemu ( ale nie asymptotycznie stabilnym).

\newpage
\begin{framed}
\textbf{Zadanie 4 - Zadania od Bauera do kolokwium II } \\ 
Dany jest system dynamiczny 
\begin{align*}
\dot{x}(t)&=x(t)- \sin (x(t))
\end{align*}
Wyznaczyć jego punkty równowagi i za pomocą II metody Lapunowa zbadać ich stabilność. 
\end{framed}
System jest nieliniowy ciągły, skończenie wymiarowy, stacjonarny, zakładam, że ma rozwiązanie.\\
Punkty równowagi: \\
\begin{align*}
& x(t) - \sin (x(t)) = 0 \\
& sin( x(t) ) = x \\
& x^{*} = 0 
\end{align*}
Jest tylko jeden punkt równowagi. \\
Kandydat na funkcjonał Lapunowa: \\
\begin{align*}
V(x)=\frac{1}{2}x^{2}
\end{align*}
Spełnia on założenia: \\
\begin{itemize}
\item \( V(x) \) jest ciągły 
\item \( V(x)>0 \) w pewnym otoczeniu \( 0 \)
\item \( V(x) \) ma ciągłe pochodne I rzędu
\item \( V(0) = 0 \) 
\end{itemize}
Obliczam pochodną: \\ 
\begin{align*}
\dot{V}(x)=x \cdot ( x - \sin x ) = x^{2}-x \sin x = x(x- \sin x) 
\end{align*}

\begin{figure}[H]
\centerline{\includegraphics[scale=0.5]{kol2_1.jpg}}
\caption{sinx i x - jeden punkt wspólny}
\label{fig:kol2_1}
\end{figure}
Zatem \( \forall x \) należącego do otoczenia \( x^{*} \) zachodzi \( \dot{V}(x)>0  \), a zatem punkt równowagi jest niestabilny. 

\newpage
\begin{framed}
\textbf{Zadanie 5 - Zadania od Bauera do kolokwium II } \\ 
Dany jest system dynamiczny 
\begin{align*}
\dot{x}(t)&=-x(t)+2x(t)^{2}
\end{align*}
Za pomocą metody La Salle'a określić przybliżony zbiór przyciągania zerowego punktu równowagi. 
\end{framed}
System jest nieliniowy ciągły, skończenie wymiarowy, stacjonarny, zakładam, że ma rozwiązanie.\\
Punkty równowagi to \( x^{*} = 0 \) ( z polecenia ). 
Kandydat na funkcjonał Lapunowa: \\
\begin{align*}
V(x)=\frac{1}{2}x^{2}
\end{align*}
Spełnia on założenia: \\
\begin{itemize}
\item \( V(x) \) jest ciągły 
\item \( V(x)>0 \) w pewnym otoczeniu \( 0 \)
\item \( V(x) \) ma ciągłe pochodne I rzędu
\item \( V(0) = 0 \) 
\end{itemize}
Obliczam pochodną: \\ 
\begin{align*}
\dot{V}(x)=x \cdot ( -x+2x^{2} ) = x^{2}(-1+2x)  
\end{align*}
\begin{figure}[H]
\centerline{\includegraphics[scale=0.5]{kol2_2.jpg}}
\caption{$ x^{2}(-1+2x) $}
\label{fig:kol2_1}
\end{figure}
Wyznaczam \( Z_{l} \) - kandydatem jest \( l = \frac{1}{8} \), bo z rysunku widać, że maksymalny zbiór będzie dla \( |x|<\frac{1}{2} \), czyli \\
\begin{align*}
|x|<\frac{1}{2} \\
x^{2} < \frac{1}{4} \\
\frac{1}{2} x^{2} < \frac{1}{8} \\
V(x) < l 
\end{align*}
Przybliżony zbiór przyciągania to: \\
\begin{align*}
Z_{l} = \{ x: \frac{1}{2}x^{2} < \frac{1}{8} \} = ( -\frac{1}{2} , \frac{1}{2} )
\end{align*}

\newpage
\begin{framed}
\textbf{Zadanie 6 - Zadania od Bauera do kolokwium II } \\ 
Dany jest system dynamiczny 
\begin{align*}
\dot{x_{1}}(t)&=x_{2}(t) \\
\dot{x_{2}}(t)&=-2x_{1}(t)-x_{2}(t)
\end{align*}
Za pomocą II metody Lapunowa pokazać asymptotyczną stabilność zerowego punktu równowagi ( wskazówka: funkcjonał Lapunowa można wyznaczyć za pomocą równania Lapunowa ). 
\end{framed}
( Po co robić Lapunowa dla systemu liniowego ??? ) \\ 
System jest ciągły, skończenie wymiarowy, stacjonarny, zakładam, że ma rozwiązanie.\\
Punkty równowagi to \( x^{*} = 0 \) ( z polecenia ). 
Kandydat na funkcjonał Lapunowa ( zgodnie ze wskazówką ): \\
\begin{align*}
V(x)=x^{T}Dx
\end{align*}
Macierz A systemu: \\
\begin{align*}
A = 
\begin{bmatrix}
0 & 1 \\
-2 & -1
\end{bmatrix}
\end{align*}
Równanie Lapunowa ma postać: \\
\begin{align*}
A^{T}D+DA = -G
\end{align*}
Ma ono rozwiązanie gdy: \\
\begin{align*}
Re \lambda(A) < 0 
\end{align*}

\begin{align*}
det( \lambda I - A ) = det
\begin{bmatrix}
\lambda & -1 \\
2 & \lambda + 1
\end{bmatrix} = 
\lambda( \lambda+1) + 2 = \lambda^{2} + \lambda + 2 
\end{align*}
\begin{align*}
\Delta = 1-4 \cdot 2 = -7 \\
\lambda_{1} = \frac{-1-\sqrt{7}i}{2} \\
\lambda_{2} = \frac{-1+\sqrt{7}i}{2} \\
\end{align*}
Zresztą widać, że równanie 2 rzędu i same współczynniki dodatnie więc musi mieć ujemne części rzeczywiste pierwiastków. \\
G wybieram arbitralnie taką, że \( G = G^{T} > 0 \), wybieram macierz jednostkową. D jest również macierzą symetryczną dodatnio określoną. \\
\begin{align*}
& D = 
\begin{bmatrix}
d_{11} & d_{12} \\
d_{12} & d_{22}
\end{bmatrix} \\
& G = 
\begin{bmatrix}
1 & 0 \\
0 & 1
\end{bmatrix} \\
\end{align*} 
\begin{align*}
\begin{bmatrix}
0 & -2 \\ 
1 & -1
\end{bmatrix}
\begin{bmatrix}
d_{11} & d_{12} \\
d_{12} & d_{22}
\end{bmatrix} 
+
\begin{bmatrix}
d_{11} & d_{12} \\
d_{12} & d_{22}
\end{bmatrix} 
\begin{bmatrix}
0 & 1 \\
-2 & -1
\end{bmatrix}
=
\begin{bmatrix}
-1 & 0 \\
0 & -1
\end{bmatrix}
\end{align*}
\begin{align*}
\begin{bmatrix}
-2d_{12} & -2d_{22} \\
d_{11}-d_{12} & d_{12}-d_{22}
\end{bmatrix} +
\begin{bmatrix}
-2d_{12} & d_{11}-d_{12} \\
-2d_{22} & d_{12}-d_{22}
\end{bmatrix} = 
\begin{bmatrix}
-1 & 0 \\
0 & -1
\end{bmatrix} 
\end{align*}
\begin{align*}
\begin{bmatrix}
-4d_{12} & d_{11}-d_{12}-2d_{22} \\
d_{11}-d_{12}-2d_{22} & 2d_{12}-2d_{22}
\end{bmatrix} =
\begin{bmatrix}
-1 & 0 \\
0 & -1
\end{bmatrix} 
\end{align*}
\begin{align*}
\begin{cases}
-4d_{12} = -1  \\
d_{11}-d_{12}-2d_{22} = 0 \\
d_{11}-d_{12}-2d_{22} = 0 \\
2d_{12}-2d_{22} = -1 \\
\end{cases}
\implies
\begin{cases}
d_{12} = \frac{1}{4}  \\
d_{11} = \frac{7}{4} \\
d_{22} = \frac{3}{4} \\
\end{cases}
\end{align*}
\begin{align*}
D = 
\begin{bmatrix}
\frac{7}{4} & \frac{1}{4} \\ \\
\frac{1}{4} & \frac{3}{4}
\end{bmatrix}
\end{align*}
Kandydat na funkcjonał Lapunowa:
\begin{align*}
V(x) = 
\begin{bmatrix}
x_{1} & x_{2}
\end{bmatrix}
\begin{bmatrix}
\frac{7}{4} & \frac{1}{4} \\ \\
\frac{1}{4} & \frac{3}{4}
\end{bmatrix}
\begin{bmatrix}
x_{1} \\
x_{2}
\end{bmatrix} = 
\begin{bmatrix}
\frac{7}{4}x_{1}+\frac{1}{4}x_{2} & \frac{1}{4}x_{1}+\frac{3}{4}x_{2}
\end{bmatrix}
\begin{bmatrix}
x_{1} \\
x_{2}
\end{bmatrix} = 
\end{align*}
\begin{align*}
& \frac{7}{4}x_{1}^{2}+\frac{1}{4}x_{1}x_{2}+\frac{1}{4}x_{1}x_{2}+\frac{3}{4}x_{2}^{2} = \\
& \frac{7}{4}x_{1}^{2}+\frac{1}{2}x_{1}x_{2}+\frac{3}{4}x_{2}^{2}
\end{align*}
Spełnia on założenia: \\
\begin{itemize}
\item \( V(x) \) jest ciągły 
\item \( V(x)>0 \quad \forall x - \{ 0 \} \) ( bo forma jest dodatnio określona )
\item \( V(x) \) ma ciągłe pochodne I rzędu
\item \( V(0) = 0 \) 
\end{itemize}
Obliczam pochodną: \\ 
\begin{align*}
\dot{V}(x)= 
\begin{bmatrix}
\frac{14}{4}x_{1}+\frac{1}{2}x_{2} & \frac{6}{4}x_{2}+\frac{1}{2}x_{1}
\end{bmatrix}
\begin{bmatrix}
x_{2} \\
-2x_{1} - x_{2}
\end{bmatrix}   
\end{align*}
\begin{align*}
\dot{V}(x) = -x_{1}^{2}-x_{2}^{2}
\end{align*}
Widać więc, z postaci pochodnej, że jest ona zawsze ujemna w otoczeniu zera, a zatem zachodzi \( V(x) > 0 \land \dot{V}(x) < 0 \) w pewnym otoczeniu 0 więc jest to punkt równowagi asymptotycznie stabilny. 

\newpage
\begin{framed}
\textbf{Zadanie 7 - Zadania od Bauera do kolokwium II } \\ 
Układ postaci
\begin{align*}
&\dot{x}(t)=Ax(t)+Bu(t) \\
&y(t)=Cx(t) 
\end{align*}
z macierzami:
\begin{align*}
A =
\begin{bmatrix}
0 & 1 \\
-1 & -1 
\end{bmatrix}
B =
\begin{bmatrix}
0  \\
-1  
\end{bmatrix}
C^{T} =
\begin{bmatrix}
1  \\
0 
\end{bmatrix}
\end{align*}
objęto ujemnym, nieliniowym, sprzężeniem zwrotnym od wyjścia \( f(y)=y^{3} \). Uzasadnić, że forma kwadratowa \( V(x)=x^{T}Hx \), gdzie \( H \) jest rozwiązaniem równania: 
\begin{align*}
A^{T}H+HA=-C^{T}C
\end{align*} 
jest funkcjonałem Lapunowa dla tego układu. Znaleźć maksymalny obszar atrakcji zerowego punktu równowagi.
\end{framed}
System jest nieliniowy, stacjonarny, skończenie wymiarowy, z czasem ciągłym. Z drugiego równania systemu:
\begin{align*}
&y(t) = x_{1} \\
&u(t) = f(y) = y^{3} = x_{1}^{3}
\end{align*}
UWAGA: system objęty jest \textit{ujemnym} sprzężeniem zwrotnym dlatego sterowanie \( u(t) \) wstawiamy ze znakiem minus do równań systemu. \\
Więc system można zapisać: \\
\begin{align*}
\begin{cases}
\dot{x_{1}}(t)= x_{2}(t) \\
\dot{x_{2}}(t)= -x_{1}(t)-x_{2}(t)+x_{1}^{3}(t)
\end{cases}
\end{align*}
Jednym z punktów równowagi tego systemu jest \( x^{*} = 0 \). 
\begin{align*}
-CC^{T}=
\begin{bmatrix}
-1 \\ 0
\end{bmatrix}
\begin{bmatrix}
1 & 0
\end{bmatrix}
=
\begin{bmatrix}
-1 & 0 \\
0 & 0
\end{bmatrix}
\end{align*}
Macierz H to macierz symetryczna dodatnio określona:
\begin{align*}
H = 
\begin{bmatrix}
h_{11} & h_{12} \\
h_{12} & h_{22}
\end{bmatrix}
\end{align*}

\begin{align*}
\begin{bmatrix}
0 & -1 \\ 
1 & -1
\end{bmatrix}
\begin{bmatrix}
h_{11} & h_{12} \\
h_{12} & h_{22}
\end{bmatrix} 
+
\begin{bmatrix}
h_{11} & h_{12} \\
h_{12} & h_{22}
\end{bmatrix} 
\begin{bmatrix}
0 & 1 \\
-1 & -1
\end{bmatrix}
=
\begin{bmatrix}
-1 & 0 \\
0 & 0
\end{bmatrix}
\end{align*}
\begin{align*}
\begin{bmatrix}
-h_{12} & -h_{22} \\
h_{11}-h_{12} & h_{12}-h_{22}
\end{bmatrix} +
\begin{bmatrix}
-h_{12} & h_{11}-h_{12} \\
-h_{22} & h_{12}-h_{22}
\end{bmatrix} = 
\begin{bmatrix}
-1 & 0 \\
0 & 0
\end{bmatrix} 
\end{align*}
\begin{align*}
\begin{bmatrix}
-2h_{12} & h_{11}-h_{12}-h_{22} \\
h_{11}-h_{12}-h_{22} & 2h_{12}-2h_{22}
\end{bmatrix} =
\begin{bmatrix}
-1 & 0 \\
0 & 0
\end{bmatrix} 
\end{align*}
\begin{align*}
\begin{cases}
-2h_{12} = -1  \\
h_{11}-h_{12}-h_{22} = 0 \\
2d_{12}-2d_{22} = 0 \\
\end{cases}
\implies
\begin{cases}
h_{12} = \frac{1}{2}  \\
h_{11} = 1 \\
h_{22} = \frac{1}{2} \\
\end{cases}
\end{align*}
\begin{align*}
H = 
\begin{bmatrix}
1 & \frac{1}{2} \\ \\
\frac{1}{2} & \frac{1}{2}
\end{bmatrix}
\end{align*}
Kandydat na funkcjonał Lapunowa:
\begin{align*}
V(x) = 
\begin{bmatrix}
x_{1} & x_{2}
\end{bmatrix}
\begin{bmatrix}
1 & \frac{1}{2} \\ \\
\frac{1}{2} & \frac{1}{2}
\end{bmatrix}
\begin{bmatrix}
x_{1} \\
x_{2}
\end{bmatrix} = 
\begin{bmatrix}
x_{1}+\frac{1}{2}x_{2} & \frac{1}{2}x_{1}+\frac{1}{2}x_{2}
\end{bmatrix}
\begin{bmatrix}
x_{1} \\
x_{2}
\end{bmatrix} = 
\end{align*}
\begin{align*}
& x_{1}^{2}+\frac{1}{2}x_{1}x_{2}+\frac{1}{2}x_{1}x_{2}+\frac{1}{2}x_{2}^{2} = \\
& x_{1}^{2}+x_{1}x_{2}+\frac{1}{2}x_{2}^{2}
\end{align*}
Spełnia on założenia: \\
\begin{itemize}
\item \( V(x) \) jest ciągły 
\item \( V(x)>0 \quad \forall x - \{ 0 \} \) ( bo forma jest dodatnio określona )
\item \( V(x) \) ma ciągłe pochodne I rzędu
\item \( V(0) = 0 \) 
\end{itemize}
Obliczam pochodną: \\ 
\begin{align*}
\dot{V}(x)= 
\begin{bmatrix}
2x_{1}+x_{2} & x_{2}+x_{1}
\end{bmatrix}
\begin{bmatrix}
x_{2} \\
-x_{1} - x_{2} + x_{1}^{3}
\end{bmatrix}   
\end{align*}
\begin{align*}
\dot{V}(x) = 2x_{1}x_{2}+x_{2}^{2}-x_{1}x_{2}-x_{2}^{2}+x_{2}x_{1}^{3}-x_{1}^{2}-x_{2}x_{1}+x_{1}^{4} = x_{1}^{4}-x_{1}^{2}+x_{2}x_{1}^{3}=x_{1}^{2}(x_{1}^{2}+x_{2}x_{1}-1)
\end{align*}
\( x_{1}^{2} \) jest zawsze dodatnie oraz istnieje takie otoczenie 0, że wyrażenie \( x_{1}^{2}+x_{2}x_{1} - 1 \) jest niedodatnie więc jest to funkcjonał Lapunowa. \\
Aby wyznaczyć obszar przyciągania korzystam z twierdzenia LaSalle'a. Poszukuje takiego obszaru aby spełnione były oba warunki : 
\begin{align*}
V(x) > 0 \\
\dot{V}(x) \leq 0
\end{align*}
w pierwszej kolejności interesuje mnie gdzie jest granica w której pochodna przestaje być niedodatnia. Tą granicę obliczam z równania:
\begin{align*}
\dot{V}(x)=0 \implies x_{1} = 0 \lor x_{1}^{2}+x_{2}x_{1}-1=0
\end{align*}
Z drugiego równania otrzymuje, że:
\begin{align*}
x_{2}=\frac{1-x_{1}^{2}}{x_{1}}
\end{align*}
co wyznacza granicę po przekroczeniu której pochodna przestaje być niedodatnia.
\begin{figure}[H]
\centerline{\includegraphics[scale=0.3]{kol2_5.jpg}}
\caption{Granica niedodatniości pochodnej}
\label{fig:kol2_5}
\end{figure}
Teraz sprawdzam czy na tej granicy funkcjonał \( V(x) \) pozostaje dodatni. W tym celu podstawiam \( x_{2} = \frac{1-x_{1}^{2}}{x_{1}}  \) do wyrażenia na \( V(x) \) i otrzymuje: 
\begin{align*}
V_{1}(x_{1})=\frac{x_{1}^{4}+1}{2x_{1}}
\end{align*}
co odpowiada wartościom funkcji \(V(x)\) na wyznaczonym brzegu. Aby sprawdzić czy funkcja jest tam dodatnio szukam jej minimum, które wynosi:
\begin{align*}
x_{1min}=-1 \lor 1 \implies x_{2min} = 0 \\
V_{min}( x_{1min} , x_{2min} )=1 
\end{align*}
co odpowiada minimalnej wartości na brzegu. A zatem maksymalny obszar przyciągania wyznaczony na podstawie twierdzenia LaSalle'a to:
\begin{align*}
Z_{l} = \{ x: V(x)<l=1 \}
\end{align*}


\newpage
\section{Kryterium Koła i Popova}
\subsection{Zbiór zadań}
\begin{framed}
\textbf{\colorbox{green}{Zadanie 1 - Zadania od Bauera do kolokwium II }} \\ 
Układ postaci
\begin{align*}
\dot{x}(t)&=-2x(t)+4u(t)
\end{align*}
z wyjściem \( y(t)=x(t) \) objęto ujemnym sprzężeniem zwrotnym od wyjścia:
\begin{align*}
u(t)=(1+e^{-t})y
\end{align*}
Zbadać asymptotyczną stabilność powstałego układu. 
\end{framed}

\begin{framed}
\textbf{Zadanie 2 - Egzamin } \\ 
Układ postaci
\begin{align*}
&\dot{x}(t)=Ax(t)+Bu(t) \\
&y=Cx(t)
\end{align*}
z macierzami:
\begin{align*}
A = 
\begin{bmatrix}
0 & 1 & 0 \\
0 & 0 & 1 \\
0 & 0 & -\gamma 
\end{bmatrix} 
\;
B = 
\begin{bmatrix}
0 \\
0 \\
1
\end{bmatrix}
\;
C = 
\begin{bmatrix}
1 \\
\beta \\
0
\end{bmatrix}^{T}
\end{align*}
objęto, ujemnym, nieliniowym sprzężeniem zwrotnym od wyjścia \( f(y) = y^3 \). Sprawdzić stabilność asymptotyczną zerowego punktu równowagi. \\
Wskazówka: Wykorzystać kryterium Popova \\
Uwaga: Zakładamy \( \beta > 0 \; , \gamma > 0 \; \beta \cdot \gamma > 0 \)
\end{framed}



\newpage
\subsection{Rozwiązania zadań ze zbioru}
\begin{framed}
\textbf{Zadanie 1 - Zadania od Bauera do kolokwium II } \\ 
Układ postaci
\begin{align*}
\dot{x}(t)&=-2x(t)+4u(t)
\end{align*}
z wyjściem \( y(t)=x(t) \) objęto ujemnym sprzężeniem zwrotnym od wyjścia:
\begin{align*}
u(t)=(1+e^{-t})y
\end{align*}
Zbadać asymptotyczną stabilność powstałego układu. 
\end{framed}

System ma sprzężenie zwrotne niestacjonarne, co wyklucza stosowanie kryterium Popova, bo jest ono zdefiniowane dla sprzężenia stacjonarnego, stosuje twierdzenie Koła. \\
Spełnione są założenia:
\begin{itemize}
\item System jest SISO, liniowy, stacjonarny, objęty nieliniowym, niestacjonarnym sprzężeniem zwrotnym
\item Macierz A części liniowej nie ma pierwiastków na osi urojonej ( nie musi być stabilna asymptotycznie ! ), bo \( \lambda = -2 = A \) dla systemu jednowymiarowego.
\end{itemize}
\begin{align*}
\dot{x}(t)=-2x(t)+4(1+e^{-t})y(t) = -2x(t)+4(1+e^{-t})y(t)
\end{align*}
Wyznaczam transmitancję części liniowej:
\begin{align*}
G(s)=C^{T}(sI-A)^{-1}B=1\cdot (s+2)^{-1} \cdot 4 = \frac{4}{s+2}
\end{align*}
Ale twierdzenie Koła ( jak też Popova ) są sformułowane dla dodatniego sprzężenia zwrotnego więc bierzemy ze znakiem minus:
\begin{align*}
G_{1}(s)=-\frac{4}{s+2}
\end{align*}
Wyznaczam charakterystykę amplitudowo-fazową:
\begin{align*}
G_{1}(j\omega)=\frac{-4}{j\omega+2}=\frac{-4(2-j\omega)}{(2+j\omega)(2-j\omega)}&=
\frac{-4(2-j\omega)}{4-(j\omega)^{2}} = \frac{-8+4j\omega}{4-j^{2}\omega^{2}}=\frac{-8+4j\omega}{4+\omega^{2}} = \\
& \frac{-8}{\omega^{2}+4}+j\frac{4\omega}{4+\omega^{2}}
\end{align*}
Wyznaczam tabelkę wartości dla charakterystyki:
\begin{table}[H]
\begin{center}
\begin{tabular}{|p{2cm}|p{2cm}|p{2cm}|p{2cm}|p{2cm}|p{2cm}|}
\hline
\( \omega \) & 0 & \( \frac{1}{2} \) & 2 & 6 & \( \infty\)  \\ \hline
\( P(\omega) \) & -2 & \( \frac{-32}{17} \) & -1 & \( \frac{-1}{5} \) & 0  \\ \hline
\( Q(\omega) \) & 0 & \( \frac{8}{17} \) & 1 & \( \frac{3}{8} \) & 0  \\ \hline
\end{tabular}
\end{center}
\end{table}
Szkicuję charakterystykę:
\begin{figure}[H]
\centerline{\includegraphics[scale=0.2]{kol2_3.jpg}}
\caption{Szkic charakterystyki amp-faz}
\label{fig:kol2_3}
\end{figure}
Widać, że wychodzi coś podobnego do koła, można jeszcze sprawdzić czy maksymalną wartością na osi urojonej jest 1 poprzez sprawdzenie zerowania pochodnej \( Q(\omega) \): 
\begin{align*}
Q^{'}(\omega) = \frac{4(\omega^{2}+4)-2\omega \cdot 4\omega}{(\omega^{2}+4)^{2}}=\frac{4(2-\omega)(2+\omega)}{(\omega^{2}+4)^{2}}
\end{align*}
Widać, że ma maksimum dla \( \omega = 2 \) więc maksymalna wartość na osi urojonej to 1. \\
Dobieram \( m_{1} \) i \(m_{2}\) (  \(m_{2}\) > \( m_{1} \) ) aby zawrzeć charakterystykę w kole\footnote{Gdyby się tak nie dało dobrać koła bo charakterystyka by była za duża, to dobieram półpłaszczyznę, wtedy \( m_{1} = 0  \) lub \( m_{2} = 0  \)}, takim które przecina oś \( P(\omega) \) w punktach \( \frac{1}{m_{1}} \) i \( \frac{1}{m_{2}}\). Charakterystyka nie może mieć punktów wspólnych z brzegiem koła więc dobieram:
\begin{align*}
\frac{1}{m_{2}} > 0 \\
\frac{1}{m_{1}} < -2
\end{align*}
skąd otrzymuje, że:
\begin{align*}
m_{2} < \infty \\
m_{1} > \frac{-1}{2}
\end{align*}
Stąd sektor dopuszczalny: 
\begin{figure}[H]
\centerline{\includegraphics[scale=0.2]{kol2_4.jpg}}
\caption{Sektor dopuszczalny}
\label{fig:kol2_4}
\end{figure}
Jeśli wykres funkcji \( u(t) = f(t,y) \) zawiera się w sektorze dopuszczalnym, to zerowe rozwiązanie układu zamkniętego jest asymptotycznie stabilne, jeśli tak nie jest, to nic nie można powiedzieć o stabilności układu. W przypadku zadania \( u(t) = (1+e^{-t})y \) więc wykres w każdej chwili czasu zawiera się w sektorze dopuszczalnym więc punkt równowagi jest asymptotycznie stabilny.

\newpage
\begin{framed}
\textbf{Zadanie 2 - Egzamin } \\ 
Układ postaci
\begin{align*}
&\dot{x}(t)=Ax(t)+Bu(t) \\
&y=Cx(t)
\end{align*}
z macierzami:
\begin{align*}
A = 
\begin{bmatrix}
0 & 1 & 0 \\
0 & 0 & 1 \\
0 & 0 & -\gamma 
\end{bmatrix} 
\;
B = 
\begin{bmatrix}
0 \\
0 \\
1
\end{bmatrix}
\;
C = 
\begin{bmatrix}
1 \\
\beta \\
0
\end{bmatrix}^{T}
\end{align*}
objęto, ujemnym, nieliniowym sprzężeniem zwrotnym od wyjścia \( f(y) = y^3 \). Sprawdzić stabilność asymptotyczną zerowego punktu równowagi. \\
Wskazówka: Wykorzystać kryterium Popova \\
Uwaga: Zakładamy \( \beta > 0 \; , \gamma > 0 \; \beta \cdot \gamma > 0 \)
\end{framed}

\newpage
\section{Systemy dyskretne}

\subsection{Zbiór zadań}
\begin{framed}
\textbf{\colorbox{green}{Zadanie 1 - Zadania od Bauera do kolokwium II} } \\ 
Do ciągłego systemu dynamicznego opisanego równaniami
\begin{align*}
\dot{x}(t)=Ax(t)+Bu(t) \\
y(t)=Cx(t)
\end{align*}
przy czym 
\begin{align*}
A = 
\begin{bmatrix}
-1 & 0 \\
0 & -3
\end{bmatrix}
B = 
\begin{bmatrix}
1 & 0 \\
0 & 1
\end{bmatrix}
C = 
\begin{bmatrix}
1 & 0
\end{bmatrix}
\end{align*}
podłączono ekstrapolator rzędu zerowego na wejściu i impulsator na wyjściu, przy czym pracują one synchronicznie z okresem próbkowania \( h = 1s \). Wyliczyć parametry systemu dyskretnego odpowiadające takiemu połączeniu.
\end{framed}

\begin{framed}
\textbf{\colorbox{green}{Zadanie 2 - Zadania od Bauera do kolokwium II }} \\ 
Do systemu
\begin{align*}
& x_{1}(i+1)=x_{2}(i) \\
& x_{2}(i+1)=x_{1}(i)+x_{2}(i)+u(i) \\
& u(i) = -k_{1}x_{1}(i)-k_{2}x_{2}(i) \\
& i = 0, 1, 2, \dots,
\end{align*}
zbadać asymptotyczną stabilność układu w zależności od \( k_{1} \) i \( k_{2} \). Zaznaczyć odpowiedni obszar na płaszczyźnie \( k_{1} \times k_{2} \).
\end{framed}

\begin{framed}
\textbf{\colorbox{green}{Zadanie 3 - Zadania od Bauera do kolokwium II} } \\ 
Do ciągłego systemu dynamicznego opisanego równaniami: 
\begin{align*}
\dot{x}(t) =
\begin{bmatrix}
-ln\frac{1}{3} & 0 \\
0 & -ln\frac{1}{2}
\end{bmatrix}
x(t)+
\begin{bmatrix}
ln9 \\
ln 8
\end{bmatrix}
u(t)
\end{align*}
podłączono ekstrapolator rzędu zerowego na wejściu i impulsator na wyjściu, przy czym pracują one synchronicznie z okresem próbkowania \( h = 1 \). Znaleźć parametry systemu dyskretnego odpowiadające temu połączeniu. Do powstałego modelu dyskretnego wpięto sterowania postaci \( u(k)= Kx(k) \). Dobrać wartości K tak aby wartościami własnymi układu zamkniętego były \( \lambda_{1} = \lambda_{2} = 0 \). \\ 
\textit{Wskazówka: Sterowania ma postać \( u(k) = K_{1}x_{1}(k)+K_{2}x_{2}(k)\)}
\end{framed}

\begin{framed}
\textbf{Zadanie 4 - Egzamin} \\ 
Do ciągłego systemu dynamicznego opisanego równaniami
\begin{align*}
\dot{x}(t)=
\begin{bmatrix}
0 & \pi & 0 \\
-\pi & 0 & 0 \\
0 & 0 & \pi
\end{bmatrix}
x(t) + 
\begin{bmatrix}
0 \\
0 \\
\pi 
\end{bmatrix}
u(t)
\end{align*}
podłączono ekstrapolator rzędu zerowego na wejściu i impulsator na wyjściu, przy czym pracują one synchronicznie z okresem próbkowania \( h = 1s \). Wyliczyć parametry systemu dyskretnego odpowiadające takiemu połączeniu. Do powstałego modelu dyskretnego wpięto sterowanie \( u(k) = Kx(k) \). Dobrać wartość \( K \) tak żeby układ zamknięty był:
\begin{itemize}
\item asymptotycznie stabilny
\item stabilny
\end{itemize}
Wskazówka: Sterowanie ma postać \( u(k) = K_1x_1(k)+K_2x_2(k)+K_3x_3(k) \).
\end{framed}

\newpage
\subsection{Rozwiązania zadań ze zbioru}
\begin{framed}
\textbf{Zadanie 1 - Zadania od Bauera do kolokwium II } \\ 
Do ciągłego systemu dynamicznego opisanego równaniami
\begin{align*}
\dot{x}(t)=Ax(t)+Bu(t) \\
y(t)=Cx(t)
\end{align*}
przy czym 
\begin{align*}
A = 
\begin{bmatrix}
-1 & 0 \\
0 & -3
\end{bmatrix}
B = 
\begin{bmatrix}
1 & 0 \\
0 & 1
\end{bmatrix}
C = 
\begin{bmatrix}
1 & 0
\end{bmatrix}
\end{align*}
podłączono ekstrapolator rzędu zerowego na wejściu i impulsator na wyjściu, przy czym pracują one synchronicznie z okresem próbkowania \( h = 1s \). Wyliczyć parametry systemu dyskretnego odpowiadające takiemu połączeniu.s 
\end{framed}
\begin{align*}
&x^{+}(i+1)=A^{+}x^{+}(i)+B^{+}u^{+}(i) \\
&y^{+}(i)=C^{+}x^{+}(i)
\end{align*}
gdzie 
\begin{align*}
x^{+}(i)=x(ih), u^{+}(i)=u(ih), y^{+}=y(ih), i=0,1,2, \dots ,
\end{align*}
\( h = 1s \)
\begin{align*}
A^{+}=e^{hA}
\end{align*}
A jest w postaci kanonicznej Jordana z wartościami własnymi \( \lambda_{1} = -1 \) i \( \lambda_{2} = -3 \).
\begin{align*}
A^{+}=e^{hA}=
\begin{bmatrix}
e^{\lambda_{1}h} & 0 \\
0 & e^{\lambda_{2}h}
\end{bmatrix}
=
\begin{bmatrix}
e^{-1} & 0 \\
0 & e^{-3}
\end{bmatrix}
\end{align*}

\begin{align*}
B^{+}=\int_{0}^{h}e^{tA}Bdt=
\int_{0}^{1}
\begin{bmatrix}
e^{-t} & 0 \\
0 & e^{-3t}
\end{bmatrix}
\begin{bmatrix}
1 & 0 \\
0 & 1
\end{bmatrix}
dt
=
\int_{0}^{1}
\begin{bmatrix}
e^{-t} & 0 \\
0 & e^{-3t}
\end{bmatrix}
dt
=
\begin{bmatrix}
\int_{0}^{1}e^{-t}dt & \int_{0}^{1}0dt \\
\int_{0}^{1}0dt & \int_{0}^{1}e^{-3t}dt
\end{bmatrix}
=
\end{align*}
\begin{align*}
\begin{bmatrix}
-e^{-t}\Bigr|_{0}^{1} & 0 \\
0 & \frac{-1}{3}e^{-3t}\Bigr|_{0}^{1}
\end{bmatrix}=
\begin{bmatrix}
-e^{-1}+1 & 0 \\
0 & \frac{-1}{3}e^{-3}+\frac{1}{3}
\end{bmatrix}
\end{align*}
\begin{align*}
C^{+}=C
\end{align*}

\newpage
\begin{framed}
\textbf{Zadanie 2 - Zadania od Bauera do kolokwium II } \\ 
Do systemu
\begin{align*}
& x_{1}(i+1)=x_{2}(i) \\
& x_{2}(i+1)=x_{1}(i)+x_{2}(i)+u(i) \\
& u(i) = -k_{1}x_{1}(i)-k_{2}x_{2}(i) \\
& i = 0, 1, 2, \dots,
\end{align*}
zbadać asymptotyczną stabilność układu w zależności od \( k_{1} \) i \( k_{2} \). Zaznaczyć odpowiedni obszar na płaszczyźnie \( k_{1} \times k_{2} \).
\end{framed}
Podstawiam za \( u(i) \) sterowanie i otrzymuje układ:
\begin{align*}
& x_{1}(i+1)=x_{2}(i) \\
& x_{2}(i+1)=x_{1}(i)(1-k_{1})+x_{2}(i)(1-k_{2})
\end{align*}
Nowa macierz układu:
\begin{align*}
M = 
\begin{bmatrix}
0 & 1 \\
1-k_{1} & 1-k_{2}
\end{bmatrix}
\end{align*}
Wielomian charakterystyczny:
\begin{align*}
det(\lambda I - M) =
det
\begin{bmatrix}
\lambda & -1 \\
k_{1}-1 & \lambda+k_{2}-1
\end{bmatrix}
=
\lambda(\lambda+k_{2}-1)+(k_{1}-1)
=
\lambda^{2}+\lambda k_{2} -\lambda+k_{1}-1 =
\end{align*}
\begin{align*}
\lambda^{2}+\lambda(k_{2}-1)+k_{1}-1
\end{align*}
Układ będzie asymptotycznie stabilny \( \Leftrightarrow \) wartości własne tego równania będę się znajdować wewnątrz koła jednostkowego o środku w zerze. \\ 
Gdyby teraz policzyć wartości własne z tego równania to nie wyszły by one zbyt ładne i ciężko by było stwierdzić kiedy leżą w kole a kiedy nie. \\
Stosując przekształcenie \( \lambda = \frac{s+1}{s-1} \) odwzorowujemy wnętrze koła jednostkowego w lewą półpłaszczyznę zespoloną. 
\begin{align*}
\frac{(s+1)^{2}}{(s-1)^{2}}+\frac{(s+1)(k_{2}-1)}{s-1}+k_{1}-1=\\
\frac{(s+1)^{2}+(s+1)(s-1)(k_{2}-1)+(k_{1}-1)(s-1)^{2}}{(s-1)^{2}} = \\
\frac{s^{2}+2s+1+(s^{2}-1)(k_{2}-1)+(k_{1}-1)(s^{2}-2s+1)}{(s-1)^{2}}
\end{align*}
Licznik:
\begin{align*}
&s^{2}+2s+1+s^{2}k_{2}-s^{2}-k_{2}+1+k_{1}s^{2}-2k_{1}s+k_{1}-s^{2}+2s-1 = \\
&s^{2}(k_{2}+k_{1}-1)+s(4-2k_{1})+1-k_{2}+k_{1}
\end{align*} 
Do tego wielomianu można zastosować kryterium Hurwitza. Warunek konieczny:
\begin{align*}
a_{2} > 0, a_{1}>0, a_{0}>0 \\
\begin{cases}
k_{2}>-k_{1}+1 \\
k_{1}<2 \\
k_{2}<k_{1}+1
\end{cases}
\end{align*}
Macierz Hurwitza:
\begin{align*}
H = 
\begin{bmatrix}
a_{1} & 0 \\
a_{2} & a_{0}
\end{bmatrix}
=
\begin{bmatrix}
4-2k_{1} & 0 \\
k_{2}+k_{1}-1 & 1-k_{2}+k_{1}
\end{bmatrix}
\end{align*}
Minory wiodące większe od 0:
\begin{align*}
\begin{cases}
4-2k_{1}>0 \\
(4-2k_{1})(1-k_{2}+k_{1})>0 \Leftrightarrow 1-k_{2}+k_{1}>0
\end{cases}
\implies
\begin{cases}
k_{1}<2 \\
k_{2}<k_{1}+1
\end{cases}
\end{align*}
Ostatecznie:
\begin{align*}
\begin{cases}
k_{1}<2 \\
k_{2}<k_{1}+1 \\
k_{2} > -k_{1}+1
\end{cases}
\end{align*}
W żółtym obszarze ( bez krawędzi ) system pozostaje asymptotycznie stabilny.
\begin{figure}[H]
\centerline{\includegraphics[scale=0.4]{kol2_6.jpg}}
\caption{Asymptotycznie stabilny ( żółty bez krawędzi )}
\label{fig:kol2_6}
\end{figure}

 
\newpage
\begin{framed}
\textbf{Zadanie 3 - Zadania od Bauera do kolokwium II } \\ 
Do ciągłego systemu dynamicznego opisanego równaniami: 
\begin{align*}
\dot{x}(t) =
\begin{bmatrix}
-ln\frac{1}{3} & 0 \\
0 & -ln\frac{1}{2}
\end{bmatrix}
x(t)+
\begin{bmatrix}
ln9 \\
ln 8
\end{bmatrix}
u(t)
\end{align*}
podłączono ekstrapolator rzędu zerowego na wejściu i impulsator na wyjściu, przy czym pracują one synchronicznie z okresem próbkowania \( h = 1 \). Znaleźć parametry systemu dyskretnego odpowiadające temu połączeniu. Do powstałego modelu dyskretnego wpięto sterowania postaci \( u(k)= Kx(k) \). Dobrać wartości K tak aby wartościami własnymi układu zamkniętego były \( \lambda_{1} = \lambda_{2} = 0 \). \\ 
\textit{Wskazówka: Sterowania ma postać \( u(k) = K_{1}x_{1}(k)+K_{2}x_{2}(k)\)}
\end{framed}
W zadaniu dany jest układ ciągły więc pierwszą czynnością jest zamiana na układ dyskretny: \\
\begin{align*}
&x^{+}(i+1)=A^{+}x^{+}(i)+B^{+}u^{+}(i) \\
&y^{+}(i)=C^{+}x^{+}(i)
\end{align*}
gdzie 
\begin{align*}
x^{+}(i)=x(ih), u^{+}(i)=u(ih), y^{+}=y(ih), i=0,1,2, \dots ,
\end{align*}
\( h = 1s \)
\begin{align*}
A^{+}=e^{hA}
\end{align*}
A jest w postaci kanonicznej Jordana z wartościami własnymi \( \lambda_{1} = -ln\frac{1}{3} = ln 3 \) i \( \lambda_{2} = -ln\frac{1}{2} = ln 2\). W przekształceniach korzystam z tożsamości:
\begin{align*}
&-\log_a b = \log_a b^{-1} \\
&a^{\log_a b} = b
\end{align*}
\begin{align*}
A^{+}=e^{hA}=
\begin{bmatrix}
e^{\lambda_{1}h} & 0 \\
0 & e^{\lambda_{2}h}
\end{bmatrix}
=
\begin{bmatrix}
e^{ln3} & 0 \\
0 & e^{ln2}
\end{bmatrix}
=
\begin{bmatrix}
3 & 0 \\
0 & 2
\end{bmatrix}
\end{align*}

\begin{align*}
B^{+}=\int_{0}^{h}e^{tA}Bdt=
\int_{0}^{1}
\begin{bmatrix}
e^{t\cdot ln3} & 0 \\
0 & e^{t\cdot ln2}
\end{bmatrix}
\begin{bmatrix}
ln9 \\
ln8
\end{bmatrix}
dt
=
\int_{0}^{1}
\begin{bmatrix}
ln9 \cdot e^{t\cdot ln3} \\
ln8 \cdot e^{t\cdot ln2} 
\end{bmatrix}
dt
=
\end{align*}
\begin{align*}
\begin{bmatrix}
ln9 \cdot \int_{0}^{1}e^{t\cdot ln3}dt  \\ \\
ln8 \cdot \int_{0}^{1} e^{t\cdot ln2}dt
\end{bmatrix}
=
\begin{bmatrix}
\frac{ln9}{ln3} \cdot 
\begin{bmatrix}
e^{t \cdot ln3}\Bigr|_{0}^{1}
\end{bmatrix} \\
\frac{ln8}{ln2}\begin{bmatrix}
e^{t \cdot ln2}\Bigr|_{0}^{1}
\end{bmatrix}
\end{bmatrix}=
\begin{bmatrix}
2 \cdot (e^{ln3}-1) \\
 3 \cdot ( e^{ln2}-1)
\end{bmatrix} = 
\begin{bmatrix}
4  \\
3 
\end{bmatrix}
\end{align*}
\begin{align*}
C^{+}=C
\end{align*}
System z obliczonymi macierzami jest postaci: 
\begin{align*}
&x^{+}(i+1)=A^{+}x^{+}(i)+B^{+}u^{+}(i) \\
&y^{+}(i)=C^{+}x^{+}(i)
\end{align*}
wiadomo, że \( u(k) = K_{1}x_{1}(k)+K_{2}x_{2}(k)\), podstawiając to równanie jako sterowanie do systemu:
\begin{align*}
\begin{bmatrix}
&x_{1}^{+}(i+1) \\
&x_{2}^{+}(i+1)
\end{bmatrix}
=
\begin{bmatrix}
3 & 0 \\
0 & 2
\end{bmatrix}
\begin{bmatrix}
x_{1}(i) \\
x_{2}(i)
\end{bmatrix}
+
\begin{bmatrix}
4 \\ 
3
\end{bmatrix}
(K_{1}x_{1}(i)+K_{2}x_{2}(i)) 
\end{align*}
\begin{align*}
\begin{cases}
x_{1}^{+}(i+1) = (3+4K_{1})x_{1}(i) + 4K_{2}x_{2}(i) \\
x_{2}^{+}(i+1) = 3K_{1}x_{1}(i) + (2+3K_{2})x_{2}(i)
\end{cases}
\end{align*}
Nowa macierz:
\begin{align*}
M = 
\begin{bmatrix}
3+4K_{1} & 4K_{2} \\
3K_{1} & 2+3K_{2}
\end{bmatrix}
\end{align*}
Wartości własne:
\begin{align*}
\det(\lambda I - M ) = 
\begin{bmatrix}
\lambda -3-4K_{1} & 4K_{2} \\
3K_{1} & \lambda -2-3K_{2}
\end{bmatrix}
=
( \lambda -3-4K_{1} )( \lambda -2-3K_{2} ) - 3K_{1} \cdot 4K_{2}
= 
\end{align*}
\begin{align*}
=
&\lambda^{2}-2\lambda-3K_{2}\lambda-3\lambda+6+9K_{2}-4K_{1}\lambda+8K_{1}+12K_{1}K_{2}-12K_{1}K_{2}
= \\
&\lambda^{2}+(-4K_{1}-3K_{2}-5 )\lambda+8K_{1}+9K_{2}+6
\end{align*}
W ogólnym przypadku można w tym miejscu skorzystać ze wzorów Viete'a, ale tutaj od razu widać, że aby \( lambda_{1} = \lambda_{2} = 0 \) musi zachodzić: 
\begin{align*}
\begin{cases}
-4K_{1}-3K_{2}-5 = 0 \\
8K_{1}+9K_{2}+6 = 0
\end{cases}
\implies
\begin{cases}
-8K_{1}-6K_{2}-10 = 0 \\
8K_{1}+9K_{2}+6 = 0
\end{cases}
\implies
3K_{2}-4 = 0
\end{align*}
\begin{align*}
K_{2} = \frac{4}{3} \\
8K_{1} = -8 - 10 \\
K_{1} = \frac{-9}{4}
\end{align*}

\newpage
\begin{framed}
\textbf{Zadanie 4 - Egzamin} \\ 
Do ciągłego systemu dynamicznego opisanego równaniami
\begin{align*}
\dot{x}(t)=
\begin{bmatrix}
0 & \pi & 0 \\
-\pi & 0 & 0 \\
0 & 0 & \pi
\end{bmatrix}
x(t) + 
\begin{bmatrix}
0 \\
0 \\
\pi 
\end{bmatrix}
u(t)
\end{align*}
podłączono ekstrapolator rzędu zerowego na wejściu i impulsator na wyjściu, przy czym pracują one synchronicznie z okresem próbkowania \( h = 1s \). Wyliczyć parametry systemu dyskretnego odpowiadające takiemu połączeniu. Do powstałego modelu dyskretnego wpięto sterowanie \( u(k) = Kx(k) \). Dobrać wartość \( K \) tak żeby układ zamknięty był:
\begin{itemize}
\item asymptotycznie stabilny
\item stabilny
\end{itemize}
Wskazówka: Sterowanie ma postać \( u(k) = K_1x_1(k)+K_2x_2(k)+K_3x_3(k) \).
\end{framed}
Tu macierz jest w postaci kanonicznej Jordana więc parametry wyznaczyć łatwo, później trochę trudniej... 






\newpage
\section{Badanie zachowania układu}
\subsection{Zbiór zadań}
\begin{framed}
\textbf{\colorbox{green}{Zadanie 1 - Zadania od Bauera do kolokwium II }} \\ 
Dla systemu
\begin{align*}
\ddot{x}(t)+2\xi \omega_{0}\dot{x}(t)+\omega^{2}x(t) = 0
\end{align*}
zbadać zachowanie się układu w zależności od \( \xi \) i \( \omega_{0} \). Zaznaczyć odpowiednie obszary na płaszczyźnie \( \xi \times \omega_{0} \). 
\end{framed}

\begin{framed}
\textbf{\colorbox{green}{Zadanie 2 - Zadania od Bauera do kolokwium I}} \\ 
Dla jakich wartości parametrów \( k_{1} \) i \( k_{2} \) system dynamiczny 
\begin{align*}
\dot{x}(t)=
\begin{bmatrix}
0 & 1 \\
-k_{1} & -k_{2}
\end{bmatrix}
x(t)
\end{align*}
będzie asymptotycznie stabilny. 
\end{framed}

\begin{framed}
\textbf{\colorbox{green}{Zadanie 3 - Zadania od Bauera do kolokwium I / Egzamin }} \\ 
Zbadać charakter pracy układu 
\begin{align*}
&\ddot{x}(t)+\dot{x}(t)+x(t) = u(t) \\
&u(t)=Kx(t) 
\end{align*}
w zależności od parametru K. Zaznaczyć wszystkie istotne rodzaje zachowań na osi liczbowej. 
\end{framed}

\begin{framed}
\textbf{\colorbox{green}{Zadanie 4 - Egzamin }} \\ 
Dany jest układ opisany następującymi równaniami 
\begin{align*}
&\dot{x}(t)=Ax(t)+Bu(t) \\
&y(t)=Cx(t) \\
&x(0)=0 \\
&u(t)=Ky(t) \\
&A =
\begin{bmatrix}
-1 & 1 \\
-1 & 0 
\end{bmatrix}
B = 
\begin{bmatrix}
0 \\
1
\end{bmatrix}
C =
\begin{bmatrix}
1 & 0
\end{bmatrix} 
\end{align*}
Dobrać takie \( K \in \mathbb{R} \) aby system zamknięty był asymptotycznie stabilny.
\end{framed}

\begin{framed}
\textbf{\colorbox{green}{Zadanie 5 - Informatyka Modelowanie }} \\ 
Dla systemu\\
	$\begin{array}{rcl}x(t)+4\ddot{x}(t)+\dot{x}(t)&=&u(t) \\ u(t)&=&k_1\dot{x}(t)-k_2x(t)\end{array}$\\
zbadać zachowanie się ukłądu w zależności od $k_1$ i $k_2$. Zaznaczyć odpowiednie obszary na płaszczyźnie $k_1 \times k_2$
\end{framed}



\begin{framed}
\textbf{Zadanie 6 - Egzamin} \\ 
Dany jest system dynamiczny opisany równaniem:
\begin{align*}
&\ddot{x}(t)=a\dot{x}(t)+bx(t)=u(t) \\
&u(t)=K_1x(t)+K_2\dot{x}(t) \\
& \text{gdzie} \\
& a, \; b \; \text{to pewne stałe rzeczywiste}
\end{align*}
Dobrać parametry regulatora \( K_1 \; K_2 \) ,tak by równanie charakterystyczne układu miało zadane z góry pierwiastki \( \lambda _1 \; \lambda _2 \). 
\end{framed}


\begin{framed}
\textbf{Zadanie 7 - Egzamin} \\ 
Korzystając z kryterium Hurwitza określić stabilność układu automatycznej regulacji składającego się z obiektu o transmitancji \( G(s) = \frac{K}{s(T_1s+1)(T_2s+1)} \) i regulatora \( G_r(s) = K_r \) , gdzie \( T_1 = 1,5 \; , T_2=0,5 \; , K =5 \; , K_r = 0,5 \).  
\begin{figure}[H]
\begin{center}
\begin{tikzpicture}[scale=1,inner sep=0.4mm]
\node (sum)[place_1, label=below right:{$-$}] at (-1,0) {};
\node (pid) [trans] at (2,0) {$G_0(s)$};
\node (feed) [trans] at (2,-2) {$G_R(s)$};
% \node (env) [trans] at (3.75,1.5) {$env_{1}$};

% \draw [->] (sum) to (pid);
\path[->] (sum) edge node [above ] {$e$} (pid);
\path[->] (-3,0) edge node [above ] {$u \quad \quad +$} (sum);
\path[->] (pid) edge node [above ] {$y$} (6,0);
%\path[->]  (10,0) - (10,-3) - (-1,-3) - (sum);
\draw [->] (4,0) -- (4,-2) -- (feed);
\draw [->] (feed) -- (-1,-2) -- (sum); 

\end{tikzpicture}
\end{center}
\end{figure}
\end{framed}



\newpage
\subsection{Rozwiązania zadań ze zbioru}
\begin{framed}
\textbf{Zadanie 1 - Zadania od Bauera do kolokwium II } \\ 
Dla systemu
\begin{align*}
\ddot{x}(t)+2\xi \omega_{0}\dot{x}(t)+\omega_{0}^{2}x(t) = 0
\end{align*}
zbadać zachowanie się układu w zależności od \( \xi \) i \( \omega_{0} \). Zaznaczyć odpowiednie obszary na płaszczyźnie \( \xi \times \omega_{0} \). 
\end{framed}
Najpierw przechodzę na opis przy pomocy równań stanu, a później wyznaczam wartości własne. 
\begin{align*}
\begin{cases}
x_{1}(t) = x(t) \\
x_{2}(t) = \dot{x}(t)
\end{cases}
\implies
\begin{cases}
\dot{x}_{1}(t)=\dot{x}(t)=x_{2}(t) \\
\dot{x}_{2}(t)=\ddot{x}(t)=-2\xi \omega_{0}x_{2}(t)-\omega_{0}^{2}x_{1}(t) 
\end{cases}
\end{align*}
Macierz A:
\begin{align*}
A = 
\begin{bmatrix}
0 & 1 \\
-\omega_{0}^{2} & -2\xi \omega_{0}
\end{bmatrix}
\end{align*}
Wielomian charakterystyczny:
\begin{align*}
&det( \lambda I - A ) =
det 
\begin{bmatrix}
\lambda & -1 \\
\omega _{0}^{2} & \lambda +2\xi \omega _{0}
\end{bmatrix}=
\lambda ( \lambda +2\xi \omega _{0} ) + \omega _{0}^{2}= \\
&\lambda ^{2} +2\lambda \xi \omega _{0} + \omega_{0}^{2}
\end{align*}
Wartości własne:
\begin{align*}
&\Delta = 4\xi ^{2} \omega _{0}^{2} - 4\omega _{0}^{2} \\
&\sqrt{\Delta } = 2\omega_{0} \sqrt{\xi ^{2} - 1} \\
&\lambda_{1} = \frac{-2\xi \omega_{0} - 2\omega_{0} \sqrt{\xi ^{2} - 1} }{2 } , \quad \lambda_{2} = \frac{-2\xi \omega_{0} + 2\omega_{0} \sqrt{\xi ^{2} - 1} }{2 } \\
&\lambda_{1} = -\xi \omega_{0} - \omega_{0} \sqrt{\xi ^{2} - 1}  , \quad \lambda_{2} = -\xi \omega_{0} + \omega_{0} \sqrt{\xi ^{2} - 1} \\
&\lambda_{1} = -\omega_{0} ( \xi  + \sqrt{\xi ^{2} - 1} ) , \quad \lambda_{2} = -\omega_{0} ( \xi  - \sqrt{\xi ^{2} - 1} ) 
\end{align*}
Część urojona będzie niezerowa ( oscylacje ) dla:
\begin{align*}
\xi ^{2} - 1 < 0 \Leftrightarrow \xi \in (-1,1) 
\end{align*}
Wyrażenie pod pierwiastkiem nie wpłynie na stabilność w żaden sposób, bo będzie albo dodatnie albo zerowe albo da część urojoną więc wpływa jedynie na oscylacje. \\
Asymptotyczna stabilność:
\begin{align*}
\begin{cases}
\omega _{0} > 0 \\
\xi > 0
\end{cases}
\lor
\begin{cases}
\omega _{0} < 0 \\
\xi < 0
\end{cases}
\end{align*}
Stabilność:
\begin{align*}
\begin{cases}
\omega _{0} = 0 \\
\xi \in \mathbb{R}
\end{cases}
\lor
\begin{cases}
\omega _{0} \in \mathbb{R} \\
\xi = 0
\end{cases}
\end{align*}
Obszar na płaszczyźnie:
\begin{figure}[H]
\centerline{\includegraphics[scale=0.8]{kol2_7.jpg}}
\caption{Zachowanie systemu}
\label{fig:kol2_7}
\end{figure}

\newpage
\begin{framed}
\textbf{Zadanie 2 - Zadania od Bauera do kolokwium I / Egzamin} \\ 
Dla jakich wartości parametrów \( k_{1} \) i \( k_{2} \) system dynamiczny 
\begin{align*}
\dot{x}(t)=
\begin{bmatrix}
0 & 1 \\
-k_{1} & -k_{2}
\end{bmatrix}
x(t)
\end{align*}
będzie asymptotycznie stabilny. 
\end{framed}
Macierz jest w postaci Frobeniusa:
\begin{align*}
c_{1} = -(-k_{1})=k_1, \; c_0 = -(-k_2)=k_2, \; c_2 = 1 \; \text{(zawsze)}
\end{align*}
Wielomian charakterystyczny:
\begin{align*}
w(\lambda)=\lambda ^{2} + k_1 \lambda + k_2
\end{align*}
Dla systemu 2 rzędu wystarczy, że wszystkie współczynniki są dodatnie:
\begin{align*}
k_{1}, \; k_{2} > 0
\end{align*}


\newpage
\begin{framed}
\textbf{Zadanie 3 - Zadania od Bauera do kolokwium I / Egzamin } \\ 
Zbadać charakter pracy układu 
\begin{align*}
&\ddot{x}(t)+\dot{x}(t)+x(t) = u(t) \\
&u(t)=Kx(t) 
\end{align*}
w zależności od parametru K. Zaznaczyć wszystkie istotne rodzaje zachowań na osi liczbowej. 
\end{framed}
$\ddot{x}+\dot{x}+x=Kx$\\
$\ddot{x}+\dot{x}+x(1-K)=0$\\
$\lambda^2+\lambda+1-K=0$ wielomian charakterystyczny\\
$\left[\begin{array}{cc}1&0\\1&1-K\end{array}\right]$ macierz Hurwitza\\
$1-K>0 \Rightarrow K<1$\\
$\Delta=1-4(1-K)=-3+4K<0\Rightarrow K<\frac 34$\\
\begin{figure}[!h]
\begin{tikzpicture}
	\draw[thick][->](-2,0)--(3,0)node[right=.2] {$K$};

	\draw (0,-0.1) -- (0,0.1) node [below=4pt]{{0}};
	\draw (2,-0.1) -- (2,0.1) node [below=4pt]{{1}};
	\draw (1.5,-0.1) -- (1.5,0.1) node [below=4pt]{$\frac34$};

	\draw(-2,1)--(1.7,1)--(2,0);
	\node at (0,.75){asymptotycznie stabilny};
	\draw[->](2.5,-1)--(2.1,-0.1);
	\node at (3.1,-1.1){stabilny};
	\draw(4,1)--(2.3,1)--(2,0);
	\node at (3.2,.75){niestabilny};
	\draw(-2,.5)--(1.2,.5)--(1.5,0);
	\node at (0,.25){oscylacje};
	\draw(4,.5)--(1.8,.5)--(1.5,0);
	\node at (3.3,.25){brak oscylacji};

\end{tikzpicture}
\end{figure}

\newpage
\begin{framed}
\textbf{Zadanie 4 - Egzamin } \\ 
Dany jest układ opisany następującymi równaniami 
\begin{align*}
&\dot{x}(t)=Ax(t)+Bu(t) \\
&y(t)=Cx(t) \\
&x(0)=0 \\
&u(t)=Ky(t) \\
&A =
\begin{bmatrix}
-1 & 1 \\
-1 & 0 
\end{bmatrix}
B = 
\begin{bmatrix}
0 \\
1
\end{bmatrix}
C =
\begin{bmatrix}
1 & 0
\end{bmatrix} 
\end{align*}
Dobrać takie \( K \in \mathbb{R} \) aby system zamknięty był asymptotycznie stabilny.
\end{framed}
\begin{align*}
&u(t)=Ky(t) \\
&y(t) = Cx(t) = 
\begin{bmatrix}
1 & 0 
\end{bmatrix}
\begin{bmatrix}
x_{1} \\
x_{2}
\end{bmatrix}=
x_{1} \\
&u(t)=Kx_1 \\
&\dot{x}(t)=
\begin{bmatrix}
-1 & 1 \\
-1 & 0 
\end{bmatrix}
\begin{bmatrix}
x_1 \\
x_2
\end{bmatrix}
+
\begin{bmatrix}
0 \\
1
\end{bmatrix}
Kx_1 \\
&\dot{x}(t) = 
\begin{bmatrix}
-1 & 1 \\
-1 + K & 0
\end{bmatrix} \\
& \lambda I - A_{new} = 
\begin{bmatrix}
\lambda & 0 \\
0 & \lambda
\end{bmatrix} +
\begin{bmatrix}
1 & -1 \\
1-K & 0
\end{bmatrix}\\
&\lambda I - A_{new} = 
\begin{bmatrix}
\lambda + 1 & -1 \\
1 - K & \lambda
\end{bmatrix} \\
&det( \lambda I - A_{new} ) = \lambda ^{2} + \lambda + 1 - K
\end{align*}
Aby system II rzędu był asymptotycznie stabilny wystarczy, że wszystkie współczynniki będą dodatnie. 
\begin{align*}
&1-K>0\\
&K<1
\end{align*}

\newpage
\begin{framed}
\textbf{Zadanie 5 - Informatyka Modelowanie } \\ 
Dla systemu\\
	$\begin{array}{rcl}x(t)+4\ddot{x}(t)+\dot{x}(t)&=&u(t) \\ u(t)&=&k_1\dot{x}(t)-k_2x(t)\end{array}$\\
zbadać zachowanie się ukłądu w zależności od $k_1$ i $k_2$. Zaznaczyć odpowiednie obszary na płaszczyźnie $k_1 \times k_2$
\end{framed}

$x+4\ddot{x}+\dot{x}=k_1\dot{x}-k_2x$\\
$4\ddot{x}+(1-k_1)\dot{x}+(1+k_2)x=0$\\
wielomian charakterystyczny:\\
$x=e^{\lambda t} \ \ \ \dot{x}=\lambda e^{\lambda t} \ \ \ \ddot{x}=\lambda^2 e^{\lambda t}$\\
$4\lambda^2+(1-k_1)\lambda+1+k_2=0$\\
macierz Hurwitza dla wielomianu stopnia drugiego:
$a_0x^2+a_1x+a_2=0$\\
$\left[\begin{array}{cc}a_1&0\\a_0&a_2\end{array}\right] \ \ $ czyli $\ \ 
\left[\begin{array}{cc}1-k_1&0\\4&1+k_2\end{array}\right]$\\
żeby układ był stabilny to $|a_1|>0$ i $\left|\begin{array}{cc}a_1&0\\a_0&a_2\end{array}\right|>0$\\
więc:\\
$1-k_1>0 \Rightarrow\boxed{ k_1<1}$\\
$(1-k_1)(1+k_2)>0 \Rightarrow 1+k_2>0 \Rightarrow \boxed{k_2>-1}$\\
dla $\Delta<0$ występują oscylacje, więc:\\
$\Delta=(1-k_1)^2-4\cdot4(1-k_2)=(1-k_1)^2-16-16k_2<0$\\
$k_2>\frac{1}{16}(1-k_1)^2-1$\\
\begin{figure}[!h]
\begin{tikzpicture}
\fill[fill=blue, opacity=.2](-6,3)--(1,3)--(1,-1)--(-6,-1);
\draw [pattern=my north east lines, line space=8pt, draw=green!50!black](-6,3)--(-6.0,2.06)--(-5.75,1.84)--(-5.5,1.64)--(-5.25,1.44)--(-5.0,1.25)--(-4.75,1.06)--(-4.5,0.89)--(-4.25,0.72)--(-4.0,0.56)--(-3.75,0.41)--(-3.5,0.26)--(-3.25,0.12)--(-3.0,0.0)--(-2.75,-0.12)--(-2.5,-0.23)--(-2.25,-0.33)--(-2.0,-0.43)--(-1.75,-0.52)--(-1.5,-0.6)--(-1.25,-0.68)--(-1.0,-0.75)--(-0.75,-0.8)--(-0.5,-0.85)--(-0.25,-0.9)--(0.0,-0.93)--(0.25,-0.96)--(0.5,-0.98)--(0.75,-0.99)--(1.0,-1.0)--(1.25,-0.99)--(1.5,-0.98)--(1.75,-0.96)--(2.0,-0.93)--(2.25,-0.9)--(2.5,-0.85)--(2.75,-0.8)--(3.0,-0.75)--(3.25,-0.68)--(3.5,-0.6)--(3.75,-0.52)--(4.0,-0.43)--(4.25,-0.33)--(4.5,-0.23)--(4.75,-0.12)--(5.0,0.0)--(5.25,0.12)--(5.5,0.26)--(5.75,0.41)--(6.0,0.56)--(6,3);
\draw [color=blue](-6.0,2.06)--(-5.75,1.84)--(-5.5,1.64)--(-5.25,1.44)--(-5.0,1.25)--(-4.75,1.06)--(-4.5,0.89)--(-4.25,0.72)--(-4.0,0.56)--(-3.75,0.41)--(-3.5,0.26)--(-3.25,0.12)--(-3.0,0.0)--(-2.75,-0.12)--(-2.5,-0.23)--(-2.25,-0.33)--(-2.0,-0.43)--(-1.75,-0.52)--(-1.5,-0.6)--(-1.25,-0.68)--(-1.0,-0.75)--(-0.75,-0.8)--(-0.5,-0.85)--(-0.25,-0.9)--(0.0,-0.93)--(0.25,-0.96)--(0.5,-0.98)--(0.75,-0.99)--(1.0,-1.0)--(1.25,-0.99)--(1.5,-0.98)--(1.75,-0.96)--(2.0,-0.93)--(2.25,-0.9)--(2.5,-0.85)--(2.75,-0.8)--(3.0,-0.75)--(3.25,-0.68)--(3.5,-0.6)--(3.75,-0.52)--(4.0,-0.43)--(4.25,-0.33)--(4.5,-0.23)--(4.75,-0.12)--(5.0,0.0)--(5.25,0.12)--(5.5,0.26)--(5.75,0.41)--(6.0,0.56);



	\draw[very thick][->](-6,0)--(6,0)node[right=.2] {$k_1$};
	\draw[very thick][->](0,-3)--(0,3)node[above=.2] {$k_2$};

	\draw (-0.1,-1) -- (0.1,-1) node [left=3pt]{{-1}};
	\draw (5,-0.1) -- (5,0.1) node [below=4pt]{{5}};
	\draw (-3,-0.1) -- (-3,0.1) node [below=4pt]{{-3}};
	\draw (1,-0.1) -- (1,0.1) ;
	\node at (1.2,-0.2) {1};

	\draw[dashed, color=red](-6,-1)--(6,-1);
	\draw[dashed, color=red](1,-3)--(1,3);
	\draw[thick](1,-1) circle(.15);
\end{tikzpicture}
\end{figure}
\\
wewnątrz niebieskiego obszaru asymptotycznie stabilny $k_1<1 \wedge k_2>-1)$\\
na czerwonych prostych granicznych stabilny $(k_1=1 \vee k_2=-1)$ bez punktu wspólnego\\
niestabilny na przecięciu prostych i w pozostałych obszarach\\
oscylacje dla zakreskowanego $k_2>\frac{1}{16}(1-k_1)^2-1$

\newpage
\begin{framed}
\textbf{Zadanie 6 - Egzamin} \\ 
Dany jest system dynamiczny opisany równaniem:
\begin{align*}
&\ddot{x}(t)=a\dot{x}(t)+bx(t)=u(t) \\
&u(t)=K_1x(t)+K_2\dot{x}(t) \\
& \text{gdzie} \\
& a, \; b \; \text{to pewne stałe rzeczywiste}
\end{align*}
Dobrać parametry regulatora \( K_1 \; K_2 \) ,tak by równanie charakterystyczne układu miało zadane z góry pierwiastki \( \lambda _1 \; \lambda _2 \). 
\end{framed}

\newpage
\begin{framed}
\textbf{Zadanie 7 - Egzamin} \\ 
Korzystając z kryterium Hurwitza określić stabilność układu automatycznej regulacji składającego się z obiektu o transmitancji \( G(s) = \frac{K}{s(T_1s+1)(T_2s+1)} \) i regulatora \( G_r(s) = K_r \) , gdzie \( T_1 = 1,5 \; , T_2=0,5 \; , K =5 \; , K_r = 0,5 \).  
\begin{figure}[H]
\begin{center}
\begin{tikzpicture}[scale=1,inner sep=0.4mm]
\node (sum)[place_1, label=below right:{$-$}] at (-1,0) {};
\node (pid) [trans] at (2,0) {$G_0(s)$};
\node (feed) [trans] at (2,-2) {$G_R(s)$};
% \node (env) [trans] at (3.75,1.5) {$env_{1}$};

% \draw [->] (sum) to (pid);
\path[->] (sum) edge node [above ] {$e$} (pid);
\path[->] (-3,0) edge node [above ] {$u \quad \quad +$} (sum);
\path[->] (pid) edge node [above ] {$y$} (6,0);
%\path[->]  (10,0) - (10,-3) - (-1,-3) - (sum);
\draw [->] (4,0) -- (4,-2) -- (feed);
\draw [->] (feed) -- (-1,-2) -- (sum); 

\end{tikzpicture}
\end{center}
\end{figure}
\end{framed}




\newpage
\section{Uchyb regulacji}
\subsection{Zbiór zadań}
\begin{framed}
\textbf{\colorbox{green}{Zadanie 1 - Egzamin}} \\ 
Dla jakich wartości parametru \( k_{f} \in \mathbb{R} \) uchyb ustalony będzie równy \( 0 \) jeśli na układ z rysunku:
\begin{figure}[H]
\begin{center}
\begin{tikzpicture}[scale=1,inner sep=0.4mm]
\node (sum)[place_1, label=below right:{$-$}] at (-1,0) {};
\node (pid) [trans] at (2,0) {$G_0(s)$};
\node (feed) [trans] at (2,-2) {$G_R(s)$};
% \node (env) [trans] at (3.75,1.5) {$env_{1}$};

% \draw [->] (sum) to (pid);
\path[->] (sum) edge node [above ] {$e$} (pid);
\path[->] (-3,0) edge node [above ] {$u \quad \quad +$} (sum);
\path[->] (pid) edge node [above ] {$y$} (6,0);
%\path[->]  (10,0) - (10,-3) - (-1,-3) - (sum);
\draw [->] (4,0) -- (4,-2) -- (feed);
\draw [->] (feed) -- (-1,-2) -- (sum); 

\end{tikzpicture}
\end{center}
\end{figure}

gdzie \( G_{0}=\frac{k_p}{s} \) oraz \( G_{R}(s)=k_f \) podano sygnał \( u(t) = t \).
\end{framed}
\begin{framed}

\textbf{\colorbox{yellow}{Zadanie 2 - Egzamin}} \\ 
Dany jest układ regulacji jak na rysunku
\begin{figure}[H]
\begin{center}
\begin{tikzpicture}[scale=1,inner sep=0.4mm]
\node (sum)[place_1, label=below right:{$-$}] at (-3,0) {};
\node (sumsum)[place_1, label=below right:{$-$}] at (-1,0) {};
\node (pid) [trans] at (2,0) {$G_0(s)$};
\node (feed) [trans] at (2,-2) {$G_K(s)$};
% \node (env) [trans] at (3.75,1.5) {$env_{1}$};

% \draw [->] (sum) to (pid);
\draw[->] (sumsum) -- (pid);
\path[->] (-5,0) edge node [above ] {$u \quad +$} (sum);
\path[->] (sum) edge node [above ] {$e \quad +$} (sumsum);
\path[->] (pid) edge node [above ] {$y$} (6,0);
%\path[->]  (10,0) - (10,-3) - (-1,-3) - (sum);
\draw [->] (4,0) -- (4,-2) -- (feed);
\draw [->] (feed) -- (-1,-2) -- (sumsum); 
\draw [->] (5,0) -- (5,-3) -- (-3,-3) -- (sum);

\end{tikzpicture}
\end{center}
\end{figure}

gdzie obiekt \( G_0\) jest opisany transmitancją:
\begin{align*}
G_0(s)=\frac{K}{(T_1s+1)(T_2s+1)}
\end{align*}
Wyznaczyć transmitancję operatorową \( G_K(s) \) członu korekcyjnego, którego umieszczenie w torze dodatkowego sprzężenia zwrotnego spowoduje zanikanie uchybu ustalonego na wymuszenie skokowe \( 1(t) \). 
\end{framed}

 
\begin{framed}
\textbf{\colorbox{yellow}{Zadanie 3 - Egzamin}} \\
Dla układu regulacji z rysunku
\begin{figure}[H]
\centering
\begin{tikzpicture}[scale=1,inner sep=0.4mm]

\node (sum)[place_1, label=below right:{$-$}] at (-1,0) {};
\node (pid) [trans] at (2,0) {$G_C(s)$};
\node (sum1)[place_1, label=above right:{$-$}] at (5,0) {};
\node (objekt) [trans] at (6.7,0) {$G_0(s)$};
% \node (env) [trans] at (3.75,1.5) {$env_{1}$};

% \draw [->] (sum) to (pid);
\path[->] (sum) edge node [above ] {$e(t)$} (pid);
\path[->] (pid) edge node [above ] {$u(t) \; \; \; \; +$} (sum1);
\path[->] (-3,0) edge node [above ] {$r(t \quad +$} (sum);
\path[->] (sum1) edge node [above ] {} (objekt);
\path[->] (objekt) edge node [above ] {$y(t)$} (10,0);
\path[->] (5,2) edge node [left ] {$z(t)$} (sum1);
%\path[->]  (10,0) - (10,-3) - (-1,-3) - (sum);
\draw [->] (8,0) -- (8,-2) -- (-1,-2) -- (sum);
\end{tikzpicture}
\caption{Schemat rozważanego układu regulacji}
\label{fig:opt_par_1}
\end{figure}

gdzie 
\begin{align*}
G_C(s)=KG_0(s)=\frac{1}{s+1}
\end{align*}
wyznaczyć wartość początkową i końcową uchybu regulacji dla wymuszeń \( u(t) = u_0 \cdot 1(t) \) oraz \( z(t) = z_0 \cdot 1(t) \) działających równocześnie na układ. 
\end{framed}

\newpage
\subsection{Rozwiązania zadań ze zbioru}
\begin{framed}
\textbf{Zadanie 1 - Egzamin} \\ 
Dla jakich wartości parametru \( k_{f} \in \mathbb{R} \) uchyb ustalony będzie równy \( 0 \) jeśli na układ z rysunku:
\begin{figure}[H]
\begin{center}
\begin{tikzpicture}[scale=1,inner sep=0.4mm]
\node (sum)[place_1, label=below right:{$-$}] at (-1,0) {};
\node (pid) [trans] at (2,0) {$G_0(s)$};
\node (feed) [trans] at (2,-2) {$G_R(s)$};
% \node (env) [trans] at (3.75,1.5) {$env_{1}$};

% \draw [->] (sum) to (pid);
\path[->] (sum) edge node [above ] {$e$} (pid);
\path[->] (-3,0) edge node [above ] {$u \quad \quad +$} (sum);
\path[->] (pid) edge node [above ] {$y$} (6,0);
%\path[->]  (10,0) - (10,-3) - (-1,-3) - (sum);
\draw [->] (4,0) -- (4,-2) -- (feed);
\draw [->] (feed) -- (-1,-2) -- (sum); 

\end{tikzpicture}
\end{center}
\end{figure}
gdzie \( G_{0}=\frac{k_p}{s} \) oraz \( G_{R}(s)=k_f \) podano sygnał \( u(t) = t \).
\end{framed}


W układzie nie ma zakłóceń więc rozważam uchyb od wartości zadanej. 
Z definicji uchyb ustalony od wartości zadanej to granica: 
\begin{align*}
\lim _{t \rightarrow \infty} e_r(t) = \lim _{s \rightarrow 0}sE_r (s)
\end{align*}
Aby ją obliczyć potrzebne jest więcej wyznaczenie transformaty Laplace'a uchybu. Rozpatrując układ w dziedzinie operatora \( s \):
\begin{align*}
&\begin{cases}
E(s)= U(s)-Y(s) \cdot G_R (s) \\
Y(s) = E(s) \cdot G_0 (s) 
\end{cases}
\implies
E(s)=U(s)-E(s)\cdot G_0(s) \cdot G_R(s) \\
&E(s)=\frac{U(s)}{1+G_0(s)G_R(s)}
\end{align*}
\begin{align*}
	& \mathscr{L}^{-1}\{\frac{a}{s}\} = a \\
	& \mathscr{L}^{-1}\{\frac{a}{s-b}\} = ae^{bt} \\
	& \mathscr{L}^{-1}\{\frac{a}{s^2}\} = a\cdot t
\end{align*}
\begin{align*}
&U(s) = \mathscr{L}\{t\}=\frac{1}{s^2} \\
&E(s) = \frac{\frac{1}{s^2}}{1+\frac{k_p k_f}{s}} = \frac{1}{s^2+k_p k_f s}\\
&\lim _{s \rightarrow 0}sE_r (s) = \lim_{s \rightarrow 0} s \cdot \frac{1}{s^2+k_p k_f s} = \lim _{s \rightarrow 0} \frac{1}{s+k_p k_f} =
\frac{1}{k_p k_f} \\
& \frac{1}{k_p k_f} = 0 
\end{align*}
Nie istnieje takie \( k_f \), ponieważ \( \frac{1}{k_f k_p} = 1 \Longleftrightarrow k_f \rightarrow \infty \). 

\newpage
\begin{framed}
\textbf{Zadanie 2 - Egzamin} \\ 
Dany jest układ regulacji jak na rysunku
\begin{figure}[H]
\begin{center}
\begin{tikzpicture}[scale=1,inner sep=0.4mm]
\node (sum)[place_1, label=below right:{$-$}] at (-3,0) {};
\node (sumsum)[place_1, label=below right:{$-$}] at (-1,0) {};
\node (pid) [trans] at (2,0) {$G_0(s)$};
\node (feed) [trans] at (2,-2) {$G_K(s)$};
% \node (env) [trans] at (3.75,1.5) {$env_{1}$};

% \draw [->] (sum) to (pid);
\draw[->] (sumsum) -- (pid);
\path[->] (-5,0) edge node [above ] {$u \quad +$} (sum);
\path[->] (sum) edge node [above ] {$e \quad +$} (sumsum);
\path[->] (pid) edge node [above ] {$y$} (6,0);
%\path[->]  (10,0) - (10,-3) - (-1,-3) - (sum);
\draw [->] (4,0) -- (4,-2) -- (feed);
\draw [->] (feed) -- (-1,-2) -- (sumsum); 
\draw [->] (5,0) -- (5,-3) -- (-3,-3) -- (sum);

\end{tikzpicture}
\end{center}
\end{figure}

gdzie obiekt \( G_0\) jest opisany transmitancją:
\begin{align*}
G_0(s)=\frac{K}{(T_1s+1)(T_2s+1)}
\end{align*}
Wyznaczyć transmitancję operatorową \( G_K(s) \) członu korekcyjnego, którego umieszczenie w torze dodatkowego sprzężenia zwrotnego spowoduje zanikanie uchybu ustalonego na wymuszenie skokowe \( 1(t) \). 
\end{framed}
Najpierw obliczam transmitancję zastępczą układu wewnętrznego składającego się z \( G_0(s) \) oraz \( G_K(s) \) ze wzoru:
\begin{align*}
G_{Z1}(s)=\frac{G_0(s)}{1+G_K(s)G_0(s)}
\end{align*}
Wówczas powstaje układ ze zwykłym sprzężeniem zwrotnym ujemnym i obliczam dla niego uchyb i przyrównuję do zera ze wzoru:
\begin{align*}
\lim_{t\rightarrow s } s\cdot E(s) = 0
\end{align*}
Dopisać obliczenia...

\newpage
\begin{framed}
\textbf{Zadanie 3 - Egzamin} \\
Dla układu regulacji z rysunku
\begin{figure}[H]
\centering
\begin{tikzpicture}[scale=1,inner sep=0.4mm]
\node (sum)[place_1, label=below right:{$-$}] at (-1,0) {};
\node (pid) [trans] at (2,0) {$G_C(s)$};
\node (sum1)[place_1, label=above right:{$-$}] at (5,0) {};
\node (objekt) [trans] at (6.7,0) {$G_0(s)$};
% \node (env) [trans] at (3.75,1.5) {$env_{1}$};

% \draw [->] (sum) to (pid);
\path[->] (sum) edge node [above ] {$e(t)$} (pid);
\path[->] (pid) edge node [above ] {$u(t) \; \; \; \;+$} (sum1);
\path[->] (-3,0) edge node [above ] {$r(t) \quad +$} (sum);
\path[->] (sum1) edge node [above ] {} (objekt);
\path[->] (objekt) edge node [above ] {$y(t)$} (10,0);
\path[->] (5,2) edge node [left ] {$z(t)$} (sum1);
%\path[->]  (10,0) - (10,-3) - (-1,-3) - (sum);
\draw [->] (8,0) -- (8,-2) -- (-1,-2) -- (sum);
\end{tikzpicture}
\caption{Schemat rozważanego układu regulacji}
\label{fig:opt_par_1}
\end{figure}
gdzie 
\begin{align*}
G_C(s)=KG_0(s)=\frac{1}{s+1}
\end{align*}
wyznaczyć wartość początkową i końcową uchybu regulacji dla wymuszeń \( u(t) = u_0 \cdot 1(t) \) oraz \( z(t) = z_0 \cdot 1(t) \) działających równocześnie na układ. 
\end{framed}

Uchyb regulacji to suma uchybów pochodzących od wartości zadanej i od zakłóceń. A więc w chwili początkowej uchyb ten wynosi \( u_0 + z_0 \). \\
W chwili końcowej uchyb będzie równy sumie granic:
\begin{align*}
e_{ustalony}=\lim_{s\rightarrow 0}s\cdot E_z(s) +  \lim_{s\rightarrow 0}s\cdot E_r(s) 
\end{align*}
Najpierw należy obliczyć \( E_z(s) \) oraz \( E_r(s) \), a następnie liczymy granicę. \\
Zapisuje równania w przestrzeni zmiennej \( s \) : \\
\begin{align*}
\begin{cases}
&E_r(s)=R(s) - Y(s) \\
&Y(s) = E_z(s)\cdot G_0(s) \\
&E_z(s) = U(s)-Z(s) \\
&U(s) = E_r(s) \cdot G_C(s)
\end{cases}
\end{align*}
gdzie \( Z(s) \; R(s) \; G_C(s) \; G_0(s) \; \) są znane. Z równań tych łatwo wyliczyć uchyby podstawiając drugie równanie do pierwszego oraz czwarte do trzeciego.  \\
Uzupełnić obliczenia...






\newpage
\section{Obliczanie transmitancji}
\subsection{Zbiór zadań}
\begin{framed}
\textbf{\colorbox{green}{Zadanie 1 - Zadania od Bauera do kolokwium II}} \\ 
Oblicz transmitancję układu opisanego równaniami
\begin{align*}
\dot{x}(t)=Ax(t)+Bu(t) \\
y(t)=Cx(t)+Du(t)
\end{align*}
przy czym: \\
\begin{align*}
A = 
\begin{bmatrix}
0 & 1 \\
0 & 0
\end{bmatrix}
B = 
\begin{bmatrix}
1 & 0 \\
0 & 2
\end{bmatrix}
C = 
\begin{bmatrix}
1 & 0 \\
0 & 1
\end{bmatrix}
D = 
\begin{bmatrix}
1 & 0 \\
0 & 1
\end{bmatrix}
\end{align*}
\end{framed}

\begin{framed}
\textbf{Zadanie 2 - Zadania od Bauera do kolokwium I } \\ 
Układ jest opisany równaniami stanu w postaci
\begin{align*}
\dot{x}(t)=Ax(t)+Bu(t) \\
y(t)=Cx(t)+Du(t)
\end{align*}
z macierzami: \\
\begin{align*}
A = 
\begin{bmatrix}
4 & 2 \\
2 & 3
\end{bmatrix}
B = 
\begin{bmatrix}
1  \\
1
\end{bmatrix}
C = 
\begin{bmatrix}
0 & 1
\end{bmatrix}
\end{align*}
Znaleźć transmitancję operatorową układu przy założeniu zerowych warunków początkowych \( x(0) = 0 \). 
\end{framed}

\begin{framed}
\textbf{Zadanie 3 - Zadania od Bauera do kolokwium I } \\ 
Układ jest opisany równaniami stanu w postaci
\begin{align*}
\dot{x}(t)=Ax(t)+Bu(t) \\
y(t)=Cx(t)+Du(t)
\end{align*}
z macierzami: \\
\begin{align*}
A = 
\begin{bmatrix}
-4 & -2 \\
0 & 3
\end{bmatrix}
B = 
\begin{bmatrix}
3  \\
3
\end{bmatrix}
C = 
\begin{bmatrix}
1 & 0
\end{bmatrix}
\end{align*}
Znaleźć transmitancję operatorową układu przy założeniu zerowych warunków początkowych \( x(0) = 0 \). 
\end{framed}

\begin{framed}
\textbf{\colorbox{green}{Zadanie 4 - Egzamin}} \\ 
Dany jest układ postaci:
\begin{figure}[H]
\centering
\begin{tikzpicture}[scale=1,inner sep=0.4mm]

\node (pid) [trans] at (2,0) {$P$};
\node (objekt) [trans] at (6.7,0) {$G$};
% \node (env) [trans] at (3.75,1.5) {$env_{1}$};

% \draw [->] (sum) to (pid);
\path[->] (-2,0) edge node [above ] {$u(t)$} (pid);
\path[->] (pid) edge node [above ] {$y(t)$} (objekt);
\path[->] (objekt) edge node [above ] {$v(t)$} (10,0);
\end{tikzpicture}
\label{fig:opt_par_1}
\end{figure}
gdzie system P jest opisany równaniem:
\begin{align*}
&\ddot{y}(t)+3\dot{y}(t)+y(t)=3\dot{u}(t)-u(t) \\
&\text{a system G równaniem} \\
&\ddot{v}(t)-6\dot{v}(t)+2v(t)=\dot{y}(t)+4y(t)
\end{align*}
Znaleźć równania stanu tego układu.
\end{framed}

\begin{framed}
\textbf{\colorbox{green}{Zadanie 5 - Zadania od Bauera do kolokwium I}} \\ 
Za pomocą transmitancji znaleźć odpowiedź układu
\begin{align*}
&\dot{x}(t)=5x(t)+u(t) \\
&y(t) = x(t)
\end{align*}
na skok jednostkowy, czyli funkcję postaci: 
\begin{align*}
u(t) =
\begin{cases}
0, \; t < 0 \\
1, \; t \geq 0
\end{cases}
\end{align*}
Przyjąć \( x(0) = 0 \)
\end{framed}

\newpage
\subsection{Rozwiązania zadań}
\begin{framed}
\textbf{Zadanie 1 - Zadania od Bauera do kolokwium II } \\ 
Oblicz transmitancję układu opisanego równaniami
\begin{align*}
\dot{x}(t)=Ax(t)+Bu(t) \\
y(t)=Cx(t)+Du(t)
\end{align*}
przy czym: \\
\begin{align*}
A = 
\begin{bmatrix}
0 & 1 \\
0 & 0
\end{bmatrix}
B = 
\begin{bmatrix}
1 & 0 \\
0 & 2
\end{bmatrix}
C = 
\begin{bmatrix}
1 & 0 \\
0 & 1
\end{bmatrix}
D = 
\begin{bmatrix}
1 & 0 \\
0 & 1
\end{bmatrix}
\end{align*}
\end{framed}

\begin{align*}
G(s) = C(sI-A)^{-1}B + D 
\end{align*}
Podstawiam do wzoru i liczę. Macierz odwrotną dla 2x2, na przekątnej zamieniamy, na drugiej przekątnej minusy, wszystko przez wyznacznik.
\begin{align*}
& G(s)=
\begin{bmatrix}
1 & 0 \\
0 & 1
\end{bmatrix}
\begin{bmatrix}
s & -1 \\
0 & s
\end{bmatrix}^{-1}
\begin{bmatrix}
1 & 0 \\
0 & 2
\end{bmatrix}
+
\begin{bmatrix}
1 & 0 \\
0 & 1
\end{bmatrix}
= \\
&
=
\frac{1}{s^{2}}
\begin{bmatrix}
1 & 0 \\
0 & 1
\end{bmatrix}
\begin{bmatrix}
s & 1 \\
0 & s
\end{bmatrix}
\begin{bmatrix}
1 & 0 \\
0 & 2
\end{bmatrix}
+
\begin{bmatrix}
1 & 0 \\
0 & 1
\end{bmatrix}=\\
&
=
\frac{1}{s^{2}}
\begin{bmatrix}
s & s+1 \\
0 & s
\end{bmatrix}
\begin{bmatrix}
1 & 0 \\
0 & 2
\end{bmatrix}
+
\begin{bmatrix}
1 & 0 \\
0 & 1
\end{bmatrix}
= \\
& =
\frac{1}{s^{2}}
\begin{bmatrix}
s & 2s+2 \\
0 & 2s
\end{bmatrix}
+
\begin{bmatrix}
1 & 0 \\
0 & 1
\end{bmatrix} =
\\
& =
\begin{bmatrix}
\frac{1}{s} & \frac{2s+2}{s^{2}} \\ \\
0 & \frac{2}{s}
\end{bmatrix}
+
\begin{bmatrix}
1 & 0 \\
0 & 1
\end{bmatrix} =
\\
& =
\begin{bmatrix}
\frac{1}{s}+1 & \frac{2s+2}{s^{2}} \\
0 & \frac{2}{s}+1
\end{bmatrix}
\end{align*}

\newpage
\begin{framed}
\textbf{Zadanie 2 - Zadania od Bauera do kolokwium I } \\ 
Oblicz transmitancję układu opisanego równaniami
\begin{align*}
\dot{x}(t)=Ax(t)+Bu(t) \\
y(t)=Cx(t)+Du(t)
\end{align*}
przy czym: \\
\begin{align*}
A = 
\begin{bmatrix}
4 & 2 \\
2 & 3
\end{bmatrix}
B = 
\begin{bmatrix}
1  \\
1
\end{bmatrix}
C = 
\begin{bmatrix}
0 & 1
\end{bmatrix}
\end{align*}

\end{framed}

\newpage
\begin{framed}
\textbf{Zadanie 3 - Zadania od Bauera do kolokwium I } \\ 
Układ jest opisany równaniami stanu w postaci
\begin{align*}
\dot{x}(t)=Ax(t)+Bu(t) \\
y(t)=Cx(t)+Du(t)
\end{align*}
z macierzami: \\
\begin{align*}
A = 
\begin{bmatrix}
-4 & -2 \\
0 & 3
\end{bmatrix}
B = 
\begin{bmatrix}
3  \\
3
\end{bmatrix}
C = 
\begin{bmatrix}
1 & 0
\end{bmatrix}
\end{align*}
Znaleźć transmitancję operatorową układu przy założeniu zerowych warunków początkowych \( x(0) = 0 \). 
\end{framed}

\newpage
\begin{framed}
\textbf{Zadanie 4 - Egzamin} \\ 
Dany jest układ postaci:
\begin{figure}[H]
\centering
\begin{tikzpicture}[scale=1,inner sep=0.4mm]

\node (pid) [trans] at (2,0) {$P$};
\node (objekt) [trans] at (6.7,0) {$G$};
% \node (env) [trans] at (3.75,1.5) {$env_{1}$};

% \draw [->] (sum) to (pid);
\path[->] (-2,0) edge node [above ] {$u(t)$} (pid);
\path[->] (pid) edge node [above ] {$y(t)$} (objekt);
\path[->] (objekt) edge node [above ] {$v(t)$} (10,0);
\end{tikzpicture}
\label{fig:opt_par_1}
\end{figure}
gdzie system P jest opisany równaniem:
\begin{align*}
&\ddot{y}(t)+3\dot{y}(t)+y(t)=3\dot{u}(t)-u(t) \\
&\text{a system G równaniem} \\
&\ddot{v}(t)-6\dot{v}(t)+2v(t)=\dot{y}(t)+4y(t)
\end{align*}
Znaleźć równania stanu tego układu.
\end{framed}
\begin{align*}
\begin{cases}
&\ddot{y}(t)+3\dot{y}(t)+y(t)=3\dot{u}(t)-u(t) \\
&\ddot{v}(t)-6\dot{v}(t)+2v(t)=\dot{y}(t)+4y(t)
\end{cases}
\end{align*}
Działam operatorem Laplace'a na każde z równań i przyjmuje zerowe warunki początkowe.
\begin{align*}
\begin{cases}
s^{2}Y(s)+3sY(s)+Y(s)=3sU(s)-U(s) \\
s^{2}V(s)-6sV(s)+2V(s)=sY(s)+5Y(s)
\end{cases}
\end{align*}
\begin{align*}
P(s) = \frac{Y(s)}{U(s)}=\frac{3s-1}{s^2+3s+1} \\
G(s) = \frac{V(s)}{Y(s)}=\frac{s+5}{s^2-6s+2}
\end{align*}
Transmitancja zastępcza całego układu:
\begin{align*}
&G_z = P(s)\cdot G(S) = \frac{(3s-1)(s+5)}{(s^2+3s+1)(s^2-6s+2)} =
\frac{3s^2+15s-s-5}{s^4-6s^3+2s^2+3s^3-18s^2+6s+s^2-6s+2} = \\
&\frac{3s^2+14s-5}{s^4-3s^3-15s^2+0s+2}
\end{align*}
Przyjmując postać kanoniczną system można przedstawić w postaci:
\begin{align*}
&A = 
\begin{bmatrix}
0 & 1 & 0 & 0 \\
0 & 0 & 1 & 0 \\
0 & 0 & 0 & 1 \\
-c_0 & -c_1 & -c_2 & -c_3
\end{bmatrix}
=
\begin{bmatrix}
0 & 1 & 0 & 0 \\
0 & 0 & 1 & 0 \\
0 & 0 & 0 & 1 \\
-2 & 0 & 15 & 3
\end{bmatrix}
\;
B = \begin{bmatrix}
0 \\
0 \\
0 \\ 
1
\end{bmatrix}
\;
C^T = 
\begin{bmatrix}
a_0 & a_1 & a_2 & a_3
\end{bmatrix}
=
\begin{bmatrix}
-5 & 14 & 3 & 0  
\end{bmatrix} \\
& D = 0
\end{align*}
Ponieważ dla transmitancji:
\begin{align*}
\frac{Y(s)}{U(s)}=\frac{a_ns^n+a_{n-1}s^{n-1}+\dots + a_1s + a_0}{c_ns^n+c_{n-1}s^{n-1}+\dots + c_1s + c_0}
\end{align*}
Macierze są postaci:
\begin{align*}
&A = 
\begin{bmatrix}
0 & 1 & 0 & 0 \\
0 & 0 & 1 & 0 \\
0 & 0 & 0 & 1 \\
-c_0 & -c_1 & \dots & -c_{n-1}
\end{bmatrix}
\;
B = \begin{bmatrix}
0 \\
0 \\
0 \\ 
1
\end{bmatrix}
 \\
&C^T = 
\begin{bmatrix}
(a_0-c_0\cdot a_n)& (a_1-c_1\cdot a_n) & (a_2-c_2\cdot a_n) & \dots & (a_{n-1}-c_{n-1}\cdot a_n )
\end{bmatrix}
\\
& D = 0
\end{align*}

\newpage
\begin{framed}
\textbf{Zadanie 5 - Zadania od Bauera do kolokwium I} \\ 
Za pomocą transmitancji znaleźć odpowiedź układu
\begin{align*}
&\dot{x}(t)=5x(t)+u(t) \\
&y(t) = x(t)
\end{align*}
na skok jednostkowy, czyli funkcję postaci: 
\begin{align*}
u(t) =
\begin{cases}
0, \; t < 0 \\
1, \; t \geq 0
\end{cases}
\end{align*}
Przyjąć \( x(0) = 0 \)
\end{framed}

\begin{gather*}
u(t) = \begin{cases}
0 & \text{dla } t < 0 \\
1 & \text{dla } t \geqslant  0
\end{cases} \\
\begin{cases}
\dot{x}(t) = 5x(t) + u(t) \\
y(t) = x(t)
\end{cases} \\
\mathbf{A} = 5, \mathbf{B} = 1, \mathbf{C} = 1
\end{gather*}

Dla $x(0) = 0$ mamy:
\[ G(s) = \mathbf{C} (s\mathbf{I} - \mathbf{A})^{-1} \mathbf{B} \]

\begin{equation*}
\begin{aligned}
& G(s) = 1 \cdot [s - 5]^{-1} \cdot 1 = \frac{1}{s-5} \\
& U(s) = \mathscr{L}\{u(t)\} = \frac{1}{s} \\
& Y(s) = G(s) \cdot U(s) = \frac{1}{s-5} \cdot \frac{1}{s} \\
& \frac{1}{(s-5)s} = \left. \frac{A}{s-5} + \frac{B}{s} \right| \cdot (s-5)s \\
& 1 = A \cdot s + B \cdot s - 5B = s(A+B) - 5B \\
\end{aligned}
\end{equation*}
\begin{equation*}
\begin{aligned}
& \begin{cases}
A+B = 0 \\
-5B = 1
\end{cases} \\
& \begin{cases}
B = \frac{-1}{5} \\
A = \frac{1}{5}
\end{cases} \\
& Y(s) = \frac{\frac15}{s-5} + \frac{-\frac15}{s} \\
& y(t) = \mathscr{L}^{-1}\{Y(s)\}
& \begin{array}{rl}
	& \mathscr{L}^{-1}\{\frac{a}{s}\} = a \\
	& \mathscr{L}^{-1}\{\frac{a}{s-b}\} = ae^{bt}
\end{array} \\
& y(t) = \mathscr{L}^{-1}\left\{ \frac{\frac15}{s-5} + \frac{-\frac15}{s} \right\} = \boxed{ \frac15e^{5t} - \frac15 }
\end{aligned}
\end{equation*}


\newpage
\section{Wyznaczanie \( e^{At} \)}
\subsection{Zbiór zadań}
\begin{framed}
\textbf{\colorbox{green}{Zadanie 1 - Zadania od Bauera do kolokwium I }} \\ 
Naszkicować rozwiązania równania różniczkowego:
\begin{align*}
&\dot{x}(t)=-x(t)+1 \\
&\text{dla} \; x(0) = x_i \; i = 1,2,3 \; t \geq 0 \\
&x_1 = 0 \; x_2 = 1 \; x_3 = 2 
\end{align*}
\end{framed}

\begin{framed}
\textbf{\colorbox{green}{Zadanie 2 - Zadania od Bauera do kolokwium I }} \\ 
Dane jest równanie różniczkowe 
\begin{align*}
&\dot{x}(t)=-x(t)+2 \\
&x(0)=0, \; t \geq 0 \\ 
\end{align*}
Po jakim czasie \( t_k \) zachodzi \( x(t_k) = 1 \)
\end{framed}

\begin{framed}
\textbf{\colorbox{green}{Zadanie 3 - Zadania od Bauera do kolokwium I }} \\ 
Wyznaczyć macierz \( e^{At} \) dla macierzy:
\begin{align*}
A = 
\begin{bmatrix}
-2 & 1 \\
-2 & 0
\end{bmatrix}
\end{align*}
\end{framed}

\begin{framed}
\textbf{\colorbox{green}{Zadanie 4 - Zadania od Bauera do kolokwium I }} \\ 
Wyznaczyć macierz \( e^{At} \) dla macierzy:
\begin{align*}
A = 
\begin{bmatrix}
-\frac{1}{2 }& -\frac{1}{2} \\ \\
\frac{1}{2} & -\frac{3}{2}
\end{bmatrix}
\end{align*}
\end{framed}

\begin{framed}
\textbf{Zadanie 5 - Zadania od Bauera do kolokwium I } \\ 
Wyznaczyć rozwiązanie \( x(t) \), \( t \geq 0 \) równania:
\begin{align*}
&\ddot{x}(t)+\dot{x}(t)+3x(t)=0 \\
&x(0)=1, \; \dot{x}(0)=0
\end{align*}
\end{framed}

\begin{framed}
\textbf{\colorbox{green}{Zadanie 6 - Zadania od Bauera do kolokwium I }} \\ 
Dany jest system opisany równaniem:
\begin{align*}
&\dot{x}_1(t)=-\pi x_2(t) \\
&\dot{x}_2(t)=\pi x_1(t) \\
\end{align*}
naszkicować zbiór punktów powstałych z trajektorii stanu systemu w chwili \( t = 0,75 s \) dla warunków początkowych branych ze zbioru :
\begin{align*}
X = \{ (x_1,x_2)\in \mathbb{R}^2 : |x_1+x_2|=1 \}
\end{align*}
\end{framed}

\begin{framed}
\textbf{\colorbox{green}{Zadanie 7 - Egzamin}} \\ 
Dany jest układ:
\begin{align*}
\dot{x}_1(t)=-x_1+x_2 \\
\dot{x}_2(t)=x_2
\end{align*}
z wyjściem \( y = x_1 + x_2 \). Znaleźć warunek początkowy \( x_0 \) jeśli wiadomo, że w przedziale \( t \in [0,1] \) wyjście układu ma postać \( y = e^{-t} \). 
\end{framed}


\newpage
\subsection{Rozwiązania zadań ze zbioru}
\begin{framed}
\textbf{Zadanie 1 - Zadania od Bauera do kolokwium I } \\ 
Naszkicować rozwiązania równania różniczkowego:
\begin{align*}
&\dot{x}(t)=-x(t)+1 \\
&\text{dla} \; x(0) = x_i \; i = 1,2,3 \; t \geq 0 \\
&x_1 = 0 \; x_2 = 1 \; x_3 = 2 
\end{align*}
\end{framed}
$\frac{dx}{dt}=-x+1$\\
$-\ln|-x+1|=t+c$\\
$ce^{-t}=-x+1$\\
$x=1-ce^{-t}$\\
$x(0)=1-c=x_i\Rightarrow c=1-x_i$\\
$x=1-(1-x_i)e^{-t}$\\
\\
$x=1-e^{-t} \ \vee \ x=1 \ \vee \ x=1+e^{-t}$\\
($t\geqslant 0$, więc tylko prawa strona)\\
\begin{figure}[!h]
\begin{tikzpicture}
\draw [color=blue, thick]%(-1.75,-2.37)--(-1.5,-1.74)--(-1.25,-1.24)--(-1.0,-0.85)--(-0.75,-0.55)--(-0.5,-0.32)--(-0.25,-0.14)--
(0.0,0.0)--(0.25,0.11)--(0.5,0.19)--(0.75,0.26)--(1.0,0.31)--(1.25,0.35)--(1.5,0.38)--(1.75,0.41)--(2.0,0.43)--(2.25,0.44)--(2.5,0.45)--(2.75,0.46)--(3.0,0.47);
\draw [color=blue, thick](0.0,.5)--(3,.5);
\draw [color=blue, thick]%(-1.75,3.37)--(-1.5,2.74)--(-1.25,2.24)--(-1.0,1.85)--(-0.75,1.55)--(-0.5,1.32)--(-0.25,1.14)--
(0.0,1.0)--(0.25,0.88)--(0.5,0.8)--(0.75,0.73)--(1.0,0.68)--(1.25,0.64)--(1.5,0.61)--(1.75,0.58)--(2.0,0.56)--(2.25,0.55)--(2.5,0.54)--(2.75,0.53)--(3.0,0.52);



	\node at(-.3,-.3){$x_1$};
	\node at(-.5,.5){$x_2$};
	\node at(-.3,1){$x_3$};


	\draw[thick][->](-3,0)--(3,0) node [right=3pt]{$t$};
	\draw[thick][->](0,-3)--(0,3) node [right=3pt]{$x$};



	\draw (-0.1,.5) -- (0.1,.5) node [left=3pt]{{1}};
	\draw (1,-0.1) -- (1,0.1) node [below=4pt]{{1}};
\end{tikzpicture}
\end{figure}



\newpage
\begin{framed}
\textbf{Zadanie 2 - Zadania od Bauera do kolokwium I } \\ 
Dane jest równanie różniczkowe 
\begin{align*}
&\dot{x}(t)=-x(t)+2 \\
&x(0)=0, \; t \geq 0 \\ 
\end{align*}
Po jakim czasie \( t_k \) zachodzi \( x(t_k) = 1 \)
\end{framed}
\begin{align*}
&x(t)=e^{At}+\int _0^t e^{(t-\tau)A}Bu(\tau) d\tau \\
&A = -1, \; B = 2 \;, u(\tau)=1 \\
&x_0 = 0 \\
&x(t)=e^{-t} \int _0^t e^{-\tau \cdot -1} \cdot 2 d \tau = e^{-t} \cdot 2 \cdot \int _0^t e^{\tau} d\tau = e^{-t} \cdot 2 [ e^{\tau} ] \bigg | _0^t = e^{-t}\cdot 2 \cdot ( e^t - 1 ) = 2-2e^{-t} \\
&x(t) = 1 \\
&2-2e^{-t}=1 \\
&e^{-t}=\frac{1}{2} \\
&-t = \text{ln} |\frac{1}{2}| \\
&t = -\text{ln}\frac{1}{2}
\end{align*}

\newpage
\begin{framed}
\textbf{Zadanie 3 - Zadania od Bauera do kolokwium I } \\ 
Wyznaczyć macierz \( e^{At} \) dla macierzy:
\begin{align*}
A = 
\begin{bmatrix}
-2 & 1 \\
-2 & 0
\end{bmatrix}
\end{align*}
\end{framed}
Macierz A nie jest w postaci kanonicznej, obliczam wartości własne.
\begin{align*}
&\lambda I - A = 
\begin{bmatrix}
\lambda+2 & -1 \\
2 & \lambda
\end{bmatrix} \\
&|\lambda I - A |= (\lambda +2)\lambda+2=\lambda ^{2}+2\lambda + 2 \\
& \Delta = -4 \\
& \sqrt{\Delta} = 2i \\
& \lambda _1 = \frac{-2+2i}{2} \; \lambda _2 = \frac{-2-2i}{2} \\
& \lambda _1 = -1 + i \; \lambda _2 = - 1-i \\
& a = -1 \; b = 1 \\
& J = 
\begin{bmatrix}
-1 & 1 \\
-1 & -1
\end{bmatrix} \\
&e^{Jt}=
e^{at}
\begin{bmatrix}
\cos bt & \sin bt \\
-\sin bt & \cos bt
\end{bmatrix}
=
e^{-t}
\begin{bmatrix}
\cos t & \sin t \\
-\sin t & \cos t
\end{bmatrix}
\end{align*}
Wektory własne ( wybieram np. \( \lambda _{1} = -1+i \) ) : 
\begin{align*}
&\begin{bmatrix}
1+i & -1 \\
2 & -1+i
\end{bmatrix}
\begin{bmatrix}
x \\
y
\end{bmatrix}
=
\begin{bmatrix}
0 \\
0
\end{bmatrix} \\
&\begin{cases}
x+ix-y=0 \\
2x-y+iy=0
\end{cases}
\implies
y=x+ix \\
& ( x , x+ix )
\implies
P = 
\begin{bmatrix}
x & 0 \\
x & x
\end{bmatrix}
=
\begin{bmatrix}
1 & 0 \\
1 & 1
\end{bmatrix} \\
&P^{-1}=\frac{1}{1}
\begin{bmatrix}
1 & 0 \\
-1 & 1 
\end{bmatrix} \\
&e^{At} = Pe^{Jt}P^{-1} = 
\begin{bmatrix}
1 & 0 \\
1 & 1
\end{bmatrix}
\begin{bmatrix}
e^{-t}\cos t & e^{-t}\sin t \\
-e^{-t}\sin t & e^{-t}\cos t
\end{bmatrix}
\begin{bmatrix}
1 & 0 \\
-1 & 1
\end{bmatrix}
=
\begin{bmatrix}
e^{-t}(\cos t - \sin t ) & e^{-t} \sin t \\
-2e^{-t}\sin t & e^{-t}(\sin t + \cos t )
\end{bmatrix}
\end{align*}



\newpage
\begin{framed}
\textbf{Zadanie 4 - Zadania od Bauera do kolokwium I } \\ 
Wyznaczyć macierz \( e^{At} \) dla macierzy:
\begin{align*}
A = 
\begin{bmatrix}
-\frac{1}{2 }& -\frac{1}{2} \\ \\
\frac{1}{2} & -\frac{3}{2}
\end{bmatrix}
\end{align*}
\end{framed}
Macierz A nie jest w postaci kanonicznej, obliczam wartości własne.
\begin{align*}
&\lambda I - A = 
\begin{bmatrix}
\lambda+\frac{1}{2} & \frac{1}{2} \\
-\frac{1}{2} & \lambda+\frac{3}{2}
\end{bmatrix} \\
&|\lambda I - A |= (\lambda +\frac{1}{2})(\lambda+\frac{3}{2})+\frac{1}{4}=\lambda ^{2}+2\lambda + 1 = (\lambda+1)^{2} 
\end{align*}
Wektory własne: 
\begin{align*}
&\begin{bmatrix}
-\frac{1}{2} & \frac{1}{2} \\
-\frac{1}{2} & \frac{1}{2}
\end{bmatrix}
\begin{bmatrix}
x \\
y
\end{bmatrix}
=
\begin{bmatrix}
0 \\
0
\end{bmatrix} \\
&-\frac{1}{2}x+\frac{1}{2}y=0 \\
&\implies
y=x \\
& ( x , x ) \\
\end{align*}
Wektory główne:
\begin{align*}
&\begin{bmatrix}
-\frac{1}{2} & \frac{1}{2} \\
-\frac{1}{2} & \frac{1}{2}
\end{bmatrix}
\begin{bmatrix}
x_1 \\
y_1
\end{bmatrix}
=
\begin{bmatrix}
x \\
x
\end{bmatrix} \\
&-\frac{1}{2}x_1+\frac{1}{2}y_1=x \\
&\implies
y_1=x_1+2x \\
& ( x , x_1+2x ) \\
\end{align*}

\begin{align*}
&P = 
\begin{bmatrix}
x & x \\
x & x_1+2x
\end{bmatrix}
=
\begin{bmatrix}
1 & 1 \\
1 & 3
\end{bmatrix} \\
&P^{-1}=\frac{1}{2}
\begin{bmatrix}
3 & -1 \\
-1 & 1 
\end{bmatrix} 
=
\begin{bmatrix}
\frac{3}{2} & -\frac{1}{2} \\
-\frac{1}{2} & \frac{1}{2}
\end{bmatrix} \\
&J=
\begin{bmatrix}
-1 & 1 \\
0 & -1
\end{bmatrix} \\
&e^{At} = Pe^{Jt}P^{-1} = 
\begin{bmatrix}
1 & 1 \\
1 & 3
\end{bmatrix}
\begin{bmatrix}
e^{-t} & te^{-t} \\
0 & e^{-t}
\end{bmatrix}
\begin{bmatrix}
\frac{3}{2} & -\frac{1}{2} \\
-\frac{1}{2} & \frac{1}{2}
\end{bmatrix}
=
\begin{bmatrix}
e^{-t}-\frac{-te^{-t}}{2} & -\frac{te^{-t}}{2} t \\
-\frac{te^{-t}}{2} & e^{-t}+\frac{-te^{-t}}{2}
\end{bmatrix}
\end{align*}




\newpage
\begin{framed}
\textbf{Zadanie 5 - Zadania od Bauera do kolokwium I } \\ 
Wyznaczyć rozwiązanie \( x(t) \), \( t \geq 0 \) równania:
\begin{align*}
&\ddot{x}(t)+\dot{x}(t)+3x(t)=0 \\
&x(0)=1, \; \dot{x}(0)=0
\end{align*}
\end{framed}

\newpage
\begin{framed}
\textbf{Zadanie 6 - Zadania od Bauera do kolokwium I } \\ 
Dany jest system opisany równaniem:
\begin{align*}
&\dot{x}_1(t)=-\pi x_2(t) \\
&\dot{x}_2(t)=\pi x_1(t) \\
\end{align*}
naszkicować zbiór punktów powstałych z trajektorii stanu systemu w chwili \( t = 0,75 s \) dla warunków początkowych branych ze zbioru :
\begin{align*}
X = \{ (x_1,x_2)\in \mathbb{R}^2 : |x_1+x_2|=1 \}
\end{align*}
\end{framed}

$\dot{x}=\left[\begin{array}{cc}0&-\pi\\\pi&0\end{array}\right]x$\\
$\left|\begin{array}{cc}-\lambda&-\pi\\\pi&-\lambda\end{array}\right|=\lambda^2+\pi^2=0\Rightarrow\lambda^2=-\pi^2 \ \ \ \ \ \ \ \lambda=\pm i\pi$\\
$J=\left[\begin{array}{cc}0&-\pi\\\pi&0\end{array}\right]$\\
$A = J$, $a=0, b=\pi$\\
$e^{tJ}=e^{t}\left[\begin{array}{cc}\cos(\pi t)&\sin(\pi t)\\-\sin(\pi t)&\cos(\pi t)\end{array}\right]$\\
$x(t)=e^{tJ}x(0)+\underbrace{\int_0^te^{(t-\tau)A}Bu(\tau)\ d\tau}_{=0 \text{, \ bo }u=0 \ \ B=0}$\\
$x(t)=e^{tJ}x(0)$\\\\
$x(t)=\left[\begin{array}{cc}\cos(\pi t)&\sin(\pi t)\\-\sin(\pi t)&\cos(\pi t)\end{array}\right]x(0)$\\
$t=\frac 34 s$\\
$x(\frac 34)=\left[\begin{array}{cc}-\frac{\sqrt{2}}{2}&\frac{\sqrt{2}}{2}\\-\frac{\sqrt{2}}{2}&-\frac{\sqrt{2}}{2}\end{array}\right]x(0)=-\frac{\sqrt{2}}{2}\left[\begin{array}{cc}1&-1\\1&1\end{array}\right]x(0)$\\
$\begin{cases}
x_1(\frac34)=-\frac{\sqrt{2}}{2}(x_1(0)-x_2(0))\\
x_2(\frac34)=-\frac{\sqrt{2}}{2}(x_1(0)+x_2(0))\\
|x_1+x_2|=1\Rightarrow \begin{array}{ccc}x_1+x_2=1&\vee&x_1+x_2=-1 \\ x_1=1-x_2&\vee& x_1=-1-x_2\end{array}
\end{cases}$\\
$\begin{cases}
x_1(\frac34)=-\frac{\sqrt{2}}{2}(-x_2(0)+1-x_2(0))=-\frac{\sqrt{2}}{2}-\sqrt{2}x_2(0)\\
x_2(\frac34)=-\frac{\sqrt{2}}{2}(-x_2(0)+1+x_2(0))=-\frac{\sqrt{2}}{2}
\end{cases}$\\
$\begin{cases}
x_1(\frac34)=-\frac{\sqrt{2}}{2}(-1-x_2(0)-x_2(0))=\frac{\sqrt{2}}{2}+\sqrt{2}x_2(0)\\
x_2(\frac34)=-\frac{\sqrt{2}}{2}(-1-x_2(0)+x_2(0))=\frac{\sqrt{2}}{2}
\end{cases}$\\

\begin{figure}[H]
\begin{tikzpicture}
	\draw [color=blue](-3,1)--(3,1);
	\draw [color=blue](-3,-1)--(3,-1);


	\draw[very thick][->](-3,0)--(3,0) node[right=.2] {$x_1(\frac34)$};
	\draw[very thick][->](0,-2.625)--(0,2.75) node[above=.2] {$x_2(\frac34)$};
\end{tikzpicture}
\end{figure}

\newpage
\begin{framed}
\textbf{Zadanie 7 - Egzamin} \\ 
Dany jest układ:
\begin{align*}
\dot{x}_1(t)=-x_1+x_2 \\
\dot{x}_2(t)=x_2
\end{align*}
z wyjściem \( y = x_1 + x_2 \). Znaleźć warunek początkowy \( x_0 \) jeśli wiadomo, że w przedziale \( t \in [0,1] \) wyjście układu ma postać \( y = e^{-t} \). 
\end{framed}
Macierz systemu \( A \) ma postać:
\begin{align*}
A = 
\begin{bmatrix}
-1 & 1 \\
0 & 1
\end{bmatrix}
\end{align*}
Wartości własne:
\begin{align*}
&\lambda I - A =
\begin{bmatrix}
\lambda + 1 & -1 \\
0 & \lambda -1
\end{bmatrix} \\
& \lambda _1 = -1 \; , \lambda _ 2 = 1 \\
\end{align*}
Macierz A nie jest w postaci kanonicznej Jordana, wyznaczam \( e^{At} \). Wektory własne: \\
\begin{align*}
&\lambda _1 = -1 \\
&\begin{bmatrix}
0 & -1 \\
0 & -2
\end{bmatrix}
\begin{bmatrix}
x \\
y
\end{bmatrix}
=
\begin{bmatrix}
0 \\
0
\end{bmatrix}
\implies 
y = 0 \; 
\text{np.} (1,0) \\
&\lambda _2 = 1 \\
 &\begin{bmatrix}
2 & -1 \\
0 & 0
\end{bmatrix}
\begin{bmatrix}
x \\
y
\end{bmatrix}
=
\begin{bmatrix}
0 \\
0
\end{bmatrix}
\implies 
2x - y = 0 \;
\text{np.} (1,2) \\
\end{align*}
Macierz \( e^{At} \):
\begin{align*}
e^{At} = 
\begin{bmatrix}
1 & 1 \\
0 & 2 
\end{bmatrix}
\begin{bmatrix}
e^{-t} & 0 \\
0 & e^t
\end{bmatrix}
\begin{bmatrix}
1 & -\frac{1}{2} \\
0 & \frac{1}{2}
\end{bmatrix} = 
\begin{bmatrix}
e^{-t} & -\frac{1}{2}e^{-t}+\frac{1}{2}e^t \\
0 & e^t
\end{bmatrix}
\end{align*}
Wiadomo, że dla układu jednorodnego: 
\begin{align*}
x = e^{At}x_0
\end{align*}
Zatem
\begin{align*}
&\begin{bmatrix}
x_1 \\
x_2
\end{bmatrix}
=
\begin{bmatrix}
e^{-t} & -\frac{1}{2}e^{-t}+\frac{1}{2}e^t \\
0 & e^t
\end{bmatrix}
\begin{bmatrix}
x_{10} \\
x_{20}
\end{bmatrix}
=
\begin{bmatrix}
e^{-t}x_{10}-\frac{1}{2}e^{-t}x_{20}+\frac{1}{2}e^tx_{20} \\
e^tx_{20}
\end{bmatrix}
\end{align*}
Z treści zadania:
\begin{align*}
&y = x_1 + x_2 = e^{-t}x_{10}-\frac{1}{2}e^{-t}x_{20}+\frac{1}{2}e^tx_{20} + e^tx_{20} \\
&\text{chcę aby} \\
&y = x_1 + x_2 = e^{-t} \\
&e^{-t}x_{10}-\frac{1}{2}e^{-t}x_{20}+\frac{1}{2}e^tx_{20} + e^tx_{20} = e^{-t} \\
\end{align*}
Wybierając np. \( (x_{10}, x_{20} ) = ( 1,0) \) otrzymam prawidłowy wynik. 




\newpage
\section{Sterowalność, obserwowalność, stabilizowalność}
\subsection{Zbiór zadań}
\begin{framed}
\textbf{\colorbox{green}{Zadanie 1 - Egzamin }} \\ 
Dla jakich parametrów \( b_{1}, b_{2} \) para \( (A,B) \) jest : \\
a) Sterowalna \\
b) Stabilizowalna
\begin{align*}
\dot{x}(t)=Ax(t)+Bu(t) \\
A =
\begin{bmatrix}
-1 & 0 \\
0 & 1 
\end{bmatrix}
B = 
\begin{bmatrix}
b_{1} \\
b_{2}
\end{bmatrix}
\end{align*}
\end{framed}

\begin{framed}
\textbf{\colorbox{green}{Zadanie 2 - Egzamin }} \\ 
Dla jakich \( b\) i \(c\) układ jest obserwowalny ?
\begin{align*}
\dddot{z}=bu(t) \\
y(t)=-cz(t)
\end{align*}
\end{framed}

\begin{framed}
\textbf{\colorbox{green}{Zadanie 3 - Egzamin} } \\ 
Podaj taką macierz \( C \) żeby układ 
\begin{align*}
&\dot{x}(t)=Ax(t) \\
&y=Cx(t)
\end{align*}
jeśli
\begin{align*}
A = 
\begin{bmatrix}
-1 & 0 & 0 \\
0 & -1 & 0 \\
0 & 0 & -1 
\end{bmatrix}
\end{align*}
był obserwowalny.
\end{framed}

\newpage
\subsection{Rozwiązania zadań ze zbioru}
\begin{framed}
\textbf{Zadanie 1 - Egzamin } \\ 
Dla jakich parametrów \( b_{1}, b_{2} \) para \( (A,B) \) jest : \\
a) Sterowalna \\
b) Stabilizowalna
\begin{align*}
\dot{x}(t)=Ax(t)+Bu(t) \\
A =
\begin{bmatrix}
-1 & 0 \\
0 & 1 
\end{bmatrix}
B = 
\begin{bmatrix}
b_{1} \\
b_{2}
\end{bmatrix}
\end{align*}
\end{framed}
Para \( (A,B) \) jest sterowalna \( \Longleftrightarrow \) \( rz [ B \; AB \; A^{2}B \dots A^{n-1}B ] = n\) 
\begin{center}
\( \Longleftrightarrow |B \; AB \; A^{2}B \dots A^{n-1}B| \neq 0 \) \\
\( A^{2\times 2} \implies n = 2 \) \\
\( rz[B \; AB ] = 2 \Longleftrightarrow | B \; AB | \neq 0 \)
\end{center}
\begin{align*}
&AB = 
\begin{bmatrix}
-1 & 0 \\
0 & 1
\end{bmatrix}
\begin{bmatrix}
b_{1} \\
b_{2}
\end{bmatrix}
=
\begin{bmatrix}
-b_{1} \\
b_{2}
\end{bmatrix} \\
&det
\begin{bmatrix}
b_{1} & -b_{1} \\
b_{2} & b_{2}
\end{bmatrix}
\neq 0 \\
& b_{1}b_{2}+b_{1}b_{2} \neq 0 \\
&2b_{1}b_{2} \neq 0 \\
&b_{1}b_{2} \neq 0 \\
&b_{1} \in \mathbb{R} - \{0\} \: b_{2} \in \mathbb{R} - \{0\}
\end{align*} 
\begin{figure}[H]
\begin{center}
\begin{tikzpicture}
	\draw[color=blue] (0,0) circle(.15);
	\draw[color=blue] (0,0.4) circle(.15);
	\draw[color=blue] (0,0.8) circle(.15);
	\draw[color=blue] (0,1.2) circle(.15);
	\draw[color=blue] (0,1.6) circle(.15);
	\draw[color=blue] (0,2) circle(.15);
	\draw[color=blue] (0,2.4) circle(.15);
	\draw[color=blue] (0,2.8) circle(.15);
	\draw[color=blue] (0,0) circle(.15);
	\draw[color=blue] (0,-0.4) circle(.15);
	\draw[color=blue] (0,-0.8) circle(.15);
	\draw[color=blue] (0,-1.2) circle(.15);
	\draw[color=blue] (0,-1.6) circle(.15);
	\draw[color=blue] (0,-2) circle(.15);
	\draw[color=blue] (0,-2.4) circle(.15);
	\draw[color=blue] (0,-2.8) circle(.15);
	\draw[color=blue] (0.2,0) circle(.15);
	\draw[color=blue] (0.4,0) circle(.15);
	\draw[color=blue] (0.8,0) circle(.15);
	\draw[color=blue] (1.2,0) circle(.15);
	\draw[color=blue] (1.6,0) circle(.15);
	\draw[color=blue] (2,0) circle(.15);
	\draw[color=blue] (2.4,0) circle(.15);
	\draw[color=blue] (2.8,0) circle(.15);
	\draw[color=blue] (-0.2,0) circle(.15);
	\draw[color=blue] (-0.4,0) circle(.15);
	\draw[color=blue] (-0.8,0) circle(.15);
	\draw[color=blue] (-1.2,0) circle(.15);
	\draw[color=blue] (-1.6,0) circle(.15);
	\draw[color=blue] (-2,0) circle(.15);
	\draw[color=blue] (-2.4,0) circle(.15);
	\draw[color=blue] (-2.8,0) circle(.15);
	
	\draw[color = gray] (-3,-3) -- (3,3);
	\draw[color = gray] (-3,-2) -- (2,3);
	\draw[color = gray] (-3,-1) -- (1,3);
	\draw[color = gray] (-3,0) -- (0,3);
	\draw[color = gray] (-2,-3) -- (3,2);
	\draw[color = gray] (-1,-3) -- (3,1);
	\draw[color = gray] (-3,1) -- (-1,3);
	\draw[color = gray] (1,-3) -- (3,-1);
	\draw[color = gray] (0,-3) -- (3,0);

	\draw[very thick][->](-3,0)--(3,0) node[right=.2] {$b_1$};
	\draw[very thick][->](0,-2.625)--(0,2.75) node[above=.2] {$b_2$};
\end{tikzpicture}
\end{center}
\caption{Sterowalny dla obszaru szarego bez osi}
\end{figure}
Jeśli para \( (A,B) \) jest sterowalna \( \implies \) jest stabilizowalna, a więc w obszarze szarym jest też stabilizowalna ( czy gdzieś więcej ? ). \\
Para \( (A,B) \) jest stabilizowalna gdy: \\
\begin{align*}
\exists K : Re\lambda(A+BK)<0 \Longleftrightarrow rz [ \lambda _{i}I-A,B]=n \; \forall \lambda _{i} \in \lambda (A) : Re \lambda _{i} \geq 0
\end{align*}
Dla każdej wartości własnej macierzy \( A \) o części rzeczywistej nieujemnej macierz blokowa ma rząd równy \( n =2 \). Dla macierzy \( A \) wartości własne o nieujemnej części rzeczywistej to \( \lambda_{1} = 1 \). 
\begin{align*}
[ \lambda _{1}I-A,B]=
\begin{bmatrix}
\begin{bmatrix}
1 & 0 \\
0 & 1 
\end{bmatrix}
-
\begin{bmatrix}
-1 & 0 \\
0  & 1
\end{bmatrix}
,
\begin{bmatrix}
b_{1} \\
b_{2}
\end{bmatrix}
\end{bmatrix}=
\begin{bmatrix}
2 & 0 & b_{1} \\
0 & 0 & b_{2}
\end{bmatrix}
\end{align*} 
Minory \( 2\times 2\) :
\begin{align*}
det\begin{bmatrix}
2 & 0 \\
0 & 0
\end{bmatrix} = 0
\;
det\begin{bmatrix}
2 & b_{1} \\
0 & b_{2}
\end{bmatrix}
= 2b_{2}
\;
det \begin{bmatrix}
0 & b_{1} \\
0 & b_{2}
\end{bmatrix}
= 0
\end{align*}
stąd aby układ był stabilizowalny:
\begin{align*}
2b_{2} \neq 0 \\
b_{2} \neq 0
\end{align*}
\begin{figure}[H]
\begin{center}
\begin{tikzpicture}
	
	\draw[color=blue] (0.2,0) circle(.15);
	\draw[color=blue] (0.4,0) circle(.15);
	\draw[color=blue] (0.8,0) circle(.15);
	\draw[color=blue] (1.2,0) circle(.15);
	\draw[color=blue] (1.6,0) circle(.15);
	\draw[color=blue] (2,0) circle(.15);
	\draw[color=blue] (2.4,0) circle(.15);
	\draw[color=blue] (2.8,0) circle(.15);
	\draw[color=blue] (-0.2,0) circle(.15);
	\draw[color=blue] (-0.4,0) circle(.15);
	\draw[color=blue] (-0.8,0) circle(.15);
	\draw[color=blue] (-1.2,0) circle(.15);
	\draw[color=blue] (-1.6,0) circle(.15);
	\draw[color=blue] (-2,0) circle(.15);
	\draw[color=blue] (-2.4,0) circle(.15);
	\draw[color=blue] (-2.8,0) circle(.15);
	
	\draw[color = gray] (-3,-3) -- (3,3);
	\draw[color = gray] (-3,-2) -- (2,3);
	\draw[color = gray] (-3,-1) -- (1,3);
	\draw[color = gray] (-3,0) -- (0,3);
	\draw[color = gray] (-2,-3) -- (3,2);
	\draw[color = gray] (-1,-3) -- (3,1);
	\draw[color = gray] (-3,1) -- (-1,3);
	\draw[color = gray] (1,-3) -- (3,-1);
	\draw[color = gray] (0,-3) -- (3,0);

	\draw[very thick][->](-3,0)--(3,0) node[right=.2] {$b_1$};
	\draw[very thick][->](0,-2.625)--(0,2.75) node[above=.2] {$b_2$};
\end{tikzpicture}
\end{center}
\caption{Stabilizowalny dla obszaru szarego bez osi}
\end{figure}

\newpage
\begin{framed}
\textbf{Zadanie 2 - Egzamin } \\ 
Dla jakich \( b\) i \(c\) układ jest obserwowalny ?
\begin{align*}
\dddot{z}=bu(t) \\
y(t)=-cz(t)
\end{align*}
\end{framed}
Przejście na równania stanu:
\begin{align*}
\begin{cases}
z_{1}=z \\
z_{2}=\dot{z} \\
z_{3}=\ddot{z}
\end{cases}
\implies
\begin{cases}
\dot{z}_{1}=z_{2} \\
\dot{z}_{2}=z_{3} \\
\dot{z}_{3}=\dddot{z}=bu(t)
\end{cases}
\end{align*}
\begin{align*}
&A =
\begin{bmatrix}
0 & 1 & 0 \\
0 & 0 & 1 \\
0 & 0 & 0 
\end{bmatrix}
B = 
\begin{bmatrix}
0 \\
0 \\
b
\end{bmatrix} \\
&y=-cz(t) = -cz_{1}\\
&y = 
\begin{bmatrix}
-c & 0 & 0
\end{bmatrix}
\begin{bmatrix}
z_{1} \\
z_{2} \\
z_{3}
\end{bmatrix}
\end{align*}
Układ jest obserwowalny \( \Longleftrightarrow \) 
\begin{align*}
rz \begin{bmatrix}
C \\
CA \\
CA^{2} \\
\dots \\
CA^{n-1}
\end{bmatrix}
= n
\end{align*}
\(A^{2\times 2} \implies n = 2\)
\begin{align*}
\begin{bmatrix}
C \\
CA \\
CA^{2} 
\end{bmatrix}
=
\begin{bmatrix}
-c & 0 & 0 \\
0 & -c & 0 \\
0 & 0 & -c
\end{bmatrix}
\end{align*}
\begin{align*}
rz
\begin{bmatrix}
-c & 0 & 0 \\
0 & -c & 0 \\
0 & 0 & -c
\end{bmatrix}
= 3
\Longleftrightarrow
-c^{3} \neq 0 \Longleftrightarrow (c \neq 0 \land b \in \mathbb{R})
\end{align*}

\newpage
\begin{framed}
\textbf{Zadanie 3 - Egzamin } \\ 
Podaj taką macierz \( C \) żeby układ 
\begin{align*}
&\dot{x}(t)=Ax(t) \\
&y=Cx(t)
\end{align*}
jeśli
\begin{align*}
A = 
\begin{bmatrix}
-1 & 0 & 0 \\
0 & -1 & 0 \\
0 & 0 & -1 
\end{bmatrix}
\end{align*}
był obserwowalny.
\end{framed}
Układ jest obserwowalny \( \Longleftrightarrow \) 
\begin{align*}
rz \begin{bmatrix}
C \\
CA \\
CA^{2} \\
\dots \\
CA^{n-1}
\end{bmatrix}
= n
\end{align*}
\(A^{3\times 3} \implies n = 3\)
\begin{align*}
C =
\begin{bmatrix}
c_{1} & c_{2} & c_{3}
\end{bmatrix}
\end{align*}
\begin{align*}
\begin{bmatrix}
C \\
CA \\
CA^{2} 
\end{bmatrix}
=
\begin{bmatrix}
c_{1} & c_{2} & c_{3} \\
-c_{1} & -c_{2} & -c_{3} \\
c_{1} & c_{2} & c_{3}
\end{bmatrix}
\end{align*}
\begin{align*}
rz
\begin{bmatrix}
c_{1} & c_{2} & c_{3} \\
-c_{1} & -c_{2} & -c_{3} \\
c_{1} & c_{2} & c_{3}
\end{bmatrix}
= 3
\Longleftrightarrow
det \neq 0
\end{align*}
Wyznacznik tej macierzy jest zawsze równy \( 0 \) więc taka macierz nie istnieje, układ nie jest obserwowalny.

\newpage
\section{LQ / Wskaźniki jakości}
\subsection{Zbiór zadań}
\begin{framed}
\textbf{\colorbox{green}{Zadanie 1 - Egzamin }} \\ 
Wyznaczyć regulator optymalny w sensie wskaźnika jakości 
\begin{align*}
J = x^2(10) + \int_0^{10}( x(t)^{2}+\frac{1}{4}u(t)^{2}) dt
\end{align*}
jeśli
\begin{align*}
&\dot{x}(t)=\frac{1}{2}u(t) \\
&\text{gdzie} \\
&x(0)= 1 
\end{align*}
Wskazówka: Regulator optymalny ma postać \( u(t) = -Kx(t), \; K \in \mathbb{R} \). 
\end{framed}

\begin{framed}
\textbf{\colorbox{green}{Zadanie 2 - Egzamin }} \\ 
Dany jest system dynamiczny \( \dot{x}(t)=-x(t)+u(t) \). Znaleźć sterowanie postaci: \( u(t)=Kx(t) \) minimalizujące wskaźnik:
\begin{align*}
J(u)= \int _0^{\infty} ||x(t)||^2+||u(t)||^2 dt
\end{align*}
\end{framed}

\begin{framed}
\textbf{\colorbox{green}{Zadanie 3 - Egzamin = Zadanie 2 - Egzamin}} \\ 
Układ opisany równaniem:
\begin{align*}
\dot{y}(t)=-y(t)+u(t)
\end{align*}
z warunkiem początkowym \( y(0)=1 \) objęto ujemnym sprzężeniem zwrotnym o wzmocnieniu \(K\). Znaleźć takie \(K\), które minimalizuje wskaźnik jakości:
\begin{align*}
J = \int _0^{\infty} y(t)^{2} + u(t)^{2} dt
\end{align*} 
\end{framed}

\newpage
\subsection{Rozwiązania zadań ze zbioru}
\begin{framed}
\textbf{Zadanie 1 - Egzamin } \\ 
Wyznaczyć regulator optymalny w sensie wskaźnika jakości 
\begin{align*}
J = x^2(10) + \int_0^{10}( x(t)^{2}+\frac{1}{4}u(t)^{2}) dt
\end{align*}
jeśli
\begin{align*}
&\dot{x}(t)=\frac{1}{2}u(t) \\
&\text{gdzie} \\
&x(0)= 1 
\end{align*}
Wskazówka: Regulator optymalny ma postać \( u(t) = -Kx(t), \; K \in \mathbb{R} \). 
\end{framed}
Jak widać na podstawie wskaźnika jakości mamy tu do czynienia z problemem liniowo-kwadratowym ze skończonym horyzontem, ponieważ jest to problem postaci:
\begin{align*}
&\dot{x}=Ax+Bu \\
&J=x^T(t_1)F(t_1)x(t_1)+\int_{t_0}^{t_1}(x^T Qx+u^T Ru+2x^T Nu)dt \\
&u = -Kx \\
\end{align*}
W problemie tym wzmocnienie regulatora optymalnego wyraża się wzorem:
\begin{align*}
&K=R^{-1}(B^TP(t)+N^T) \\
&\text{gdzie P znajduje się przez rozwiązanie równania Ricattiego:} \\
&A^TP(t)+P(t)A-(P(t)B+N)R^{-1}(B^TP(t)+N^T)+Q=-\dot{P}(t) \\
\end{align*}
W zadaniu:
\begin{align*}
t_1=10, \;
F(t_1)=F(10)=1, \;
Q=1, \;
R=\frac{1}{4}, \;
N=0
\end{align*}
a także jeśli \( \dot{x}(t)=\frac{1}{2}u(t)  \) to \( A = 0 \) oraz \( B = \frac{1}{2} \). \\ 
Podstawiając więc te wartości do równania na \( \dot{P}(t) \) : 
\begin{align*}
&\underbrace{A^TP(t)}_{=0}+\underbrace{P(t)A}_{=0}-(\underbrace{P(t)B+N}_{\frac{1}{2}P(t)})\underbrace{R^{-1}}_{=4}(\underbrace{B^TP(t)+N^T}_{\frac{1}{2}P(t)})+Q=-\dot{P}(t) \\
&-P(t)^2+1=-\dot{P}(t)
\end{align*}
Równanie ma rozwiązanie stałe \( P(t) = 1 \). 
Podstawiając do równania na \( K \) otrzymujemy: \( K = 4 \cdot \frac{1}{2} \cdot 1 + 0 = 2 \). 

\newpage
\begin{framed}
\textbf{Zadanie 2 - Egzamin } \\ 
Dany jest system dynamiczny \( \dot{x}(t)=-x(t)+u(t) \). Znaleźć sterowanie postaci: \( u(t)=Kx(t) \) minimalizujące wskaźnik:
\begin{align*}
J(u)= \int _0^{\infty} ||x(t)||^2+||u(t)||^2 dt
\end{align*}
\end{framed}
Jak widać na podstawie wskaźnika jakości mamy tu do czynienia z problemem liniowo-kwadratowym z nieskończonym horyzontem, ponieważ jest to problem postaci:
\begin{align*}
&\dot{x}=Ax+Bu \\
&J=\int_{0}^{\infty}(x^T Qx+u^T Ru+2x^T Nu)dt \\
&u = -Kx \\
\end{align*}
Wskaźnik jest postaci:
\begin{align*}
&J=\int_{0}^{\infty}(x^2+u^2)dt 
\end{align*}
W problemie tym wzmocnienie regulatora optymalnego wyraża się wzorem:
\begin{align*}
&K=R^{-1}(B^TP+N^T) \\
&\text{gdzie P znajduje się przez rozwiązanie równania Ricattiego:} \\
&A^TP+PA-(PB+N)R^{-1}(B^TP+N^T)+Q=0 \\
\end{align*}
W zadaniu:
\begin{align*}
Q=1, \;
R=1, \;
N=0
\end{align*}
a także \( A = -1 \) oraz \( B = 1 \). \\ 
Podstawiając więc te wartości do równania na \( P \) : 
\begin{align*}
&\underbrace{A^TP}_{-P}+\underbrace{PA}_{-P}-(\underbrace{PB+N}_{P})\underbrace{R^{-1}}_{=1}(\underbrace{B^TP+N^T)}_{P}+Q=0 \\
&-2P - P^2  + 1=0
\end{align*}
Równanie ma rozwiązania stałe \( P = 2\sqrt{2} \; \lor \; -2\sqrt{2} \). \\
Wtedy \( K = 1(1\cdot 2\sqrt{2}+0)=2\sqrt{2} \)






\newpage
\begin{framed}
\textbf{Zadanie 3 - Egzamin} \\ 
Układ opisany równaniem:
\begin{align*}
\dot{y}(t)=-y(t)+u(t)
\end{align*}
z warunkiem początkowym \( y(0)=1 \) objęto ujemnym sprzężeniem zwrotnym o wzmocnieniu \(K\). Znaleźć takie \(K\), które minimalizuje wskaźnik jakości:
\begin{align*}
J = \int _0^{\infty} y(t)^{2} + u(t)^{2} dt
\end{align*} 
\end{framed}
Patrz \( \rightarrow \) Rozwiązanie zadania 2 ( LQ / Wskaźniki jakości )



\newpage
\section{Definicje \& Twierdzenia}
\subsection{Zbiór zadań}
\begin{framed}
\textbf{Zadanie 1 - Pytanie własne } \\ 
Czym jest trajektoria równania, trajektoria fazowa, a czym portret fazowy ?
\end{framed}
\begin{framed}
\textbf{Zadanie 2 - Pytanie własne } \\ 
O czym mówi twierdzenie Grabman-Hartman ?
\end{framed}
\begin{framed}
\textbf{Zadanie 3 - Pytanie własne } \\ 
Czym jest sterowalność układu ?
\end{framed}
\begin{framed}
\textbf{Zadanie 4 - Pytanie własne } \\ 
Czym jest stabilizowalność układu ?
\end{framed}
\begin{framed}
\textbf{Zadanie 5 - Pytanie własne } \\ 
Czym jest obserwowalność układu ?
\end{framed}
\begin{framed}
\textbf{Zadanie 6 - Egzamin } \\ 
Proszę sformułować twierdzenie Nyquista ( dla układu z jednym wejściem i jednym wyjściem ) ?
\end{framed}
\begin{framed}
\textbf{Zadanie 7 - Pytanie własne } \\ 
Proszę sformułować twierdzenie Michajłowa ?
\end{framed}
\begin{framed}
\textbf{Zadanie 8 - Pytanie własne } \\ 
Czym jest wykrywalność ?
\end{framed}
\begin{framed}
\textbf{Zadanie 9 - Pytanie własne } \\ 
Podać trzy definicje stabilności.
\end{framed}
\begin{itemize}
\item Stabilność w sensie Lapunowa: rozwiązanie \( x^* \) nazywamy stabilnym w sensie Lapunowa wtedy i tylko wtedy, gdy dla każdego \( t_0 \in [0, \infty) \) i dla każdego \( \epsilon \) istnieje \( \delta >0 \) taka, że każda trajektoria stanu \( x \) spełniająca warunek \( ||x(t_0)-x^*(t_0)||<\delta \) jest określona na \( [t_0, \infty ) \) oraz \( || x(t)-x^*(t) || < \epsilon \) w przedziale \( [t_0, \infty ) \).
\item Asymptotyczna stabilność: rozwiązanie \( x^* \) nazywamy stabilnym asymptotycznie wtedy i tylko wtedy, gdy jest stabilne w sensie Lapunowa oraz \( \exists \delta > 0 \) taka, że dla każdego rozwiązania \( x^*\) określonego na przedziale  \( [t_0, \infty ) \) zachodzi \( \lim_{t \rightarrow \infty}||x(t)-x^*(t)||=0 \).  
\end{itemize}
\newpage
\subsection{Rozwiązania zadań ze zbioru}
\begin{framed}
\textbf{Zadanie 1 - Pytanie własne } \\ 
Czym jest trajektoria równania, trajektoria fazowa, a czym portret fazowy ?
\end{framed}
Trajektoria równania jest krzywą w przestrzeni \( \mathbb{R}^{N} \times
 t \). Trajektoria fazowa to rzut trajektorii na przestrzeń stanu \( \mathbb{R}^{N} \) przy danych warunkach początkowych. Portret fazowy to rodzina trajektorii fazowych dla różnych warunków początkowych.

\begin{framed}
\textbf{Zadanie 2 - Pytanie własne } \\ 
O czym mówi twierdzenie Grabman-Hartman ?
\end{framed}
Twierdzenie mówi, że jeżeli macierz A systemu zlinearyzowanego nie ma wartości własnych o zerowych częściach rzeczywistych, to trajektorie systemu nieliniowego są lokalnie podobne do trajketorii systemu liniowego.
\begin{framed}
\textbf{Zadanie 3 - Pytanie własne } \\ 
Czym jest sterowalność układu ?
\end{framed}
Sterowalność to własność układu polegająca na tym, że istnieje sterowanie przeprowadzające układ w pewnym skończonym przedziale czasu do zadanego stanu. 
\begin{framed}
\textbf{Zadanie 4 - Pytanie własne } \\ 
Czym jest stabilizowalność układu ?
\end{framed}
Stabilizowalność to własność układu mówiąca o tym, że istnieje takie sterowanie \( u(t)=Kx(t) \) dla którego układ opisany macierzą stanu A oraz macierzą wejść B jest stabilny, czyli macierz \( A + BK \) jest stabilna. 
\begin{framed}
\textbf{Zadanie 5 - Pytanie własne } \\ 
Czym jest obserwowalność układu ?
\end{framed}
Obserwowalność to własność układu, która pozwala na określenie stanu układu na podstawie obserwacji wejścia oraz wyjścia.
\begin{framed}
\textbf{Zadanie 6 - Egzamin } \\ 
Proszę sformułować twierdzenie Nyquista ( dla układu z jednym wejściem i jednym wyjściem ) ?
\end{framed}
Kryterium Nyquista pozwala na określenie stabilności układu zamkniętego na podstawie charakterystyki amplitudowo-fazowej układu otwartego. \\
Założenia:
\begin{itemize}
\item Rozłączamy sprzężenie zwrotne 
\item \( G_{0} \) to transmitancja otrzymanego układu otwartego
\item Układ otwarty ma \( m \) biegunów w prawej półpłaszczyźnie i \( n-m \) w lewej ( nie ma pierwiastków na osi urojonej )
\end{itemize}
Przy spełnieniu powyższych założeń układ zamknięty jest stabilny  asymptotycznie wtedy i tylko wtedy gdy:
\begin{align*}
&1+G_{0}(j\omega) \neq 0 \; \forall \omega \in \mathbb{R} \\
&\text{oraz} \\
&\Delta \text{arg}(1+G_{0}(j\omega))=m\pi \\
&\text{dla} \; \omega \in [ 0, \infty ] 
\end{align*}
\begin{framed}
\textbf{Zadanie 7 - Pytanie własne } \\ 
Proszę sformułować twierdzenie Michajłowa ?
\end{framed}
Służy do określania stabilności dowolnego układu którego transmitancja zastępcza jest znana. 
Założenia:
\begin{itemize}
\item \( M(s) \) to mianownik transmitancji zastępczej układu.
\item \( n\) to stopień wielomianu \( M(s) \)
\end{itemize}
Układ jest asymptotycznie stabilny wtedy i tylko wtedy gdy przyrost argumentu \( \Delta \text{arg} \; M(j\omega) = n \frac{\pi}{2} \), gdzie \( \omega \in [ 0 , \infty ] \).
\begin{framed}
\textbf{Zadanie 8 - Pytanie własne } \\ 
Czym jest wykrywalność ?
\end{framed}
Wykrywalność to własność układu mówiąca o tym, że dla macierzy \(A,C\) istnieje taka macierz \(L\), że macierz \(A+LC\) jest stabilna asymptotycznie. 
\begin{framed}
\textbf{Zadanie 9 - Pytanie własne } \\ 
Podać trzy definicje stabilności.
\end{framed}
\end{document}
