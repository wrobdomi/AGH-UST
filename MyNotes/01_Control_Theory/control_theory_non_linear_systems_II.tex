\documentclass[a4paper,11pt]{article}
\usepackage{amssymb}
\usepackage{amsmath}
\usepackage[english, polish]{babel}
\usepackage[utf8]{inputenc}   % lub utf8
\usepackage[T1]{fontenc}
\usepackage{graphicx}
\usepackage{anysize}
\usepackage{enumerate}
\usepackage{times}
\usepackage{titlesec}
\usepackage{float}
\usepackage[justification=centering]{caption}
\titlelabel{\thetitle.\quad}
\usepackage{titlesec}
\usepackage{titleps,kantlipsum}

\newpagestyle{mypage}{%
  \headrule
  \sethead{\MakeUppercase{\thesection\quad \sectiontitle}}{}{\thesubsection\quad \subsectiontitle}
  \setfoot{}{}{\thesubsubsection\quad \subsubsectiontitle}
}
\settitlemarks{section,subsection,subsubsection}
\pagestyle{mypage}
%\marginsize{left}{right}{top}{bottom}
\marginsize{3cm}{3cm}{3cm}{3cm}
\sloppy
 
\begin{document}
\begin{table}
\begin{center}
\begin{tabular}{|l|l|l|}
\hline
\multicolumn{3}{|c|}{\textbf{Druga metoda Lapunowa}} \\ \hline Dominik Wróbel & \textbf{17 IV 2018} & \textbf{Wt 09:30} \\ \hline

\end{tabular}
\end{center}
\end{table}
\tableofcontents

\section{Cel ćwiczenia}
Celem ćwiczenia jest zapoznanie się z działaniem i stosowaniem drugiej metody Lapunowa do badania nieliniowych układów dynamicznych. Rozważane przykłady pozwolą na obserwacje działania drugiej metody Lapunowa oraz wyznaczenie jej wad i zalet, a także porównanie do metody pierwszej.
\section{Przebieg ćwiczenia}
\subsection{Zadanie 4.2}
W zadaniu rozważany jest nieliniowy układ dynamiczny opisany układem równań: 
\begin{equation*}
\begin{cases}
\dot{x_{1}}(t)=x_{2}(t)-x_{1}(t)+x_{1}^{3}(t) \\
\dot{x_{2}}(t)=-x_{1}(t)
\end{cases}
\end{equation*}
\newpage
W zadaniu zostaną wykonane następujące punkty :
\begin{itemize}
\item Znalezienie punktów równowagi systemu,
\item Zbadanie stabilności znalezionych punktów równowagi przy pomocy II metody Lapunowa, 
\item Zbadanie stabilności znalezionych punktów równowagi przy pomocy metody linearyzacji i porównanie z II metodą Lapunowa,
\item Wyznaczenie obszaru atrakcji punktów równowagi przy pomocy twierdzenia LaSalle'a,
\item Wyznaczenie obszaru atrakcji punktów równowagi przy pomocy eksperymentów numerycznych
\end{itemize}
\subsubsection{Wyznaczenie punktów równowagi}
Punkty równowagi: \\
\begin{equation*}
\begin{cases}
-x_{1}=0 \\
x_{2}-x_{1}+x_{1}^{3}=0
\end{cases} \implies
\begin{cases}
x_{1}=0 \\
x_{2}=0
\end{cases}
\end{equation*}
Układ ma jeden, zerowy punkt równowagi.

\subsubsection{Badanie stabilności za pomocą II metody Lapunowa}
Zgodnie z zaleceniem w poleceniu do badania stabilności zostanie wykorzystany funkcjonał energetyczny:
\begin{equation*}
V(x)=\frac{1}{2}(x_{1}^{2}+x_{2}^{2})
\end{equation*}
Funkcjonał ten spełnia założenia :
\begin{itemize}
\item \( V : R^{2} \rightarrow R \)
\item Funkcjonał jest ciągły wraz z pierwszymi pochodnymi cząstkowymi względem \(x_{1}\) oraz \(x_{2}\) w pewnym otoczeniu zerowego punktu równowagi systemu, 
\item V(0) = 0 oraz V(0) > 0 w pewnym otoczeniu \(\Omega_{1}\) zera, z wyłączeniem zera 
\end{itemize}
Spełnienie tych warunków jest oczywiste dla funkcjonału energetycznego. Aby móc wnioskować o tym czy funkcjonał ten jest funkcjonałem Lapunowa dla badanego systemu, a później o stabilności tego systemu, poza powyższymi warunkami potrzebne jest jeszcze zbadanie znaku wyrażenia \( \dot{V}(x) \). \\ Funkcjonał będzie funkcjonałem Lapunowa badanego systemu jeśli spełniony będzie warunek :
\begin{itemize}
\item \( \dot{V}(x)\leq 0 \) w pewnym otoczeniu \( \Omega_{2}\) zera
\end{itemize} 
\begin{equation*}
\dot{V}(x)=
\begin{bmatrix}
\frac{\partial V}{\partial x_{1}} \quad
\frac{\partial V}{\partial x_{2}} 
\end{bmatrix}
\begin{bmatrix}
f_{1}(x) \\
f_{2}(x)
\end{bmatrix}=
\begin{bmatrix}
x_{1} \quad
x_{2}
\end{bmatrix}
\begin{bmatrix}
x_{2}-x_{1}+x_{1}^{3} \\ 
x_{1}
\end{bmatrix}=
x_{1}^{4}-x_{1}^{2}
=x_{1}^{2}(x_{1}^{2}-1)=
x_{1}^{2}(x_{1}-1)(x_{1}+1)
\end{equation*}
Z powyższego równania od razu widać, że powyższy warunek jest spełniony dla \( x \in [-1, 1] \) oraz dla dowolnego \( x_{2} \). Wiadomo więc, że funkcjonał ten jest funkcjonałem Lapunowa badanego systemu. \\
Korzystając z twierdzenia Lapunowa można teraz określić stabilność tego punktu równowagi. Przed przystąpieniem do wyznaczania stabilności zauważyć należy, że równanie \( \dot{V}(x) = 0 \) jest spełnione dla wszystkich punktów postaci \( (0,x_{2}) \). Nie można zatem na podstawie twierdzenia Lapunowa wnioskować o stabilności asymptotycznej, a jedynie o stabilności, ponieważ nie można znaleźć otoczenia punktu \( (0,0) \) dla którego spełniona byłaby nierówność ostra \( \dot{V}(x) < 0 \) .
\begin{center}
\textit{Punkt (0,0) jest stabilnym punktem równowagi systemu, ponieważ w pewnym otoczeniu zera istnieje funkcjonał Lapunowa V(x) i zachodzi tam nierówność słaba \( \dot{V}(x) \leq 0 \), dla każdego x należącego do tego otoczenia i \( x \neq 0 \) } .
\end{center}
\subsubsection{Badanie stabilności metodą linearyzacji - porównanie metod}
Zastosowanie metody linearyzacji pozwoli na porównanie działania obu metod.
Macierz Jacobiego : 
\begin{equation*}
J=\begin{bmatrix}
-1+3x_{1}^{2} & 1 \\
-1 & 0 
\end{bmatrix}
\end{equation*}
Dla x = ( 0,0 ) : \\
\begin{equation*}
A=\begin{bmatrix}
-1 & 1 \\
-1 & 0 
\end{bmatrix}
\end{equation*}
Równanie charakterystyczne: 
\begin{equation*}
\det{(\lambda I-A)}=\lambda^{2}+\lambda+1
\end{equation*}
Wartości własne:
\begin{equation*}
\lambda_{1}=\frac{-1}{2}-\frac{\sqrt{3}i}{2}
\end{equation*}
\begin{equation*}
\lambda_{2}=\frac{-1}{2}+\frac{\sqrt{3}i}{2}
\end{equation*}
Części rzeczywiste wszystkich wartości własnych macierzy stanu A liniowego przybliżenia nieliniowego systemu w punkcie równowagi są ujemne więc punkt równowagi nieliniowego systemu jest asymptotycznie stabilny. \\ 
I metoda Lapunowa dała wiec w tym przypadku inny, dokładniejszy rezultat niż metoda II. 
\subsubsection{Wyznaczenie obszaru atrakcji przy pomocy twierdzenia LaSalle'a}
Z punktu 2.1.2 wiadomo, że w przypadku rozważanego funkcjonału spełnione są założenia: 
\begin{itemize}
\item \( V : R^{2} \rightarrow R \)
\item Funkcjonał jest ciągły wraz z pierwszymi pochodnymi cząstkowymi względem \(x_{1}\) oraz \(x_{2}\) w pewnym otoczeniu zerowego punktu równowagi systemu, 
\item V(0) = 0 oraz V(0) > 0 w pewnym otoczeniu \(\Omega_{1}\) zera, z wyłączeniem zera, 
\item \( \dot{V}(x) \leq 0 \) w pewnym otoczeniu \( \Omega_{2}\) zera
\end{itemize}
Nie można więc zastosować uproszczonej wersji twierdzenia LaSalle'a. Zgodnie z zaleceniem w poleceniu przyjęto, że stała \( l = \frac{1}{2} \). Wyznaczenie zbioru \( Z_{l} \) będącego podzbiorem faktycznego obszaru przyciągania odbywa się przy pomocy rozwiązania nierówności \( V(x) < l \). 
\begin{equation*}
Z_{l}: \quad \frac{1}{2}(x_{1}^{2}+x_{2}^{2})<\frac{1}{2}
\end{equation*}
\begin{equation*}
Z_{l}: \quad  x_{1}^{2}+x_{2}^{2}<1
\end{equation*}
Zbiór \(Z_{l}\) opisany powyższą nierównością jest kołem bez punktów brzegowych o promieniu 1.
Następnie poszukiwany jest zbiór E, taki, że : 
\begin{equation*}
E = \{ x \in Z_{l} : \dot{V}(x)=\frac{\partial V(x)}{\partial x} f(x) =0 \}
\end{equation*}
\begin{equation*}
x_{1}^{2}(x_{1}-1)(x_{1}+1)=0
\end{equation*}
Zauważyć należy, że choć rozwiązaniem powyższego równania są punkty dla których \( x_{1} = 1 \) lub \( x_{1} = -1 \), to nie należą one do zbioru \( Z_{l} \). Dlatego zbiór E to zbiór 
\begin{equation}
E = \{ (0,x_{2}), \quad x_{2} \in ( -1 , 1 ) \}
\end{equation} 
Zbiór ten to zbiór wszystkich punktów na osi \( x_{2} \), których wartość jest większa od -1 i mniejsza od 1. 
Kolejną czynnością jest poszukiwanie największego zbioru inwariantnego \( M \subset E \).
Jeżeli rozważymy niezerowy punkt należący do zbioru \( E \), to na podstawie pierwszego równania systemu na \( \dot{x_{1}}(t) \) otrzymamy, że  \( \dot{x}_{1}(t) \neq 0 \), a więc  \( \dot{x_{1}}(t) \) zmienia się z upływem czasu, co oznacza, trajektoria startująca z tego punktu nie pozostaje w zbiorze \( E \). 

Jedynym punktem należącym do zbioru M jest więc punkt 0. Na podstawie twierdzenia LaSalle'a można więc stwierdzić, że każde rozwiązanie równania systemu startujące z punktu należącego do \( Z_{l} \) dąży do punktu 0 dla \( t \rightarrow \infty \). Przy pomocy twierdzenia LaSalle'a udało się więc pokazać asymptotyczną stabilność zerowego punktu równowagi, czyli własność silniejszą niż udało uzyskać się przy pomocy II metody Lapunowa. 
\subsubsection{Wyznaczenie obszaru atrakcji przy pomocy eksperymentu}
W celu eksperymentalnego wyznaczenia obszaru atrakcji zerowego punktu równowagi zbudowano model w programie Matlab, który przedstawia Rysunek \ref{fig:simulink_1}.
\begin{figure}[H]
\centerline{\includegraphics[scale=0.8]{simulink_1.jpg}}
\centering
\caption{Model rozważanego systemu}
\label{fig:simulink_1}
\end{figure}
Uzyskany portret fazowy przedstawia Rysunek \ref{fig:lap2_1}. 
\begin{figure}[H]
\centerline{\includegraphics[scale=1.2]{lap2_1.jpg}}
\caption{Portret fazowy badanego systemu, kolorem zielonym oznaczono granicę wyznaczonej analitycznie estymaty obszaru przyciągania, a kolorem brązowym wyznaczony zbiór E.}
\label{fig:lap2_1}
\end{figure}
\newpage
Na Rysunku \ref{fig:lap2_2} przedstawiono estymatę rzeczywistego obszaru przyciągania wyznaczonego na podstawie eksperymentu. Na rysunku widać, że obszar wyznaczony analitycznie jest podzbiorem rzeczywistego obszaru przyciągania.
\begin{figure}[H]
\centerline{\includegraphics[scale=1.2]{lap2_2.jpg}}
\caption{Kolorem pomarańczowym zaznaczono estymatę rzeczywsitego obszaru przyciągania wyznaczonego na podstawie przeprowadzonego eksperymentu}
\label{fig:lap2_2}
\end{figure}

\subsection{Zadanie 4.1}
W zadaniu rozważany jest nieliniowy układ dynamiczny opisany układem równań: 
\begin{equation*}
\begin{cases}
\dot{x_{1}}(t)=-x_{1}(t)+2x_{1}^{2}(t)x_{2}(t) \\
\dot{x_{2}}(t)=-x_{2}(t)
\end{cases}
\end{equation*}
\newpage
W zadaniu zostaną wykonane następujące punkty :
\begin{itemize}
\item Znalezienie punktów równowagi systemu,
\item Zbadanie stabilności znalezionych punktów równowagi przy pomocy II metody Lapunowa dla dwóch różnych funkcjonałów podanych w zadaniu, 
\item Analityczne wyznaczenie obszaru atrakcji punktów równowagi przy pomocy twierdzenia LaSalle'a dla obu funkcjonałów,
\item Wyznaczenie obszaru atrakcji punktu równowagi przy pomocy eksperymentów numerycznych
\end{itemize}
\subsubsection{Wyznaczenie punktów równowagi}
Wyznaczenie punktów równowagi: \\
\begin{equation*}
\begin{cases}
-x_{2}=0 \\
-x_{1}+2x_{1}^{2}x_{2}=0
\end{cases} \implies
\begin{cases}
x_{1}=0 \\
x_{2}=0
\end{cases}
\end{equation*}
Układ ma jeden, zerowy punkt równowagi.
\subsubsection{Badanie stabilności za pomocą II metody Lapunowa - funkcjonał I}
Zgodnie z zaleceniem w poleceniu do badania stabilności zostanie wykorzystany funkcjonał postaci:
\begin{equation*}
V(x)=\frac{1}{2}x_{1}^{2}+x_{2}^{2}
\end{equation*}
Funkcjonał ten spełnia założenia :
\begin{itemize}
\item \( V : R^{2} \rightarrow R \)
\item Funkcjonał jest ciągły wraz z pierwszymi pochodnymi cząstkowymi względem \(x_{1}\) oraz \(x_{2}\) w pewnym otoczeniu zerowego punktu równowagi systemu, 
\item V(0) = 0 oraz V(x) > 0 w pewnym otoczeniu \(\Omega_{1}\) zera, z wyłączeniem zera 
\end{itemize}
Spełnienie tych warunków jest oczywiste dla przyjętego funkcjonału. Aby móc wnioskować o tym czy funkcjonał ten jest funkcjonałem Lapunowa dla badanego systemu, a później o stabilności tego systemu, poza powyższymi warunkami potrzebne jest jeszcze zbadanie znaku wyrażenia \( \dot{V}(x) \). \\ Funkcjonał będzie funkcjonałem Lapunowa badanego systemu jeśli spełniony będzie warunek :
\begin{itemize}
\item \( \dot{V}(x)\leq 0 \) w pewnym otoczeniu \( \Omega_{2}\) zera
\end{itemize} 
\begin{equation*}
\dot{V}(x)=
\begin{bmatrix}
\frac{\partial V}{\partial x_{1}} \quad
\frac{\partial V}{\partial x_{2}} 
\end{bmatrix}
\begin{bmatrix}
f_{1}(x) \\
f_{2}(x)
\end{bmatrix}=
\begin{bmatrix}
x_{1} \quad
2x_{2}
\end{bmatrix}
\begin{bmatrix}
-x_{1}+2x_{1}^{2}x_{2} \\ 
-x_{2}
\end{bmatrix}=
-x_{1}^{2}+2x_{1}^{3}x_{2}-2x_{2}^{2}
\end{equation*}
Z powyższego równania widać, że dla x należących do II lub IV ćwiartki układu współrzędnych (x1,x2) oraz dla punktów leżących na osiach tego układu całe wyrażenie jest ujemne.
Rozważmy teraz sytuacje w której x należy do I lub III ćwiartki. Rozwiązanie nierówności w tej sytuacji nie jest łatwe dlatego posłużono się szacowaniem rozwiązania. W tym przypadku mamy gwarancję, że wyrażenie \( \frac{1}{x_{1}x_{2}} > 0 \) . Mnożąc przez to wyrażenie obie strony nierówności otrzymamy : 
\begin{equation*}
-x_{1}^{2}-2x_{2}^{2}+2x_{1}^{3} x_{2} < 0 
\end{equation*} 
\begin{equation*}
\frac{-x_{1}}{x_{2}}-2\frac{x_{2}}{x_{1}}+2x_{1}^{2} < 0
\end{equation*}
\begin{equation*}
2x_{1}^{2} < 2\frac{x_{2}}{x_{1}}+\frac{x_{1}}{x_{2}}
\end{equation*}
Rozważając powyższą nierówność dla liczb \(|x_{1}|<\frac{1}{\sqrt{2}}\) oraz \(|x_{2}|<\frac{1}{\sqrt{2}}\) należących do I lub III ćwiartki układu współrzędnych zauważamy, że nierówność ta jest zawsze spełniona, ponieważ:
\begin{itemize}
\item wyrażenie po lewej stronie musi mieć wartość mniejszą od 1, gdyż \(|x_{1}|<\frac{1}{\sqrt{2}}\) , 
\item w wyrażeniu po prawej stronie mamy dzielenie przez siebie dwóch liczb mniejszych od 1 w wyniku czego jedno z tych dzieleń daje liczbę większą od 1, co gwarantuje, że prawa strona jest większa od lewej, a ponadto wyrażenia po stronie prawej są zawsze dodatnie w rozważanych ćwiartkach układu współrzędnych,
\item w przypadku gdy x1 = x2 otrzymuje się wartość 3 po prawej stronie, co również spełnia nierówność
\end{itemize}
Nierówność wyjściowa jest więc spełniona w I i III ćwiartce w pewnym otoczeniu 0. \\
Ostatecznie więc nierówność \( -x_{1}^{2}-2x_{2}^{2}+2x_{1}^{3} x_{2} < 0 \) jest spełnialna w każdej z ćwiartek układu w pewnym otoczeniu 0 \( \left( np.  \quad \Omega = \{ (x_{1},x_{2} ) : |x_{1}| < \frac{1}{\sqrt{2}}, \quad |x_{2}| < \frac{1}{\sqrt{2}} \} \right) \) , co dowodzi, że istnieje otoczenie 0 dla którego \( \dot{V}(x) < 0 \), a tym samym badany funkcjonał jest funkcjonałem Lapunowa tego systemu.\\ 
Stosując teraz twierdzenie Lapunowa do badanego systemu otrzymamy wniosek : 
\begin{center}
\textit{Punkt (0,0) jest asymptotycznie stabilnym punktem równowagi systemu, ponieważ w pewnym otoczeniu zera istnieje funkcjonał Lapunowa V(x) i zachodzi tam nierówność silna \( \dot{V}(x)<0\), dla \( x \neq 0 \) . Punkt (0,0) nie jest globalnie asymptotycznie stabilny.} 
\end{center}
\subsubsection{Wyznaczenie obszaru atrakcji przy pomocy twierdzenia LaSalle'a - funkcjonał I}
Z punktu 2.2.2 wiadomo, że w przypadku rozważanego systemu spełnione są założenia: 
\begin{itemize}
\item \( V : R^{2} \rightarrow R \)
\item Funkcjonał jest ciągły wraz z pierwszymi pochodnymi cząstkowymi względem \(x_{1}\) oraz \(x_{2}\) w pewnym otoczeniu zerowego punktu równowagi systemu, 
\item V(0) = 0 oraz V(0) > 0 w pewnym otoczeniu \(\Omega_{1}\) zera, z wyłączeniem zera, 
\item \( \dot{V}(x)< 0 \) w pewnym otoczeniu \( \Omega_{2}\) zera
\end{itemize}
Można więc zastosować uproszczoną wersję twierdzenia LaSalle'a. Wyznaczenie zbioru dla którego spełnione są założenia twierdzenia zostanie przeprowadzone na podstawie rozważań z punktu 2.1.2, z którego wiadomo, że:
\begin{itemize}
\item Dla każdego punktu znajdującego się w II lub IV ćwiartce lub na osiach układu współrzędnych (x1,x2) są spełnione założenia twierdzenia LaSalle'a. 
\item Założenia twierdzenia LaSalle'a są na pewno spełnione w ćwiartce I i III, o ile \( | x_{1} | < \frac{1}{\sqrt{2}} \) oraz \( | x_{2} | < \frac{1}{\sqrt{2}} \) , gdzie (x1,x2) to punkt z I lub III ćwiartki. 
\end{itemize}
Jeżeli zatem znaleziony zostanie zbiór \( Z_{l} \) zawierający się w zbiorze A opisanym warunkami : 
\begin{equation*}
A= \{ (x_{1},x_{2}) : \quad |x_{1}|<\frac{1}{\sqrt{2}} \quad , |x_{2}|<\frac{1}{\sqrt{2}} \}
\end{equation*}
to wówczas mamy gwarancję, że warunki twierdzenia LaSalle'a są spełnione dla zbioru \( Z_{l} \), ponieważ jest on podzbiorem zbioru dla którego założenia twierdzenia są spełnione. 
Przyjmując \( l = \frac{1}{4} \) otrzymamy zbiór \( Z_{l} \) : 
\begin{equation*}
Z_{l} = \{ (x_{1},x_{2}) : \quad \frac{x_{1}^{2}}{2}+x_{2}^{2} < \frac{1}{4} \}
\end{equation*}
Zbiór ten spełnia założenia ponieważ jest podzbiorem zbioru A. Nierówność opisująca zbiór to elipsa, której połowa dłuższej półosi ma długość \( \frac{1}{\sqrt{2}} \) . Nie jest to w tym przypadku maksymalna możliwa wartość \( l \), ponieważ nie udało się rozwiązać w sposób dokładny nierówności \( \dot{V}(x) < 0 \) .
\subsubsection{Badanie stabilności za pomocą II metody Lapunowa - funkcjonał II}
Zgodnie z zaleceniem w poleceniu do badania stabilności zostanie wykorzystany funkcjonał postaci:
\begin{equation*}
V(x)=\frac{x_{1}^{2}}{1-x_{1}x_{2}}+x_{2}^{2}
\end{equation*}
Dziedzina tego funkcjonału : \\
\( D = \{ (x_{1},x_{2}) \in R^{2} : 1-x_{1}x_{2} \neq 0 \} \) \\
\( D = \{ (x_{1},x_{2}) \in R^{2} : x_{2} \neq \frac{1}{x_{1}} \} \) 
\\
Aby funkcjonał V(x) mógł być funkcjonałem Lapunowa muszą być spełnione założenia : 
\begin{itemize}
\item \( V : R^{n} \supset \Omega \rightarrow R \)
\item Funkcjonał jest ciągły wraz z pierwszymi pochodnymi cząstkowymi względem \(x_{1}\) oraz \(x_{2}\) w pewnym otoczeniu zerowego punktu równowagi systemu, 
\item V(0) = 0 oraz V(x) > 0 w pewnym otoczeniu \(\Omega_{1}\) zera, z wyłączeniem zera 
\end{itemize}
Założenie \( V(x)>0 \) można łatwo uzasadnić rozważając wartość wyrażenia \( -x_{1}x_{2} \). Wyrażenie to:
\begin{itemize}
\item Dla \(x_{1}, x_{2} \) należących do II lub IV ćwiartki ma wartość dodatnią więc V(x) ma wartość dodatnią,
\item Dla \(x_{1}, x_{2} \) leżących na osiach układu współrzędnych ma wartość zerową więc V(x) ma wartość dodatnią,
\item dla \(x_{1}, x_{2} \) należących do I lub III ćwiartki ma wartość ujemną, V(x) pozostanie dodatnie o ile spełnione będą warunki \(x_{2}<\frac{1}{x_{1}} \) w pierwszej ćwiartce oraz \(x_{2}>\frac{1}{x_{1}}\) w III ćwiartce. Warunki te wynikają z nierówności stanowiącej o tym, że mianownik pierwszego składnika V(x) ma być dodatni  
\end{itemize}
Dla spełnienia wszystkich założeń ograniczamy zbiór dziedziny do zbioru \( \Omega \): \\
\( \Omega = \{ (x_{1},x_{2}): x_{2} \leq \frac{1}{x_{1}} \) dla \( x \in II,III \), \( x_{2}>\frac{1}{x_{1}} \) dla \( x \in I,IV \} \)
Graficznie zbiór \( \Omega \) jest zaznaczony na Rysunku \ref{fig:lap2_3}.
\begin{figure}[H]
\centerline{\includegraphics[scale=0.8]{lap2_3.jpg}}
\centering
\caption{Zbiór \( \Omega \) - bez czerwonych linii funkcji \(x_{2}=\frac{1}{x_{1}} \) }
\label{fig:lap2_3}
\end{figure}
Aby móc wnioskować o tym czy funkcjonał ten jest funkcjonałem Lapunowa dla badanego systemu, a później o stabilności tego systemu, poza powyższymi warunkami potrzebne jest jeszcze zbadanie znaku wyrażenia \( \dot{V}(x) \). \\ Funkcjonał będzie funkcjonałem Lapunowa badanego systemu jeśli spełniony będzie warunek :
\begin{itemize}
\item \( \dot{V}(x)\leq 0 \) w pewnym otoczeniu \( \Omega_{2}\) zera
\end{itemize} 
\begin{equation*}
\dot{V}(x)=
\begin{bmatrix}
\frac{\partial V}{\partial x_{1}} \quad
\frac{\partial V}{\partial x_{2}} 
\end{bmatrix}
\begin{bmatrix}
f_{1}(x) \\
f_{2}(x)
\end{bmatrix}=
\begin{bmatrix}
\frac{2x_{1}-x_{1}^{2}x_{2}}{(1-x_{1}x_{2})^{2}} \quad
\frac{x_{1}^{3}}{(1-x_{1}x_{2})^{2}}+2x_{2}
\end{bmatrix}
\begin{bmatrix}
-x_{1}+2x_{1}^{2}x_{2} \\ 
-x_{2}
\end{bmatrix}=
\end{equation*}
\begin{equation*}
\frac{(2x_{1}-x_{1}^{2}x_{2})(-x_{1}+2x_{1}^{2}x_{2})}{(1-x_{1}x_{2})^{2}}+\frac{-x_{1}^{3}x_{2}}{(1-x_{1}x_{2})^{2}}-2x_{2}^{2}=
\end{equation*}
\begin{equation*}
\frac{-2x_{1}^{4}x_{2}^{2}-2x_{2}^{2}x_{2}^{4}+4x_{1}x_{2}^{3}+4x_{1}^{3}x_{2}-2x_{1}^{2}-2x_{2}^{2}}{(1-x_{1}x_{2})^{2}}=
\end{equation*}
\begin{equation*}
2\frac{-x_{1}^{4}x_{2}^{2}-x_{2}^{2}x_{2}^{4}+2x_{1}x_{2}^{3}+2x_{1}^{3}x_{2}-x_{1}^{2}-x_{2}^{2}}{(1-x_{1}x_{2})^{2}}=
\end{equation*}
\begin{equation*}
2\frac{ -x_{1}^{4}x_{2}^{2}-x_{2}^{2}x_{2}^{4}  +  2x_{1}x_{2}(x_{2}^{2}+x_{1}^{2}) - (x_{1}^{2}+x_{2}^{2})}{(1-x_{1}x_{2})^{2}}=
\end{equation*}
\begin{equation*}
2\frac{ -x_{1}^{2}x_{2}^{2}(x_{1}^{2}+x_{2}^{2})  +  2x_{1}x_{2}(x_{2}^{2}+x_{1}^{2}) - (x_{1}^{2}+x_{2}^{2})}{(1-x_{1}x_{2})^{2}}=
\end{equation*}
\begin{equation*}
2\frac{(x_{1}^{2}+x_{2}^{2}) (-x_{1}^{2}x_{2}^{2}  +  2x_{1}x_{2} - 1 )}{(1-x_{1}x_{2})^{2}}=
\end{equation*}
\begin{equation*}
-2\frac{(x_{1}^{2}+x_{2}^{2})(1-x_{1}x_{2})^{2}}{(1-x_{1}x_{2})^{2}}=
\end{equation*}
\begin{equation*}
-2(x_{1}^{2}+x_{2}^{2}) \leq 0 \quad \forall x \in \Omega
\end{equation*}
Z powyższego równania widać, że dla dowolnych x \( \in \Omega \) różnych od 0 wyrażenie \(\dot{V}(x)\) ma wartość ujemną, a zatem badany funkcjonał jest funkcjonałem Lapunowa tego systemu.\\ 
Stosując teraz twierdzenie Lapunowa do badanego systemu otrzymamy wniosek : 
\begin{center}
\textit{Punkt (0,0) jest asymptotycznie stabilnym punktem równowagi systemu, ponieważ w pewnym otoczeniu zera istnieje funkcjonał Lapunowa V(x) i zachodzi tam nierówność silna \( \dot{V}(x)<0\), dla \( x \neq 0 \) . Punkt (0,0) nie jest globalnie asymptotycznie stabilny, ponieważ rozważania są ograniczone do podzbioru \( R^{n} \)} 
\end{center}
\subsubsection{Wyznaczenie obszaru atrakcji przy pomocy zmodyfikowanego twierdzenia LaSalle'a - funkcjonał II}
Z punktu 2.2.4 wiadomo, że w przypadku rozważanego systemu spełnione są założenia: 
\begin{itemize}
\item \( R^{n} \supset \Omega \rightarrow R \)
\item Funkcjonał jest ciągły wraz z pierwszymi pochodnymi cząstkowymi względem \(x_{1}\) oraz \(x_{2}\) w pewnym otoczeniu zerowego punktu równowagi systemu, 
\item V(0) = 0 oraz V(0) > 0 w pewnym otoczeniu \(\Omega_{1}\) zera, z wyłączeniem zera, 
\item \( \dot{V}(x)< 0 \) w pewnym otoczeniu \( \Omega_{2}\) zera
\end{itemize}
Zbiór \( \Omega \) został wyznaczony w poprzednim punkcie. W tym przypadku konieczne jest zastosowanie zmodyfikowanego twierdzenia LaSalle'a, ponieważ \( \Omega \neq R^{n} \). Poszukiwany zbiór \( Z_{l} \) nie może mieć punktów wspólnych z wyznaczonym zbiorem \( \Omega \). \\
\( \partial Z_{l} \cap \partial \Omega = \emptyset \) . 
Z poprzedniego punktu wiadomo, że w zbiorze \( \Omega \) dla dowolnego \( x \in \Omega \) są spełnione warunki 
\begin{itemize}
\item \( V(x)>0 \)
\item \( \dot{V}(x) < 0 \)
\end{itemize}
Wynika stąd, że ograniczając rozważania do tego zbioru można przyjąć dowolnie duże \( l \), a rozwiązania nierówności \( V(x) < l \) należące do zbioru \( \Omega \) będą spełniać powyższe warunki. 
Pozostaje sprawdzić czy zbiory \( \Omega \) oraz \( Z_{l} \) mogą mieć punkty wspólne, a tym samym czy \( Z_{l} \) zawiera się w \( \Omega \).
Zbiór \( Z_{l} \) dla dowolnie dużego \( l \) jest ograniczany przez poziomice opisaną równaniem :
\begin{equation*}
\frac{x_{1}^{2}}{1-x_{1}x_{2}}+x_{2}^{2}=l
\end{equation*}
Poszukujemy punktów wspólnych tej poziomicy oraz brzegu zbioru \( \Omega \) : 
\begin{equation*}
\begin{cases}
\frac{x_{1}^{2}}{1-x_{1}x_{2}}+x_{2}^{2}=l \\
x_{2}=\frac{1}{x_{1}}
\end{cases}
\end{equation*}
Po podstawieniu drugiego z równań do pierwszego otrzymamy sprzeczność, dla dowolnego \( l \), co dowodzi, że brzegi zbiorów nie mają punktów wspólnych.
Zatem zbiór \( Z_{l} \) opisany warunkami : 
\begin{equation*}
Z_{l}= \{ (x_{1},x_{2}) \in \Omega : \frac{x_{1}^{2}}{1-x_{1}x_{2}}+x_{2}^{2}<l \}
\end{equation*}
dla dowolnie dużego \(l\) jest estymatą obszaru atrakcji badanego systemu wyznaczoną analitycznie. 
\subsubsection{Wyznaczenie obszaru atrakcji przy pomocy eksperymentu}
W celu eksperymentalnego wyznaczenia obszaru atrakcji zerowego punktu równowagi zbudowano model w programie Matlab, który przedstawia Rysunek \ref{fig:simulink_2}
\begin{figure}[H]
\centerline{\includegraphics[scale=0.8]{simulink_2.jpg}}
\centering
\caption{Model rozważanego systemu}
\label{fig:simulink_2}
\end{figure}
Na Rysunku \ref{fig:lap2_4} przedstawiono zachowanie systemu nieliniowego w pewnym otoczeniu punktu równowagi. Czerwonym kolorem zaznaczono brzeg obszaru \( \Omega \). 
\begin{figure}[H]
\centerline{\includegraphics[scale=0.8]{lap2_4.jpg}}
\caption{Portret fazowy rozważanego systemu, kolor czerwony to brzeg zbioru \( \Omega \).}
\label{fig:lap2_4}
\end{figure}
\newpage
Na Rysunku \ref{fig:lap2_4_kopia} przedstawiono ten sam portret z zaznaczonymi wyznaczonymi analitycznie obszarami przyciągania dla obu użytych funkcjonałów. Czerwonym kolorem zaznaczono brzeg obszaru \( \Omega \). Wnętrze obszaru zielonego to zbiór \(Z_{l}\) wyznaczony dla funkcjonału numer I, wnętrze obszaru pomarańczowego to zbiór \(Z_{l}\) wyznaczony dla funkcjonału numer II.
\begin{figure}[H]
\centerline{\includegraphics[scale=1.2]{lap2_4_kopia.jpg}}
\caption{Portret fazowy rozważanego systemu, kolor czerwony to brzeg zbioru \( \Omega \), wnętrze obszaru zielonego to zbiór \(Z_{l}\) wyznaczony dla funkcjonału numer I, wnętrze obszaru pomarańczowego to zbiór \(Z_{l}\) wyznaczony dla funkcjonały numer II ).}
\label{fig:lap2_4_kopia}
\end{figure}


\section{Wnioski końcowe}
Badanie systemów nieliniowych przy użyciu I metody Lapunowa jest łatwe, ale metodę tą można stosować tylko w określonych przypadkach. Nie można z niej korzystać gdy system liniowy nie poddaje się linearyzacji lub wśród wartości własnych macierzy stanu systemu zlinearyzowanego jest przynajmniej jedna o zerowej części rzeczywistej, a pozostałe mają części rzeczywiste ujemne. Oprócz tego wadą tej metody jest też to, że nie daje informacji o obszarze przyciągania. 

Gdy nie można stosować I metody Lapunowa, często rozwiązaniem problemu jest zastosowanie II metody Lapunowa. Zaletą tej metody jest to, że może być stosowana w większej liczbie przypadków niż metoda pierwsza. Ponadto przy jej pomocy można wyznaczyć przybliżenie obszaru przyciągania. Wadą tej metody jest konieczność poszukiwania odpowiedniego funkcjonału oraz rozwiązywanie nierówności. Stosowanie tej metody jest w związku z tym często trudniejsze niż stosowanie metody pierwszej.

Przeprowadzone eksperymenty i obliczenia pokazały, że dla różnych funkcjonałów stopień skomplikowania obliczeń może być różny, a także różne mogą być rezultaty stosowania metody LaSalle'a. Dla stosowanych funkcjonałów należy zawsze pamiętać o spełnieniu przez funkcjonały odpowiednich założeń.
\end{document}