\documentclass[a4paper,15pt]{article}
\usepackage{amssymb}
\usepackage{amsmath}
\usepackage[english, polish]{babel}
\usepackage[utf8]{inputenc}   % lub utf8
\usepackage[T1]{fontenc}
\usepackage{graphicx}
\usepackage{anysize}
\usepackage{enumerate}
\usepackage{times}
\usepackage{titlesec}
\usepackage{float}
\usepackage{titleps,kantlipsum}
\usepackage{listings}
\usepackage{color}
\usepackage{hyperref}
\lstloadlanguages{Matlab}
 
\usepackage[justification=centering]{caption}
\titlelabel{\thetitle.\quad}


% Definicja nowego stylu strony
\newpagestyle{mypage}
{
  \headrule
  
  \sethead
  { \MakeUppercase{\thesection\quad \sectiontitle} } 
  {}
  {\thesubsection\quad \subsectiontitle}
  
  \setfoot
  {}
  {\thepage}
  {}
}

\newpagestyle{mypage_1}
{
	\headrule
	
	\sethead
	{  }
	{\MakeUppercase{Identyfikacja procesów przemysłowych }}
	{}
	
	\setfoot
	{}
	{}
	{}
}

\settitlemarks{section,subsection,subsubsection}

\pagestyle{mypage_1}

%\marginsize{left}{right}{top}{bottom}
\marginsize{3cm}{3cm}{3cm}{3cm}
\sloppy
\titleformat{\section}
  {\normalfont\Large\bfseries}{\thesection}{1em}{}[{\titlerule[0.8pt]}]
 
 \definecolor{darkred}{rgb}{0.9,0,0}
\definecolor{grey}{rgb}{0.4,0.4,0.4}
\definecolor{orange}{rgb}{1,0.6,0.05}
\definecolor{darkgreen}{rgb}{0.2,0.5,0.05}
 
\lstset{
language=Matlab,
basicstyle=\ttfamily\small,
keywordstyle=\color{darkgreen}\ttfamily\bfseries\small,
stringstyle=\color{red}\ttfamily\small,
commentstyle=\color{grey}\ttfamily\small,
numbers=left,
numberstyle=\color{darkred}\ttfamily\scriptsize,
identifierstyle=\color{blue}\ttfamily\small,
showstringspaces=false,
morekeywords={}
}


\begin{document}

%\thispagestyle{mypage_1}

\begin{table}
\begin{center}
\begin{tabular}{|l|l|l|}
\hline
\multicolumn{3}{|c|}{\textbf{Identyfikacja procesów przemysłowych}} \\ \hline Dominik Wróbel & 30 X 2018 & Wt. 13:45, s. 111 \\ \hline
\multicolumn{3}{|c|}{Laboratorium 3} \\ \hline 
\end{tabular}
\end{center}
\end{table}


\section{Przebieg ćwiczenia}
\subsection{Zadanie 1 }

Zadanie polega na badaniu zachowania układu inercyjnego pierwszego rzędu z wymuszeniem losowym w postaci białego szumu, które jest dodawane do sterowania na wejściu do układu. 
\begin{figure}[H]
\centerline{\includegraphics[scale=0.7]{01_schemat.jpg}}
\centering
\caption{Układ rozważany w zadaniu}
\label{fig:01_schemat}
\end{figure}
Układ taki opisany jest równaniem Ito:
\begin{align*}
& dx = (-ax + bu)dt + \sqrt{g}dw,\quad u=\sin \omega_1 t \\
& a = 2, \ b = 3, \ g = 0,01, \ \omega_1 = \pi
\end{align*}
Warunek początkowy to zmienna losowa o rozkładzie \( N(x_0, K_0) \), gdzie \( x_0=10 \), \(K_0=4\).

\subsubsection{Wyznaczenie średniej i wariancji procesu}

Pierwszą czynnością w zadaniu było rozwiązanie numeryczne równań opisujących wartość średnią oraz wariancje procesu:
\begin{align*}
\dot{\mu} = -a\mu + bu, \ \mu(0)=x_0, \\
\dot{K} = -2Ka+g, \ K(0)=K_0
\end{align*}
Równania zostały rozwiązane przy użyciu solvera ode45 dla czasu \( t \in [0,6]\).
\begin{figure}[H]
\centerline{\includegraphics[scale=0.4]{02_solution.jpg}}
\centering
\caption{Rozwiązania równań opisujących średnią i wariancję}
\label{fig:02_solution}
\end{figure}

\subsubsection{Wyznaczenie odchylenie standardowego}
Następnie wyznaczono odchylenie standardowe korzystając ze wzoru \( \sigma (t) = \sqrt{K(t)} \). Na jednym wykresie przedstawiono średnią procesu oraz linie \( m(t) = \mu (t) \pm 3\sigma (t) \ , t \in [0,6] \). Rezultat przedstawia \mbox{Rysunek \ref{fig:03_odchylenie}}.
\begin{figure}[H]
\centerline{\includegraphics[scale=0.45]{03_odchylenie.jpg}}
\centering
\caption{Wartość średnia wraz z liniami \( m(t) \).}
\label{fig:03_odchylenie}
\end{figure}
Rozważany układ jest jednowymiarowy, wzór na rozkład prawdopodobieństwa procesu w chwili t wyraża się wzorem:
\begin{align*}
& p(x,t) = \frac{1}{\sigma\sqrt{2\pi}}e^{\frac{-(x-\mu)^2}{2\sigma ^2}}
\end{align*}

Prawdopodobieństwo, że \( x(t) \in [\mu(t) - 3\sigma(t), \mu(t) + 3\sigma(t) ] \) oblicza się przy pomocy dystrybuanty rozkładu normalnego:
\begin{align*}
P(  E - 2 \sigma < X < E + 3\sigma ) = \Phi (3) - \Phi (-3) = 2\Phi (3) - 1 = 99,7\%
\end{align*}

Asymptotyczne odchylenie standardowe \( \lim _{ t \rightarrow \infty } \sqrt{K(t)} \) można obliczyć na podstawie równania różniczkowego przy założeniu, że \( \dot{K} = 0 \) dla \( t \rightarrow \infty \). Wtedy \( K = \frac{g}{2a} = 0,0025 \).

\subsubsection{Badania zachowania układu dla różnych warunków początkowych}
Następnie na jednym wykresie przedstawiono kilka różnych przebiegów procesu dla różnych warunków początkowych. Jak widać, wszystkie z eksperymentów zawierają się w wyznaczonych liniach \( m(t) \). Wyniki przedstawia Rysunek \ref{fig:04_symulacja}. 
Kolorem szarym oznaczono sygnał wejściowy.

\begin{figure}[H]
\centerline{\includegraphics[scale=0.45]{04_symulacja.jpg}}
\centering
\caption{Przeprowadzone eksperymenty dla różnych warunków początkowych.}
\label{fig:04_symulacja}
\end{figure}


\subsubsection{Kod programu}
Kod programu prezentuje listing poniżej.
\begin{lstlisting}[caption=Zadanie 1, captionpos=b,label=lis1, firstnumber=1,frame=single]
close all;
clear all;

tspan = 0:0.01:6;

a=2;
b=3;
g = 0.01;
w1 = pi;

mi0 = 10;
[t1,mi] = ode45(@(t,mi) -a*mi + b*sin(w1*t), tspan, mi0);

figure();
plot(t1,mi);

K0 = 4;
[t2,K] = ode45(@(t,K) -2*a*K + g, tspan, K0);

hold on;
plot(t2,K);


sigma = sqrt(K);
sigmaPlus = mi + 3*sigma;
sigmaMinus = mi - 3*sigma;

figure();
plot(t2,mi);
hold on;
plot(t2,sigmaPlus)
hold on;
plot(t2,sigmaMinus)

A = -a;
B = 1;
C = 1;
D = 0;

figure()
uSin = sin(w1*t1);
noise = randn(length(t1),1);

uControl = b*uSin + sqrt(g)*noise; 


sys = ss(A,B,C,D);

x0 = 10;
lsim(sys,uControl,tspan,x0) 

hold on;
plot(t2,sigmaPlus,'m');
hold on;
plot(t2,sigmaMinus,'m');

hold on;
x0 = 5;
lsim(sys,uControl,tspan,x0) 

hold on;
x0 = 7;
lsim(sys,uControl,tspan,x0)

hold on;
x0 = 14;
lsim(sys,uControl,tspan,x0) 

\end{lstlisting}

\subsection{Zadanie 2 }
Zadanie jest analogiczne do zadania 1, które schematycznie przedstawia Rysunek \ref{fig:01_schemat}, tym razem analizowanym układem jest układ II rzędu, opisany równaniem Ito:
\begin{align*}
& dx = ( Ax + Bu ) dt + Gdw \\
& A = 
\begin{bmatrix}
0 & 1  \\
- \omega _0^2 & -2\xi \omega _0 
\end{bmatrix} \
B = 
\begin{bmatrix}
0 \\
b
\end{bmatrix} \
G = 
\begin{bmatrix}
0 \\
\sqrt{g}
\end{bmatrix} \
u = 0 \\
& \omega _0 = 1, \ \xi = 0,1, \ b=1, \ g=1
\end{align*}
Warunkiem początkowym jest zmienna losowa o rozkładzie normalnym \( N(x_0, K_0) \), przy czym \mbox{\( x_0 = \begin{bmatrix} 15 & 0 \end{bmatrix} \)}, \( K_0 = 10^{-1}diag(1,1)\).

\subsubsection{Średnia i macierz kowariancji}
Zadanie rozpoczęto od numerycznego rozwiązania równań opisujących średnią oraz macierz kowariancji:
\begin{align*}
& \dot{\mu} = A\mu + Bu, \ \mu(0)=x_0 \\
& \dot{K} = KA^T + AK +GG^T, \ K(0)=K_0
\end{align*}
Macierz K jest symetryczna więc równanie opisujące macierz kowariancji sprowadza się do układu równań:
\begin{align*}
& \dot{K_{11}} = 2K_{12}, \ K_{11}(0)=10^{-3} \\
& \dot{K_{12}} = K_{22} - 2\xi\omega _0 K_{12} - \omega _{0}^{2}K_{12}, \ K_{12}(0)=0 \\
& \dot{K_{22}} = -2\omega _0^2 K_{12} - 4\xi \omega_0 K_{22} +g , \ K_{22}(0)=10^{-3}
\end{align*}
Równanie opisujące średnią oraz układ równań opisujący macierz kowariancji rozwiązano numerycznie przy użyciu solvera ode45. Otrzymane rozwiązania przedstawiają Rysunki \ref{fig:05_srednia2} oraz \ref{fig:06_kowariancja2}.
s

\begin{figure}[H]
  \begin{minipage}[b]{0.45\textwidth}
    \includegraphics[width=\textwidth]{05_srednia2.jpg}
    \caption{Średnia}
    \label{fig:05_srednia2}
  \end{minipage}
  \begin{minipage}[b]{0.45\textwidth}
    \includegraphics[width=\textwidth]{06_kowariancja2.jpg}
    \caption{Kowariancja \( K_{11}, K_{12}, K_{22} \) } 
    \label{fig:06_kowariancja2}
  \end{minipage}
\end{figure}

\subsubsection{Portret fazowy i rozkład prawdopodobieństwa}
Rozkład prawdopodobieństwa procesu w chwili t opisuje wzór:
\begin{align*}
p(t,x) = \frac{1}{2\pi|K(t)|}e^{0,5(x-\mu(t))^TK(t)^{-1}(x-\mu(t))}
\end{align*}
Na podstawie tego wzoru można narysować poziomice rozkładu w chwilach \( t = 0,1,2,3,4,5,6 \) wraz z portretem fazowym średniej. Wyniki prezentuje Rysunek \ref{fig:07_portret}.

\begin{figure}[H]
\centerline{\includegraphics[scale=0.45]{07_portret.jpg}}
\centering
\caption{Portret fazowy wraz z poziomicami rozkładu prawdopodobieństwa w kolejnych chwilach czasu \( t = 0,1,2,3,4,5,6 \). }
\label{fig:07_portret}
\end{figure}

Na podstawie zadania 1 obliczono prawdopodobieństwo \( P(|x_{1}(t)-\mu _1(t)| > 3\sqrt{K_{11}(t)}), \ t=6 \). Z zadania 1 wiadomo, że 
\begin{align*}
& P(  E - 2 \sigma < X < E + 3\sigma ) = \Phi (3) - \Phi (-3) = 2\Phi (3) - 1 = 99,7\%
\end{align*}
więc na tej podstawie łatwo stwierdzić, że \( P(|x_{1}(t)-\mu _1(t)| > 3\sqrt{K_{11}(t)})=0,03\%, \ t=6 \)

\subsubsection{Asymptotyczna kowariancja i średnia procesu}
Podobnie jak w zadaniu 1, zakłada się, że w nieskończoności zmiany średniej i kowariancji są równe zero, a zatem w celu obliczenia asymptotycznych wartości tych wielkości przyrównano równania je opisujące do 0.
\begin{align*}
& 0 = A\mu \rightarrow \mu = \begin{bmatrix}
0 \\ 0
\end{bmatrix}
\end{align*}
\begin{align*}
& 0 = 2K_{12} \\
& 0 = K_{22} - 2\xi\omega _0 K_{12} - \omega _{0}^{2}K_{12} \\
& 0 = -2\omega _0^2 K_{12} - 4\xi \omega_0 K_{22} +g 
\end{align*}
\begin{align*}
K_{asm} = \begin{bmatrix}
\frac{g}{4\xi \omega _0^3} & 0 \\
0 & \frac{g}{4\xi \omega _0}
\end{bmatrix}
=
\begin{bmatrix}
2,5 & 0 \\
0 & 2,5
\end{bmatrix}
\end{align*}

\subsubsection{Badanie układu dla \( \xi = 0 \).}
Dla tej wartości współczynnika tłumienia narysowano przebieg średniej położenia oraz prędkości. Z wykresów widać, że wielkości te mają przebieg sinusoidalny przy czym wartości prędkości odpowiadające położeniom są sinusami różniącymi się w fazie. Takie zachowanie jest uzasadnione ponieważ układ dla \( \xi =0 \) jest układem nietłumionym i wykonuje drgania o niezmieniającej się amplitudzie.

\begin{figure}[H]
\centerline{\includegraphics[scale=0.45]{08_xi0.jpg}}
\centering
\caption{Położenie oraz prędkość w dziedzinie czasu dla \( \xi = 0 \). }
\label{fig:08_xi0}
\end{figure}

\subsection{Kod programu}
Kod programu prezentuje listing poniżej.
\begin{lstlisting}[caption=Zadanie 1, captionpos=b,label=lis1, firstnumber=1,frame=single]
clear all;
close all;

tspan = 0:0.01:6;

w0 = 1;
ksi = 0; % dla ksi = 0
b = 1;
g = 1;

A = [ 0 1 ; -w0^2 -2*ksi*w0 ];
B = [ 0 ; b ];
G = [ 0 ; sqrt(g) ];
u = 0; 

x0 = [15, 0]';
K0 = (10^-1) * diag(1,1);

mi0 = x0;

[t1,miMatrix] = ode45(@(t,mi) A*mi + B*u, tspan, mi0);

for i = 1:length(t1)-1
    v(i,1) = (miMatrix(i+1,1)-miMatrix(i,1))/(t1(i+1)-t1(i)) ;
    v(i,2) = (miMatrix(i+1,2)-miMatrix(i,2))/(t1(i+1)-t1(i)) ;
end
tv = t1(1:600);

figure()
plot(t1, miMatrix(:,1));
hold on;
plot(t1, miMatrix(:,2));



[t2,kMatrix] = ode45(@vdp1,tspan,[0.01; 0; 0.01]);
figure()
plot(t2, kMatrix(:,1));
hold on;
plot(t2, kMatrix(:,2));
hold on;
plot(t2, kMatrix(:,3));


figure();
plot(t1, miMatrix(:,1), 'b');
hold on;
plot(t1, miMatrix(:,2), 'r');
hold on;
plot(tv, v(:,1), 'm');
hold on;
plot(tv, v(:,2), 'g');

hold on;
mu = [miMatrix(1,1) miMatrix(1,2)];
Sigma = [kMatrix(1,1) kMatrix(1,2); kMatrix(1,2)  kMatrix(1,3)];
x1 = -15:0.01:20; x2 = -15:.01:20;
[X1,X2] = meshgrid(x1,x2);
F = mvnpdf([X1(:) X2(:)],mu,Sigma);
F = reshape(F,length(x2),length(x1));

mvncdf([0 0],[1 1],mu,Sigma);
contour(x1,x2,F,[0.000001 0.00001 0.0001 0.001 0.01 1 2 10 50 100 1000]);
xlabel('x'); ylabel('y');



\end{lstlisting}

\section{Wnioski końcowe}
Przeprowadzone eksperymenty pozwalają na uzyskanie informacji o wyjściu układu przy zakłóceniach obecnych na wejściu układu. Informacja taka pozwala stwierdzić w jakim przedziale znajdować będzie się wartość zmiennej procesowej w danej chwili czasu. 

 
\end{document}