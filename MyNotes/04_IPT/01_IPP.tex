\documentclass[a4paper,15pt]{article}
\usepackage{amssymb}
\usepackage{amsmath}
\usepackage[english, polish]{babel}
\usepackage[utf8]{inputenc}   % lub utf8
\usepackage[T1]{fontenc}
\usepackage{graphicx}
\usepackage{anysize}
\usepackage{enumerate}
\usepackage{times}
\usepackage{titlesec}
\usepackage{float}
\usepackage{titleps,kantlipsum}
\usepackage{listings}
\usepackage{color}
\usepackage{hyperref}
\lstloadlanguages{Matlab}
 
\usepackage[justification=centering]{caption}
\titlelabel{\thetitle.\quad}


% Definicja nowego stylu strony
\newpagestyle{mypage}
{
  \headrule
  
  \sethead
  { \MakeUppercase{\thesection\quad \sectiontitle} } 
  {}
  {\thesubsection\quad \subsectiontitle}
  
  \setfoot
  {}
  {\thepage}
  {}
}

\newpagestyle{mypage_1}
{
	\headrule
	
	\sethead
	{  }
	{\MakeUppercase{Identyfikacja procesów przemysłowych }}
	{}
	
	\setfoot
	{}
	{}
	{}
}

\settitlemarks{section,subsection,subsubsection}

\pagestyle{mypage_1}

%\marginsize{left}{right}{top}{bottom}
\marginsize{3cm}{3cm}{3cm}{3cm}
\sloppy
\titleformat{\section}
  {\normalfont\Large\bfseries}{\thesection}{1em}{}[{\titlerule[0.8pt]}]
 
 \definecolor{darkred}{rgb}{0.9,0,0}
\definecolor{grey}{rgb}{0.4,0.4,0.4}
\definecolor{orange}{rgb}{1,0.6,0.05}
\definecolor{darkgreen}{rgb}{0.2,0.5,0.05}
 
\lstset{
language=Matlab,
basicstyle=\ttfamily\small,
keywordstyle=\color{darkgreen}\ttfamily\bfseries\small,
stringstyle=\color{red}\ttfamily\small,
commentstyle=\color{grey}\ttfamily\small,
numbers=left,
numberstyle=\color{darkred}\ttfamily\scriptsize,
identifierstyle=\color{blue}\ttfamily\small,
showstringspaces=false,
morekeywords={}
}


\begin{document}

%\thispagestyle{mypage_1}

\begin{table}
\begin{center}
\begin{tabular}{|l|l|l|}
\hline
\multicolumn{3}{|c|}{\textbf{Identyfikacja procesów przemysłowych}} \\ \hline Dominik Wróbel & 16 X 2018 & Wt. 13:45, s. 111 \\ \hline
\multicolumn{3}{|c|}{Laboratorium 1} \\ \hline 
\end{tabular}
\end{center}
\end{table}


\section{Przebieg ćwiczenia}
\subsection{Zadanie 1 - błąd oszacowania średniej }

W pierwszej kolejności zbadano czy dane z zadania mają rozkład normalny. W tym celu najpierw obliczono średnie napięcie w każdej chwili czasu poprzez obliczenie wartości średniej każdego wiersza otrzymanych pomiarów. Tak otrzymane średnie wartości napięć poddano dwóm testom - Kołmogorowa - Smirnowa oraz Lillieforsa. Dla testów otrzymano następujące wyniki:
\begin{itemize}
\item Kołmogorowa - Smirnowa : 1, hipoteza o rozkładzie normalnym danych została odrzucona
\item Lillieforsa : 0, hipoteza o rozkładzie normalnym danych została potwierdzona
\end{itemize}

Aby sprawdzić rozkład danych narysowano także histogram wraz z dopasowaną krzywą rozkładu normalnego:

\begin{figure}[H]
\centerline{\includegraphics[scale=0.6]{01_hist.jpg}}
\centering
\caption{Histogram dla danych z eksperymentu wraz z dopasowaną krzywą rozkładu normalnego}
\label{fig:01_hist}
\end{figure}

Na podstawie wyników otrzymanych testów oraz Rysunku \ref{fig:01_hist} można stwierdzić, że dane mają rozkład normalny. 

Kolejną czynnością było wyznaczenie średniej wartości prądu płynącego przez rezystor oraz oszacowanie błędu tej średniej ( odchylenie standardowe ). Otrzymaną średnią uzyskano dzieląc średnie napięcie ze wszystkich chwil czasu przez wartość rezystora 120 \( \Omega \). 
\begin{align*}
& I_{sr} = 8,2805 mA \\
& \sigma = 10,5451 mA
\end{align*}

Następnie obliczono przedział ufności dla oszacowania średniej przy użyciu funkcji matlaba \textit{paramci} na poziomie \( \frac{1}{100} \). Otrzymane wartości przedziału to \( I_1 = 8,0088 mA, \quad I_2 = 8.5521 mA \). 

Dla obliczonej średniej oraz odchylenia standardowego narysowano rozkład Gaussa wraz z przedziałami ufności. 

\begin{figure}[H]
\centerline{\includegraphics[scale=0.6]{02_gauss.jpg}}
\centering
\caption{Rozkład Gaussa dla wartości prądu w mA, przedział ufności na poziomie istotności 0,01 to <8,0088 ;  8.5521 > }
\label{fig:02_gauss}
\end{figure}

Kod programu prezentuje listing poniżej.
\begin{lstlisting}[caption=Zadanie 1, captionpos=b,label=lis1, firstnumber=1,frame=single]
clear all;

sredniaWektorNapiecie = 0; 

load('data_01.mat');


for j = 1:10000 
    sredniaWektorNapiecie(j) = mean(u(j,:));
end


histfit(sredniaWektorNapiecie);


kstestResult = kstest(sredniaWektorNapiecie);
liltestResult = lillietest(sredniaWektorNapiecie);
kstestResult
liltestResult


sredniaSrednichNapiecie = mean(sredniaWektorNapiecie);
sredniaWektorPrad = sredniaWektorNapiecie / 120;
sredniPrad = sredniaSrednichNapiecie / 120;

sredniPradMa = sredniPrad * 1000;
sredniaWektorPradMa = 1000 * sredniaWektorPrad;
odchylenieSrednichMa = std(sredniaWektorPradMa);


pd = fitdist(sredniaWektorPradMa', 'Normal')
ci = paramci(pd,'Alpha',.01)


x = [-15:.1:30];
norm = normpdf(x,sredniPradMa,odchylenieSrednichMa);
figure;
plot(x,norm)
hold on;
line([8.0088  8.0088 ], [0 0.04]);
line([  8.5521     8.5521  ], [0 0.04]);

\end{lstlisting}




\subsection{Zadanie 2 - aproksymacja wielomianami }

Zadanie polega na zastosowaniu metody najmniejszych kwadratów do zbioru danych z czujnika ciśnieniowego. Współczynniki są wyznaczane dla różnego stopnia wielomianów \( n = 1,2,3,4,5 \) korzystając z równania:
\begin{align*}
& \Phi ^T\Phi a=\Phi ^T Y \\
& a = ( \Phi ^T\Phi ) ^ {-1} \Phi ^T Y
\end{align*}
Jakość dopasowania została zbadana przy pomocy testu chi2 na poziomie ufności \( \frac{1}{100} \), badając odchyłki wartości zmierzonych od aproksymowanych. Wektor podlegający testowi opisuje równanie:
\begin{align*}
& e = Y - \Phi a
\end{align*}

\begin{table}[H]
\caption{Współczynniki wielomianu oraz wyniki testu ch2 dla różnych stopni wielomianu}
\begin{center}
\begin{tabular}{|l|l|l|}
\hline
n & współczynniki wielomianu & Test chi2 \\ \hline
1 & 0.4361 -1.7532 & 1  \\ \hline
2 & 0.0096    0.1007    0.1830 & 0  \\ \hline
3 & 0.0001    0.0064    0.1451    0.0571 & 0  \\ \hline
4 & -0.0000    0.0004   -0.0008    0.2002   -0.0351 & 0  \\ \hline
5 & -0.0000    0.0001   -0.0014    0.0225    0.0861    0.0887 & 0 \\ \hline
\end{tabular}
\end{center}
\end{table}
Hipoteza testu chi2 została odrzucono tylko w przypadku \( n = 1 \). Najmniejsze n dla którego testchi2 pozwala przyjąć hipotezę, że reszty modelu pochodzą z rozkładu normalnego wynosi 2. 

Odchylenie standardowe dla pomiarów obliczono na podstawie wariancji korzystając ze wzoru:
\begin{align*}
& \sigma ^2 = \frac{e^Te}{N-n-1}\\
\end{align*} 
Natomiast odchylenia standardowe parametrów na podstawie wariancji parametrów, które są elementami diagonalnymi macierzy:
\begin{align*}
& cov a = \sigma ^ 2 ( \Phi^T \Phi)^{-1}
\end{align*} 

\begin{table}[H]
\caption{Odchylenia standardowe dla poziomu cieczy oraz parametrów }
\begin{center}
\begin{tabular}{|l|l|l|l|l|}
\hline
n & \( \sigma ^2 \) & \( \sigma \) & \( diag(cova) \) & \( \sigma_p \) \\ \hline
1 & 0.8507 &    0.9223 &  0.0335, 0.0001  &   0.1831    0.0090\\ \hline
2 &    0.0387   & 0.1967  &  0.0033,  0.0001, 0.0000  &     0.0579    0.0076    0.0002 \\ \hline
3 & 0.0364 & 0.1909 &  0.0054,  0.0003, 0.0000,  0.0000 &  0.0736    0.0183    0.0012    0.0000 \\ \hline
4 &  0.0356 & 0.1887 & 0.0079, 0.0013, 0.0000,  0.0000,  0.0000  &   0.0890    0.0356    0.0042    0.0002    0.0000 \\ \hline
5 &  0.0340 & 0.1844 & 0.0103,   0.0036, 0.0001, 0.0000, 0.0000, 0.0000  & 0.1016,    0.0597,    0.0107,    0.0008,   0.0000,    0.0000 \\ \hline
\end{tabular}
\end{center}
\end{table}


\newpage
Dane oraz aproksymacje wraz z krzywymi odchylonymi od charakterystyki o 3 wartości odchylenia przedstawiają wykresy poniżej:

\begin{figure}[H]

\minipage{0.5\textwidth}

  \includegraphics[width=\linewidth]{03_n1.jpg}
  \caption{n = 1}
  \label{fig:opt_wiel_2}

\endminipage\hfill
\minipage{0.5\textwidth}%

  \includegraphics[width=\linewidth]{04_n2.jpg}
  \caption{n = 2}
  \label{fig:opt_wiel_3}

\endminipage

\end{figure}

\begin{figure}[H]

\minipage{0.5\textwidth}

  \includegraphics[width=\linewidth]{05_n3.jpg}
  \caption{n = 3}
  \label{fig:opt_wiel_2}

\endminipage\hfill
\minipage{0.5\textwidth}%

  \includegraphics[width=\linewidth]{06_n4.jpg}
  \caption{n = 4}
  \label{fig:opt_wiel_3}

\endminipage

\end{figure}

\begin{figure}[H]

\minipage{0.5\textwidth}

  \includegraphics[width=\linewidth]{07_n5.jpg}
  \caption{n = 5}
  \label{fig:opt_wiel_2}

\endminipage\hfill


\end{figure}

Kod programu prezentuje listing poniżej.
\begin{lstlisting}[caption=Zadanie 2, captionpos=b,label=lis1, firstnumber=1,frame=single]
clear all;
load('data_02.mat')


% n = 1,2,3,4,5
n = 1;
for i = 1:n
    
    matrixPhi(:,1) = ones(length(u),1);
    
    for j = 1:i
        matrixPhi(:,j+1) = u.^(j); 
    end
    
    %matrixPhiT = matrixPhi';
    a = ( matrixPhi' * matrixPhi ) \ matrixPhi' * y(:,5);
    
end



for i = 0:length(a)-1
    p(i+1) = a(length(a)-i);
end
p;

val = polyval(p,u);

plot(u,y(:,5),'*');
hold on;
plot(u,val,'r');

e = y(:,5) - matrixPhi * a;

estWar = ( e' * e ) / ( length(u) - n - 1 )
odchStan = sqrt(estWar)

val1 = val + 3*odchStan;
val2 = val - 3*odchStan;

hold on;
plot(u,val1,'r');
hold on;
plot(u,val2,'r');

[h,p] = chi2gof(e,'Alpha',0.01);

estWar = ( e' * e ) / ( length(u) - n - 1 )
odchStan = sqrt(estWar)

ma1 = inv( matrixPhi' * matrixPhi );  
macKow = estWar * ma1
for i = 1:length(macKow(:,1))
   odch(i) = sqrt(macKow(i,i)); 
end
odch

\end{lstlisting}

\section{Zadanie 3 - oscylator z tłumieniem}
Zadnie polega na zastosowaniu metody najmniejszych kwadratów w celu doboru parametrów systemu dynamicznego przy minimalizacji odchyłek od rzeczywistych pomiarów. 
Rozważanym w zadaniu układem jest oscylator z tłumieniem dla którego pomiary są wykonywane w dyskretnych chwilach czasu. Model taki może być opisany równaniem różnicowym:
\begin{align*}
y_k = \theta_1 y_{k-1} + \theta_2 y_{k-2} + e_k
\end{align*}
Parametry zostały oszacowane na podstawie minimalizacji funkcji:
\begin{align*}
& V(\theta)=\frac{1}{2}(\Phi \theta - Y )^T(\Phi \theta - Y ) \rightarrow min \\
& \theta ^* = (\Phi ^T \Phi)^{-1} \Phi ^T Y \\
& \theta^* = 
\begin{bmatrix}
1,9081 \\
-0,9973
\end{bmatrix}
\end{align*}

Oszacowanie błędów dla wyznaczonych parametrów obliczono na podstawie macierzy kowariancji, której elementy diagonalnej są wariancjami parametrów.
\begin{align*}
& cov \theta ^* = \frac{(\Phi\theta ^* - Y)^T (\Phi\theta ^* - Y)}{N-2} \\
& \sigma_{\theta _1} = 0,0005967 \\
& \sigma_{\theta _2} = 0,0005953
\end{align*}

Zakłócenie e obliczono ze wzoru:
\begin{align*}
e = Y - \Phi \theta ^ *
\end{align*}
Wariancja tego zakłócenia wynosi  0,0000057. 

Porównanie modelu oraz danych pomiarowych przedstawia Rysunek poniżej.
\begin{figure}[H]
\centerline{\includegraphics[scale=0.6]{08_por.jpg}}
\centering
\caption{Porównanie wyników danych pomiarowych ( kolor niebieski ) oraz modelu ( kolor zielony )}
\label{fig:08_por}
\end{figure}
Z Rysunku \ref{fig:08_por} wynika, że parametry udało się wyznaczyć z dużą dokładnością. 
Kod programu: 
\begin{lstlisting}[caption=Zadanie 3, captionpos=b,label=lis1, firstnumber=1,frame=single]
clear all;
load('data_03.mat')

phiMatrix(:,1) = y(2:end-1);  
phiMatrix(:,2) = y(1:end-2); 
Y = y(3:length(y));

theta = ( phiMatrix' * phiMatrix ) \ phiMatrix' * Y;

figure();
plot(t,y);

yApr(1:2) = y(1:2); 
for i = 3:length(y)
    yApr(i) = theta(1) * yApr(i-1) + theta(2) * yApr(i-2); 
end

hold on;
plot(t,yApr,'g');

vTheta = 0.5*(phiMatrix * theta  - Y)' * (phiMatrix * theta - Y );
estWar = 2*vTheta / ( length(y) - 2 );
mul = inv ( phiMatrix' * phiMatrix );
macKow = estWar * mul;
odchTheta(1) = sqrt(macKow(1,1));
odchTheta(2) = sqrt(macKow(2,2));

e = Y - phiMatrix * theta;
mean(e)
wariancja = var(e)


\end{lstlisting}


\section{Wnioski końcowe}
Przeprowadzone aproksymacje prezentują sposoby wyznaczania parametrów dla różnych zagadnień. Z eksperymentów wynika, że do dokładnego wyznaczenia parametrów modelu konieczna jest duża liczba próbek z rzeczywistego eksperymentu. Duże znaczenie ma również implementacja metody użytej do aproksymacji, stopień wielomianu aproksymującego ma duży wpływa na dopasowanie do danych pomiarowych co uwidacznia się w zadaniu 2. Przy wyznaczaniu parametrów należy też pamiętać o oszacowaniu błędów. 
 
\end{document}