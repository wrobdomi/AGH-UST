\documentclass[a4paper,15pt]{article}
\usepackage{amssymb}
\usepackage{amsmath}
\usepackage[english, polish]{babel}
\usepackage[utf8]{inputenc}   % lub utf8
\usepackage[T1]{fontenc}
\usepackage{graphicx}
\usepackage{anysize}
\usepackage{enumerate}
\usepackage{times}
\usepackage{titlesec}
\usepackage{float}
\usepackage{titleps,kantlipsum}
\usepackage{listings}
\usepackage{color}
\usepackage{hyperref}
\lstloadlanguages{Matlab}
 
\usepackage[justification=centering]{caption}
\titlelabel{\thetitle.\quad}


% Definicja nowego stylu strony
\newpagestyle{mypage}
{
  \headrule
  
  \sethead
  { \MakeUppercase{\thesection\quad \sectiontitle} } 
  {}
  {\thesubsection\quad \subsectiontitle}
  
  \setfoot
  {}
  {\thepage}
  {}
}

\newpagestyle{mypage_1}
{
	\headrule
	
	\sethead
	{  }
	{\MakeUppercase{Identyfikacja procesów przemysłowych }}
	{}
	
	\setfoot
	{}
	{}
	{}
}

\settitlemarks{section,subsection,subsubsection}

\pagestyle{mypage_1}

%\marginsize{left}{right}{top}{bottom}
\marginsize{3cm}{3cm}{3cm}{3cm}
\sloppy
\titleformat{\section}
  {\normalfont\Large\bfseries}{\thesection}{1em}{}[{\titlerule[0.8pt]}]
 
 \definecolor{darkred}{rgb}{0.9,0,0}
\definecolor{grey}{rgb}{0.4,0.4,0.4}
\definecolor{orange}{rgb}{1,0.6,0.05}
\definecolor{darkgreen}{rgb}{0.2,0.5,0.05}
 
\lstset{
language=Matlab,
basicstyle=\ttfamily\small,
keywordstyle=\color{darkgreen}\ttfamily\bfseries\small,
stringstyle=\color{red}\ttfamily\small,
commentstyle=\color{grey}\ttfamily\small,
numbers=left,
numberstyle=\color{darkred}\ttfamily\scriptsize,
identifierstyle=\color{blue}\ttfamily\small,
showstringspaces=false,
morekeywords={}
}


\begin{document}

%\thispagestyle{mypage_1}

\begin{table}
\begin{center}
\begin{tabular}{|l|l|l|}
\hline
\multicolumn{3}{|c|}{\textbf{Identyfikacja procesów przemysłowych}} \\ \hline Dominik Wróbel & 27 XI 2018 / 04 XII 2018 & Wt. 13:45, s. 111 \\ \hline
\multicolumn{3}{|c|}{Laboratorium 7} \\ \hline 
\end{tabular}
\end{center}
\end{table}


\section{Przebieg ćwiczenia}

Laboratorium polega na identyfikacji parametrów modelu serwomechanizmu z silnikiem prądu stałego. Rozważany w zadaniu model jest uproszczonym modelem, który otrzymywany jest z równań mechanicznego i elektrycznego po pominięciu indukcyjności.
  
\subsection{Zadanie 1 }
Celem zadania jest wyznaczenia parametrów \( H \) oraz \( K \), których znaczenie wynika z poniższych wzorów uzyskanych z równania wyjściowego. 
Wyjściowe uproszczone równanie serwomechanizmu z silnikiem prądu stałego wyraża się wzorem
\begin{align*}
& \frac{d\omega}{dt} = \frac{k_m}{JR}(u-k_e\omega -\frac{R}{k_m}f_t(\omega)) \\
& k_m - \text{stała mechanicza} \\
& J - \text{ moment bezwładności } \\
& R - \text{rezystancja silnika} \\
& k_e - \text{stała elektryczna} \\
& f_t - \text{funkcja opisująca wypadkowy moment sił oporu}
\end{align*}

Wielkością mierzoną w laboratorium której pomiary użyte są w obliczeniach jest kąt obrotu wału silnika. 

Wprowadzając oznaczenia:
\begin{align*}
& x_1 = \varphi  - \text{kąt obrotu w radianach} \\
& x_2 = \omega
\end{align*}
model przekształcony może zostać do postaci:
\begin{align*}
& \dot{x}_1 = x_2, \ x_1(0)=x_{10} \\
& \dot{x}_2 = K(u-H(x_2)), x_2(0)=x_{20} \\
& H(\omega) = k_e\omega-\frac{R}{k_m} \\
& K = \frac{k_m}{JR}
\end{align*}

Zadanie rozpoczęto od wyznaczenia wartości funkcji H metodą najmniejszych kwadratów. W pierwszej kolejności na podstawie pomiarów wyznaczono zależność obrotów silnika od sterowania w stanie ustalonym. Na jej podstawie przy pomocy metody najmniejszych kwadratów ( \textit{polyfit} ) wyznaczono współczynniki wielomianu drugiego stopnia który przybliża funkcję H:
\begin{align*}
H(\omega) = 
\begin{cases}
a^+\omega^2 + b^+\omega + c^+, & \omega>10^{-2}rad/s \\
a^-\omega^2 + b^-\omega + c^-, & \omega<10^{-2}rad/s \\
0, & poza tym
\end{cases}
\end{align*}

Otrzymane wartości współczynników to:
\begin{align*}
& a^+ = -1,56e-10 \	b^+ = 2,93e-05	\ c^+ = 0,10 \\
& a^- = -1,86	\ b^- = 2,17 \ c^- = -0,042
\end{align*}

Wykres funkcji wraz z naniesionymi wartościami pomiarów przedstawia Rysunek \ref{fig:7_01}.


\begin{figure}[H]
\centerline{\includegraphics[scale=0.5]{7_01.jpg}}
\centering
\caption{Funkcja \( H(\omega) \) wraz z naniesionymi punktami pomiarowymi}
\label{fig:7_01}
\end{figure}

Kolejnym krokiem w rozwiązaniu było wyznaczenie parametru K. Parametr ten wyznaczono minimalizując wskaźnik jakości:
\begin{align*}
Q(K,x_{10},x_{20}) = \frac{1}{2N}\sum _{k=1}^N(x_1(kT_0)-y(kT_0))^2
\end{align*}
Minimalizację przeprowadzono korzystając z funkcji matlaba \textit{lsqnonlin} oraz całkowania metodą stało-krokową Rungego-Kutty. Porównanie wartości prędkości i położenia dla danych z modelu i pomiarów przedstawia Rysunek \ref{fig:7_02}. Uzyskana wartość parametru K = 5.45e+04.

\begin{figure}[H]
\centerline{\includegraphics[scale=0.25]{7_02.jpg}}
\centering
\caption{Porównanie modelu oraz pomiarów dla prędkości i położenia. Kolor czerwony - model.}
\label{fig:7_02}
\end{figure}

W celu porównania, wszystkie wykonane czynności powtórzono dla modelu liniowego:
\begin{align*}
& \dot{x}_1 = x_2 \\
& \dot{x}_2 = -\frac{1}{T}x_2+\frac{k}{T}u
\end{align*}

Otrzymano wartości parametrów T = 2.91e-04 oraz k = 13.85.
\begin{figure}[H]
\centerline{\includegraphics[scale=0.4]{7_04.jpg}}
\centering
\caption{Funkcja \( H(\omega) \) wraz z naniesionymi punktami pomiarowymi - model liniowy}
\label{fig:7_04}
\end{figure}

\begin{figure}[H]
\centerline{\includegraphics[scale=0.25]{7_03.jpg}}
\centering
\caption{Porównanie modelu liniowego oraz pomiarów dla prędkości i położenia. Kolor czerwony - model.}
\label{fig:7_03}
\end{figure}



\subsubsection{Kod programu}
Kod programu prezentuje listing poniżej.
\begin{lstlisting}[caption=Zadanie 1, captionpos=b,label=lis1, firstnumber=1,frame=single]
close all;
clear all;

load('data_01.mat')

timeInterval = 0.01;

for n=1:20
    
    % reading data
    un = data(n).u;
    yn = data(n).y;
    
    % Y Axis - U %
    % last 20 samples 
    un_20 = un(end-19:end);
    % mean of last 20 samples
    un_20Mean = mean(un_20);
    % add mean of last 20 samples to vector 
    uLast20(n) = un_20Mean; 
    
    % X Axis - Velocity %
    
    % diff of samples
    yn_diff = diff(yn);
    % calculating velocity
    %yn_velocity = yn_diff / timeInterval;
    yn_velocity = yn_diff./timeInterval;
    % last 20 samples 
    yn_velocity = yn_velocity(end-20:end);
    
    % mean velocity 
    meanVelocity(n) = mean(yn_velocity);
    
end

figure();
plot( meanVelocity, uLast20, 'o');


% Calculating a+, b+, c+ coef
pPlus = polyfit(meanVelocity(12:20),uLast20(12:20),2)
% Calculating a-, b-, c- coef
pMinus = polyfit(meanVelocity(2:10),uLast20(2:10),2)

xPlus = 0:0.1:50000;
xMinus = -50000:0.1:0;

yPlusResult = pPlus(1)*xPlus.^2 + pPlus(2)*xPlus + pPlus(3);

yMinusResult = pMinus(1)*xMinus.^2 + pMinus(2)*xMinus + pMinus(3);

hold on;
grid on;
xlabel('\omega');
ylabel('U');
plot(xPlus, yPlusResult, 'r');
plot(xMinus, yMinusResult, 'r');

LB=-inf;
UB=inf;
X0=1;  
xopt1=lsqnonlin('cel',X0,LB,UB,[],data(22).u(1:5000),
data(22).t(1:5000),(data(22).y(1:5000))); 
tf=data(22).t(end); 
[t,x]=rk42([0;0],data(22).u,tf,K1); 
\end{lstlisting}

\subsection{Zadanie 2 }

W zadaniu tym identyfikowanym obiektem jest silnik prądu stałego napędzający śmigło. Prędkość obrotowa śmigła jest mierzona za pomocą tachoprądnicy o stałej \( \frac{0.52V}{1000obr.} \). Napięcie jest wzmacniane i mierzone przy pomocy przetwornika A/D. Model silnika i śmigła ma postać:
\begin{align*}
& \dot{x}=K(u-H(x)), \ x(0)=x_0
\end{align*}

Zależność pomiędzy prędkością obrotową śmigła, a wyjściem przetwornika A/D ma postać:
\begin{align*}
\omega = au + b
\end{align*}
W pierwszej kolejności wyznaczono parametry a i b prostej. Prędkość otrzymano korzystając ze zmierzonego napięcia oraz stałej tachoprądnicy. Następnie skorzystano w metody najmniejszych kwadratów w celu wyznaczenia parametrów a oraz b. Otrzymano wartości a = 319,59 oraz b = -20,08. Prostą wraz z pomiarami prezentuje Rysunek \ref{fig:7_05}.
\begin{figure}[H]
\centerline{\includegraphics[scale=0.5]{7_05.jpg}}
\centering
\caption{Prosta \( \omega = au + b \)}
\label{fig:7_05}
\end{figure}

Kolejną czynnością było wyznaczenie współczynników funkcji \( H(\omega) \). W tym zadaniu funkcja ta jest modelowana wielomianem 3 stopnia:
\begin{align*}
& H(\omega)=az^3+bz^2+cz+d \\
& z = 10^{-3}\omega
\end{align*}
W tym celu ponownie skorzystano z metody najmniejszych kwadratów oraz wcześniej wyznaczonej prędkości. Otrzymano następujące wartości współczynników:
\begin{align*}
& a = 0,022 \\
& b = 0,00027 \\
& c = 0,149 \\
& d = 0,00018 \\
\end{align*}

Otrzymaną funkcję przedstawia Rysunek \ref{fig:7_06}.
\begin{figure}[H]
\centerline{\includegraphics[scale=0.5]{7_06.jpg}}
\centering
\caption{Prosta \( H(\omega) \)}
\label{fig:7_06}
\end{figure}

Następnie wyznaczono parametr K metodą opisaną w Zadaniu 1. Otrzymano wartość parametru k równą 7,064. Porównanie modelu z pomiarami przedstawia Rysunek \ref{fig:7_07}.

\begin{figure}[H]
\centerline{\includegraphics[scale=0.45]{7_07.jpg}}
\centering
\caption{Prosta \( Pomiar - kolor czerwony, Model - kolor niebieski. \)}
\label{fig:7_07}
\end{figure}


\subsubsection{Kod programu}
Kod programu prezentuje listing poniżej.
\begin{lstlisting}[caption=Zadanie 2, captionpos=b,label=lis1, firstnumber=1,frame=single]
clear all;
close all;

load('data_01_ident.mat')

% --- READING DATA ---- % 
u = [-1.560 -1.490 -1.375 -1.321 -1.204 -1.087 -0.945 -0.783 
-0.587 -0.347 0 0.350 0.583 0.765 0.924 1.054 1.172 1.263 1.347 1.429 1.493];
u_ad = [-9.5 -8.9 -8.3 -7.6 -7.1 -6.4 -5.6 -4.6 -3.5 -2.1 0 2.1 
3.4 4.6 5.6 6.4 7.1 7.7 8.2 8.8 9.1];


% --- CALCULATING a and b --- %
scale = 0.52 / 1000;
velocity = u ./ scale;
abCoefs = polyfit(u_ad,velocity,1);
% --- CALCULATING f(u_ad) = Omega --- %
Hw = control;
figure();
hold on;
grid;
xlabel('u_{ad}');
ylabel('\omega')
plot(u_ad,velocity,'or');
plot(u_ad, polyval(abCoefs, u_ad),'-');
hold off;


% --- CALCULATING H(w) = f(w) --- %
% --- CALCULATING a, b, c and d H(w) --- %
wH = velocity;
wH = 10^-3 .* wH;
abcdCoefs = polyfit(wH,control,4);

figure()
hold on;
grid;
xlabel('\omega');
ylabel('H(\omega)');
plot(wH, control, 'or');
plot(wH, polyval(abcdCoefs, wH),'-');
hold off;


load('data_02_ident.mat')


figure()
wID=polyval(velocity,u_ad)/1000;
hold on;

plot(t,u,'g');
plot(t,wID,'r');


LB = [-100 -100]; 
UB = [100 100]; 
x0 = 3;
K = 7;
Kopt = lsqnonlin('cel', [x0 K]', LB, UB, [], u, t, wID, abcdCoefs);

\end{lstlisting}



\section{Wnioski końcowe}
W badanym zagadnieniu zgodnie z oczekiwaniami sprawdziły się modele wyższego rzędu.  Ważnym i ciekawym aspektem rozważanego zagadnienia było modelowanie funkcji na dwóch rozłącznych przedziałach. Takie podejście daje przewagę nad rozważaniem jednego przedziału w przypadku gdy funkcja jest silnie nieliniowa w danym przedziale. Między innymi z tego też powodu funkcja liniowa dała gorsze rezultaty w Zadaniu 1.

 
\end{document}