\documentclass[a4paper,15pt]{article}
\usepackage{amssymb}
\usepackage{amsmath}
\usepackage[english, polish]{babel}
\usepackage[utf8]{inputenc}   % lub utf8
\usepackage[T1]{fontenc}
\usepackage{graphicx}
\usepackage{anysize}
\usepackage{enumerate}
\usepackage{times}
\usepackage{titlesec}
\usepackage{float}
\usepackage{titleps,kantlipsum}
\usepackage{listings}
\usepackage{color}
\usepackage{hyperref}
\lstloadlanguages{Matlab}
 
\usepackage[justification=centering]{caption}
\titlelabel{\thetitle.\quad}


% Definicja nowego stylu strony
\newpagestyle{mypage}
{
  \headrule
  
  \sethead
  { \MakeUppercase{\thesection\quad \sectiontitle} } 
  {}
  {\thesubsection\quad \subsectiontitle}
  
  \setfoot
  {}
  {\thepage}
  {}
}

\newpagestyle{mypage_1}
{
	\headrule
	
	\sethead
	{  }
	{\MakeUppercase{Identyfikacja procesów przemysłowych }}
	{}
	
	\setfoot
	{}
	{}
	{}
}

\settitlemarks{section,subsection,subsubsection}

\pagestyle{mypage_1}

%\marginsize{left}{right}{top}{bottom}
\marginsize{3cm}{3cm}{3cm}{3cm}
\sloppy
\titleformat{\section}
  {\normalfont\Large\bfseries}{\thesection}{1em}{}[{\titlerule[0.8pt]}]
 
 \definecolor{darkred}{rgb}{0.9,0,0}
\definecolor{grey}{rgb}{0.4,0.4,0.4}
\definecolor{orange}{rgb}{1,0.6,0.05}
\definecolor{darkgreen}{rgb}{0.2,0.5,0.05}
 
\lstset{
language=Matlab,
basicstyle=\ttfamily\small,
keywordstyle=\color{darkgreen}\ttfamily\bfseries\small,
stringstyle=\color{red}\ttfamily\small,
commentstyle=\color{grey}\ttfamily\small,
numbers=left,
numberstyle=\color{darkred}\ttfamily\scriptsize,
identifierstyle=\color{blue}\ttfamily\small,
showstringspaces=false,
morekeywords={}
}


\begin{document}

%\thispagestyle{mypage_1}

\begin{table}
\begin{center}
\begin{tabular}{|l|l|l|}
\hline
\multicolumn{3}{|c|}{\textbf{Identyfikacja procesów przemysłowych}} \\ \hline Dominik Wróbel & 23 X 2018 & Wt. 13:45, s. 111 \\ \hline
\multicolumn{3}{|c|}{Laboratorium 2} \\ \hline 
\end{tabular}
\end{center}
\end{table}


\section{Przebieg ćwiczenia}
\subsection{Zadanie 1 }

Zadanie polega na wyznaczeniu parametrów obiektu opisanego transmitancją 
\begin{align*}
& G(s)=\frac{k}{Ts+1}
\end{align*}
który pobudzany jest białym szumem Gaussa o wariancji \( \sigma ^2 = 1 \). Częstotliwość próbkowania jest równa \( f_0 = 1kHz \). Sygnałem mierzonym jest sygnał wyjściowy układu. Parametry wyznaczone zostaną na podstawie funkcji autokorelacji oraz widma mocy sygnału wyjściowego. 

Zadanie rozpoczęto od wyznaczenia teoretycznego widma mocy sygnału. 
\begin{align*}
& S(\omega) = |G(i\omega)|^2 \\
& |G(i\omega)|^2 = \frac{|k|^2}{|T\omega j + 1|^2} = \frac{\sqrt{k^2}^2}{\sqrt{T^2\omega ^2 + 1}^2} = \frac{k^2}{1+T^2\omega ^2}
\end{align*}

Następnie przy użyciu matlaba narysowano widmo mocy na podstawie pomiarów z zadania przy użyciu estymatora Yule’a-Walker’a . Na podstawie tych danych wybrano dwa punktu z widma na podstawie których wyznaczone zostały parametry k oraz T modelu. 
\begin{align*}
& 10\log _{10} S(\omega) = 10 \log _{10} \frac{k^2}{1+T^2 \omega ^2} \\
& S(\omega ) = \frac{k^2}{T^2 \omega ^2 + 1} \\
& \omega = 0 \rightarrow \sqrt{S(\omega)} = \frac{k}{1+0} = k \\
& \omega = 1 \rightarrow S(\omega) = \frac{k^2}{1+T^2} \rightarrow T = \sqrt{\frac{k^2}{S(\omega)}-1}
\end{align*}
Z danych otrzymanych metodą Yule’a-Walker’a odczytano wyznaczone wartości widma dla częstotliwości \( \omega = 0 \) oraz \( \omega = 1 \).
\begin{align*}
& \omega = 0, \quad S(\omega) = 0,0084 \\
& \omega = 1,0071, \quad S(\omega) = 0,00012985 \\
\end{align*}
Podstawiając te wartości do wzorów na wartość k oraz T otrzymano:
\begin{align*}
& k = 0,0917 \\
& T = 7,9811
\end{align*}
Dla wyznaczonych parametrów k oraz T narysowano na jednym wykresie estymowane oraz teoretyczne widmo mocy. Wykresy przedstawia Rysunek \ref{fig:01_widmo}

\begin{figure}[H]
\centerline{\includegraphics[scale=0.47]{01_widmo.jpg}}
\centering
\caption{Teoretyczne ( kolor czerwony ) oraz estymowane ( kolor niebieski ) widmo mocy.}
\label{fig:01_widmo}
\end{figure}

W zadaniu wyznaczona została również funkcja autokorelacji badanego sygnału, została ona narysowana przy użyciu funkcji matlaba xcorr.

\begin{figure}[H]
\centerline{\includegraphics[scale=0.47]{02_auto.jpg}}
\centering
\caption{Funkcja autokorelacji}
\label{fig:02_auto}
\end{figure}

Kod programu prezentuje listing poniżej.
\begin{lstlisting}[caption=Zadanie 1, captionpos=b,label=lis1, firstnumber=1,frame=single]
close all;
clear all;
load('data_01.mat');
Fs = 1000; % czestotliwosc probkowania w Hz
% x - sygnal mierzony

h = spectrum.yulear;
Hpsd = psd(h,x,'Fs',Fs);
values = Hpsd.Data;
freq = Hpsd.Frequencies;

semilogy(freq, values);


k = sqrt(values(1));
T = sqrt(k^2/values(34) - 1);

w=0:0.01:500;
s = k^2 ./ ( 1 + T^2*w.^2 );

hold on;
% wyrysowanie teoretycznego widma mocy
semilogy(w, s, 'r');

% wyrysowanie funkcji autokorelacji
figure();
plot(xcorr(x));

\end{lstlisting}


\newpage
\subsection{Zadanie 2 }
Zadanie ma takie same założenia jak zadanie numer 1, zmianie ulega tylko badany obiekt. W zadaniu tym przyjęto tą samą metodykę wyznaczania parametrów obiektu, co w zadaniu numer 1, tym razem badany jest obiekt o transmitancji:
\begin{align*}
G(s) = \frac{k}{s^2+2\xi \omega _0 s + \omega _0 ^2}
\end{align*}

Zadanie rozpoczęto od wyznaczenia teoretycznego widma mocy sygnału. 
\begin{align*}
& S(\omega) = |G(i\omega)|^2 \\
& G(i\omega) = \frac{k}{i^2\omega ^2 + 2\xi \omega _0 \omega i + \omega _0^2} =
\frac{k}{-\omega ^2 + 2\xi\omega _0 \omega i + \omega _0^2} = \\
& = \frac{k}{\omega _0^2 - \omega^2 + 2\xi\omega _0 \omega i} \\
& |G(i\omega)| = \frac{\sqrt{k^2}}{\sqrt{(\omega _0 ^2 - \omega ^2) ^2 + (2\xi\omega _0 \omega)^2}} \\
& |G(i\omega)|^2 = \frac{k^2}{(\omega _0 - \omega ^2 ) ^2 + 4\xi ^2 \omega _0 ^2 \omega ^2}
\end{align*}

W zadaniu częstotliwość własna \( \omega _0 \) została wyznaczona w oparciu o analizę estymowanego widma mocy sygnału. Odpowiada ona częstotliwości dla której widmo osiąga największą wartość. Estymowane widmo przedstawia Rysunek \ref{fig:03_widmo}. 

\begin{figure}[H]
\centerline{\includegraphics[scale=0.47]{03_widmo.jpg}}
\centering
\caption{Estymowane widmo mocy sygnału.}
\label{fig:03_widmo}
\end{figure}

Znaleziona przy pomocy programu wartość \( \omega _0 \) jest równa  246.6125 Hz.


\begin{align*}
& 10\log _{10} S(\omega) = 10 \log _{10} \frac{k^2}{(\omega _0 - \omega ^2 ) ^2 + 4\xi ^2 \omega _0 ^2 \omega ^2} \\
& S(\omega ) = \frac{k^2}{(\omega _0 - \omega ^2 ) ^2 + 4\xi ^2 \omega _0 ^2 \omega ^2} \\
& \omega = 0 \rightarrow k =  \omega _0^2 \sqrt{S(\omega)} \\
& \xi = \sqrt{\frac{k^2 - S(\omega)(\omega _0^2 - \omega^2)^2}{4S(\omega)(\omega _0^2)\omega^2}}
\end{align*}
Z danych otrzymanych metodą Yule’a-Walker’a odczytano wyznaczone wartości widma dla częstotliwości \( \omega = 0 \) oraz \( \omega = 250 \).
Podstawiając te wartości do wzorów na wartość k oraz \( \xi \) otrzymano:
\begin{align*}
& k = 0.022124 \\
& \xi = 0.026941
\end{align*}

Następnie narysowano widmo teoretyczne i estymowane na jednym wykresie w celu ich porównania.
\begin{figure}[H]
\centerline{\includegraphics[scale=0.47]{04_widmo.jpg}}
\centering
\caption{Estymowane ( kolor niebieski ) oraz teoretyczne ( kolor czerwony ) widmo mocy sygnału.}
\label{fig:04_widmo}
\end{figure}


Kod programu prezentuje listing poniżej.
\begin{lstlisting}[caption=Zadanie 2, captionpos=b,label=lis1, firstnumber=1,frame=single]
close all;
clear all;
load('data_02.mat');
Fs = 1000; % czestotliwosc probkowania w Hz
% x - sygnal mierzony

h = spectrum.yulear;
Hpsd = psd(h,x,'Fs',Fs);

values = Hpsd.Data;
freq = Hpsd.Frequencies;

semilogy(freq, values);

% wyznacznie w0
w0 = freq(find(values == max(values)));
k = w0^2 * sqrt(values(1));
licz = k^2 -  values(8193)*( w0^2 - freq(8193)^2 )^2 ;
mian = 4 * values(8193) * w0^2 * freq(8193)^2;
ksi = sqrt(licz/mian);

w=0:0.01:500;
s = (k^2)./( (w0^2 - w.^2).^2 + 4*ksi^2*w0^2*w.^2 );

figure()
semilogy(freq, values);
hold on;
% wyrysowanie teoretycznego widma mocy
semilogy(w, s, 'r');




\end{lstlisting}



\section{Wnioski końcowe}
Z przeprowadzonych eksperymentów wynika, że przyjęta metodyka wyznaczania parametrów modeli daje zadowalające rezultaty. Ograniczeniem tej metody jest jednak zakres częstotliwości w których widmo estymowane jest dobrze dopasowane do widma teoretycznego. Zarówno w zadaniu 1 jak i w zadaniu 2 najlepsze dopasowanie osiągnięto w okolicach częstotliwości dla których odczytane zostały punkty pomiarowe, a wiec dla \mbox{ \( \omega = 1 \) dla zadania 1 oraz \( \omega = 250 \) dla zadania  2.} 

 
\end{document}