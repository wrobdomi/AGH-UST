\documentclass[a4paper,15pt]{article}
\usepackage{amssymb}
\usepackage{amsmath}
\usepackage[english, polish]{babel}
\usepackage[utf8]{inputenc}   % lub utf8
\usepackage[T1]{fontenc}
\usepackage{graphicx}
\usepackage{anysize}
\usepackage{enumerate}
\usepackage{times}
\usepackage{titlesec}
\usepackage{float}
\usepackage{titleps,kantlipsum}
\usepackage{listings}
\usepackage{color}
\usepackage{hyperref}
\lstloadlanguages{Matlab}
 
\usepackage[justification=centering]{caption}
\titlelabel{\thetitle.\quad}


% Definicja nowego stylu strony
\newpagestyle{mypage}
{
  \headrule
  
  \sethead
  { \MakeUppercase{\thesection\quad \sectiontitle} } 
  {}
  {\thesubsection\quad \subsectiontitle}
  
  \setfoot
  {}
  {\thepage}
  {}
}

\newpagestyle{mypage_1}
{
	\headrule
	
	\sethead
	{  }
	{\MakeUppercase{Identyfikacja procesów przemysłowych }}
	{}
	
	\setfoot
	{}
	{}
	{}
}

\settitlemarks{section,subsection,subsubsection}

\pagestyle{mypage_1}

%\marginsize{left}{right}{top}{bottom}
\marginsize{3cm}{3cm}{3cm}{3cm}
\sloppy
\titleformat{\section}
  {\normalfont\Large\bfseries}{\thesection}{1em}{}[{\titlerule[0.8pt]}]
 
 \definecolor{darkred}{rgb}{0.9,0,0}
\definecolor{grey}{rgb}{0.4,0.4,0.4}
\definecolor{orange}{rgb}{1,0.6,0.05}
\definecolor{darkgreen}{rgb}{0.2,0.5,0.05}
 
\lstset{
language=Matlab,
basicstyle=\ttfamily\small,
keywordstyle=\color{darkgreen}\ttfamily\bfseries\small,
stringstyle=\color{red}\ttfamily\small,
commentstyle=\color{grey}\ttfamily\small,
numbers=left,
numberstyle=\color{darkred}\ttfamily\scriptsize,
identifierstyle=\color{blue}\ttfamily\small,
showstringspaces=false,
morekeywords={}
}


\begin{document}

%\thispagestyle{mypage_1}

\begin{table}
\begin{center}
\begin{tabular}{|l|l|l|}
\hline
\multicolumn{3}{|c|}{\textbf{Identyfikacja procesów przemysłowych}} \\ \hline Dominik Wróbel & 13 XI 2018 & Wt. 13:45, s. 111 \\ \hline
\multicolumn{3}{|c|}{Laboratorium 5} \\ \hline 
\end{tabular}
\end{center}
\end{table}


\section{Przebieg ćwiczenia}

Laboratorium polega na identyfikacji parametrów trzech różnych zbiorników z wodą. Zbiorniki różnią się pomiędzy sobą kształtem i wymiarami. Równania opisujące wypływ wody ze zbiorników to równania Torricellego. W zadaniach zostaną zbadanego dwa różne modele, pozwoli to na porównanie ich dokładności. Modele zostaną również zastosowane do porównania ich z danymi rzeczywistymi z eksperymentu. 
  
\subsection{Zadanie 1 }

W zadaniu pierwszym równanie Torricellego mają stałą potęgą po prawej stronie równania równą \( \frac{1}{2} \).

\begin{align*}
& \dot{x_1}= - \frac{c_1\sqrt{x_1}}{S_1}, \ \dot{x_2}=-\frac{c_2\sqrt{x_2}}{S_2(x_2)}, \ \dot{x_3} = - \frac{c_3\sqrt{x_3}}{S_3(x_3)}, \ x_i(0) = x_{0,i}, \ i=1,2,3, \\
& S_1 = cw, \ S_2(x_2) = w(\frac{b-a}{H}x_2+a), \ S_3(x_3)=w\sqrt{2Rx_3-x_3^2} \\
& a=10cm, \ b=44,5cm, \ c=25, \ w=3,5cm, \ H=35cm, \ R=36,4cm
\end{align*}

Identyfikacji podlegają parametry \( c_i \) każdego z równań. Wyznaczane są także poziomy początkowe cieczy. Okres próbkowania wynosi \( T=0,01s \). Do rozwiązania obu zadań zastosowano minimalizację funkcji 
\begin{equation*}
Q(c_i,\alpha _i, x_{0,i} ) =\frac{1}{2}\sum _{k=1}^{N} (x_k(kT_0, c_i,\alpha _i, x_{0,i}) - y_i(kT_0))^2 \rightarrow min
\end{equation*}
z parametrami \( c_i, \alpha _i, x_{0,i} \). Minimalizacja została wykonana przy pomocy funkcji \textit{lsqnonlin} matlaba. 


Otrzymane wartości parametrów oraz porównanie modelu z danymi z eksperymentu przedstawiają rysunki poniżej. Dane z eksperymentów zostały odpowiednio skrócone w celu uwzględnienie tylko właściwych pomiarów.

\begin{figure}[H]
\centerline{\includegraphics[scale=0.5]{05_pierwszy_zad1.jpg}}
\centering
\caption{Porównanie modelu oraz wyników eksperymentu - Zbiornik I. Parametry:
\( c_1 = 19,01, \ x_0 = 30,34, \alpha = 0,5 \)}
\label{fig:05_pierwszy_zad1}
\end{figure}

\begin{figure}[H]
\centerline{\includegraphics[scale=0.5]{05_pierwszy_zad1_hist.jpg}}
\centering
\caption{Różnice pomiędzy modelem a eksperymentem - Zbiornik I}
\label{fig:05_pierwszy_zad1_hist}
\end{figure}

\begin{figure}[H]
\centerline{\includegraphics[scale=0.5]{03_drugi_zad1.jpg}}
\centering
\caption{Porównanie modelu oraz wyników eksperymentu - Zbiornik II. Parametry:
\( c_2 = 22,40, \ x_0 = 34,05, \alpha = 0,5 \)}
\label{fig:03_drugi_zad1}
\end{figure}

\begin{figure}[H]
\centerline{\includegraphics[scale=0.5]{03_drugi_zad1_hist.jpg}}
\centering
\caption{Różnice pomiędzy modelem a eksperymentem - Zbiornik II}
\label{fig:03_drugi_zad1_hist}
\end{figure}


\begin{figure}[H]
\centerline{\includegraphics[scale=0.5]{04_trzeci_zad1.jpg}}
\centering
\caption{Porównanie modelu oraz wyników eksperymentu - Zbiornik III. Parametry:
\( c_3 = 17,57 , \ x_0 = 35,16, \alpha = 0,5 \)}
\label{fig:04_trzeci_zad1}
\end{figure}

\begin{figure}[H]
\centerline{\includegraphics[scale=0.5]{04_trzeci_zad1_hist.jpg}}
\centering
\caption{Różnice pomiędzy modelem a eksperymentem - Zbiornik III}
\label{fig:04_trzeci_zad1_hist}
\end{figure}

Przy pomocy testu chi kwadrat ( funkcja matlaba \textit{chi2gof} ) sprawdzono jakość dopasowania na poziomie istotności 0,001. Dla każdego ze zbiorników test ten dał negatywny wynik. 




\subsubsection{Kod programu}
Kod programu prezentuje listing poniżej.
\begin{lstlisting}[caption=Zadanie 1, captionpos=b,label=lis1, firstnumber=1,frame=single]
clear all;
close all;

load('data_01.mat');

% dane wejsciowe
figure();
plot(t1, x1, 'r');
hold on;
plot(t2, x2, 'g');
hold on;
plot(t3, x3, 'b');

% obcinanie danych
x1 = x1(186:4050);
t1 = t1(186:4050);
x2 = x2(186:3550);
t2 = t2(186:3550);
x3 = x3(230:5093);
t3 = t3(230:5093);

% dane po obrobce
figure();
plot(t1, x1, 'r');
hold on;
plot(t2, x2, 'g');
hold on;
plot(t3, x3, 'b');

% parametry zbiornikow 
a = 10;
b = 44.5;
c = 25;
w = 3.5;
H = 35;
R = 36.4;

T0 = 0.01;

S1 = c*w;


LB = [0.1 0.2 30]';
UB = [200 0.5 40]';
% punkt startowy 
c=30; al=0.3; x0 = x2(1);
xopt = lsqnonlin('cel',[c al x0]', LB, UB, [], t2, x2);
% model - zbiornik 2
[t2mod,x2mod] = ode45(@(t,x2m) -1 * xopt(1) * x2m^(1/2) / 
( w * ( b - a ) / H * x2m + a*w ), t2, xopt(3));
figure()
plot(t2mod,x2mod, 'r');
hold on;
plot(t2, x2, 'b');
title('Drugi zbiornik');
figure();
histfit(x2-x2mod);
test2 = chi2gof(x2-x2mod, 'alpha', 0.01);

c=30; al=0.3; x0 = x1(1);
xopt = lsqnonlin('celOne',[c al x0]', LB, UB, [], t1, x1);
% model - zbiornik 1 
[t1mod,x1mod] = ode45(@(t,x1m) -1 * xopt(1) * x1m^(1/2) / S1, t1, xopt(3));
figure()
plot(t1mod,x1mod, 'r');
hold on;
plot(t1, x1, 'b');
title('Pierwszy zbiornik');
figure();
histfit(x1-x1mod);
test1 = chi2gof(x1-x1mod, 'alpha', 0.01);


c=30; al=0.3; x0 = x3(1);
xopt = lsqnonlin('celThree',[c al x0]', LB, UB, [], t3, x0);
% model - zbiornik 3 
[t3mod,x3mod] = ode45(@(t,x3m) -1 * xopt(1) * x3m^(1/2) / 
( w * sqrt(2*R*x3m-x3m^2)), t3, xopt(3));
figure()
plot(t3mod,x3mod, 'r');
hold on;
plot(t3, x3, 'b');
title('Trzeci zbiornik');
figure();
histfit(x3-x3mod);
test3 = chi2gof(x3-x3mod, 'alpha', 0.01);
\end{lstlisting}

\subsection{Zadanie 2 }

W zadaniu drugim badane są te same zbiorniki. Przyjęto tą samą metodykę do identyfikacji parametrów modelu. W tym zadaniu jednak identyfikowany jest jeden parametr więcej, jest nim wykładnik \( \alpha \).

Otrzymane wartości parametrów oraz porównanie modelu z danymi z eksperymentu przedstawiają rysunki poniżej.

\begin{figure}[H]
\centerline{\includegraphics[scale=0.5]{09_pierwszy_zad2.jpg}}
\centering
\caption{Porównanie modelu oraz wyników eksperymentu - Zbiornik I. Parametry:
\( c_1 = 30,06, \ x_0 = 30,00, \alpha = 0,33 \)}
\label{fig:05_pierwszy_zad1}
\end{figure}

\begin{figure}[H]
\centerline{\includegraphics[scale=0.5]{09_pierwszy_zad2_hist.jpg}}
\centering
\caption{Różnice pomiędzy modelem a eksperymentem - Zbiornik I}
\label{fig:05_pierwszy_zad1_hist}
\end{figure}

\begin{figure}[H]
\centerline{\includegraphics[scale=0.5]{08_drugi_zad2.jpg}}
\centering
\caption{Porównanie modelu oraz wyników eksperymentu - Zbiornik II. Parametry:
\( c_2 = 40,37, \ x_0 = 33,19, \alpha = 0,29 \)}
\label{fig:03_drugi_zad1}
\end{figure}

\begin{figure}[H]
\centerline{\includegraphics[scale=0.5]{08_drugi_zad2_hist.jpg}}
\centering
\caption{Różnice pomiędzy modelem a eksperymentem - Zbiornik II}
\label{fig:03_drugi_zad1_hist}
\end{figure}


\begin{figure}[H]
\centerline{\includegraphics[scale=0.5]{10_trzeci_zad2.jpg}}
\centering
\caption{Porównanie modelu oraz wyników eksperymentu - Zbiornik III. Parametry:
\( c_3 = 32.90 , \ x_0 = 33.15, \alpha = 0.28 \)}
\label{fig:04_trzeci_zad1}
\end{figure}

\begin{figure}[H]
\centerline{\includegraphics[scale=0.5]{10_trzeci_zad2_hist.jpg}}
\centering
\caption{Różnice pomiędzy modelem a eksperymentem - Zbiornik III}
\label{fig:04_trzeci_zad1_hist}
\end{figure}

Przy pomocy testu chi kwadrat ( funkcja matlaba \textit{chi2gof} ) sprawdzono jakość dopasowania na poziomie istotności 0,001. Dla każdego ze zbiorników test ten dał negatywny wynik. 





\subsubsection{Kod programu}
Kod programu prezentuje listing poniżej.
\begin{lstlisting}[caption=Zadanie 2, captionpos=b,label=lis1, firstnumber=1,frame=single]
clear all;
close all;

load('data_01.mat');

% dane wejsciowe
figure();
plot(t1, x1, 'r');
hold on;
plot(t2, x2, 'g');
hold on;
plot(t3, x3, 'b');

% obcinanie danych
x1 = x1(186:4050);
t1 = t1(186:4050);
x2 = x2(186:3550);
t2 = t2(186:3550);
x3 = x3(230:4672);
t3 = t3(230:4672);

% dane po obrobce
figure();
plot(t1, x1, 'r');
hold on;
plot(t2, x2, 'g');
hold on;
plot(t3, x3, 'b');

% parametry zbiornikow 
a = 10;
b = 44.5;
c = 25;
w = 3.5;
H = 35;
R = 36.4;

T0 = 0.01;

S1 = c*w;


LB = [0.1 0.2 30]';
UB = [200 0.5 40]';
c=30; al=0.3; x0 = x2(1);
xopt = lsqnonlin('cel',[c al x0]', LB, UB, [], t2, x2);
zbiornik2 = xopt;
[t2mod,x2mod] = ode45(@(t,x2m) -1 * xopt(1) * x2m^(xopt(2)) / 
( w * ( b - a ) / H * x2m + a*w ), t2, xopt(3));
figure()
plot(t2, x2, 'b');
hold on;
plot(t2mod,x2mod, 'r','LineWidth',2);
title('Drugi zbiornik');
figure();
histfit(x2-x2mod);
test2 = chi2gof(x2-x2mod, 'alpha', 0.001);

c=30; al=0.3; x0 = x1(1);
xopt = lsqnonlin('celOne',[c al x0]', LB, UB, [], t1, x1);
zbiornik1 = xopt; 
[t1mod,x1mod] = ode45(@(t,x1m) -1 * xopt(1) * x1m^(xopt(2)) / S1, t1, xopt(3));
figure()
plot(t1, x1, 'b');
hold on;
plot(t1mod,x1mod, 'r','LineWidth',2);
title('Pierwszy zbiornik');
figure();
histfit(x1-x1mod);
test1 = chi2gof(x1-x1mod, 'alpha', 0.001);


c=30; al=0.3; x0 = x3(1);
xopt = lsqnonlin('celThree',[c al x0]', LB, UB, [], t3, x0);
zbiornik3 = xopt;
[t3mod,x3mod] = ode45(@(t,x3m) -1 * xopt(1) * x3m^(xopt(2)) / 
( w * sqrt(2*R*x3m-x3m^2)), t3, xopt(3));
figure()
plot(t3, x3, 'b');
hold on;
plot(t3mod,x3mod, 'r');
title('Trzeci zbiornik');
figure();
histfit(x3-x3mod);
test3 = chi2gof(x3-x3mod, 'alpha', 0.001);

\end{lstlisting}



\section{Wnioski końcowe}
W badanym zagadnieniu lepiej sprawdził się model zastosowany w zadaniu 2, uwzględniający więcej parametrów. Model uwzględniający większą liczbę parametrów procesu fizycznego sprawdza się zazwyczaj lepiej, ponieważ z większą dokładnością oddaje rzeczywisty proces. Mimo to żadnemu ze zidentyfikowanych modeli nie udało się uzyskać zgodności z danymi na poziomie 0,001 testu chi kwadrat. Warto również zauważyć, że na przebieg eksperymentu miały wpływ zakłócenia, co ma negatywny wpływ na dokładność danych pomiarowych.

 
\end{document}