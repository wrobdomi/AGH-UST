\documentclass{standalone}
\usepackage[T1]{fontenc}
\usepackage[latin2]{inputenc}
\usepackage[english]{babel}
\usepackage{tikz}
\usepackage{times}

% Packages needed to draw 
\usetikzlibrary{calc,through,backgrounds,positioning,fit}
\usetikzlibrary{shapes,arrows,shadows}
 
\begin{document}
 
% Other things needed to draw
\tikzstyle{place}=[shape=circle, draw, minimum height=10mm]
\tikzstyle{trig}=[shape=circle, draw, dashed, minimum height=10mm]
\tikzstyle{trans}=[shape=rectangle, draw, minimum height=6mm, minimum width=12mm]
 
\centering

%OPTIONS of tikzpicture:
% scale option scales the whole picture 


%GENERAL TIPS
% cordinate system starts from left bottom corner

\begin{tikzpicture}[scale=2,inner sep=0.4mm]
%\node (e1) [place,label=below left:{$e_1$}] at (3,0) {$f(e_1)$};


% DRAWING LINES %%%%%%%%%%%%%%%%%%%%%%%%%%%%%%%%%%%%%%%%%%%%%%%%%%%%%%%%

% Drawing a straight line 
% Specify points and put -- between them
\draw (0.5,0) -- (3,2); % Draw a line from point (0,0) to (1,0)

% Drawing a broken line
% Specify points and put -- between them
\draw (0,0) -- (2,2.5) -- (5,2.5) -- (1,0);

% Drawing a curve line
% Specify point, after a point put .. and controls point, 
% then .. or last point and so on
\draw (0,0) .. controls(1,1) .. (2,3) .. controls(3,6) .. (3,0);

% Drawing help lines from point (0,0) to point (2,3)
% These are gray lines on the picture
\draw[help lines] (0,0) grid (2,3);

% Adding arrows to the lines is simple...
\draw [->] (0,6) -- (6,5);
\draw [<-] (0,4) -- (5,3);
\draw [|->] (0,3) -- (6,1);

% Adding options to lines

% width
\draw [thick] (0,2) -- (6,3);
\draw [ultra thick] (0,1) -- (6,2);

% type of line & colors
\draw [dashed, green] (4,1) -- (2,0.5);
\draw [dotted, ultra thick, red] (0,2) -- (3.5,5);


% DRAWING SHAPES %%%%%%%%%%%%%%%%%%%%%%%%%%%%%%%%%%%%%%%%%%%%%%%%%%%%%%%

%Drawing different shapes

%Circle - specify middle and radius
\draw (0,0) circle (7pt);
%Ellipse - specify middle and axes
\draw (5,5) ellipse (20pt and 7pt);
%Rectangle - Square - specify bottom left point first and up right corner next
\draw (0,0) rectangle (5,5);


% NODES %%%%%%%%%%%%%%%%%%%%%%%%%%%%%%%%%%%%%%%%%%%%%%%%%%%%%%%%%%%%%%%%%

% specify name of node i brackets ( needed to appeal to node )
% then specify its center, then set its shape in options
% and put draw to draw node, put string in curly brackets 
% to be put into node 
\node (name_a) at (3.5,3.5) [shape=circle, draw] {$x_1$};
\node (b) at (5.5,5.5) [shape=rectangle, draw] {$x_2$};

% Drawing connections between nodes
% We simply draw lines/arrows like we did with drawing lines, but 
% this time we use names of nodes instead of points, and we specify
% how to reach the node
\draw [->] (name_a.east) -- (b.center);

% Using 'to' to draw connections between nodes allows us to specify angle
% of input and output
\draw [->] (name_a) to [out=135, in=45] (b);


% Adding label to node
\node (c) [place, label=below left:{$e_1$}] at (2,5) [shape=rectangle] {node}; 
   


\end{tikzpicture}
 
\end{document}