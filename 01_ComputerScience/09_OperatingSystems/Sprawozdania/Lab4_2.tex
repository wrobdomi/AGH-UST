\documentclass[a4paper,15pt]{article}
\usepackage{amssymb}
\usepackage{amsmath}
\usepackage[english, polish]{babel}
\usepackage[utf8]{inputenc}   % lub utf8
\usepackage[T1]{fontenc}
\usepackage{graphicx}
\usepackage{anysize}
\usepackage{enumerate}
\usepackage{times}
\usepackage{caption}
\usepackage{titlesec}
\usepackage{float}
\usepackage{titleps,kantlipsum}
\usepackage{listings}
\usepackage{xcolor}
\usepackage{hyperref}
\usepackage{framed}
\usepackage{tcolorbox}
\lstloadlanguages{Matlab}
 
\usepackage[justification=centering]{caption}
\titlelabel{\thetitle.\quad}

\pagenumbering{arabic}

\DeclareCaptionFont{white}{\color{white}}
\DeclareCaptionFormat{listing}{%
  \parbox{\textwidth}{\colorbox{darkgreen}{\parbox{\textwidth}{#1#2#3}}\vskip-4pt}}
\captionsetup[lstlisting]{format=listing,labelfont=white,textfont=white}
\lstset{frame=lrb,xleftmargin=\fboxsep,xrightmargin=-\fboxsep}

% Definicja nowego stylu strony
\newpagestyle{mypage}
{
  \headrule
  
  \sethead
  { \MakeUppercase{\thesection\quad \sectiontitle} } 
  {}
  {\thesubsection\quad \subsectiontitle}
  
  \setfoot
  {}
  {}
  {\thepage}
}

\newpagestyle{mypage_1}
{
	\headrule
	
	\sethead
	{  }
	{\MakeUppercase{Systemy Operacyjne - Laboratorium}}
	{}
	
	\setfoot
	{}
	{\thepage}
	{}
}

\settitlemarks{section,subsection,subsubsection}

\pagestyle{mypage_1}

\newcommand{\ask}[2]{
    \begin{tcolorbox}[colback=black!5!white,colframe=gray,title={Pytanie #1}]
        #2
    \end{tcolorbox}
}

\newcommand{\assignment}[2]{
    \begin{tcolorbox}[colback=black!5!white,colframe=black,title={Zadanie #1}]
        #2
    \end{tcolorbox}
}

%\marginsize{left}{right}{top}{bottom}
\marginsize{3cm}{3cm}{3cm}{3cm}
\sloppy
\titleformat{\section}
  {\normalfont\Large\bfseries}{\thesection}{1em}{}[{\titlerule[0.8pt]}]
 
 \definecolor{darkred}{rgb}{0.9,0,0}
\definecolor{grey}{rgb}{0.4,0.4,0.4}
\definecolor{orange}{rgb}{1,0.6,0.05}
\definecolor{darkgreen}{rgb}{0.2,0.5,0.05}
 
\definecolor{mGreen}{rgb}{0,0.6,0}
\definecolor{mGray}{rgb}{0.5,0.5,0.5}
\definecolor{mPurple}{rgb}{0.58,0,0.82}
\definecolor{mKeyword}{RGB}{0,0,242}
\definecolor{backgroundColour}{RGB}{242,242,242}

\lstdefinestyle{CStyle}{
    backgroundcolor=\color{backgroundColour},   
    commentstyle=\color{mGreen},
    keywordstyle=\color{mKeyword},
    numberstyle=\tiny\color{mGray},
    stringstyle=\color{mPurple},
    basicstyle=\footnotesize,
    breakatwhitespace=false,         
    breaklines=true,                 
    %captionpos=b,                    
    keepspaces=true,                 
    numbers=left,                    
    numbersep=5pt,                  
    showspaces=false,                
    showstringspaces=false,
    showtabs=false,                  
    tabsize=2,
    language=C
}


\newcommand{\Hilight}{\makebox[0pt][l]{\color{cyan}\rule[-4pt]{0.65\linewidth}{14pt}}}


\begin{document}

\begin{table}
\begin{center}
\begin{tabular}{|c|c|c|}
\hline
\multicolumn{3}{|c|}{\textbf{Procesy 1 - Ustawianie limitów procesu}} \\ \hline Dominik Wróbel & 28 III 2019 & Czw. 17:00 \\ \hline

\end{tabular}
\end{center}
\end{table}

\tableofcontents

\newpage
\section{Ustawiania limitów procesu}

\assignment{}{Proszę pobrać, skompilować i uruchomić program save.c}

\subsection{Zadania 1,2 i 4}
\assignment{1}{Proszę rozbudować program tak aby wartość zmiennej filename mogła być zadana jako (obowiązkowy) argument wywołania programu}
\assignment{2}{Proszę rozbudować program tak aby wartość zmiennej s mogła być zadana jako (opcjonalny) argument wywołania programu. Docelowo wywołanie programu save powinno mieć taką składnię: \\ ./save [bytes] file}
\assignment{4}{Rozbuduj program o sprawdzanie czy podano wystarczającą liczbę argumentów. Jeżeli nie to zakończ program zwracając 1.}

\begin{lstlisting}[style=CStyle, label=some-code, caption=Zadania 1 2 i 4 - plik save.c - poczatek funkcji main]
...
int main(int argc, char* argv[])
    {

    char* filename;
    int s = 100;
	  int fd;	
	    
	  if(argc == 1 || argc > 3){ // Zadanie 4
          printf("Invalid arguments. Please check options.");
          return 1;
      }
      if(argc == 2){ // Zadanie 1 - one argument -> filename is obligatory
          printf("One argument");
          filename = argv[1];
          fd = open(filename, O_WRONLY | O_CREAT | O_TRUNC);
          if(fd == -1){
              printf("Error opening file");
              exit(EXIT_FAILURE);
          }
      } else if(argc == 3){ // Zadanie 2 - two arguments -> filename is obligatory
          printf("Two arguments");
          filename = argv[2];
          fd = open(filename, O_WRONLY | O_CREAT | O_TRUNC);
          if(fd == -1){
              printf("Error opening file");
              exit(EXIT_FAILURE);
          }
          s = atoi(argv[1]); // second argument is number of bytes
          printf("s is %d", s);
      }
...
\end{lstlisting}

\subsection{Zadanie 3}
\assignment{}{Proszę przetestować działanie programu:
\begin{itemize}
\item Zapisz 100 bajtów do pliku o nazwie tmp1.txt
\item Zapisz 53 bajty do pliku o nazwie tmp2.txt
\end{itemize}
}
Program działa zgodnie z oczekiwaniami, zapisuje do pliku 100 bajtów (w przypadku braku podania jawnej liczby bajtów) lub przekazaną jako opcje liczbę bajtów. 
\begin{lstlisting}[style=CStyle, label=some-code, caption=Testowanie działania save.c]
~/OperatingSystems/Lab4$ ./save tmp1.txt
One argument
RLIMIT_FSIZE: cur=-1, max=-1
Writing 100 bytes into tmp1.txt file...
Returning zero


~/OperatingSystems/Lab4$ cat tmp1.txt | wc -c
100


~/OperatingSystems/Lab4$ ./save 53 tmp2.txt
Two arguments
s is 53
RLIMIT_FSIZE: cur=-1, max=-1
Writing 53 bytes into tmp2.txt file...
Returning zero


~/OperatingSystems/Lab4$ cat tmp2.txt | wc -c
53
\end{lstlisting}

\newpage
\subsection{Zadanie 5}
\assignment{5}{
Przetestuj działanie:
\begin{itemize}
\item Po wykonaniu tej komendy \underline{nie powinien} pojawić się komunikat FAIL: \\ ./save tmp1.txt || echo FAIL
\item Po wykonaniu tej komendy \underline{powinien} pojawić się komunikat OK: \\ ./save 200 tmp1.txt \&\& echo OK 
\item Po wykonaniu tej komendy \underline{powinien} pojawić się komunikat FAIL: \\ ./save || echo FAIL
\item Po wykonaniu tej komendy \underline{nie powinien} pojawić się komunikat OK: \\ ./save \&\& echo OK
\end{itemize}
}

\begin{lstlisting}[style=CStyle, label=some-code, caption=Testowanie działania save.c w przypadku wywołań z \&\& oraz ||]
~/OperatingSystems/Lab4$ ./save tmp1.txt || echo FAIL
One argument
RLIMIT_FSIZE: cur=-1, max=-1
Writing 100 bytes into tmp1.txt file...
Returning zero
// nie wyswietla FAIL

~/OperatingSystems/Lab4$ ./save 200 tmp1.txt && echo OK
Two arguments
s is 200RLIMIT_FSIZE: cur=-1, max=-1
Writing 200 bytes into tmp1.txt file...
Returning zero
OK // wyswietla OK


~/OperatingSystems/Lab4$ ./save || echo FAIL
Not enough arguments. Please enter filename.FAIL // wyswietla FAIL


~/OperatingSystems/Lab4$ ./save && echo OK
Not enough arguments. Please enter filename.
// nie wyswietla OK
\end{lstlisting}

\newpage
\subsection{Zadanie 6}
\assignment{6}{
Zmodyfikuj funkcje main programu shell ustawiając przed główną pętlą miękki limit maksymalnej wielkości tworzonych plików przez proces (RLIMIT\_FSIZE) na wartość 50 bajtów.
}
\begin{lstlisting}[style=CStyle, label=some-code, caption=plik shell.c - przed główną pętlą]
	...
	    	  
  	  struct rlimit r; // ustawianie limitu procesu
      getrlimit(RLIMIT_FSIZE, &r);
      r.rlim_cur = 50;
      setrlimit(RLIMIT_FSIZE, &r);


      while(1){
        printprompt();
        if(readcmd(cmd, MAXCMD) == RESERROR) continue;
        res = parsecmd(cmd, MAXCMD, &cmds);
        printparsedcmds(&cmds);
        executecmds(&cmds);
        dealocate(&cmds);
      }
     
	return 0;
	
	...
\end{lstlisting}

\newpage
\subsection{Zadanie 7}
\assignment{7}{
Sprawdź czy ustawiony limit jest dziedziczony przez procesy potomne wykorzystując w tym celu pierwszą aplikację save. Wpisz w swojej konsoli następujące polecenia i zaobserwuj, czy ich działanie jest jak na poniższym listingu.
\begin{framed}
@ ./save 10 tmp3.txt                    \# plik tmp3.txt zostaje utworzony i zawiera 10 bajtów \\
@ wc -c tmp3.txt \\
10 tmp.txt \\
@ ./save tmp4.txt \&\& ./save tmp5.txt   \# plik tmp4.txt zostaje utworzony i zawiera 50 bajtów, natomiast plik tmp5.txt nadal nie istnieje \\
@ wc -c tmp4.txt \\
50 tmp4.txt \\
@ wc -c tmp5.txt \\
wc: tmp5.txt: Nie ma takiego pliku ani katalogu 
\end{framed}
}

Wynik działania programu zgadza się z listingiem. Ustawiony limit \textbf{jest dziedziczony} przez procesy potomne. Wywołania funkcji systemowych takich jak fork czy exec dziedziczą ustawione limity. 
\begin{lstlisting}[style=CStyle, label=some-code, caption=plik shell.c - przed główną pętlą]
@ ./save tmp1.txt || echo FAIL
Parsed command(s):
Command 1:
argv[0]: ./save
argv[1]: tmp1.txt
argv[2]: (null)
Command 1:
argv[0]: echo
argv[1]: FAIL
argv[2]: (null)
One argument
RLIMIT_FSIZE: cur=50, max=-1
Writing 100 bytes into tmp1.txt file...

@ ./save tmp4.txt && ./save tmp5.txt
Parsed command(s):
Command 1:
argv[0]: ./save
argv[1]: tmp4.txt
argv[2]: (null)
Command 1:
argv[0]: ./save
argv[1]: tmp5.txt
argv[2]: (null)
One argument
RLIMIT_FSIZE: cur=50, max=-1
Writing 100 bytes into tmp4.txt file...

@ wc -c tmp4.txt
Parsed command(s):
Command 1:
argv[0]: wc
argv[1]: -c
argv[2]: tmp4.txt
argv[3]: (null)
50 tmp4.txt

@ wc -c tmp5.txt
Parsed command(s):
Command 1:
argv[0]: wc
argv[1]: -c
argv[2]: tmp5.txt
argv[3]: (null)
wc: tmp5.txt: No such file or directory
\end{lstlisting}




\end{document}
