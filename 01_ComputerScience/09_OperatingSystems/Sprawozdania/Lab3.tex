\documentclass[a4paper,15pt]{article}
\usepackage{amssymb}
\usepackage{amsmath}
\usepackage[english, polish]{babel}
\usepackage[utf8]{inputenc}   % lub utf8
\usepackage[T1]{fontenc}
\usepackage{graphicx}
\usepackage{anysize}
\usepackage{enumerate}
\usepackage{times}
\usepackage{caption}
\usepackage{titlesec}
\usepackage{float}
\usepackage{titleps,kantlipsum}
\usepackage{listings}
\usepackage{xcolor}
\usepackage{hyperref}
\usepackage{framed}
\usepackage{tcolorbox}
\lstloadlanguages{Matlab}
 
\usepackage[justification=centering]{caption}
\titlelabel{\thetitle.\quad}


\DeclareCaptionFont{white}{\color{white}}
\DeclareCaptionFormat{listing}{%
  \parbox{\textwidth}{\colorbox{darkgreen}{\parbox{\textwidth}{#1#2#3}}\vskip-4pt}}
\captionsetup[lstlisting]{format=listing,labelfont=white,textfont=white}
\lstset{frame=lrb,xleftmargin=\fboxsep,xrightmargin=-\fboxsep}

% Definicja nowego stylu strony
\newpagestyle{mypage}
{
  \headrule
  
  \sethead
  { \MakeUppercase{\thesection\quad \sectiontitle} } 
  {}
  {\thesubsection\quad \subsectiontitle}
  
  \setfoot
  {}
  {\thepage}
  {}
}

\newpagestyle{mypage_1}
{
	\headrule
	
	\sethead
	{  }
	{\MakeUppercase{Systemy Operacyjne - Laboratorium}}
	{}
	
	\setfoot
	{}
	{}
	{}
}

\settitlemarks{section,subsection,subsubsection}

\pagestyle{mypage_1}

\newcommand{\ask}[2]{
    \begin{tcolorbox}[colback=black!5!white,colframe=gray,title={Pytanie #1}]
        #2
    \end{tcolorbox}
}

\newcommand{\assignment}[2]{
    \begin{tcolorbox}[colback=black!5!white,colframe=black,title={Zadanie #1}]
        #2
    \end{tcolorbox}
}

%\marginsize{left}{right}{top}{bottom}
\marginsize{3cm}{3cm}{3cm}{3cm}
\sloppy
\titleformat{\section}
  {\normalfont\Large\bfseries}{\thesection}{1em}{}[{\titlerule[0.8pt]}]
 
 \definecolor{darkred}{rgb}{0.9,0,0}
\definecolor{grey}{rgb}{0.4,0.4,0.4}
\definecolor{orange}{rgb}{1,0.6,0.05}
\definecolor{darkgreen}{rgb}{0.2,0.5,0.05}
 
\definecolor{mGreen}{rgb}{0,0.6,0}
\definecolor{mGray}{rgb}{0.5,0.5,0.5}
\definecolor{mPurple}{rgb}{0.58,0,0.82}
\definecolor{mKeyword}{RGB}{0,0,242}
\definecolor{backgroundColour}{RGB}{242,242,242}

\lstdefinestyle{CStyle}{
    backgroundcolor=\color{backgroundColour},   
    commentstyle=\color{mGreen},
    keywordstyle=\color{mKeyword},
    numberstyle=\tiny\color{mGray},
    stringstyle=\color{mPurple},
    basicstyle=\footnotesize,
    breakatwhitespace=false,         
    breaklines=true,                 
    %captionpos=b,                    
    keepspaces=true,                 
    numbers=left,                    
    numbersep=5pt,                  
    showspaces=false,                
    showstringspaces=false,
    showtabs=false,                  
    tabsize=2,
    language=C
}


\newcommand{\Hilight}{\makebox[0pt][l]{\color{cyan}\rule[-4pt]{0.65\linewidth}{14pt}}}


\begin{document}

\begin{table}
\begin{center}
\begin{tabular}{|c|c|c|}
\hline
\multicolumn{3}{|c|}{\textbf{Zaawansowane operacje wejścia-wyjścia dla plików}} \\ \hline Dominik Wróbel & 14 III 2019 & Czw. 17:00 \\ \hline

\end{tabular}
\end{center}
\end{table}

\tableofcontents

\newpage
\section{Pozyskiwanie i wyświetlanie metadanych pliku}
\subsection{Odpowiedz na pytania}

\ask{}{Proszę przejrzeć manual do funkcji z rodziny stat(2). Czym różnią się te funkcje ?}

\begin{itemize}
\item  Funkcja stat przyjmuje ścieżkę jako argument i szuka i-node podążając tą ścieżką. 
\item Funkcja lstat jest identyczna jak stat, ale w przypadku gdy ścieżka prowadzi do linku symbolicznego, wyświetla metadane powiązane z tym linkiem, a nie z tym do czego on wskazuje, jak w przypadku stat.
\item  Funkcja fstat przyjmuje deskryptor, a nie ścieżkę do pliku, znajduje i-node w tablicy aktywnych i-node w jądrze systemu
\end{itemize}

\ask{}{
Co reprezentuje flaga S\_IFMT zdefiniowana dla pola st\_mode ? 
}

Jest to makro reprezentujące maskę dzięki której można sprawdzić typ pliku. Aby przy pomocy tego makra uzyskać typ pliku należy wykonać logiczny AND z polem st\_mode struktury stat. 



\ask{}{
Zmienna sb jest wypełnioną strukturą typu struct stat. Czy można sprawdzić typ pliku (np. czy plik jest urządzeniem blokowym) w następujący sposób ?
Odpowiedź uzasadnij.
}
\begin{lstlisting}[style=CStyle, label=some-code]
if (( sb.st_mode & S_IFBLK) == S_IFBLK) { /* plik jest urzadzeniem blokowym */}
\end{lstlisting}

Nie, nie można sprawdzić typu pliku w taki sposób. Najpierw należy wykonać logiczny AND z maską pliku w celu uzyskania typu pliku, a następnie przyrównać wynik do odpowiedniej flagi. Prawidłowa wersja to:
\begin{lstlisting}[style=CStyle, label=some-code]
if (( sb.st_mode & S_IFMT) == S_IFBLK) { /* plik jest urzadzeniem blokowym */}
\end{lstlisting}

\subsection{Program stat\_info.c}
\assignment{}{
Proszę pobrać, skompilować i uruchomić poniższy program. stat\_info.c
}

\newpage
\assignment{}{
4. Proszę zmodyfikować funkcję print\_type tak aby badała wszystkie możliwe typy pliku i wyświetlała odpowiednią informację. \\
5. Proszę zmodyfikować funkcję print\_perms tak aby wyświetlała prawa dostępu do pliku w postaci numerycznej, np. 755. \\
6. Proszę zmodyfikować funkcję print\_owner tak aby wyświetlała nazwę właściciela i grupy właścicieli oraz podawała identyfikatory w nawiasach, np. wta(1234) iastaff(5678) \\
7. Proszę zmodyfikować funkcję print\_perms tak aby wyświetlone zostały prawa dostępu do pliku użytkownika uruchamiającego program np. Your permisions: read: yes, write: no, execute: no. \\
8. Proszę zmodyfikować funkcję print\_size tak aby wyświetlała informacje o rozmiarze pliku w sformatowany sposób - w kilo/megabajtach itp., np. 1MB zamiast 1024bytes \\
9. Proszę zmodyfikować funkcje print\_laststch, print\_lastacc i print\_lastmod tak aby wyświetlały czas, który minął od daty ostatniej zmiany statusu/dostępu/modyfikacji, np. 3 day ago \\
10. Proszę zmodyfikować funkcje print\_name tak, aby w przypadku gdy podany jako argument plik jest linkiem symbolicznym wyświetlał jego nazwę w formacie: nazwa\_linku  nazwa\_plik\_na\_ktory\_wskazuje. 
}

\begin{lstlisting}[style=CStyle, label=some-code,caption=Zadania 4-10]
static void print_file_info(struct stat *sb, char *name){

	  // 4.
      printf("File type:                ");
      switch (sb->st_mode & S_IFMT) {
      case S_IFBLK:  printf("block device\n");            break;
      case S_IFCHR:  printf("character device\n");        break;
  	  case S_IFDIR:	 printf("directory\n");				  break;
  	  case S_IFIFO:	 printf("named or unnamed pipe\n");   break;
	  	case S_IFLNK:  printf("symbolic link\n");			  break;
  	  case S_IFREG:  printf("regular file\n");		      break;
      case S_IFSOCK: printf("Socket\n");			      break;
      default:       printf("unknown?\n");                break;
      }
      
      
      // 5. i 7.
      char *perms = (char *) malloc(sizeof(char) * 128);
	  strcat(perms, "Your permissions: read: ");	
	  
	  int usr = 0;
	  int grp = 0;
	  int oth = 0; 
      printf("Mode:                     %lo (octal)\n", (unsigned long) sb->st_mode);
	  if( (sb->st_mode & S_IRUSR) == S_IRUSR){
	  	usr += 4;
		strcat(perms, "yes, write: ");
	  }
	  else{
	  	strcat(perms, "no, write: ");
	  }
	  if( (sb->st_mode & S_IWUSR) == S_IWUSR){
	  	usr += 2;
		strcat(perms, "yes, execute: ");
	  }
	  else{
	  	strcat(perms, "no, execute: ");
	  }
	  if( (sb->st_mode & S_IXUSR) == S_IXUSR){
	  	usr += 1;
		strcat(perms, "yes");
	  }
	  else {
	  	strcat(perms, "no");
	  }
	  if( (sb->st_mode & S_IRGRP) == S_IRGRP){
	  	grp += 4;
	  }
	  if( (sb->st_mode & S_IWGRP) == S_IWGRP){
	  	grp += 2;
	  }
	  if( (sb->st_mode & S_IXGRP) == S_IXGRP){
	  	grp += 1;
	  }	
	  if( (sb->st_mode & S_IROTH) == S_IROTH){
	  	oth += 4;
	  }
	  if( (sb->st_mode & S_IWOTH) == S_IWOTH){
	  	oth += 2;
	  }
	  if( (sb->st_mode & S_IXOTH) == S_IXOTH){
	  	oth += 1;
	  }		
	  printf("Permission: %d %d %d \n", usr, grp, oth);
	  printf("%s\n", perms);
	  
	  
	  // 6.	  
	  char *userName =(char *)malloc(64*sizeof(char));
	  char *groupName =(char *)malloc(64*sizeof(char));  
		
	  long uid = (long) sb->st_uid;
	  long gid = (long) sb->st_gid;
	  struct passwd *pwd = getpwuid(uid);
	  struct group *grp = getgrgid(gid);
	 
	  userName = pwd->pw_name;
	  groupName = grp->gr_name;

	  printf("Ownership:                UID=%ld   GID=%ld\n", (long) sb->st_uid, (long) sb->st_gid);
	  printf("Ownership:                %s(%ld) %s(%ld)\n", userName, (long) sb->st_uid, groupName, (long) sb->st_gid );
	  
	  
	  // 7. rozwiazanie w 5.
	  
	  
	  // 8.
	  printf("Preferred I/O block size: %ld bytes\n", (long) sb->st_blksize);
      
	  printf("File size:                %lld bytes\n",(long long) sb->st_size);
 
      printf("Kb File size:           %f kb\n",(float)sb->st_size/1024);
  
	  printf("MB File size:           %f Mb\n",(float)sb->st_size/(1024*1024));
		
      printf("Blocks allocated:         %lld\n",(long long) sb->st_blocks);
      
      // 9. 
	  printf("Last status change:       %s", ctime(&sb->st_ctime));	  
	  time_t currentTime;
  	  time(&currentTime);
  	  long long diffTime;
	  diffTime = difftime(currentTime,sb->st_ctime);
		
	  printf("Last status change:         %lld days ago\n",diffTime/(60*60*24) );
	  
	  printf("Last file access:         %s", ctime(&sb->st_atime));
		
	  time_t currentTime;
  	  time(&currentTime);
  	  long long diffTime;
	  diffTime = difftime(currentTime,sb->st_atime);
		
	  printf("Last file access:         %lld days ago\n",diffTime/(60*60*24) );
	  
	  printf("Last file modification:   %s", ctime(&sb->st_mtime));
		
	  time_t currentTime;
  	  time(&currentTime);
  	  long long diffTime;
	  diffTime = difftime(currentTime,sb->st_mtime);
		
	  printf("Last file modification:         %lld days ago\n",diffTime/(60*60*24) );
	  
	  
	  // 10.
	  char* bname = basename(name);
      printf("Name of the file:         %s\n", bname);
		
	  char* lname = (char*) malloc(256 * sizeof(char));
	  ssize_t bytesRead;
	  
	  if( (sb->st_mode & S_IFMT) == S_IFLNK ){
		printf("Is Link");
		if((bytesRead = readlink(name,lname, sizeof(lname)) != -1)){
			if(lname != NULL){
		  		lname[bytesRead] = '\0';
		  		printf("SymLink %s -> File %s\n",lname,bname);
			}
			else{
		  		exit(EXIT_FAILURE);
			}
	  	} 
	}
	  
	  
}
\end{lstlisting}

\newpage
\section{Wejście / wyjście asynchroniczne}
\assignment{}{
Do programu z poprzedniego ćwiczenia dopisz funkcję print\_content, której deklaracja może wyglądać następująco static void print\_content(char *name). Powyższa funkcja implementuje następującą funkcjonalność: 
\begin{itemize}
\item Pyta użytkownika czy chce wypisać zawartość podanego jako argument pliku.
\item Jeżeli tak, to otwiera i przy pomocy funkcji czytania asynchronicznego aio\_read(3) odczytuje zawartość pliku i wypisuje ją na ekran.
\end{itemize}
Powyższe zadanie najlepiej wykonać stosując taki oto scenariusz:
\begin{itemize}
\item Otwieramy plik open(2)
\item Inicjalizujemy wczytywanie pierwszej porcji danych aio\_read(3).
\item Przy pomocy funkcji aio\_error(3) czekamy do momentu aż odczyt się zakończy.
\item Wyświetlamy wczytane dane i wracamy do 2 jeżeli nie osiągnęliśmy końca pliku.
\end{itemize}
}

\begin{lstlisting}[style=CStyle, label=some-code,caption=print\_content]
static void print_content(char * name){
		
		char * choice = (char*) malloc(1*sizeof(char));
		
		printf("Would you like to print content of your file ?\n");
		printf("Press Y - Yes, any key otherwise");
		
		int fd;
		struct aiocb cb;
		char check[256];
		
		scanf("%c", choice);
		if(*choice == 'y' || *choice == 'Y'){
			
			fd = open(name, O_RDONLY); // open file
			memset(&cb, 0, sizeof(struct aiocb));
			cb.aio_fildes = fd; // configure struct
			cb.aio_buf = check;
			cb.aio_nbytes = 256;
			printf("reading aio");
			int n;
			while(1){
				n = aio_read(&cb); // read 
				if(n==-1){
					printf("AIO Error"); // handle errors
					exit(EXIT_FAILURE);
				}
				int err;
				printf("reading aio 2");
				while((err = aio_error (&cb)) == EINPROGRESS); // wait for asynchronous read to complete
				printf("%s\n", check);
				if(n==0){
					break;
				}
			}
				
		} else {
			printf("Printing cancelled");
		}
		
		return;
		
}
\end{lstlisting}





\end{document}
