\documentclass[a4paper,15pt]{article}
\usepackage{amssymb}
\usepackage{amsmath}
\usepackage[english, polish]{babel}
\usepackage[utf8]{inputenc}   % lub utf8
\usepackage[T1]{fontenc}
\usepackage{graphicx}
\usepackage{anysize}
\usepackage{enumerate}
\usepackage{times}
\usepackage{caption}
\usepackage{titlesec}
\usepackage{float}
\usepackage{titleps,kantlipsum}
\usepackage{listings}
\usepackage{xcolor}
\usepackage{hyperref}
\usepackage{framed}
\usepackage{tcolorbox}
\lstloadlanguages{Matlab}
 
\usepackage[justification=centering]{caption}
\titlelabel{\thetitle.\quad}

\pagenumbering{arabic}

\DeclareCaptionFont{white}{\color{white}}
\DeclareCaptionFormat{listing}{%
  \parbox{\textwidth}{\colorbox{darkgreen}{\parbox{\textwidth}{#1#2#3}}\vskip-4pt}}
\captionsetup[lstlisting]{format=listing,labelfont=white,textfont=white}
\lstset{frame=lrb,xleftmargin=\fboxsep,xrightmargin=-\fboxsep}

% Definicja nowego stylu strony
\newpagestyle{mypage}
{
  \headrule
  
  \sethead
  { \MakeUppercase{\thesection\quad \sectiontitle} } 
  {}
  {\thesubsection\quad \subsectiontitle}
  
  \setfoot
  {}
  {}
  {\thepage}
}

\newpagestyle{mypage_1}
{
	\headrule
	
	\sethead
	{  }
	{\MakeUppercase{Systemy Operacyjne - Laboratorium}}
	{}
	
	\setfoot
	{}
	{\thepage}
	{}
}

\settitlemarks{section,subsection,subsubsection}

\pagestyle{mypage_1}

\newcommand{\ask}[2]{
    \begin{tcolorbox}[colback=black!5!white,colframe=gray,title={Pytanie #1}]
        #2
    \end{tcolorbox}
}

\newcommand{\assignment}[2]{
    \begin{tcolorbox}[colback=black!5!white,colframe=black,title={Zadanie #1}]
        #2
    \end{tcolorbox}
}

%\marginsize{left}{right}{top}{bottom}
\marginsize{3cm}{3cm}{3cm}{3cm}
\sloppy
\titleformat{\section}
  {\normalfont\Large\bfseries}{\thesection}{1em}{}[{\titlerule[0.8pt]}]
 
 \definecolor{darkred}{rgb}{0.9,0,0}
\definecolor{grey}{rgb}{0.4,0.4,0.4}
\definecolor{orange}{rgb}{1,0.6,0.05}
\definecolor{darkgreen}{rgb}{0.2,0.5,0.05}
 
\definecolor{mGreen}{rgb}{0,0.6,0}
\definecolor{mGray}{rgb}{0.5,0.5,0.5}
\definecolor{mPurple}{rgb}{0.58,0,0.82}
\definecolor{mKeyword}{RGB}{0,0,242}
\definecolor{backgroundColour}{RGB}{242,242,242}

\lstdefinestyle{CStyle}{
    backgroundcolor=\color{backgroundColour},   
    commentstyle=\color{mGreen},
    keywordstyle=\color{mKeyword},
    numberstyle=\tiny\color{mGray},
    stringstyle=\color{mPurple},
    basicstyle=\footnotesize,
    breakatwhitespace=false,         
    breaklines=true,                 
    %captionpos=b,                    
    keepspaces=true,                 
    numbers=left,                    
    numbersep=5pt,                  
    showspaces=false,                
    showstringspaces=false,
    showtabs=false,                  
    tabsize=2,
    language=C
}


\newcommand{\Hilight}{\makebox[0pt][l]{\color{cyan}\rule[-4pt]{0.65\linewidth}{14pt}}}


\begin{document}

\begin{table}
\begin{center}
\begin{tabular}{|c|c|c|}
\hline
\multicolumn{3}{|c|}{\textbf{Komunikacja międzyprocesowa - łącza nienazwane}} \\ \hline Dominik Wróbel & 11 IV 2019 & Czw. 17:00 \\ \hline

\end{tabular}
\end{center}
\end{table}

\tableofcontents

\newpage
\section{Tworzenia prostych łączy jednokierunkowych}


\assignment{1.1 1.2 1.3}{ Zmodyfikuj kod programu:
\begin{itemize}
\item Podziel program na 2 procesy (użyj fork()).
\item Korzystając z funkcji close zamknij zbędne łącza (WAŻNE: które?/dlaczego?).
\item Przekaż komunikat pomiędzy procesami.
\end{itemize}
}
\ask{1.3}{
Korzystając z funkcji close zamknij zbędne łącza (WAŻNE: które?/dlaczego?).
}
Po rozdziale programu na dwa procesy musimy zamknąć deskryptory:
\begin{itemize}
\item Odpowiadający operacji czytania w procesie, który zapisuje dane, ponieważ powinien istnieć tylko jeden deskryptor do czytania we wszystkich procesach
\item Odpowiadający operacji zapisywania w procesie, który czyta dane, ponieważ jeden proces nie powinien mieć możliwości czytania i zapisywania jednocześnie
\end{itemize}

\begin{lstlisting}[style=CStyle, label=some-code, caption=simple\_pipe.c]
    #include <stdio.h>
    #include <unistd.h>
    #include <string.h>
    #include <stdlib.h>
    #include <sys/types.h>
     
    int main(int argc, char *argv[]){
     
        int pfd[2]; // file descriptors 
        size_t nread;
        char buf[100];
        pid_t newProcessPID;

        pipe(pfd); // create pipe

        char * str ="Ur beautiful pipe example";
        
        newProcessPID = fork(); // create child process

        if(newProcessPID == 0){ // child process
            close(pfd[1]); // close writing descriptor
            nread=read(pfd[0],buf,sizeof(buf));
            (nread!=0)?printf("%s (%ld bytes)\n",buf,(long)nread):printf("No data\n");
            _exit(EXIT_SUCCESS);
        } else{ // parent process
            close(pfd[0]); // close reading descriptor
            write(pfd[1], str, strlen(str));
            _exit(EXIT_SUCCESS);
        }
     
        return 0;
}
\end{lstlisting}

\assignment{1.4}{
Korzystając z funkcji fpathconf sprawdź wielkość bufora dla łączy komunikacyjnych na Twoim systemie. Odp. na pytanie: do czego potrzebna jest ta informacja?
}
\begin{lstlisting}[style=CStyle, label=some-code, caption=zadanie1\_4.c]
#include <stdio.h>
#include <unistd.h>
#include <string.h>
#include <stdlib.h>
#include <sys/types.h>

int main(){
    
    int pfd[2];
    long v;

    pipe(pfd);

    v = fpathconf(pfd[0], _PC_PIPE_BUF);
    printf("PIPE_BUF = %ld\n", v);
    
}

// WYNIK DZIALANIA:
// dominik@dominik-VirtualBox:~/OperatingSystems/Lab6$ ./check_buf
// PIPE_BUF = 4096
\end{lstlisting}

\ask{1.4}{
Korzystając z funkcji fpathconf sprawdź wielkość bufora dla łączy komunikacyjnych na Twoim systemie. Odp. na pytanie: do czego potrzebna jest ta informacja?
}

Jest to stała definiująca ile danych można zapisać do łącza w operacji atomicznej. Wartość ta jest nie mniejsza niż 512, oznacza to, że przy zapisywaniu zawsze 512 bajtów danych nie będzie przerwane przez inne dane. Jest to ważne z punktu widzenia działania funkcji read, write na łączach, ponieważ zbyt duża ilość danych może doprowadzić do zablokowania funkcji write.

\newpage
\section{Praca z łączami komunikacyjnymi}
\assignment{2}{
 \begin{itemize}
 \item Funkcja pipe w przypadku błędu ustawia zmienną errno. Zaimplementuj obsługę błędów wyświetlającą informację dla użytkownika dot. błędu. 
 \item Do powyższego programu dodaj niezbędne funkcje close zamykające w odpowiednich miejscach otwarte deskryptory plików, a zamiast funkcji dup użyj dup2. 
 \end{itemize}
}

\ask{2}{Do czego w procesie potomnym użyta jest funkcja close(0).}
W tym przypadku 0 odnosi się do deskryptora standardowego wejścia. Deskryptor ten jest zamykany i jest dzięki temu zwalniany. Wywołanie close dla deskryptorów łączy zapisujących powoduje, że read otrzymuje znak EOF, a w wypadku zamknięcia deskryptora czytającego zwracany jest błąd dla deskryptorów zapisujących. 

\begin{lstlisting}[style=CStyle, label=some-code, caption=dup-example.c]
#include <stdio.h>
#include <string.h>
#include <unistd.h>
#include <sys/types.h>
#include <stdlib.h>
    
int main(void)
{
    int     pfd[2];
    pid_t   pid;
    char    string[] = "Test";
    char    buf[10];

    if(pipe(pfd) == -1){ // 1. error handling
        perror("Pipe call failed.");
        exit(EXIT_FAILURE);
    }

    pid = fork();

    if(pid == 0) {
        close(0);              
        // dup(pfd[0]);
        dup2(pfd[0],0);
        read(STDIN_FILENO, buf, sizeof(buf));
        close(pfd[0]);
        printf("Wypisuje: %s", buf);
    } else {
        close(pfd[0]);
        write(pfd[1], string, (strlen(string)+1));    
        close(pfd[1]);    
    }

    return 0;
}

\end{lstlisting}

\section{Komunikacja z kilkoma procesami}
\subsection{Komunikacja z kilkoma procesami}
\assignment{3.1}{
Napisz program, który utworzy proces czytający dane z pliku (słownika), a następnie utworzy dwa podprocesy, z których jeden policzy liczbę linii (liczbę słów w słowniku), a drugi policzy liczbę linii, zawierających podciąg pipe (liczbę słów, które  zawierają słowo pipe). 
}
Komentarze opisujące wykonane w programie działania zamieszczono na listingu.

\begin{lstlisting}[style=CStyle, label=some-code, caption=Komunikacja z kilkoma procesami.]
#include<stdio.h>
#include<fcntl.h>
#include<stdlib.h>
#include<string.h>
#include<unistd.h>

#define BUFFSIZE 4096

int main()
{
    // ---------------------------------------------
    // parent --> child1
    // communication
    int pfd1[2];

    // child1 --> parent
    // communication
    int pfd2[2];
    
    pipe(pfd1);
    pipe(pfd2);
    // ---------------------------------------------

    // ---------------------------------------------
    // parent --> child2
    // communication
    int ppfd1[2];

    // child2 --> parent
    // communication
    int ppfd2[2];
    
    pipe(ppfd1);
    pipe(ppfd2);
    // ---------------------------------------------

    int childOne;
    int childTwo;

    char childOneReceived[BUFFSIZE];
    char parentSent[BUFFSIZE];
    char childOneToParent[BUFFSIZE];
    char parentFromChildOneReceived[BUFFSIZE];

    char childTwoReceived[BUFFSIZE];
    char childTwoToParent[BUFFSIZE];
    char parentFromChildTwoReceived[BUFFSIZE];

    const char * lookForThisWord = "pipe";

    int linesNumber = 0;

    int pipesWordNumber = 0;

    // open file
    int fdDictionary = open("dictionary.txt", O_RDONLY);
    if(fdDictionary == -1){
        printf("Error opening dictionary.txt");
        exit(EXIT_FAILURE);
    }

    // printf("BreakPoint1");

    childOne = fork();

    if(childOne == 0) // from parent to childOne
    {
        int j = 0;

        close(pfd1[1]);
        close(pfd2[0]);

        // parent --> childOne
        // receive and count lines
        while(read(pfd1[0], childOneReceived, BUFFSIZE) != 0) {
            for(j=0; j < BUFFSIZE; j++){
                if(childOneReceived[j]=='\n'){
                    linesNumber++;
                }
            }
        }
        
        // childOne --> parent
        // send number of counted lines
        printf("There are %d lines", linesNumber);

        sprintf(childOneToParent, "%d", linesNumber);
       
        write(pfd2[1],childOneToParent,BUFFSIZE);
        close(pfd2[1]);
        
        // printf("\nchild recieved is %s and lines are %d", childOneReceived, linesNumber);

        exit(EXIT_SUCCESS);
    }

    childTwo = fork();

    // printf("Child two is %d", childTwo);

    if(childTwo == 0) { // from parent to childTwo

        printf("BreakPoint2.1");

        close(ppfd1[1]);
        close(ppfd2[0]);

        char * s;
        const char * start = NULL;

        // parent --> childTwo
        // receive and count lines
        while(read(ppfd1[0], childTwoReceived, BUFFSIZE) != 0) {
                while ( (s = strstr(start, lookForThisWord)) != NULL ) {
                    pipesWordNumber++;
                }
        }

        // childTwo --> parent
        // send number of counted lines
        printf("Lines are %d", pipesWordNumber);

        sprintf(childTwoToParent, "%d", pipesWordNumber);
       
        write(ppfd2[1], childTwoToParent, BUFFSIZE);
        close(ppfd2[1]);

        exit(EXIT_SUCCESS);
        
        // printf("\nchild recieved is %s and lines are %d", childOneReceived, linesNumber);
    }


    if(childOne > 0 && childTwo > 0) { // parent
        
        close(pfd1[0]);
        close(pfd2[1]);
        close(ppfd1[0]);
        close(ppfd2[1]);

    // printf("BreakPoint4");
    // send from parent to children
        while(read(fdDictionary, parentSent, BUFFSIZE) != 0){
            // printf("BreakPoint5");
            write(pfd1[1], parentSent, BUFFSIZE);
            write(ppfd1[1], parentSent, BUFFSIZE);
            // printf("BreakPoint7");
        }
        close(pfd1[1]);
        close(ppfd1[1]);

    // printf("BreakPoint10");
    // receive from children
        read(pfd2[0], parentFromChildOneReceived, BUFFSIZE);

        printf("\nFrom childOne recieved is %s", parentFromChildOneReceived);

        read(ppfd2[0], parentFromChildTwoReceived, BUFFSIZE);

        printf("\nFrom childTwo recieved is %s", parentFromChildTwoReceived);

        exit(EXIT_SUCCESS);

    }


}

\end{lstlisting}

\newpage
\subsection{Przekazywanie danych przez kilka łączy}

\assignment{3.2}{
Proszę napisać program, który utworzy dwa procesy potomne p1 i p2.
Proces potomny p1 generuje kolejne liczby od 1 do 10 (można użyć seq).
Proces p1 powinien za pomocą łącza nienazwanego przekazać je do procesu p2, który pomnoży każdą z liczb razy 5.
Następnie proces p2 przekaże liczby do procesu macierzystego p0, który wyświetli otrzymane liczby. . 
}

W kodzie proces childOne odpowiedzialny jest za generację liczb w pętli od 1 do 10. Liczby są wysyłane 'po jednej' do procesu childTwo, który odbiera dane w postaci string, konwertuje je na int i mnoży przez 5, a następnie wysyła do procesu rodzica w postaci string, rodzic wyświetla otrzymane dane. Program działa poprawnie, ponieważ funkcję read czekają na zamknięcie deskryptorów plików zapisujących oraz na dane do odczytania, co umożliwia poprawną kolejność przepływu danych.
\begin{lstlisting}[style=CStyle, label=some-code, caption=Przekazywanie danych przez kilka łączy.]
#include<stdio.h>
#include<fcntl.h>
#include<stdlib.h>
#include<string.h>
#include<unistd.h>

#define BUFFSIZE 4096

int main()
{

    // ---------------------------------------------
    // child1 --> child2
    int ch1ch2Pipe[2];

    // child2 --> parent
    int ch2pPipe[2];
    
    pipe(ch1ch2Pipe);
    pipe(ch2pPipe);
    // ---------------------------------------------

    int childOne;
    int childTwo;

    char childOneToChildTwo[BUFFSIZE];
    char childTwoReceived[BUFFSIZE];

    // char childTwoToParent[BUFFSIZE];
    char parentReceived[BUFFSIZE];
    
    childOne = fork();

    if(childOne == 0){ // child one

        printf("Inside child one ! \n");

        int i = 0;
        close(ch1ch2Pipe[0]);

        for(i=0; i<10; i++){
            sprintf(childOneToChildTwo, "%d", (i+1));
            write(ch1ch2Pipe[1],childOneToChildTwo,BUFFSIZE);
        }

        close(ch1ch2Pipe[0]);
        exit(EXIT_SUCCESS);
    }
    

    childTwo = fork();

    if(childTwo == 0){ // child two

        printf("Inside child two ! \n");
        int numReceived = 0;

        close(ch1ch2Pipe[1]);
        close(ch2pPipe[0]);

        while(read(ch1ch2Pipe[0], childTwoReceived, BUFFSIZE) != 0) {
            printf("Child two received %s \n", childTwoReceived);
            numReceived = (atoi(childTwoReceived) * 5);
            sprintf(childTwoReceived, "%d", numReceived);
            write(ch2pPipe[1], childTwoReceived, BUFFSIZE);
        }
        
        close(ch2pPipe[1]);
      
        exit(EXIT_SUCCESS);
    }
    else { // parent
       printf("I am inside parent ! \n");

       close(ch2pPipe[1]);
     
        while(read(ch2pPipe[0], parentReceived, BUFFSIZE) != 0) {
            printf("Parent Received %s \n", parentReceived);
        }

        exit(EXIT_SUCCESS);
    }

    return 0;

}

\end{lstlisting}


\end{document}
