\documentclass[a4paper,15pt]{article}
\usepackage{amssymb}
\usepackage{amsmath}
\usepackage[english, polish]{babel}
\usepackage[utf8]{inputenc}   % lub utf8
\usepackage[T1]{fontenc}
\usepackage{graphicx}
\usepackage{anysize}
\usepackage{enumerate}
\usepackage{times}
\usepackage{caption}
\usepackage{titlesec}
\usepackage{float}
\usepackage{titleps,kantlipsum}
\usepackage{listings}
\usepackage{xcolor}
\usepackage{hyperref}
\usepackage{framed}
\usepackage{tcolorbox}
\lstloadlanguages{Matlab}
 
\usepackage[justification=centering]{caption}
\titlelabel{\thetitle.\quad}


\DeclareCaptionFont{white}{\color{white}}
\DeclareCaptionFormat{listing}{%
  \parbox{\textwidth}{\colorbox{darkgreen}{\parbox{\textwidth}{#1#2#3}}\vskip-4pt}}
\captionsetup[lstlisting]{format=listing,labelfont=white,textfont=white}
\lstset{frame=lrb,xleftmargin=\fboxsep,xrightmargin=-\fboxsep}

% Definicja nowego stylu strony
\newpagestyle{mypage}
{
  \headrule
  
  \sethead
  { \MakeUppercase{\thesection\quad \sectiontitle} } 
  {}
  {\thesubsection\quad \subsectiontitle}
  
  \setfoot
  {}
  {\thepage}
  {}
}

\newpagestyle{mypage_1}
{
	\headrule
	
	\sethead
	{  }
	{\MakeUppercase{Systemy Operacyjne - Laboratorium}}
	{}
	
	\setfoot
	{}
	{}
	{}
}

\settitlemarks{section,subsection,subsubsection}

\pagestyle{mypage_1}

\newcommand{\ask}[2]{
    \begin{tcolorbox}[colback=black!5!white,colframe=gray,title={Pytanie #1}]
        #2
    \end{tcolorbox}
}

\newcommand{\assignment}[2]{
    \begin{tcolorbox}[colback=black!5!white,colframe=black,title={Zadanie #1}]
        #2
    \end{tcolorbox}
}

%\marginsize{left}{right}{top}{bottom}
\marginsize{3cm}{3cm}{3cm}{3cm}
\sloppy
\titleformat{\section}
  {\normalfont\Large\bfseries}{\thesection}{1em}{}[{\titlerule[0.8pt]}]
 
 \definecolor{darkred}{rgb}{0.9,0,0}
\definecolor{grey}{rgb}{0.4,0.4,0.4}
\definecolor{orange}{rgb}{1,0.6,0.05}
\definecolor{darkgreen}{rgb}{0.2,0.5,0.05}
 
\definecolor{mGreen}{rgb}{0,0.6,0}
\definecolor{mGray}{rgb}{0.5,0.5,0.5}
\definecolor{mPurple}{rgb}{0.58,0,0.82}
\definecolor{mKeyword}{RGB}{0,0,242}
\definecolor{backgroundColour}{RGB}{242,242,242}

\lstdefinestyle{CStyle}{
    backgroundcolor=\color{backgroundColour},   
    commentstyle=\color{mGreen},
    keywordstyle=\color{mKeyword},
    numberstyle=\tiny\color{mGray},
    stringstyle=\color{mPurple},
    basicstyle=\footnotesize,
    breakatwhitespace=false,         
    breaklines=true,                 
    %captionpos=b,                    
    keepspaces=true,                 
    numbers=left,                    
    numbersep=5pt,                  
    showspaces=false,                
    showstringspaces=false,
    showtabs=false,                  
    tabsize=2,
    language=C
}


\newcommand{\Hilight}{\makebox[0pt][l]{\color{cyan}\rule[-4pt]{0.65\linewidth}{14pt}}}


\begin{document}

\begin{table}
\begin{center}
\begin{tabular}{|c|c|c|}
\hline
\multicolumn{3}{|c|}{\textbf{Podstawowe operacje wejścia-wyjścia dla plików}} \\ \hline Dominik Wróbel & 07 III 2019 & Czw. 17:00 \\ \hline

\end{tabular}
\end{center}
\end{table}

\tableofcontents

\newpage
\section{Podstawy obsługi plików w systemie UNIX}
\subsection{Odpowiedz na pytania}

\ask{}{Co to są deskryptory ?}

To małe liczby całkowite, które stanowią dane powiązane z procesem, określające pliki otwarte przez proces. Deskryptory służą do realizowania operacji na plikach w wywołaniach systemowych. 

\ask{}{
Jakie są standardowe deskryptory otwierane dla procesów ? 
}

Są to deskryptory o numerach 0-2, odpowiadają one standardowemu wejściu, wyjściu i wyjściu błędów. 

\ask{}{
Jakie flagi trzeba ustawić w funkcji open aby otrzymać funkcjonalność creat ? 
}

O\_WRONLY | O\_CREAT | O\_TRUNC

\ask{}{
W wyniku wykonania polecenia umask otrzymano 0022. Jakie prawa dostępu będzie miał plik otwarty w następujący sposób:
open(pathname, O\_RDWR | O\_CREAT, S\_IRWXU | S\_IRWXG | S\_IRWXO) 
}

Będą to prawa 0755 

\ask{}{
Co oznaczają flagi: O\_WRONLY | O\_CREAT | O\_TRUNC? 
}

O\_WRONLY - otwarcie pliku tylko do zapisu, O\_CREAT - flaga konieczna gdy chcemy utworzyć plik, O\_TRUNC - flaga usuwająca zawartość pliku podczas otwierania go

\ask{}{
Co oznacza flaga O\_APPEND?
} 

Przy każdym zapisie do pliku funkcją write() dane będą dodawane na jego koniec.

\ask{}{
Co oznacza zapis: S\_IRUSR | S\_IWUSR? 
}

Prawa dostępu dla usera: read oraz write. 

\newpage
\subsection{Program copy1.c}
\assignment{}{
Proszę rozbudować program o obsługę błędów. Proszę wyszczególnić następujące problemy:
\begin{itemize}
\item brak pliku
\item złe prawa dostępu do pliku
\item Dla tych błędów proszę wyświetlić odpowiedni komunikat
\item Podpowiedź: sprawdź wartości zwracane przez funkcję oraz zmienną errno
\end{itemize}
}


\begin{lstlisting}[style=CStyle, label=some-code,caption=copy1.c]
#include <fcntl.h>
#include <unistd.h>
#include <stdio.h>
#include <errno.h> /* add errno.h to get errno variable */
#include <stdlib.h>

#define BUFSIZE 512
 
void copy(char *from, char *to) 
{
	int fromfd = -1, tofd = -1;
	ssize_t nread;
	char buf[BUFSIZE];
 
	fromfd = open(from, O_RDONLY);
	if(fromfd == -1){
		if(errno == ENOENT ){
			printf("fromfd - Errno is %d", errno); /* No such file or directory */
			perror("Error: ");
			exit(EXIT_FAILURE);
		}
		if(errno == EACCES){
			printf("fromfd - Errno is %d", errno); /* Permission denied */
			perror("Error: ");
			exit(EXIT_FAILURE);
		}
	}
	
	
	tofd = open(to, O_WRONLY | O_CREAT | O_TRUNC,
				S_IRUSR | S_IWUSR);
	if(tofd == -1){
		if(errno == ENOENT )
		{
			printf("to - Errno is %d", errno); /* No such file or directory */
			perror("Error: ");
			close(fromfd);		
			exit(EXIT_FAILURE);
		}
		if(errno == EACCES)
		{
			printf("to - Errno is %d", errno); /* Permission denied */
			perror("Error: ");
			close(fromfd);
			close(tofd);
			exit(EXIT_FAILURE);
		}
	}

	
	
	while ((nread = read(fromfd, buf, sizeof(buf))) > 0)
	    write(tofd, buf, nread);	
	if(nread == -1){
		if(errno == ENOENT )
		{
			printf("read - Errno is %d", errno); /* No such file or directory */
			perror("Error: ");
			close(fromfd);
			close(tofd);
			exit(EXIT_FAILURE);
		}
		if(errno == EACCES)
		{
			printf("read - Errno is %d", errno); /* Permission denied */
			perror("Error: ");
			close(fromfd);
			close(tofd);
			exit(EXIT_FAILURE);
		}
	}
	
    close(fromfd);
	close(tofd);
	return;
}
 
int main(int argc, char **argv){
	if (argc != 3)
	{
		fprintf(stderr,"usage: %s from_pathname to_pathname\n", argv[0]);
		return 1;
	}
	copy(argv[1], argv[2]);
	return 0;
}
\end{lstlisting}


\section{Operacje pisania i czytania z pliku}
\subsection{Odpowiedz na pytania}
\ask{}{
Czy w momencie powrotu z funkcji write dane są już zapisana na urządzenie wyjściowe ?
} 
Nie, dane po wywołaniu write są przechowywane w buforze, zapisywane są dopiero w odpowiedniej dla systemu chwili. 

\newpage
\ask{}{
	Przeanalizować kod writeall.c, co robi ta funkcja ? Jakiej sytuacji dotyczy wartość EINTR ? 
} 
Funkcja \textit{writeall(int fd, const * buf,  size\_t nbytes)} zapisuje ilość bajtów określoną przez nbytes z bufora buf do pliku o deskryptorze fd. EINTR dotyczy przerwania wykonywania funkcji, np. przez inny proces.

\subsection{Funkcja readall.c}

\assignment{}{
Proszę napisać funkcję readall. Funkcja ma zwracać wartości jak read ale obsługiwać błędy analogicznie jak writeall. Proszę uwzględnić możliwość przeczytania końca pliku.
}

\begin{lstlisting}[style=CStyle, label=some-code, caption=readall.c]
#include <fcntl.h>
#include <stdio.h>
#include <stdlib.h>
#include <errno.h>
#include <unistd.h>

#define BUFSIZE 512

ssize_t readall(int fd, void *buf, size_t nbyte)
{
     ssize_t nread = 0;
	 ssize_t n;	
	
     do {
		if ( (n = read(fd, &((char *)buf)[nread], nbyte - nread)) == -1 ) {
             if (errno == EINTR)
                 continue;
             else
                 return -1;
         }
		 if(n == 0){ // if end of file is reached
		 	return 0;
		 }
         nread += n;
     } while (nread < nbyte);
     
	return nread;
	
}



    int main(int argc, char **argv){

		int toRead = -1;
		char buf[BUFSIZE];
		
    	if (argc != 2)
    	{
    		fprintf(stderr,"usage: %s from_pathname to_pathname\n", argv[0]);
    		return 1;
    	}

        toRead = open(argv[1], O_RDONLY);
    	if(toRead == -1){
            if(errno == ENOENT){
                perror("File does not exists: ");
                exit(EXIT_FAILURE);
            }
            if(errno == EACCES){
                perror("Permission denied: ");
                exit(EXIT_FAILURE);
            }
        }

        readall(toRead, buf, BUFSIZE);
		
		printf("%s", buf);
    	
		return 0;
}	
\end{lstlisting}


\ask{}{
Jaki znak jest czytany gdy osiągnięty zostanie koniec pliku ? 
} 
Czytany jest wtedy EOF, wówczas wartość zwracana przed read to 0.

\subsection{Program realizujący funkcjonalność cat - cat.c}


\assignment{}{
Proszę napisać program realizujący funkcjonalność polecenia cat bez opcji. 
\begin{itemize}
\item Polecenie powinno działać bez podania pliku oraz z podaniem pliku
\end{itemize}
}

\begin{lstlisting}[style=CStyle, label=some-code, caption=cat.c]
#include <fcntl.h>
#include <stdio.h>
#include <stdlib.h>
#include <errno.h>
#include <unistd.h>

#define BUFSIZE 512

ssize_t readall(int fd, void *buf, size_t nbyte)
{
     ssize_t nread = 0;
	 ssize_t n;	
	
     do {
		if ( (n = read(fd, &((char *)buf)[nread], nbyte - nread)) == -1 ) {
             if (errno == EINTR)
                 continue;
             else
                 return -1;
         }
		 if(n == 0){
		 	return 0;
		 }
         nread += n;
     } while (nread < nbyte);
     
	return nread;
	
}



    int main(int argc, char **argv){

		int toRead = -1;
		char buf[BUFSIZE];
		
    	if (argc != 2)
    	{
    		printf("Podaj tekst dla cat: ");
			scanf("%s", buf); // standardowe wejscie
			printf("%s", buf);
			return 0;
    	}

        toRead = open(argv[1], O_RDONLY);
    	if(toRead == -1){
            if(errno == ENOENT){
                perror("File does not exists: ");
                exit(EXIT_FAILURE);
            }
            if(errno == EACCES){
                perror("Permission denied: ");
                exit(EXIT_FAILURE);
            }
        }

        readall(toRead, buf, BUFSIZE); // plik
		
		printf("%s", buf);
    	
		return 0;
}
\end{lstlisting}



\newpage
\section{3. Wskaźnik pliku i sygnalizator O\_APPEND}

\subsection{Odpowiedz na pytania}

\ask{}{
Dwa deskryptory: fd1 i fd2 użyto do otwarcia pliku podając tę samę ścieżkę dostępu do pliku. Wskaźnik pliku ustawiony jest na początku pliku. Następnie korzystając z deskryptora fd1 wykonano operację zapisania 100b do pliku. Następnie przy użyciu deskryptora fd2 wykonano operację czytania z pliku. Pytanie: Na jakiej pozycji jest wskaźnik pliku? Jakie dane odczytano przy użyciu fd2?
} 
Wskaźnik pliku jest na pozycji po 100b. fd2 odczyta dane, które zapisał fd1, bo dwa różne deskryptory mają niezależne offsety plików.

\ask{}{
Do otwarcia pliku użyto jednego deskryptora fd3. Następnie wykonano kolejno operację pisania 100b i czytania 100b. Na jakiej pozycji jest wskaźnik pliku? Co zostało przeczytane? 
} 
Zostały przeczytane dane, które znajdują się po 100b, które zapisaliśmy, EOF jeśli plik się kończył po 100b, ponieważ fd3 ma jeden offset pliku, który przy zapisaniu jest przesuwany o 100b, a read zaczyna od tego miejsca.


\ask{}{
Czy każdorazowe poprzedzenie operacji pisania ustawieniem wskaźnika pliku na końcu pliku za pomocą funkcji lseek daje taki sam rezultat jak otwarcie pliku w trybie z ustawioną flagą O\_APPEND? Odpowiedź uzasadnij. 
} 
Nie, O\_APPEND gwarantuje każdorazowe zapisywanie na końcu pliku, nawet w sytuacji gdy na pliku działa wiele procesów, co nie jest prawdą w przypadku wywoływanie osobno lseek i write.

\ask{}{
Jak wygląda wywołanie funkcji lseek które:
\begin{itemize}
\item ustawia wskaźnik na zadanej pozycji
\item znajduje koniec pliku
\item zwraca bieżącą pozycję wskaźnika
\end{itemize}
} 
\begin{itemize}
\item ustawia wskaźnik na zadanej pozycji - lseek(fd, pos, SEEK\_SET); 
\item znajduje koniec pliku - lseek(fd, 0, SEEK\_END)
\item zwraca bieżącą pozycję wskaźnika - lseek(fd, 0, SEEK\_CUR)
\end{itemize}

\subsection{Funkcja copyappend.c}
\assignment{}{
Napisz funkcję copy, która będzie za każdym razem dodawać zawartość na końcu pliku.
}

\begin{lstlisting}[style=CStyle, label=some-code, caption=copyappend.c]
void copyAppend(char *from, char *to) 
    {
    	int fromfd = -1, tofd = -1;
    	ssize_t nread;
    	char buf[BUFSIZE];
     
    	fromfd = open(from, O_RDONLY | O_APPEND); // dodanie flagi append
    	if(fromfd == -1){
            if(errno == ENOENT){
                perror("File does not exists: ");
                exit(EXIT_FAILURE);
            }
            if(errno == EACCES){
                perror("Permission denied: ");
                exit(EXIT_FAILURE);
            }
        }

        
        tofd = open(to, O_WRONLY | O_CREAT | O_TRUNC,
    				S_IRUSR | S_IWUSR | O_APPEND); // dodanie flagi append
        if(tofd == -1){
            if(errno == ENOENT){
                perror("File does not exists: ");
                exit(EXIT_FAILURE);
            }
            if(errno == EACCES){
                perror("Permission denied: ");
                exit(EXIT_FAILURE);
            }
        }


    	while ((nread = read(fromfd, buf, sizeof(buf))) > 0)
    	    write(tofd, buf, nread);	
        if(nread == -1){
            perror("Reading failed: ");
        }
            
        close(fromfd);
    	close(tofd);
    	return;
    }

\end{lstlisting}

\ask{}{
Czy zmiana wywołań lseek i read na pread jest równoważna ? 
} 
Nie do końca, ponieważ pread ignoruje offset, jest on ustawiany jawnie przez argument offset.


\section{4. Buforowanie operacji I/O}

\subsection{Porównanie czasu działania kopiowania}

\assignment{}{
Proszę pobrać poniższy program wykorzystujący funkcje ze standardowej biblioteki wejścia-wyjścia. \\
Proszę stworzyć duży plik, np. plik 1MB wypełniony zerami: \\
dd if=/dev/zero of=/path/to/desired/big/file count=10k \\ \\
Korzystając z funkcji mierzących czas z poprzedniego laboratorium proszę napisać program testujący ile czasu zajmują funkcje kopiujące copy2 (copy1 z napisaną przez nas obsługą błędów) i copy3 \\ \\
Proszę napisać program, który liczy czas wykonania funkcji copy2 przy różnych rozmiarach bufora (proszę przyjąć przynajmniej następujące wartości rozmiaru bufora: 1, 512, 1024, 1100)
}


\begin{lstlisting}[style=CStyle, label=some-code, caption=timecmp.c]
compare: t.o timecmp.o
	gcc timecmp.o t.o -o compare -I.

t.o : t.c t.h
	gcc -c -Wall -pedantic t.c -I.

timecmp.o: timecmp.c t.h
	gcc -c -Wall -pedantic timecmp.c -I.

#include <fcntl.h>
#include <unistd.h>
#include <stdio.h>
#include <errno.h> /* add errno.h to get errno variable */
#include <stdlib.h>
#include <t.h>

#define BUFSIZE 1100
 
void copy(char *from, char *to)  /* has a bug */
{
	int fromfd = -1, tofd = -1;
	ssize_t nread;
	char buf[BUFSIZE];
 
	fromfd = open(from, O_RDONLY);
	if(fromfd == -1){
		if(errno == ENOENT ){
			printf("fromfd - Errno is %d", errno); /* No such file or directory */
			perror("Error: ");
			exit(EXIT_FAILURE);
		}
		if(errno == EACCES){
			printf("fromfd - Errno is %d", errno); /* Permission denied */
			perror("Error: ");
			exit(EXIT_FAILURE);
		}
	}
	
	
	tofd = open(to, O_WRONLY | O_CREAT | O_TRUNC,
				S_IRUSR | S_IWUSR);
	if(tofd == -1){
		if(errno == ENOENT )
		{
			printf("to - Errno is %d", errno); /* No such file or directory */
			perror("Error: ");
			close(fromfd);		
			exit(EXIT_FAILURE);
		}
		if(errno == EACCES)
		{
			printf("to - Errno is %d", errno); /* Permission denied */
			perror("Error: ");
			close(fromfd);
			close(tofd);
			exit(EXIT_FAILURE);
		}
	}

	
	
	while ((nread = read(fromfd, buf, sizeof(buf))) > 0)
	    write(tofd, buf, nread);	
 	
	if(nread == -1){
		if(errno == ENOENT )
		{
			printf("read - Errno is %d", errno); /* No such file or directory */
			perror("Error: ");
			close(fromfd);
			close(tofd);
			exit(EXIT_FAILURE);
		}
		if(errno == EACCES)
		{
			printf("read - Errno is %d", errno); /* Permission denied */
			perror("Error: ");
			close(fromfd);
			close(tofd);
			exit(EXIT_FAILURE);
		}
	}
	
    close(fromfd);
	close(tofd);
	return;
}


void copy3(char *from, char *to)
{
	FILE *stfrom, *stto;
	int c;
 
	if ((stfrom = fopen(from, "r") ) == NULL){}
	if(( stto = fopen(to, "w") ) == NULL) {}
	while ((c = getc(stfrom)) != EOF)
		putc(c, stto);
	fclose(stfrom);
	fclose(stto);
	return;
 
}



int main(int argc, char **argv){
	if (argc != 3)
	{
		fprintf(stderr,"usage: %s from_pathname to_pathname\n", argv[0]);
		return 1;
	}
	
	timestart();
	
	copy(argv[1], argv[2]);
	
	timestop("copy2");
	
	
	timestart();
	
	copy3(argv[1], argv[2]);
	
	timestop("copy3");
	
	
	return 0;
}

\end{lstlisting}


Wyniki uzyskane z eksperymentu: 

\begin{lstlisting}[style=CStyle, label=some-code, caption=Logi z programu timecmp.c]

26 lines (21 sloc) 433 Bytes
512: 
copy2:
	"Total (user/sys/real)", 0, 5, 12
	"Child (user/sys)", 0, 0
copy3:
	"Total (user/sys/real)", 50, 1, 78
	"Child (user/sys)", 0, 0


1024: 
copy2:
	"Total (user/sys/real)", 0, 3, 8
	"Child (user/sys)", 0, 0
copy3:
	"Total (user/sys/real)", 50, 1, 74
	"Child (user/sys)", 0, 0


1100: 
copy2:
	"Total (user/sys/real)", 0, 3, 8
	"Child (user/sys)", 0, 0
copy3:
	"Total (user/sys/real)", 53, 0, 78
"Child (user/sys)", 0, 0
\end{lstlisting}

Czas dla 1100 jest zgodnie z przypuszczeniami większy niż czas osiągany dla 512 czy 1024.


\end{document}
